\mychapter{13}{Lesson 13} %181109

\section{Number theory}

\begin{theorem}[Fermat's last theorem]
    \[
        \forall x, y, z \in \integer, n > 2 \implies x^n + y^n \neq z^n
    \]
\end{theorem}

\begin{lemma}
    \[
        \forall a \in \integer_n : \gcd(a, n) > 1 \implies a \notin \integer_n^\times
    \]
\end{lemma}

\begin{proof}
    Assume there exists $b \in \integer_n$ such that $ab \equiv_n 1$. Then, there exists a quotient for the division $by$ between a and b with remainder 1. Observe that $\gcd(a, n)$ divides $ab + qn$, which is equal to 1. It entails that $\gcd(a, n) = 1$, which is a contradiction.
\end{proof}

\begin{lemma}
    \[
        \forall a, b \in \nonneg : a \geq b \neq 0 \implies \gcd(a, b) = \gcd(b, a \mod b)
    \]
\end{lemma}

\begin{proof}
    Not given.
\end{proof}

\begin{theorem}
    Given two integers $a$ and $b$, their greatest common divisor can be computed efficiently with respect to their lengths. Additionally, two other numbers u and v can be efficiently computed in order to satisfy B\'ezout's identity: $\gcd(a, b) = au + bv$
\end{theorem}

\begin{proof}
    Hint: Use previous lemma recursively; we stop at $r_{t + 1} = 0$ for some $t$, thus:
    \[
        \gcd(a, b) = \dots = \gcd(r_t, r_{t + 1}) = r_t
    \]
\end{proof}

\begin{claim}
    $r_{i + 2} \leq r_i/2 \forall 0 \leq i \leq t - 2 \implies \#steps = \lambda - 1 \text{ if } |b| \in \binary^\lambda$
\end{claim}

The hypothesis is a natural consequence of the repeated modulo operations.

\begin{observation}[Exponentiation mod n: Square and multiply]
    Let $b \in \binary^l$, where by writing $b_i$ we denote b's i-th bit. Then:
    \[
        a^b \equiv_n a^{\sum_{i = 0}^l 2^ib_i} \equiv_n \prod_{i = 0}^l a^{2^ib_i} 
    \]
\end{observation}

%AP190913 - Unproven, can't find it anywhere
\begin{theorem}
    The number of primes lesser than or equal to $x$ is a number greater than or equal to $\frac{x}{3\log_2 x}$
\end{theorem}

There exist many algorithms that solve the problem of primality testing, with time complexity polynomial in the length of their numerical representation; of most relevance are those of Miller-Rabin, which is probabilistic, but consistently used in practice, and the completely deterministic Agrawal-Kayal-Saxena (\textsc{aks}), which has a much greater polynomial rank, and has been deemed impractical for most uses.

What remained as a conjecture is the intractability of determining the factors of a $\lambda$-bit composite number, which is widely known as the \emph{factorization hardness}; this in turn would imply that integer multiplication of two $\lambda$-bit primes is a \owf.

\begin{definition}
    Given a group $G$, its order is the least $i$ such that $a^i \equiv_n 1$ 
\end{definition}

\begin{corollary}
    \[
        \forall a \in \integer_m^\times \implies a^{\phi(n)} \equiv_n 1 \wedge a^b \equiv_n a^{b \mod \phi(n)} \wedge a^{p - 1} \equiv_p 1
    \]
\end{corollary}

\begin{proof}
    $(\integer_n^\times, \cdot)$ is a group with $\phi(n)$ elements. Take $\left< a \right> = \{a^0, \dots, a^{d + 1}\}$, where $d$ is the order. We must have $\phi(n) = kd$ for some $k$. Therefore, $a^{\phi(n)} = a^{dk} \equiv_n 1$.

    Also, $a^b \equiv_n a^{b\phi(n) + b \mod \phi(n)} \equiv_n 1 \cdot 2^{b \mod \phi(n)}$
\end{proof}

\begin{theorem}
    Let G, H be two groups such that $H < G$, meaning the order of H divides the order of G
\end{theorem}

\section{Standard model assumptions}

We now turn our attention to some conjectures that form the basis for most of the cryptographic schemes that will follow. We will start from the weakest, and go up to the strongest. For all our purposes, let $\groupgen(1^\lambda)$ be a ``group generator'' with security parameter $\lambda$, and let a random sample $(G, g, q) \pickUAR \groupgen(1^\lambda)$ be a triplet composed of the group itself $G$, one of its generators $g$, and its order $q$.

% CDH -> DDH -> DL
\subsubsection{Discrete logarithm}

Given $g$ and $g^x$ in a $\lambda$-bit group, there is no efficient algorithm for computing $y$ such that $g^y = g^x$ without knowing $x$ beforehand:
\[
    \forall \adversary \in \ppt \implies \Pr[\cryptog{dl}(\lambda) = 1] \in \negl(\lambda)
\]

\begin{cryptogame}
    {dlass}
    {$\cryptog{dl}(\lambda)$}
    {dl}

    \receive{\shortstack[l]{
        %AP190913: Actually, the adversary knows this!
        $(G, g, q) \pickUAR \groupgen(1^\lambda)$ \\
        $x \pickUAR \integer_q$
    }}{$y = g^x$}{}

    \send{}{$x'$}{\textsc{Output 1 iff} $y^{x'} \equiv_G y$}
    
\end{cryptogame}

This means that the \dl{} for a generic group G yields a \owf, whereas in a multiplicative group $\integer_p^\times$, we obtain a \owp.

\subsubsection{Computational Diffie-Hellman}

%AP190913: Probably it is not necessary to compute it directly, we may only need a congruent value
Given a group $G$ and two elements in it $g^x$, $g^y$, it is impractical to compute $g^{xy}$ without knowing $x$ or $y$.

\begin{cryptogame}
    {cdhass}
    {$\cryptog{cdh}(\lambda)$}
    {cdh}

    \receive{\shortstack[l]{
        %AP190913: Actually, the adversary knows this!
        $(G, g, q) \pickUAR \groupgen(1^\lambda)$ \\
        $x, y \pickUAR \integer_q$
    }}{$(g^x, g^y)$}{}

    \send{}{$h$}{\textsc{Output 1 iff} $h = g^{xy}$}
    
\end{cryptogame}

\subsubsection{Decisional Diffie-Hellman}

Given a group $G$ and three elements in it $g^x$, $g^y$, $g^z$, it is impractical to distinguish $g^{xy}$ from $g^z$ by only knowing $g^x$ and $g^y$, along with the originating triad $(G, g, q))$.

\begin{cryptogame}
    {ddhass}
    {$\cryptog{ddh}(\lambda)$}
    {ddh}

    \cseqdelay

    \receive{\shortstack[l]{
        %AP190913: Actually, the adversary knows this!
        $(G, g, q) \pickUAR \groupgen(1^\lambda)$ \\
        $x, y, z_1 \pickUAR \integer_q$ \\
        $z_0 = xy \mod q$ \\
        $b \pickUAR \binary$
    }}{$(g^x, g^y, g^{z_b})$}{}

    \cseqdelay

    \send{}{$b'$}{\textsc{Output 1 iff} $b' = b$}
    
\end{cryptogame}

All these assumptions helped in constructing the \emph{Diffie-Hellman key exchange} protocol, which is a way to establish a \ske{} channel from an unsafe channel, with any adversary unable to efficiently break the channel's secrecy. Do note that authentication is left out of the picture here.

\begin{cryptosequence}
    {dhkex}
    {The Diffie-Hellman Key Exchange protocol}
    
    \cseqentity{A}{Alice}
    \cseqentity[2.2]{B}{Bob}

    \cseqmessagel{B}{\shortstack[l]{
        $(G, g, q) \pickUAR \groupgen(1^\lambda)$ \\
        $x \pickUAR G$ 
    }}{$(G, g, q, g^x)$}{A}{}

    \cseqdelay

    \cseqmessager{A}{\shortstack[l]{
        $y \pickUAR G$ \\
        $k = (g^x)^y$
    }}{$g^y$}{B}{$k = (g^y)^x$}

\end{cryptosequence}

Some relationships have been established between these assumptions: it is known that $\ddh \implies \cdh \implies \dl$; and that $\cdh \notimplies \ddh$.
