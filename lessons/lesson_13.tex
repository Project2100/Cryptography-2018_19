\mychapter{13}{Lesson 13} %181109

\section{Number theory}

\begin{theorem} \emph{Fermat's last theorem}
    \[
        \forall x, y, z \in \integer, n > 2 \implies x^n + y^n \neq z^n
    \]
\end{theorem}

\begin{lemma}
    \[
        \forall a \in \integer_n : \gcd(a, n) > 1 \implies a \notin \integer_n^\times
    \]
\end{lemma}

\begin{proof}
    Assume there exists $b$ in group such that $ab \equiv_n 1$. Then, there exisits a quotient for the division between a and b with remainder 1. Observe that $\gcd(a, n)$ divides $ab + qn$, which is equal to 1. It entails that $\gcd(a, n) = 1$, which is a contradiction.
\end{proof}

\begin{lemma}
    \[
        \forall a, b \in \integer : a \geq b > 0 \implies \gcd(a, b) = \gcd(b, \textsc{rem}_b(a))
    \]
\end{lemma}

\begin{proof}
    
\end{proof}

\begin{theorem}
    Given $a, b$ integers, their greatest common divisor can be computed efficiently wrt a and b's lengths. Additionally, two other numbers u and v can be computed in order to satisfy bezout's identity: $\gcd(a, b) = au + bv$
\end{theorem}

\begin{proof}
    Hint: Use previous lemma recursively...
\end{proof}

Claim: $r_{i+2} \leq r_i/2 \forall o \leq i \leq t-2 \implies \#steps = \lambda - 1 \text{ if } |b| \in \binary^\lambda$

Proof: ...


\begin{definition}
    Exponentiation mod n: Square and multiply

    Let $b in \binary^l$, where by writing $b_i$ we denote b's i-th bit. Then:
    \[
        a^b \equiv_n a^{\sum_{i=0}^l 2^ib_i} \equiv_n \prod_{i=0}^l a^{2^ib_i} 
    \]
\end{definition}

\begin{theorem}
    The number of primes lesser than or equal to x is a number greater than or equal to $x/3\log_2x$
\end{theorem}

\begin{theorem} (Milner-Rabin - AKS)
    We can est in polytime if a random $\lambda$-bit number is prime
\end{theorem}

Conjecture: Integer multiplication of two lambda-bit primes is a \owf. (is this the factorization assumption?)

\section{Standard model assumptions}

Given a group G, its order is the least i such that $a^i \equiv_n 1$ 

Corollary: $\forall a \in \integer_m^\times \implies a^{\phi(n)} \equiv_n 1 \wedge a^b \equiv_n a^{rem_{\phi(n)}(b)}$

\begin{theorem}
    Let G, H be two groups such that $H < G$, meaning the order of H divides the order of G
\end{theorem}


% CDH -> DDH -> DL
\subsubsection{Discrete logarithm}

Given $g$ and $g^x$ in a n-bit group, there is no efficient algorithm for computing $y$ such that $g^y = g^x$ withouut knowing $x$ beforehand.

\begin{cryptogame}
    {dlass}
    {}
    {dl}

    \receive{\shortstack[l]{
        $(G, g, q) \pickUAR \Group\Gen(1^\lambda)$ \\
        $x \pickUAR \integer_q$
    }}{$y = g^x$}{}

    \send{}{$x'$}{\textsc{Output 1 iff} $y^{x'} \equiv_G y$}
    
\end{cryptogame}

Meaning, DL for a generic group G is a \owf, whereas in a multiplicative group $\integer_p^\times$, DL is a \owp.

\subsubsection{Computational Diffie-Hellman}

Statement: given a group, and two elements of it $g^x$, $g^y$, it is impractical to compute $g^{xy}$ without knowing both $x$ and $y$.

\begin{cryptogame}
    {cdhass}
    {}
    {cdh}

    \receive{\shortstack[l]{
        $(G, g, q) \pickUAR \Group\Gen(1^\lambda)$ \\
        $x, y \pickUAR \integer_q$
    }}{$(g^x, g^y)$}{}

    \send{}{$h$}{\textsc{Output 1 iff} $h = g^{xy}$}
    
\end{cryptogame}

\subsubsection{Decisional Diffie-Hellman}

Statement: see game

\begin{cryptogame}
    {ddhass}
    {}
    {ddh}

    \cseqdelay

    \receive{\shortstack[l]{
        $(G, g, q) \pickUAR \Group\Gen(1^\lambda)$ \\
        $x, y \pickUAR \integer_q$ \\
        $z_0 = xy$ \\
        $z_1 \pickUAR G$ \\
        $b \pickUAR \binary$
    }}{$(g^x, g^y, g^{z_b})$}{}

    \cseqdelay

    \send{}{$b'$}{\textsc{Output 1 iff} $b' = b$}
    
\end{cryptogame}

All these assumptions arose in definining what is the Diffie Hellman key exchange, which is a way to establish a \ske{} channel from an unsafe channel, with any adversary unable to efficiently break the channel's secrecy. Party authentication is not considered here.

% AP190208: This is not a game (Magritte ahoy)
\begin{cryptogame}
    {dhkex}
    {The Diffie-Hellman Key Exchange protocol}
    {}

    \receive{\shortstack[l]{
        $(G, g, q) \pickUAR \Group\Gen(1^\lambda)$ \\
        $x \pickUAR G$ 
    }}{$(G, g, q, g^x)$}{}

    \cseqdelay

    \send{\shortstack[l]{
        $y \pickUAR G$ \\
        $k = (g^x)^y$
    }}{$g^y$}{$k = (g^y)^x$}
    
\end{cryptogame}

Some relationships have been established between these assumptions: it is known that $\ddh \implies \cdh \implies \dl$. Also, $\cdh \notimplies \ddh$.
