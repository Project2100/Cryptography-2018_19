\mychapter{16}{Lesson 16} %181121

\section{\textsc{Pke} schemes over \ddh{} assumption}

\subsection{ElGamal scheme}

Let's define a new $\Pi = (\keygen, \Enc, \Dec)$. Generate the needed public parameters $(G,g,q) \pickUAR \groupgen(1^{\lambda})$\footnotemark, then:

\footnotetext{G could be any valid group such as $\quadres_p$, or an elliptic curve group}

\begin{itemize}
    \item Key generation: $(\pk, \sk) = (g^x, x)$, where $x \pickUAR \integer_q$
    \item Encryption: $\Enc(\pk, m) = (g^r, \pk^r \cdot m)$, where $r \pickUAR \integer_q$\footnote{We need $r$ because we want to re-randomize $c$}
    \item Decryption: $\Dec(\sk, (c_1, c_2))= c_1^{-\sk} \cdot c_2$
\end{itemize}

The correctness of the scheme follows from some algebraic steps:
\begin{align*}
    \hat{m} &= \Dec(\sk, \Enc(\pk, m))          \\
            &= \Dec(x, \Enc(g^x, m))            \\
            &= \Dec(x, (g^r, (g^x)^r \cdot m))  \\
            &= (g^r)^{-x} \cdot (g^x)^r \cdot m \\
            &= m                                \\
\end{align*}

\begin{theorem}
    Assuming \textsc{ddh}, the ElGamal scheme is \textsc{cpa}-secure.
\end{theorem}

\begin{proof}
    Consider the two following games $\hybridg{0}(\lambda, b)$ and $\hybridg{1}(\lambda, b)$ defined as follows. Observe that $b$ can be fixed without loss of generality.
    
    \begin{cryptogame}
        {pke1}
        {$\hybridg{0}(\lambda, b) = \cryptog{gamal}(\lambda, b)$}
        {gamal}

        \receive{\shortstack[l]{
            $x \pickUAR \integer_q$ \\
            $\pk = g^x$
        }}{$\pk$}{}

        \cseqdelay
        \send{}{$m_0, m_1$}{}
        \cseqdelay
        \receive{\shortstack[l]{
            $r \pickUAR \integer_q$ \\
            $b \pickUAR \binary$ \\
            $c = \Enc(\pk, m_b) = (g^r, (g^x)^r m_b)$
        }}{$c$}{}
        \cseqdelay

        \send{}{$b'$}{\textsc{Output 1 iff} $b' = b$}

    \end{cryptogame}

    \begin{cryptogame}
        {pke2}
        {$\hybridg{1}(\lambda, b)$}
        {$\neg$gamal}

        \receive{\shortstack[l]{
            $x \pickUAR \integer_q$ \\
            $\pk = g^x$
        }}{$\pk$}{}

        \cseqdelay
        \send{}{$m_0, m_1$}{}
        \cseqdelay
        \receive{\shortstack[l]{
            $r, z \pickUAR \integer_q$ \\
            $c = (g^r, g^z m_b)$
        }}{$c$}{}
        \cseqdelay

        \send{}{$b'$}{\textsc{Output 1 iff} $b' = b$}

    \end{cryptogame}
    

    % The reasoning was a bit cloudy, hope now it is clearer
    %AP190915: Not so sure...
    \textbf{Note:} it is important to note that we can measure the advantage of $\adversary$, so fixed its output $Adv_{\adversary}(\lambda) = |\underbrace{\Pr[\overbrace{\adversary = 1}^{b' = 1} \knowing b = 0}_{\adversary\ loses}] - \underbrace{\Pr[\overbrace{\adversary = 1}^{b' = 1} \knowing b = 1]}_{\adversary\ wins}|$. Since $b$ is fixed the above formula will give a value $\lambda \in negl$, generally the advantage of an adversary is: $\frac{1}{2}+\lambda$ (random guessing + a negligible factor). %maybe this is a repetition but I think I never wrote it anywhere else

    \todo{Here the 1/2 value may refer to how the Katz-Lindell book exposes their proofs by leaving an arbitrary choice of $b$ to the challengers, thus limiting the probability of the adversaries' success by 1/2.
    
    Venturi prefers to fix $b$ beforehand, meaning the whole proof can be stated by setting b to either 0 or 1, and maintain its validity without changing anything else. In this way, he somehow bounds the probability of success to be $\leq \negl(\lambda)$}

    We will prove that:
    \[
        \hybridg{0}(\lambda, 0) \compindist \hybridg{1}(\lambda, 0) \equiv \hybridg{1}(\lambda, 1) \compindist \hybridg{0}(\lambda, 1)
    \]

    \begin{claim}
        $\forall b \in \binary \implies \hybridg{0}(\lambda, b) \compindist \hybridg{1}(\lambda, b)$
    \end{claim}

    \begin{proof}
        This proof reduces to disproving \ddh. Fix $b$ without loss of generality, and assume there exists a distinguisher $\distinguisher$ able to distinguish $H_0(\lambda,b)$ and $H_1(\lambda,b)$. Consider the game in figure \ref{cryptoredux:gamalddh}:

        \begin{cryptoredux}
            {gamalddh}
            {---}
            {ddh}
            {gamal}

            \receive{\shortstack[l]{
                $X = g^x, Y = g^y$ \\
                $Z_0 = g^{xy}, Z_1 = g^z$ \\
                $b \pickUAR \binary$
            }}{$(X, Y, Z_b)$}{}

            \invoke{}{$\pk = X$}{}
            \cseqdelay
            \return{}{$(m_0, m_1)$}{}
            \invoke{$\boxed{a = 0}$ }{$(Y, Z_b \cdot m_a)$}{}
            \cseqdelay
            \return{}{$a'$}{}

            \cseqdelay

            \send{$b' = \begin{cases}
                    0 & \textsc{iff} \boxed{a' = 0} \\
                    1 (coin?) & \textsc{else} \\
                \end{cases}
            $}{$b'$}{\textsc{Output 1 iff} $b' = b$}

        \end{cryptoredux}

        If $\distinguisher^{\textsc{gamal}}$ is able to break the cipher, then \adversary{} is in turn able to distinguish a \textsc{dh} triple from a random one, falsifying \ddh.
    \end{proof}

    \begin{claim}
        $\hybridg{1}(\lambda, 0) \equiv \hybridg{1}(\lambda, 1)$
    \end{claim}

    \begin{proof}
        This follows from the fact that:
        \[
            (g^x, (g^r, g^z m_0)) \equiv (g^x, (g^r, U_\lambda) \equiv (g^x, (g^r, g^z m_1)))
        \]
    \end{proof}
    
    Composing the two claims, the proof is complete.

\end{proof}

\subsubsection{Properties of of El Gamal \textsc{pke} scheme}

Some useful observations can be made about this scheme:
\begin{itemize}
    \item It is \textbf{homomorphic}: Given two ciphertexts $(c_1, c_2)$ and $(c_1', c_2')$, then doing the product between them yields another valid ciphertext:
    % I'd like to expand a bit more here, or else it may stay a bit obscure...
    \begin{align*}
        & (c_1 \cdot c_1', c_2 \cdot c_2') \\
        =& (g^{r + r'}, h^{r + r'}(m \cdot m'))
    \end{align*}
    thus, decrypting $c \cdot c'$, gives $m \cdot m'$.
    
    \item It is \textbf{re-randomizable}: Given a ciphertext $(c_1, c_2)$, and $r' \pickUAR \integer_q$, then computing $(g^{r'} \cdot c_1, h^{r'} \cdot c_2)$ results in a ``fresh'' encryption for the same message: the random value used at the encryption step will change from the original $r$ to $r + r'$
\end{itemize}
These observations lead to the conclusion that this scheme is not \cca-secure. However, such properties can be desirable in some use cases, where a message must be kept secret to the second party. In fact, there are some \pke{} schemes which are designed to be \textbf{fully homomorphic}, i.e. they are homomorphic for any kind of function.

Consider the following use case: a client $C$ has an object $x$ and wants to apply a function $f$ over it, but it lacks the computational power to execute it. There is another subject $S$, which is able to efficiently compute $f$, so the goal is to let it compute $f(x)$ but the client wishes to keep $x$ secret from him. This can be achieved using a \textsc{fh-pke} scheme as follows:

% QUESTION: Why use a PKE scheme (and not an SKE)? Are there some advantages-shortcomings?

\begin{cryptosequence}
    {delseccomp}
    {Delegated secret computation}

    \cseqentity{C}{Client}
    \cseqentity[2.2]{S}{Server}

    \cseqdelay

    % Client encrypts object and sends application request
    \cseqmessager{C}{\shortstack[r]{
        Object: $x$ \\
        Function: $f$ \\
        $(\pk, \sk) \pickUAR \mathcal{KG}en$ \\
        $c \pickUAR \Enc(\pk, x)$
    }}{$f, c$}{S}{}

    \cseqdelay
    \cseqdelay

    \cseqmessagel{S}{}{$f(c)$}{C}{$f(x) = \Dec(\sk, f(c))$}

\end{cryptosequence}


However one important consideration must be made: All these useful characteristics expose an inherent malleability of any fully homomorphic scheme: any attacker can manipulate ciphertexts efficiently, and with some predictable results. This compromises even \textsc{cpa} security of such schemes.

%\subsection{Cramer-Shoup \textsc{pke} scheme}
\section{Proof systems}

% Not sure if CS is defined over DDH, or this is the DDH variant of CS, making CS more abstract
This new scheme assumes \ddh{} like its ElGamal cousin, and has the advantage of being \cca-secure. A powerful tool, called \emph{Designated Verifier Non-Interactive Zero-Knowledge} (\textsc{dv-nizk} in short), or alternatively \emph{Hash-Proof System}, is used here.


\todo{the Ali Baba example would be useful here. Also, the coloured balls is a good example too (long live Wikipedia)}

Let $L \subseteq Y$ be a Turing-recognizable language in \textsc{np}, and a predicate $V \in X \times Y \to \binary$ such that:
\[
    L := \{y \in Y : \exists x \in X \implies V(x, y) = 1\}
\]
where $x$ is called a \emph{witness} of $y$.

\todo{to review and understand/better}

% AP181122-1554: CAUTION FRAGMENTED
In our case, let $x = (p, q), y = pq$, and define a scheme $\Pi$ as follows:

\[
    \Pi = (\mathcal{S}etup, \mathcal{P}rove, \mathcal{V}erify)
\]

% AP190914: Is (omega, tau) the (pk, sk) couple?
\begin{itemize}
    \item Setting up: $(\omega, \tau) \pickUAR \mathcal{S}etup(1^\lambda)$, where $\omega$ is the \emph{common reference string}, and $\tau$ is the \emph{trapdoor}
    \item Proving a statement: $\pi = \mathcal{P}rove(\omega, y, x)$
    \item Verifying: $\widehat{\pi} = \mathcal{V}erify(\tau, y)$
\end{itemize}

Correctness is defined as $\mathcal{P}rove(\omega, y, x) = \mathcal{V}erify(\tau, y)$. Some more observations:

\begin{itemize}
    \item $\omega$ is public ($ = pk$)
    \item $\tau$ is part of the secret key
    \item $\tau = (x, y) : V(x, y) = 1$ %May be false
    \item There is presumably a common third-party, which samples from the setup and publishes $\omega$, while giving $\tau$ to only B.
    \todo{Why? Can't the verifier do the setup and publish omega directly? Is honesty an issue here?}
\end{itemize}

\begin{cryptosequence}
    {csoverview}
    {Overview of Cramer-Shoup operation}
    
    \cseqentity{P}{Prover}
    \cseqentity[2.2]{V}{Verifier}

    \cseqdelay

    \cseqmessager{P}{\shortstack[r]{
        $\omega$\\
        $(x,y)$ s.t. $R(x,y)=1$
    }}{$\pi = \mathcal{P}rove(\omega, y, x)$}{V}{\shortstack[l]{
        $\tau, y' \in L$\\
        $Ver(\tau,y)=\tilde{\pi}\in P$\\
        check if $\pi = \tilde{\pi}$
        %Object: $x$ \\
        %Function: $f$ \\
        %$(pk, sk) \pickUAR \mathcal{KG}en$ \\
        %$c \pickUAR Enc(pk, x)$
    }}
    
\end{cryptosequence}

The purpose of a proof system is to give a way to convince someone (the ``verifier'') that someone else (the ``prover'') knows something (the ``statement'' $y$), and nothing more, to no one else. The proof can be computed in two different ways, this is the core notion of \emph{zero-knowledge}, $\neg \tau \implies \textsc{zk}$

\subsubsection{Properties}

\begin{itemize}
    \item \textit{honest people} \textbf{Completeness}:
    \[
        \forall y \in L, \forall (\omega, \tau) \pickUAR \mathcal{S}etup(1^\lambda) \implies \mathcal{P}rove(\omega, y, x) = \mathcal{V}erify(\tau, y)
    \]
    \item \textit{(stronger, against malicious prover)} \textbf{Soundness}: It is hard to produce a valid proof for any $y \notin L$
    \item \textit{(characteristic, against malicious verifier)} \textbf{Zero-knowledge}: Proof for $x$ can be simulated without knowing $x$ itself
\end{itemize}


\begin{definition}[$t$-universality, or $t - 1$ simulation soundness]
    % Designated verifier non-interactive zero-knowledge
    Let $\Pi$ be a \textsc{dv-nizk} scheme; it is \textit{t-universal} iff for any distinct $y_{1 \upto t} \notin L$ we have: 
    \[
        (\omega, \mathcal{V}erify(\tau, y_1), \ldots, \mathcal{V}erify(\tau, y_{t})) = (\omega, v_1, \ldots, v_t)
    \]
    where $(\omega, \tau) \pickUAR \mathcal{S}etup(1^\lambda)$ and $v_{1 \upto t} \pickUAR \mathcal{P}roof$, where $\mathcal{P}roof$ is the proofs' space.
\end{definition}

\subsubsection{Enriching a proof system}

We can ``enrich'' a \textsc{dv-nizk} scheme with labels $l \in \binary^*$. Suppose to have the following:
\[
    L' = L \| \binary^* = \{(y, l) \in L \times \binary^*\}
\]
Then our scheme changes, because the statements are now compositions of a label $l$ along with te actual statement $y$; for the \textit{t-universality} property, we can now consider two distinct $(y_i, l_i)$.

\subsubsection{Membership-hard language}

\begin{definition}
    Language $L$ is \emph{membership-hard} iff there exists a language $\overline{L}$ such that:
    \begin{enumerate}
        \item $L \cap \overline{L} = \emptyset$
        \item $\exists \textsf{\textup{Sample}} \in \ppt$ outputting $y \pickUAR \mathcal{Y}$ together with $x \in \overline{X}$ such that $R(y, x) = 1$
        
        (therefore $(y, x) \pickUAR \textsf{\textup{Sample}}(1^{\lambda})$)
        \item $\exists \overline{\textsf{\textup{Sample}}} \in \ppt$ outputting $y \pickUAR \overline{L}$
        \item $\{y : (y, x) \pickUAR \textsf{\textup{Sample}}(1^{\lambda})\} \compindist \{y : y \pickUAR \overline{\textsf{\textup{Sample}}}(1^{\lambda})\}$
    \end{enumerate}
    
\end{definition}

