\mychapter{16}{Lesson 16} %181121
\section{\textsc{Pke} schemes over \textsc{ddh} assumption}
\subsection{El Gamal scheme}
Let's define a new $\Pi=$ (KGen, Enc, Dec). Generate the needed (\textbf{public}) parameters $(G,g,q)\pickUAR GroupGen(1^{\lambda})$\footnote{G could be any "valid" group such as $\QR[p]$ or an Elliptic Curve}.

The KeyGen algorithm is defined as follows:
\begin{itemize}
    \item Pick $x \pickUAR \mathbb{Z}_q$
    \item Output the key pair $(pk, sk)$ as $(g^x, x)$
\end{itemize}
The encryption routine Enc($pk, m$) will:
\begin{itemize}
    \item Pick $r\pickUAR \mathbb{Z}_q$
    \item Output $c=(c_1, c_2)=(g^r, pk^r \cdot m)$\footnote{We need $r$ because we want to re-randomize $c$}
\end{itemize}
The decryption routine Dec($sk, c$) will:
\begin{itemize}
    \item Compute $\hat{m} = c_1^{-sk} \cdot c_2$
\end{itemize}
Correctness of this scheme follows from some algebric steps:
\begin{align*}
    \hat{m} &= Dec(sk, Enc(pk, m)) \\
    &= Dec(x, Enc(g^x, m)) \\
    &= Dec(x, (g^r, (g^x)^r \cdot m)) \\
    &= (g^r)^{-x} \cdot (g^x)^r \cdot m \\
    &= m
\end{align*}

\begin{theorem}
    Assuming \textsc{ddh}, the El Gamal scheme is \textsc{cpa}-secure.
\end{theorem}

\begin{proof}
    Consider the two following games $H_0(\lambda,b)$ and $H_1(\lambda, b)$ defined as follows. Observe that b can be fixed without loss of generality.
    
    \newpage
    $H_0(\lambda,b):$
    \begin{figure}[htp]
        \centering
        \sdinit{}
        \begin{tikzpicture}
            \sdbegin{}
            \newinst{A}{$ \A $}
            \newinst[4]{B}{$ \C^\textsc{pke} $}
           
            \mess{B}{$pk$}{A}
            \node[anchor=west] at (mess from) { \shortstack[l]{
                $x \pickUAR \mathbb{Z}_q$ \\
                $pk = g^x$
            }};

            \postlevel
            \mess{A}{$(m_0,m_1)\in \G$}{B}
            \node[anchor=west] at (mess to) { };
            \mess{B}{$ c = (c_1, c_2)$}{A}
            \node[anchor=west] at (mess from) {  \shortstack[l]{
                $r \pickUAR \mathbb{Z}_q$ \\
                $b \pickUAR \{0, 1\}$ \\
                $(c_1,c_2)=(g^r,(g^x)^r m_b)$
            }};

            \postlevel
            \mess{A}{$b'$}{B}
            \node[anchor=west] at (mess to) { \shortstack[l]{
                \textsc{Output} 1 iff $b=b'$
            }};

            \sdend{}
        \end{tikzpicture}
    \end{figure}

    $H_1(\lambda,b):$
    \begin{figure}[h]
        \centering
        \sdinit{}
        \begin{tikzpicture}
            \sdbegin{}
            \newinst{A}{$ \A $}
            \newinst[4]{B}{$ \C^\textsc{pke} $}
           
            \mess{B}{$pk$}{A}
            \node[anchor=west] at (mess from) { \shortstack[l]{
                $x \pickUAR \mathbb{Z}_q$ \\
                $pk = g^x$
            }};

            \postlevel
            \mess{A}{$(m_0,m_1)\in \G$}{B}
            \node[anchor=west] at (mess to) { };
            \mess{B}{$ c = (c_1, c_2)$}{A}
            \node[anchor=west] at (mess from) {  \shortstack[l]{
                $r \pickUAR \mathbb{Z}_q$ \\
                $z \pickUAR \mathbb{Z}_q$ \\
                $(c_1,c_2)=(g^r,g^z m_b)$
            }};

            \postlevel
            \mess{A}{$b'$}{B}
            \node[anchor=west] at (mess to) { \shortstack[l]{
                \textsc{Output} 1 iff $b=b'$
            }};

            \sdend{}
        \end{tikzpicture}
    \end{figure}

    % The reasoning was a bit cloudy, hope now it is clearer
    \textbf{Note:} it is important to note that we can measure the advantage of $\A$, so fixed its output $Adv_{\A}(\lambda) = |\underbrace{Pr[\overbrace{\A\rightarrow 1}^{b'=1}|b=0}_{"\A\ loses"}]-\underbrace{Pr[\overbrace{\A\rightarrow 1}^{b'=1}|b=1]}_{"\A\ wins"}|$. Since $b$ is fixed the above formula will give a value $\lambda \ in negl$, generally the advantage of an adversary is: $\frac{1}{2}+\lambda$ (random guessing + a negligible factor). %maybe this is a repetition but I think I never wrote it anywhere else

    % \todo{Here the 1/2 value may refer to how the Katz-Lindell book exposes their proofs by leaving arbitrary choice of $b$ to the challengers, thus limiting the probability of the adversaries' success by 1/2.
    
    % Venturi prefers to fix $b$ beforehand, without loss of generality, meaning the whole proof can be stated by setting b to either 0 or 1, and mantain its validity without changing anything else. In this way, he somehow bounds the probability of success to be $\leq \mathcal{N}egl(\lambda)$}

    \underline{Proof techinique:}
    $H_0(\lambda,0) \approx_c H_0(\lambda,1)\equiv H_1(\lambda,0) \approx_c H_1(\lambda,1)$
    $$\implies H_0(\lambda,0) \approx_c H_1(\lambda,1)$$

    \begin{lemma}
        $\forall b \in \{0,1\}, H_0(\lambda,0) \approx_c H_0(\lambda,1)$\\
        Fix b. (Reduction to DDH)\\
        Assume $\exists$ PPT D which is able to distinguish $H_0(\lambda,b)$ and $H_1(\lambda,b)$ with non negl. probability.
    \end{lemma}

    \begin{proof}

        Consider the following Game:

        \begin{figure}[ht!]
            \centering
            \sdinit{}
            \begin{tikzpicture}
               \sdbegin{}
               \newinst{A}{$ \D $}
               \newinst[2]{B}{$ \D'_{DDH} $}
               \newinst[2]{C}{$ \C_{DDH} $}

               \postlevel
               \mess{C}{$(X,Y,Z)$}{B}
               \node[anchor=west] at (mess from) { \shortstack[l]{
                   $X=g^x, Y=g^y$\\
                   $Z=g^{xr}$ or $Z=g^z$}};
               \postlevel
               \mess{B}{$h=X$}{A}
               \node[anchor=west] at (mess to) {  };
               \postlevel
               \mess{A}{$ (m_0,m_1) $}{B}
               \node[anchor=west] at (mess from) { };
               \postlevel
               \mess{B}{$(Y,Z m_b)$}{A}
               \node[anchor=west] at (mess to) {  };
               \postlevel
               \mess{A}{$b'$}{B}
               \node[anchor=west] at (mess to) {  };
               \postlevel
               \mess{B}{$b'$}{C}
               \node[anchor=west] at (mess to) {  };
              
               \sdend{}
               \sdend{}
            \end{tikzpicture}
        \end{figure}

        \textbf{Contradiction}: $\D$ should be able to compute $log_g$ to distinguish the message.

    \end{proof}

    \begin{lemma}
        $H_1(\lambda,0)\equiv H_1(\lambda,1)$
    \end{lemma}
    \begin{proof}
        This follows from the fact that:
        $(g^x,(g^r,g^z m_0))\equiv (g^x,(g^r,U_{\lambda}) \equiv \\ \equiv (g^x,(g^r,g^z m_1)))$
    \end{proof}
    
    \begin{lemma}
        $H_1(\lambda,1)\equiv H_0(\lambda,1)$
    \end{lemma}
    This is proved in the exact same way as \textbf{Lemma 20.} As a matter of fact it is the second part of the proof (where $b$ is fixed to 1).
    
\end{proof}

\subsubsection{Properties of of El Gamal \textsc{pke} scheme}

Some useful observations can be made about this scheme:
\begin{itemize}
    \item It is \textbf{homomorphic}: Given two ciphertexts $(c_1, c_2)$ and $(c_1', c_2')$, then doing the product between them yields another valid ciphertext:
    % I'd like to expand a bit more here, or else it may stay a bit obscure...
    \begin{align*}
        & (c_1\cdot c_1', c_2\cdot c_2') \\
        =& (g^{r+r'}, h^{r+r'}(m\cdot m'))
    \end{align*}
    thus, decrypting $c\cdot c'$, gives $m\cdot m'$.
    
    \item It is \textbf{re-randomizable}: Given a ciphertext $(c_1, c_2)$, and $r' \pickUAR \mathbb{Z}_q$, then computing $(g^{r'}\cdot c_1, h^{r'}\cdot c_2)$ results in a ``fresh'' encryption for the same message: the random value used at the encryption step will change from the original $r$ to $r+r'$
\end{itemize}
\textbf{So this is not CCA-Secure!}

These properties of the El Gamal scheme can be desirable in some use cases, where a message must be kept secret to the second party. In fact, there are some \textsc{pke} schemes which are designed to be \textbf{fully homomorphic}, i.e. they are homomorphic for any kind of function.

Consider the following use case: a client $C$ has an object $x$ and wants to apply a function $f$ over it, but it lacks the computational power to execute it. There is another subject $S$, which is able to efficiently compute $f$, so the goal is to let it compute $f(x)$ but the client wishes to keep $x$ secret from him. This can be achieved using a \textsc{fh-pke} scheme as follows:

% QUESTION: Why use a PKE scheme (and not a SKE)? Are there some advantages-shortcomings?

\begin{figure}[h!]
    \centering
    \sdinit{}
    \begin{tikzpicture}
        \sdbegin{}

        % Parties
        \newinst{Client}{$C$}     %Client
        \newinst[4]{Server}{$S$}  %Server

        \postlevel

        % Client encrypts object and sends application request
        \mess{Client}{$f, c$}{Server}
        \node[anchor=east] at (mess from) { \shortstack[r]{
            Object: $x$ \\
            Function: $f$ \\
            $(pk, sk) \pickUAR \mathcal{KG}en$ \\
            $c \pickUAR Enc(pk, x)$
        }};

        \postlevel
        \postlevel

        % Server replies with the obtained image, then client decrypts the result
        \mess{Server}{$f(c)$}{Client}
        \node[anchor=east] at (mess to) { \shortstack[r]{
            $f(x) = Dec(sk, f(c))$
        }};

        \sdend{}
    \end{tikzpicture}
    \caption{Delegated secret computation}
    \label{seq:delseccomp}
\end{figure}

However one important consideration must be made: All these useful characteristics expose an inherent malleability of any fully homomorphic scheme: any attacker can manipulate ciphertexts efficiently, and with some predictable results. This compromises even \textsc{cpa} security of such schemes.

\subsection{Cramer-Shoup \textsc{pke} scheme}

% Not sure if CS is defined over DDH, or this is the DDH variant of CS, making CS more abstract
This scheme is based on the standard \textsc{ddh} assumption, and has the advantage of being \textsc{cca} secure. A powerful tool, called \textbf{Designated Verifier Non-Interactive Zero-Knowledge (DVNIZK)}, or alternatively \textbf{Hash-Proof System}, is used here.

\subsubsection{Proof systems}

Let $L$ be a Turing-recognizable language in $NP$, and a predicate $\mathcal{R} \in X \times Y \to \{0, 1\}$ such that:
\begin{equation*}
    L := \{y \in Y : \exists x \in X \: \mathcal{R}(x, y) = 1\}
\end{equation*}
where $x$ is called a ``witness'' of $y$.

% AP181122-1554: CAUTION FRAGMENTED
In our instance, let $y = pq, x = (p,q)$

$\Pi = (\mathcal{S}etup, \mathcal{P}rove, \mathcal{V}erify)$

$(\omega, \tau) \pickUAR \mathcal{S}etup(1^\lambda)$, where $\omega$ is the \textbf{Common Reference String}, and $\tau$ is the \textbf{trapdoor}.

Additional notes:
\begin{itemize}
    \item $\omega$ is public ($ = pk$)
    \item $\tau$ is part of the secret key
    \item $\tau = (x, y) : \mathcal{R}(x, y) = 1$ %May be false
    \item There is presumably a common third-party, which samples from the setup and publishes $\omega$, while giving $\tau$ to only B. %Why? Can't B do the setup and publish omega directly?
\end{itemize}

\begin{figure}[h!]
    \centering
    \sdinit{}
    \begin{tikzpicture}
        \sdbegin{}

        % Parties
        \newinst{A}{$A$}     %Prover
        \newinst[4]{B}{$B$}  %Verifier

        \postlevel

        % Client encrypts object and sends application request
        \mess{A}{$\pi = \mathcal{P}rove(\omega, y, x)$}{B}
        \node[anchor=east] at (mess from) { \shortstack[r]{
            $\omega$\\
            $(x,y)$ s.t. $R(x,y)=1$
        }};
        \node[anchor=west] at (mess to) { \shortstack[l]{
            $\tau, y' \in L$\\
            $Ver(\tau,y)=\tilde{\pi}\in P$\\
            check if $\pi = \tilde{\pi}$
            %Object: $x$ \\
            %Function: $f$ \\
            %$(pk, sk) \pickUAR \mathcal{KG}en$ \\
            %$c \pickUAR Enc(pk, x)$
        }};

        \sdend{}
    \end{tikzpicture}
    \caption{Overview of Cramer-Shoup operation}
    \label{seq:csoverview}
\end{figure}

Proof system - purpose: a way to convince $B$ that $A$ knows something

Can compute the proof in two different ways, this is the core notion of $ZK$

No $\tau \implies ZK$

\subsubsection{Properties}

\begin{itemize}
    \item \textit{(implicit, against malicious Bob)} \textbf{Zero-knowledge}: Proof for $x$ can be simulated without knowing $x$ itself
    \item \textit{(stronger, against malicious ALice)} \textbf{Soundness}: It is hard to produce a valid proof for any $y \notin L$
    \item \textit{honest people} \textbf{Completeness}: $\forall y \in L, \forall (\omega, \tau) \pickUAR \mathcal{S}etup(1^\lambda):$
    
    $\mathcal{P}rove(\omega, x, y) = \mathcal{V}erify(\tau, y)$
\end{itemize}

\todo{to review and understand/better}

\subsubsection{t-universality}
\begin{definition}
    Let $\Pi$ be DV-NIKZ\footnote{Designated verifier non-interactive
    zero-knowledge}.\\
We say it is \textit{t-universal} if for any distinct 
\[
    y_{1}, \ldots, y_{t} \text{ s.t. } y_{i}\not\in L ( \forall i \in [t])
\]
we have 
\[
    ( \omega, Ver(\tau, y_{1}), \ldots, Ver(\tau, y_{t}))=(\omega, v_{1},
    \ldots, v_{t})
\]
where $(\omega, \tau) \pickUAR Setup(1^{\lambda})$ and $v_{1}, \ldots, v_{t}
\pickUAR \P$ where $\P$ should be the proofs' space.
\end{definition}

\subsubsection{Enriching DV-NIKZ}
Can we enrich DV-NIKZ with labels $l \in \{0,1\}^{*}$?

Suppose to have the following:
\[
    L'=L \| \{0,1\}^{*}=\{(y,l): y \in L \wedge l \in \{0,1\}^{*}\}
\]
Then our scheme changes :
$Prove(\omega, (y,l), x)=\Pi; Ver(\tau, (y, l)) $
and , for \textit{t-universality} , now we can consider 2 distinct $(y_{i},
l_{i})$.
\subsubsection{Membership Hard Language (MH)}
\begin{definition}
    Language $L$ is \textit{ \textbf{MH}}if $ \exists  \bar{L}$ such that:
    \begin{enumerate}
        \item $L \cap \bar{L} = \emptyset$
        \item $ \exists$ PPT \textit{Samp} outputting $y \pickUAR \Y$
            together with $x \in \bar{\X}$ such that 
            \[
              R(y,x)=1  
            \]
            (it's possible
            to say that $Samp(1^{\lambda}) \pickUAR (y,x)$)
        \item $ \exists$ PPT $\bar{Samp}$ outputting $y \pickUAR \bar{L}$
        \item $\{y:(y,x) \pickUAR Samp(1^{\lambda})\} \approx_{c} \{y:y
            \pickUAR \bar{Samp}(1^{\lambda})\}$
    \end{enumerate}
    
\end{definition}

