\mychapter{18}{Lesson 18} %181128

\subsubsection{Cramer-Shoup scheme construction}

From the above two proof systems we can construct a \pke{} scheme, which is attributed to Cramer and Shoup:

\todo{split definition from correctness}

\begin{itemize}
    \item $(\pk, \sk) \pickUAR \keygen$, where:
    \begin{itemize}
        \item $\pk := (h_1, h_2, h_3) = (g_1^{x_1} g_2^{x_2}, g_1^{x_3} g_2^{x_4}, g_1^{x_5} g_2^{x_6})$
        \item $\sk := (x_1, x_2, x_3, x_4, x5, x_6)$
    \end{itemize}
    \item Encryption procedure:
    \begin{itemize}
        \item $r \pickUAR \integer_q$
        \item $\beta = h_s(c_1, c_2, c_3) = (g_1^r, g_2^r, h_1^r m)$
        \item $\Enc(\pk, m) = (c_1, c_2, c_3, (h_2 h_3^\beta)^r)$
    \end{itemize}
    \item Decryption procedure:
    \begin{itemize}
        \item Check that $c_1^{x_3 + \beta x_5} c_2^{x_4 + \beta x_6} = c_4$. If not, output $\perp$.
        \item Else, output $\widehat{m} = c_3 c_1^{-x_1} c_2^{-x_2}$
    \end{itemize}
\end{itemize}

\todo{All the proofs here...}

\section{Digital signatures}

In this section we explore the solutions to the problem of authentication using an asymmetric key method. Some observations are in order:

\begin{figure}
    \centering

    \begin{tikzpicture}[node distance = 2cm, auto, >=latex']

        \node (s) {};
        \node (a) [box] [right of = s] {Alice};
        \node (k) [above of = a, node distance = 1cm] {$sk$};
        \node (b) [box] [right of = a, node distance = 5cm] {Bob};
        \node (p) [above of = b, node distance = 1cm] {$pk$};
        \node (e) [right of = b] {};

        \path[->] (s) edge node {$m$} (a);
        \path[->] (k) edge (a);
        \path[->] (a) edge node {$(m, \sigma)$} (b);
        \path[->] (p) edge (b);
        \path[->] (b) edge node {$b$} (e);

    \end{tikzpicture}
    \caption{Asymmetric authentication}
    \label{fig:digisign}
\end{figure}

\begin{itemize}
    \item In a symmetric setting, a verifier routine could be banally implemented as recomputing the signature using the shared secret key and the message. Here, Bob cannot recompute $\sigma$ as he's missing Alice's secret key (and for good reasons too...). Thus, the verifying routine must be defined otherwise;
    \item In a vaguely similar manner to how an attacker could encrypt messages by itself in the asymmetric scenario, because the public key is known to everyone, any attacker can verify any signed messages, for free.
\end{itemize}

Nevertheless, proving that a \textsc{ds} scheme is secure is largely defined in the same way as in the symmetric scenario, with the \textsc{uf-cma} property:

\begin{cryptogame}{dsufcma}{Unforgeable digital signatures}{uf-cma}
    
    \receive{$(\pk, \sk) \pickUAR \keygen(1^\lambda)$}{$\pk$}{}

    \cseqdelay

    \send{$m \in M$}{$m$}{}
    \receive{$\sigma \pickUAR \textsf{\textup{Sign}}(sk, m)$}{$\sigma$}{}

    \cseqdelay

    \send{$m^* \notin M$}{$(m^*, \sigma^*)$}{\textsc{Output} $\textsf{\textup{Verify}}(\pk, m^*, \sigma^*)$}

\end{cryptogame}

\subsection{Public Key Infrastructure}

The problem now is that Alice has a public key, but she wants some sort of ``certificate of validity'' for it, so that Bob will be sure that whenever he receives Alice's public key, he can be sure it's the right one by checking such certificate.

For certificates to be useful, the parties need an universally-trusted third party, called \textit{Certification Authority}. It will provide a special \textit{signature} to Alice for proving her identity to Bob, as exemplified by the sequence in figure \ref{cryptosequence:certissue}.

Whenever Bob wants to check the validity of the Alice's public key, he can query the authority for the certificate, and verify the public key he just received, as shown in figure \ref{cryptosequence:certcheck}

\begin{cryptosequence}
    {certissue}
    {}

    \cseqentity{CA}{CA}
    \cseqentity[2.2]{A}{Alice}

    \cseqmessager{CA}{$(\pk_C, \sk_C) \pickUAR \keygen$}{$\pk_C$}{A}{}

    \cseqdelay

    \cseqmessagel{A}{\shortstack[l]{
        $(\pk_A, \sk_A) \pickUAR \keygen$ \\
        $\textsc{id}_A = \textup{``Alice''} \| \pk_A$
        }}{$\textsc{id}_A$}{CA}{}
    
    \cseqdelay

    \cseqmessager{CA}{$\textsc{cert}_A \pickUAR \textsf{\textup{Sign}}(\sk_C, \textsc{id}_A)$}{$\textsc{cert}_A$}{A}{}
    
\end{cryptosequence}

\begin{cryptosequence}
    {certcheck}
    {}

    \cseqentity{CA}{CA}
    \cseqentity[0.8]{A}{Alice}
    \cseqentity[1.4]{B}{Bob}

    \cseqmessager{CA}{\shortstack[r]{
        \small{$(\pk_C, \sk_C) \pickUAR$} \\
        \small{$\pickUAR \keygen$}
    }}{$\pk_C$}{A}{}
    \cseqmessager{CA}{}{$\pk_C$}{B}{}

    \cseqdelay

    \cseqmessagel{A}{\shortstack[l]{
        $(\pk_A, \sk_A) \pickUAR \keygen$ \\
        $\textsc{id}_A = \textup{``Alice''} \| \pk_A$
    }}{$\textsc{id}_A$}{CA}{}

    \cseqdelay

    \cseqmessager{CA}{\shortstack[r]{
        \small{$\textsc{cert}_A \pickUAR$} \\
        \small{$\pickUAR \textsf{\textup{Sign}}(\sk_C, \textsc{id}_A)$}
    }}{$\textsc{cert}_A$}{A}{}
    
    \cseqdelay

    \cseqmessager{A}{}{$(\textsc{id}_A, \textsc{cert}_A)$}{B}{\small{$b = \textsf{\textup{Auth}}(\pk_C, \textsc{id}_A, \textsc{cert}_A)$}}

\end{cryptosequence}

How can Bob recognize a valid certificate from an expired/invalid one? The infrastructure provides some servers which contain the lists of the currently valid certificates, such as $\textsc{cert}_A$, in the case of Alice.

\begin{theorem}
    Signatures are in \textit{\textbf{Minicrypt}}.
\end{theorem}

This is a counterintuitive result, not proven during the lesson, but very interesting because it implies that we can create valid signatures only with hash functions, without considering at all public key encryption.


Up next:
\begin{itemize}
    \item Digital Signatures from TDP*
    \item Digital Signatures from ID Scheme*
    \item Digital Signatures from CDH
\end{itemize}

Where * appears, something called \textit{Random Oracle Model} is used in the proof. Briefly, this model assumes the existance of an ideal hash function which behaves like a truly random function (outputs a random y as long as x was never queried, otherwise gives back the already taken y).
