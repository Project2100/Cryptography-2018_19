\mychapter{4}{Lesson 4} %181005

\section{Negligible function}

What is exactly a negligible function? Below here there is a possible interpretation of this notion, taken from an answer to a question in the Cryptography Stack Exchange website:
\begin{quotation}
    ``[...] in modern cryptographic schemes, we generally do not try to achieve perfect secrecy [...]. Instead, we define security against a specific set of adversaries whose computational power is bounded. Generally, we assume an adversary that is bounded to run in time polynomial to $n$, where $n$ is the security parameter given to the key generation algorithm [...].

    So consider a scheme $\Pi$ where the only attack against it is brute-force attack. We consider $\Pi$ to be secure if it cannot be broken by a brute-force attack in polynomial time.

    The idea of \emph{negligible probability} encompasses this exact notion. In $\Pi$, let's say that we have a polynomial-bounded adversary. Brute force attack is not an option. But instead of brute force, the adversary can try (a polynomial number of) random values and hope to guess the right one. In this case, we define security using negligible functions: The probability of success has to be smaller than the reciprocal of any polynomial function.

    And this makes a lot of sense: if the success probability for an individual guess is a reciprocal of a polynomial function, then the adversary can try a polynomial amount of guesses and succeed with high probability. If the overall success rate is $\oneover{\poly(n)}$ then we consider this attempt a feasible attack to the scheme, which makes the latter insecure.

    So, we require that the success probability must be less than the reciprocal of every polynomial function. This way, even if the adversary tries $\poly(n)$ guesses, it will not be significant since it will only have tried:
    \[
        \frac{\poly(n)}{superpoly(n)}
    \]
    As $n$ grows, the denominator grows far faster than the numerator and the success probability will not be significant.''\footnote{\ding{226} \href
        {https://crypto.stackexchange.com/questions/5832/what-exactly-is-a-negligible-and-non-negligible-function}
        {\textsf{``What exactly is a negligible (and non-negligible) function?'' --- Cryptography Stack Exchange}}}
\end{quotation}


\begin{definition}
    Let $f : \nonneg \to \nonneg$ be a function. Then it is deemed \emph{polynomial}, and denoted as $f \in \poly(\lambda)$, iff:
    \[
        \exists c \in \nonneg : f(\lambda) \in O(\lambda^c) \qedhere
    \]
\end{definition}

% AP190904: Big O or small O?!?
% ^^^^^^^^: Big O, as specified here: https://en.wikipedia.org/wiki/Randomness_extractor#Randomness_extractors_in_cryptography
\begin{definition}
    Let $\nu : \nonneg \to \real$ be a function. Then it is deemed \emph{negligible}, and denoted as $\nu \in \negl(\lambda)$, iff:
    \[
        \forall f \in \poly(\lambda)  \implies \nu(\lambda) \in O\left(\oneover{f(\lambda)}\right) \qedhere
    \]
\end{definition}

Note that these actually represent upper bounds for functions: a negligible function adheres to the polyomial function definition, whereas the opposite isn't true. To sum it up: $\negl \subset \poly$.

\begin{exercise}
    Let $p(\lambda), p'(\lambda) \in \poly(\lambda)$ and $\nu(\lambda), \nu'(\lambda) \in \negl(\lambda)$. Then prove the following:

    \begin{enumerate}
        \item $p(\lambda) \cdot p'( \lambda) \in  \poly(\lambda)$
        \item \label{ex:negl} $\nu(\lambda) + \nu'(\lambda) \in \negl(\lambda)$
    \end{enumerate} 
\end{exercise}

\begin{solution}[\ref{ex:negl}]

    \todo{Questa soluzione usa disuguaglianze deboli; per essere negligibile una funzione dev'essere strettamente minore di un polinomiale inverso. Da approfondire}
    
    We need to show that for any $c \in \nonneg $, then there is $n_0$ such that $\forall n > n_{0} \implies \nu(n) + \nu'(n) < \oneover{n^c}$.
    
    Consider an arbitrary $c \in \nonneg$. Then, since $c + 1 \in \nonneg$, and both $\nu$ and $\nu'$ are negligible, there exist $n_\nu$ and $n_{\nu'}$ such that:
    
    \begin{align*}
        \forall n \geq n_{\nu} \implies& \nu(n) \leq n^{-(c+1)} \\
        \forall n \geq n_{\nu'} \implies& \nu'(n) \leq n^{-(c+1)}
    \end{align*}

    Fix $n_{0} = \max(n_{\nu}, n_{\nu'})$. Then, since $n_0 \geq 2$, $\forall n \geq n_0$ we have:
    \begin{align*}
        & \nu(n) + \nu'(n) \\
        \leq& n^{-(c+1)} + n^{-(c+1)} \\
        =& 2n^{-(c+1)} \\
        \leq& n^{-c}
    \end{align*}
    Therefore, we conclude that $\nu(n) + \nu'(n) \in \negl(\lambda)$.
\end{solution}


\section{One-way functions}

From here, we start defining an object that is fundamental to everyday cryptography: the \emph{one-way} function, or \owf{} in short. Colloquially, a one-way function is a function that is ``easy to compute'', while being ``hard to invert'' a the same time, the concept of hardness being borrowed by complexity theory.

\begin{definition}    
    Let $f : \binary^{n(\lambda)} \to \binary^{n(\lambda)}$ be a function. Then it is a \owf{} iff:
    \[
        \forall \adversary \in \ppt\, \exists \nu(\lambda) \in \negl(\lambda) : \Pr\left[\cryptog{owf}[f](\lambda) = 1 \right] \leq \nu(\lambda) \qedhere
    \]
\end{definition}


\begin{cryptogame}
    {owfdef}
    {One-Way Function hardness}
    {owf}

    \receive{\shortstack[l]{
        $x \pickUAR \binary^n$ \\
        $y = f(x)$
    }}{$y$}{}

    \cseqdelay

    \send{}{$x'$}{\textsc{Output 1 iff} $f(x') = y$}

\end{cryptogame}

The structure of the ``game'' appearing in the definition is depicted in figure \ref{cryptogame:owfdef}. Do note that the game does not check for $x = x'$, but rather for $f(x) = f(x')$; in a sense, the adversary is not trying to guess what the original $x$ was: its goal is to find any value such that its image is $y$ according to $f$, and such value may very well not be unique.

\begin{exercise} \label{ex:owf}
    Prove the following claims:
    \begin{enumerate}
        \item There exists an inefficient adversary that wins $\cryptog{owf}[f]$ with probability $1$
        \item There exists an efficient adversary that wins $\cryptog{owf}[f]$ with probability $2^{-n}$
    \end{enumerate}
\end{exercise}

\begin{solution}[\ref{ex:owf}]\
    \begin{enumerate}
        \item Adversary uses a brute-force attack.
        \item Adversary makes a random guess.
    \end{enumerate}
\end{solution}

A one-way function can be thought as a function which is very efficient in generating ``puzzles'' that are very hard to solve from scratch. Furthermore, given a candidate solution, one can efficiently verify its validity. In a twist of perspective, for a given couple $(\mathcal{P}_\textsc{gen},\mathcal{P}_\textsc{ver})$ of a puzzle generator and a puzzle verifier, another ``game'' can be drawn as in figure \ref{cryptogame:owpuzzle}.

\begin{cryptogame}
    {owpuzzle}
    {The puzzle game}
    {puzzle}

    \receive{$(x, y) \pickUAR \mathcal{P}_\textsc{gen}$}{$y$}{}

    \cseqdelay

    \send{}{$x'$}{\textsc{Output} $\mathcal{P}_\textsc{ver}(x', y)$}

\end{cryptogame}

It can also be said that the one-way puzzle problem is in \textsc{np}, because witness checking is easy, but not in \textsc{p} because finding a solution to begin with is hard.

\subsubsection{Impagliazzo's Worlds}

Suppose to have Gauss, a genius child, and his professor. The professor gives to Gauss some mathematical problems, and Gauss wants to solve them all.

Imagine now that, if using one-way functions, the problem is $f(x)$, and its solution is $x$. According to Impagliazzo, we live in one of these possible worlds:
\begin{itemize}
    \item \textit{Algorithmica}: $\textsc{P} = \textsc{NP}$, meaning all efficiently verifiable problems are also efficiently solvable. 
    
    The professor can try as hard as possible to create a hard scheme, but he won't succeed because Gauss will always be able to efficiently break it using the verification procedure to compute the solution

    \item \textit{Heuristica}: \textsc{NP} problems are hard to solve in the worst case but easy on average. 
    
    The professor, with some effort, can create a game difficult enough, but Gauss will solve it anyway; here there are some problems that the professor cannot find a solution to

    \item \textit{Pessiland}: \textsc{NP} problems are hard on average but no one-way functions exist
    
    \item \textit{Minicrypt}: One-way functions exist but public-key cryptography is impractical
    
    \item \textit{Cryptomania}: Public-key cryptography is possible: two parties can exchange secret messages over open channels
\end{itemize}
    
\section{Computational Indistinguishability}

Distribution ensembles $X = \{X_{\lambda \in \mathbb{N}}\}$ and $Y = \{Y_{\lambda \in \mathbb{N}}\}$ are distribution sequences.

\begin{definition}[\emph{Comp. indist.}]
    Let $X$ and $Y$ be two distribution sequences; they are deemed \emph{computationally indistinguishable}, written as ``$X \compindist Y$'' iff:
    \[
        \forall \adversary \in \ppt \exists \nu(\lambda) \in \negl(\lambda) : \left|\Pr[\adversary(X_\lambda) = 1] - \Pr[\adversary(Y_\lambda) = 1]\right| \leq \nu(\lambda) \qedhere
    \]
\end{definition}

In words: any \emph{efficient} adversary attempting to distinguish outputs between the two ensembles will succeed with a probability that is negligibly different than randomly guessing. Note the emphasis on ``efficient'', which makes this relationship between ensembles weaker than what would be a purely statistical one.

With the purpose of making these new concepts clearer, it is presented this mental game.

\todo{AP181129-2344: There may be room for improvement, but I like how it's worded: it puts some unusual perspective into the cryptographic game, and it could be a good thing since it closely precedes our first reduction, and the whole hybrid argument mish-mash.}

A \emph{challenger} \challenger{} chooses a value $z$ among $X_\lambda$ and $Y_\lambda$, and gives it to a \emph{distinguisher} \distinguisher{}. In turn, \distinguisher{} has to correctly guess which was the source of $z$: either $X_\lambda$ or $Y_\lambda$. 

If we let $X_\lambda$ and $Y_\lambda$ to be \emph{computationally indistinguishable}, then, fixed 1 as one of the sources, the probability that \distinguisher{} says ``1!'' when \challenger{} picks $z$ from $X_\lambda$ is \emph{not so far} from the probability that \distinguisher{} says ``1!'' when \challenger{} picks $z$ from $Y_\lambda$.

So, this means that, when this property is verified by two random variables, there isn't too much \textit{difference} between the two variables in terms of information avaliable to \distinguisher{}, otherwise the distance between the two probabilities should be much more than a negligible quantity.

\begin{lemma} \label{lem:compmall}
    Let $f$ be a function that has polynomial time-complexity. Then, for any two ensembles $X$ and $Y$:
    \[
        X \compindist Y \implies f(x) \compindist f(y) \qedhere
    \]
\end{lemma}

\begin{proof}
    This proof is by contradiction and uses a reduction. Let $X \compindist Y$ be two indistinguishable ensembles, and $f \in \ppt$ an arbitrary poly-time complex function. Assume there exists an adversary \adversary{} to the challenge of distinguishing the ensembles' images $f(X)$ from $f(Y)$ that does efficiently succeed, as shown in figure \ref{cryptogame:fdistin}. 
    
    \begin{cryptogame}
        {fdistin}
        {A distinguisher for $f$}
        {$f$-\textsc{dist}}

        % Adversary asks for the challenge. Challenger chooses evenly between X and Y, and samples a value; then it sends that value's image by f to the adversary
        \receive{\shortstack[l]{
            $z_0 \pickUAR X$ \\
            $z_1 \pickUAR Y$ \\
            $b \pickUAR \binary$
        }}{$f(z_b)$}{}

        \cseqdelay

        % Adversary attempts to guess from which distribution the value's image came from
        \send{}{$b'$}{\textsc{Output 1 iff} $b' = b$}

    \end{cryptogame}
    
    Fix this adversary to be the distinguisher $\distinguisher_f$. From here, another adversary $\adversary \in \ppt$ can use $\distinguisher_f$ to effectively distinguish the original ensembles, as depicted in figure \ref{cryptoredux:fdistin}:
    \begin{enumerate}
        \item \adversary{} asks for the original sample from the challenger
        \item \adversary{} applies $f$ on the sample
        \item \adversary{} relays the resulting image to $\distinguisher_f$
        \item $\distinguisher_f$ replies with his outcome
        \item \adversary{} relays the outcome to the challenger
    \end{enumerate}
    All of this is done in polynomial time, since all functions and machines involved in the process opeate in \ppt. This contradicts the computational indistinguishablilty of $X$ and $Y$.

    \begin{cryptoredux}
        {fdistin}
        {Distinguisher reduction}
        {dist}
        {f-\textsc{dist}}
        
        % The distinguisher asks for the challenge
        \receive{\shortstack[l]{
            $z_0 \pickUAR X$ \\
            $z_1 \pickUAR Y$ \\
            $b \pickUAR \binary$
        }}{$z_b$}{}
      
        % The distinguisher applies f to z and relays the image to A
        \invoke{}{$f(z_b)$}{}
        
        % Adversary distinguishes betweeen the two distributions
        \return{}{$b'$}{}

        \send{}{$b'$}{\textsc{Output 1 iff} $b' = b$}
        
    \end{cryptoredux}

\end{proof}


\section{Pseudo-random generators}

A deterministic function $G \in \binary^\lambda \to \binary^{\lambda + l(\lambda)}$ is called a \emph{pseudo-random generator}, or \prg{} in short, iff:

\begin{itemize}
    \item $G \in \ppt(\lambda)$
    \item $|G(s)| = \lambda + l(\lambda)$ % Well duh, we can see it by def...
    \item Given $U_n$ to be a distribution ensemble of $n$ uniform random variables:
    \[
        G(U_{\lambda}) \compindist U_{\lambda + l(\lambda)}
    \]
\end{itemize}

\begin{cryptogame}
    {prg}
    {The pseudorandom game}
    {prg}
    
    \receive{\shortstack[l]{
        $x_0 \pickUAR G(U_\lambda)$ \\
        $x_1 \pickUAR U_{\lambda+l(\lambda)}$ \\
        $b \pickUAR \binary$
    }}{$x_b$}{}

    \cseqdelay

    \send{}{$b'$}{\textsc{Output 1 iff} $b = b'$}
    
\end{cryptogame}

So, if we take $s \pickUAR U_\lambda$, the output of $G$ will be indistinguishable from a random pick from $U_{\lambda + l(\lambda)}$.
