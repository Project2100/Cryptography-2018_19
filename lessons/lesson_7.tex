\mychapter{7}{Lesson 7} %181017

\subsubsection{\textsc{Ggm}-tree (cont'd)}

% Reminders: k is the scheme key
As stated in the previous lesson, given a \prg{} $G \in \binary^\lambda \to \binary^{2\lambda}$, we can build a function family $f_k$ by repeatedly taking halves of $G$'s images, and plugging them back into $G$. Our goal is to prove the following theorem:

\begin{theorem}
    If $G$ is a \prg, then $f_k$ is a \prf.
\end{theorem}

% AP190103: All notation is still left in an inconsistent state, it's difficult here to establish which things can go where...
\begin{proof}

    Before starting to prove the \prf-ness of $f_k$, we make a brief consideration about its time complexity: computing $f_k(x)$ consists in computing $G$ and taking half of the resulting image as many times as is the length of $x$. Since the length of $x$ is polynomial in $\lambda$, so is the number of $G$'s iterations; combine this with the fact that $G$ is itself polynomial by definition, and we conclude that $f_k$ is polynomial too.

    Having cleared any doubts about $f_k$'s time complexity, we now turn to the essential point of interest: its pseudo-randomness. The proof will proceed by induction over the lentgth of $x$, which is also the height of the tree-like structure modeling the algorithm.
    
    \textbf{Base case} ($n = 1$): $f_k$'s domain is restricted to $\binary$, meaning that its images will be respectively the two halves on a single iteration of $G(k)$; they are, of course, pseudorandom by $G$'s definition:
    \[
        (f_k(0), f_k(1)) = (G_0(k), G_1(k)) \compindist U_{2\lambda}
    \]
    therefore, in this case, $F_k$ is pseudorandom.

    % AP190103: Here too...
    \textbf{Inductive step}: Let $f'_k : \binary^{n - 1} \to \binary^\lambda$ be a \prf. Define $f_k$ as follows:
    \[
        f_k : \binary^n \to \binary^\lambda : (b, x) \in \binary \times \binary^{n - 1} \mapsto G_b(f'_k(x))
    \]
    It must be proven that if $f_k'$ is a \prf, then so is $f_k$. To help ourselves, we'll define some hybrid games as usual. We will proceed by hybrid games, depicted in figures \ref{cryptogame:ggmhyb1} and \ref{cryptogame:ggmhyb2}:

    \begin{cryptogame}
        {ggmhyb1}
        {$\hybridg{1}[f_k](\lambda, b)$: The \ggm{} construct is put against randomly driven \prg}
        {n-th}

        \cseqchallenger{\shortstack[l]{
            $k \pickUAR \binary^\lambda$ \\
            $\overline{R} \pickUAR \mathfrak{R}(n-1, \lambda)$ \\
            $b \pickUAR \binary$
        }}

        \cseqdelay
        \cseqbeginloop
        \send{}{$r = (r_0, r_{1 \text{---} n})$}{}
        \receive{\shortstack[l]{
            $y_0 = G_{r_0}(f'_k(r_{1 \text{---} n})) = f_k(r)$ \\
            $y_1 = G_{r_0}(\overline{R}(r_{1 \text{---} n}))$
        }}{$y_b$}{}
        \cseqendloop
        \cseqdelay

        \send{}{$b'$}{\textsc{Output 1 iff} $b = b'$}

    \end{cryptogame}

    \begin{cryptogame}
        {ggmhyb2}
        {$\hybridg{2}[\overline{R} \circ G](\lambda, b)$: The randomly driven \prg{} against a true random function}
        {n-th}

        \cseqchallenger{\shortstack[l]{
            $\overline{R} \pickUAR \mathfrak{R}(n-1, \lambda)$ \\
            $R \pickUAR \mathfrak{R}(n, \lambda)$ \\
            $b \pickUAR \binary$
        }}

        \cseqdelay
        \cseqbeginloop
        \send{}{$r = (r_0, r_{1 \text{---} n})$}{}
        \receive{\shortstack[l]{
            $y_0 = G_{r_0}(\overline{R}(r_{1 \text{---} n}))$ \\
            $y_1 = R(r)$
        }}{$y_b$}{}
        \cseqendloop
        \cseqdelay

        \send{}{$b'$}{\textsc{Output 1 iff} $b = b'$}

    \end{cryptogame}
    
    %AP190103: Actually, we're going to be inductive twice, one time for each hybrid couple

    \begin{lemma}
        $\hybridg{1}[f_k](\lambda, 0) \compindist \hybridg{1}[f_k](\lambda, 1)$       
    \end{lemma}

    \begin{proof}
        Assume $\exists \distinguisher^{\textsc{n-th}} \in \ppt$ that can distinguish $f_k$ from $\overline{R} \circ G$ at the $n$-th step; then an adversary \adversary{} can use it to break in turn $f'_k$'s \prf-ness, and distingush it from $\overline{R} \circ G$ as shown in figure \ref{cryptoredux:ggmhyb1}:

        % AP190103: It is not immediately clear by the sequence that we're simulating two games in two different "inductive" steps, instead of trying to break one standard assumption
        % The reasoning is: distinguish the hybrids, simulate the induction steps
        \begin{cryptoredux}
            {ggmhyb1}
            {Using $\distinguisher^{\textsc{n-th}}$ to break $f'_k$}
            {f'-prf}
            {n-th}

            \cseqchallenger{\shortstack[l]{
                $k \pickUAR U_{\lambda - 1}$ \\
                $\overline{R} \pickUAR \mathfrak{R}(n - 1, \lambda)$ \\
                $b \pickUAR \binary$
            }}

            \cseqbeginloop
            \return{}{$(r_0, r_{1 \dots n})$}{}

            \send{}{$r_{1 \dots n}$}{}

            \receive{\shortstack[l]{
                $z_0 = f_k'(r_{1 \dots n})$ \\
                $z_1 = \overline{R}(r_{1 \dots n})$
            }}{$z_b$}{}

            \cseqdelay

            \invoke{\shortstack[l]{
                $y_b = G_{r_0}(z_b)$  % AP190909 - IMPORTANT : Make a note here: y is either f_k or RcircG
            }}{$y_b$}{}

            \cseqendloop

            \return{}{$b'$}{}

            \send{}{$b'$}{\textsc{Output 1 iff} $b' = b$}

        \end{cryptoredux}

    \end{proof}

    Before tackling $\hybridg{2}[\overline{R}](\lambda, b)$, it is best to introduce another lemma:

    \begin{lemma}
        If $G : \binary^\lambda \to \binary^{2 \lambda}$ is a \prg, then:
        \[
            \forall K_i \sim \unifdist(\lambda) \iid \implies (G(K_1), \dots , G(K_t)) \compindist (U_{2 \lambda}, \dots , U_{2\lambda})
        \]
    \end{lemma}

    \begin{proof}
        \todo{Idea: all values are independent and pseudorandom on their own, hybridize progressively...}
        % AP190909: Wait: isn't that a property of a PRG already? No, the definition states indist for a single sample, not for an arbitrary sequence; hence the idea, btw
    \end{proof}

    Now for the final lemma:

    \begin{lemma}
        $\hybridg{2}[\overline{R} \circ G](\lambda, 0) \compindist \hybridg{2}[\overline{R} \circ G](\lambda, 1)$       
    \end{lemma}
    
    %AP190912: May need to flesh out some images here, especially a cryptoredux drawing...
    \begin{proof}

        This proof is trickier: the problem lies in the inherent difference between the task of detecting a \prg{} from that of detecting a \prf{}. The aforemenioned lemma helps us with the aspect of polynomial queries on a \prg{}, bringing it in line with the \prf{} game; yet, we're far from done.

        Let's delve into the details: from \adversary's perspective, the game of distinguishing $\overline{R} \circ G$ from $R$ , where \distinguisher{} performs a number of queries $q$ polynomial in $\lambda$, is perfectly modeled by the act of distinguishing a sequence of evaluations $(G(K_1), \dots, G(K_q))$ from $(U_{2\lambda}, \dots, U_{2\lambda})$. Therefore, the game can be twisted in a way that \adversary{} receives either one of the two whole sequences beforehand, and respects the rules by sending to $D$ the right piece of information on each of its queries, in order to simulate the $\overline{R} \circ G$/$R$ game correctly.

        At this point, the simple way for A to send the right info on each query $i$ would be to check whether the query's first bit $r_0$ would be $0$ or $1$, and give back to \distinguisher{} either the first or the second half of the i-th value of the sequence he received from \challenger{}; this is exactly how \ggm{} operates. However, there's a catch: \distinguisher{} might make two queries where the r1---n parts are the same, the only difference is in the first bit: essentially, D is asking for both parts of G(r) from two distinct queries. In this case, if \adversary{} just sends the i-th value's corresponding half, the simulation would break, because the halves come from different samples.
        
        Thus, the adversary must keep track of which suffixes he has been already queried about, and make sure to send the value's other half (there are only two) whenever he is queried on a suffix for a second time. This final touch solves the problem of simulating the game for D, and now the task of A is changed to an equivalent one of breaking the lemma stated at the beginning.
        
        \todo{Prata's notes:
        
        Let T1, T2 be empty tables.

        Given query x input x1---n let xbar=x1---i

        if xbar notin T1 then x <-\$ binarylambda and add k\_xbar to T2

        if xbar in T2 let k\_xbar = T2[T1[xbar]]

        output y = G\_x\_n(g\_x\_n-1(...G\_x\_i+1(k\_xbar)))

        H\_0: GGM treee
        
        H\_n: random function

        exploit lemma prg => prf
        }
        
    \end{proof}

    In the end the hybrids are proven to mutually indistinguishable, therefore the inductive step is correct, proving the theorem.

\end{proof}

\section{\textsc{Cpa}-security}

Now it's time to define a stronger notion of security, which is widely used in cryptology for first assessments of cryptographic strength. Let $\Pi := (\Enc, \Dec)$ be a \ske{} scheme, and consider the game depicted in figure \ref{cryptogame:cpadef}. Observe that this time, the adversary can ``query'' the challenger for the ciphertexts of any messages of his choice, with the only reasonable restriction that the query amount must be polynomially bound by $\lambda$. This kind of game/attack is called the \emph{Chosen Plaintext Attack}, because of the adversary's capability of obtaining ciphertexts from messages. The usual victory conditions found in n-time security games, which are based on ciphertext distinguishability, apply.

\begin{cryptogame}
    {cpadef}
    {The \cpa-security game: $\cryptog{cpa}(\lambda, b)$}
    {cpa}

    \cseqchallenger{$k \pickUAR \binary^\lambda$}

    \cseqbeginloop
    \send{}{$m$}{}
    \receive{$c \pickUAR \Enc(k, m)$}{$c$}{}
    \cseqendloop

    \cseqdelay

    \send{}{$m_0^*, m_1^*$}{}
    \receive{\shortstack[l]{
        $b \pickUAR \binary$ \\
        $c^* \pickUAR \Enc(k, m_b^*)$
    }}{$c^*$}{}

    \cseqdelay

    \cseqbeginloop
    \send{}{$m$}{}
    \receive{$c \pickUAR \Enc(k, m)$}{$c$}{}
    \cseqendloop

    \cseqdelay

    \send{}{$b'$}{\textsc{Output 1 iff} $b' = b$}

\end{cryptogame}

\begin{definition}
    A scheme is \cpa-secure if $\cryptog{cpa}(\lambda, 0) \compindist \cryptog{cpa}(\lambda, 1)$
\end{definition}


Having given this definition of security, recall the $\Pi_\xor$ scheme defined in the previous lesson. It is easy to see that $\Pi_\xor$ is not \cpa-secure for the same reasons that it is not computationally 2-time secure; however this example sheds some new light about a deeper problem:

%AP181230: Refer to page 72 of Katz-Lindell for a good explanation
\begin{observation}
    No deterministic scheme can achieve \cpa-security.
\end{observation}

This is true, because nothing prevents the adversary from asking the challenger to encrypt either $m_0$ or $m_1$, or even both, before starting the actual challenge; just as in the 2-time case for $\Pi_\xor$, he will know the messages' ciphertexts in advance, so he will be able to tell which message the challenger has encrypted every time. The solution for obtaining a \cpa-secure encryption scheme consists of returning different ciphertexts for the same message, even better if they look random. This can be achieved by using \prf{}s.

Consider the following \ske{} scheme $\Pi_{f_k}$, where $f_k$ is a \prf{} structured as follows:

\begin{itemize}
    \item $\Enc(k, m) = (c_1, c_2) = (r, f_k(r) \xor m)$, where $k \pickUAR \binary^\lambda$ and $r \pickUAR \binary^n$
    
    \item $\Dec(k, (c_1, c_2)) = f_k(c_1) \xor c_2$
\end{itemize}

Observe that the random value $r$ is part of the ciphertext, making it long $n+l$ bits; also more importantly, the adversary can and will always see $r$. The key $k$ though, which gives a \textit{flavour} to the \prf, is still secret.

\begin{theorem} \label{thm:prfcpa}
    If $f_k$ is a \prf, then $\Pi_{f_k}$ is \cpa-secure.
\end{theorem}

\begin{proof}
    We have to prove that $\cryptog{cpa}[\Pi_{f_k}](\lambda, 0) \compindist \cryptog{cpa}[\Pi_{f_k}](\lambda, 1)$; to this end, the hybrid argument will be used. Let the first hybrid $\hybridg{0}$ be the original game, the second hybrid $\hybridg{1}$ will have a different encryption routine:

    \begin{itemize}
        \item $r \pickUAR \binary^n$
        \item $R \pickUAR \mathfrak{R}(n, l)$
        \item $c = (r, R(r) \xor m)$, where $m$ is the plaintext to be encrypted
    \end{itemize}

    and then the last hybrid $\hybridg{2}$ will simply output $(r_1, r_2) \pickUAR U_{n + l}$.

    \begin{lemma}
        $\forall b \in \binary \implies \hybridg{0}(\lambda, b) \compindist \hybridg{1}(\lambda, b)$.
    \end{lemma}

    \begin{proof}
        As usual, the proof is by reduction: suppose there exists a distinguisher \distinguisher{} capable of telling the two hybrids apart; then \distinguisher{} can be used to break $f_k$'s property of being a \prf. The way to use \distinguisher{} is to make it play a \cpa-like game, as shown in figure \ref{cryptoredux:prfcpa}\footnotemark, where the adversary attempting to break $f_k$ decides which message to encrypt between $m_0$ and $m_1$ beforehand, and checks whether it guesses which message has been encrypted. Either way, the adversary can get a sensible probability gain in guessing if the received values from the challenger were random, or generated by $f_k$. Thus, assuming such \distinguisher{} exists, \adversary{} can efficiently break $f_k$, contradictiong its \prf-ness.

        \footnotetext{An observant student may notice a striking similarity with a previously exposed reduction in figure \ref{cryptoredux:xorotprg}}

        % AP190101: Potrei includere una definizione alternativa di prf più su, in modo tale che questa riduzionie sia un po' più chiara

        \begin{cryptoredux}
            {prfcpa}
            {Breaking a \prf, for fixed message choice of $m_0$}
            {prf}
            {cpa}

            \cseqchallenger{\shortstack[l]{
                $k \pickUAR \binary^\lambda$ \\
                $R \pickUAR \mathfrak{R}(n, l)$ \\
                $b \pickUAR \binary$
            }}
        
            \cseqbeginloop
            \return{}{$m$}{}
            \send{$r \pickUAR \binary^n$}{$r$}{}
            \receive{\shortstack[l]{
                $z_0 \pickUAR f_k(r)$ \\
                $z_1 \pickUAR R(r)$
            }}{$z_b$}{}
            \invoke{$c = (r, z_b \xor m)$}{$c$}{}
            \cseqendloop

            \cseqdelay
        
            \return{}{$m_0^*, m_1^*$}{}
            \send{$r^* \pickUAR \binary^n$}{$r^*$}{}
            \receive{\shortstack[l]{
                $z_0^* \pickUAR f_k(r)$ \\
                $z_1^* \pickUAR R(r)$
            }}{$z_b^*$}{}
            \invoke{$c^* = (r^*, z_b^* \xor m_0^*)$}{$c^*$}{}

            \cseqdelay
        
            \cseqbeginloop
            \return{}{$m$}{}
            \send{$r \pickUAR \binary^n$}{$r$}{}
            \receive{\shortstack[l]{
                $z_0 \pickUAR f_k(r)$ \\
                $z_1 \pickUAR R(r)$
            }}{$z_b$}{}
            \invoke{$c = (r, z_b \xor m)$}{$c$}{}
            \cseqendloop
            
            \cseqdelay
        
            \return{}{$b'$}{}

            \cseqdelay

            \send{$b'' = \begin{cases}
                0 &\textsc{iff } b' = 0 \\
                1 &\textsc{else}
            \end{cases}$
            }{$b''$}{\textsc{Output 1 iff} $b'' = b$}
        
        \end{cryptoredux}

    \end{proof}

    \begin{lemma}
        $\forall b \in \binary \implies \hybridg{1}(\lambda, b) \compindist \hybridg{2}(\lambda, b)$.
    \end{lemma}

    \begin{proof}
        Firstly, it can be safely assumed that any ciphertext $(r_i, R(r_i) \xor m_b)$ distributes equivalently with its own sub-value $R(r_i)$, because of $R$'s true randomness, and independency from $m_b$.

        %AP190102: An image visualizing the repeat event might be very useful
        Having said that, the two hybrids apparently distribute uniformly, making them perfectly equivalent; however there is a caveat: if both games are run and one value $\overline{r}$ is queried twice in both runs, then on the second query the adversary will receive the same image in $\hybridg{1}$, but almost certainly a different one in $\hybridg{2}$. This is because the first hybrid uses a function, which is deterministic by its nature, whereas the image in the second hybrid is picked completely randomly from the codomain. Nevertheless, this sneaky issue about "collisions" can be proven to happen with negligible probability.

        Call \textsc{Repeat} this collision event on $\overline{r}$ between 2 consecutive games. Then:
        
        % AP190102: Could use some clarifications? Especially about what the indices represent
        \begin{align*}
        \Pr[\textsc{Repeat}] &= \Pr[\exists i, j \in q \text{ such that } r_i = r_j] \\
            &\leq \sum_{i \neq j} \Pr[r_i=r_j] \\
            &= Col(U_n) \\
            &= \sum_{i \neq j} \sum_{e \in \binary^n} \Pr[r_1 = r_2 = e] \\
            &= \sum_{i \neq j} \sum_{e \in \binary^n} \Pr[r = e]^{2} \\
            &= \binom{q}{2} 2^{n} \frac{1}{2^{2n}} \\
            &= \binom{q}{2} 2^{-n} \\
            &\leq q^{2}2^{-n} \in \negl{\lambda} 
        \end{align*} 


        which proves that the \textsc{Repeat} influences negligibly on the two hybrids' equivalence. Thus $\hybridg{1}(\lambda, b) \compindist \hybridg{2}(\lambda, b)$\footnotemark.
        
        \footnotetext{Do note that the hybrids lose their originally supposed perfect equivalence ($\hybridg{1}(\lambda, b) \equiv \hybridg{2}(\lambda, b)$) because of the \textsc{Repeat} event. Nevertheless, the lemma is still proven because it includes computational bounds into \adversary}
    \end{proof}

    With the above lemmas, and observing that $\hybridg{2}(\lambda, 0) \equiv \hybridg{2}(\lambda, 1)$, we can reach the conclusion that $\hybridg{0}(\lambda, 0) \compindist \hybridg{0}(\lambda, 1)$, which is what we wanted to demonstrate.

\end{proof}
