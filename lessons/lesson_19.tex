\mychapter{19}{Lesson 19} %181130

\todo{E' venuto fuori un casino in questa lezione, ho cercato di riordinare le cose}

\subsubsection{Bilinear Map}

\begin{definition}
    
    Let's define a \emph{bilinear group} as $(G, G_t, q, g, \hat{e}) \pickUAR \mathcal{B}ilin(1^\lambda)$, where:
    \begin{itemize}
        \item $G, G_t$ are prime order groups (order $q$).
        \item $g$ is a generator of $G$ picked \uar.
        \item $(G, \cdot)$ is a multiplicative group and $G_t$ is the ``target'' group.
        \item $\hat{e}$ is an efficiently computable \emph{bilinear} map: $G \times G \rightarrow G_t$ defined as such:
        \[
            \forall h \in G \forall a, b \in \integer_q \implies \hat{e}(g^a, h^b) = \hat{e}(g, h)^{ab} = \hat{e}(g^{ab}, h)
        \]
        provided that $\hat{e}(g, g) \neq 1$, that is, $\hat{e}$ is \emph{non-degenerative}
    \end{itemize}

    To put it in simple words, the exponents can move.
\end{definition}

\todo{Venturi said something here related to Weil pairing over an elliptic curve. I found \href{https://www.math.auckland.ac.nz/~sgal018/crypto-book/ch26.pdf}{this}. Interesting but not useful.}

It must be noted that the \ddh{} assumption is easy for the group $G$: suffice to see that $\hat{e}(g^a, g^b) = \hat{e}(g, g^c)$ is true iff $c = ab$. On the other hand:

\begin{proposition}
    \cdh{} is hard for $G$.
\end{proposition}

\begin{proof}

    Now $\keygen(1^\lambda)$ will:
    \begin{itemize}
        \item Generate some params: $(G, G_t, g, q, \hat{e}) \pickUAR \mathcal{B}ilin(1^\lambda)$ 
        \item $a \pickUAR \integer_q$, then $g_1 = g^a$
        \item Pick $g_2 = g^b$ and $g_2, u_{0 \upto k} \pickUAR G$.
        \item Then output: 
        \begin{itemize}
            \item $\pk = (\text{params}, g_1, g_2, u_0, ..., u_k)$
            \item $\sk = g_2^a = g^{ab}$
        \end{itemize}
    \end{itemize}

    $\textsf{\textup{Sign}}(\sk, m)$:
    \begin{itemize}
        \item Divide the message $m$ of length $k$ in single bits as follows: $m = (m_{1 \upto k})$
        \item Now define $\alpha(m) = u_0\prod_{i = 1}^k u_i^{m_i}$
        \item Pick $r \pickUAR \integer_q$ and output the signature $\sigma = (\sk \cdot \alpha(m)^r,g^r)=(\sigma_1,\sigma_2)$
    \end{itemize}

    $\textsf{\textup{Ver}}(\pk, m, (\sigma_1, \sigma_2))$:
    \begin{itemize}
        \item Check $\hat{e}(g, \sigma_1) = \hat{e}(\sigma_2, \alpha(m)) = \hat{e}(g_1, g_2)$
    \end{itemize}

    The scheme's correctness is proven my ``moving'' the exponents:
    \begin{align*}
        \hat{e}(g, \sigma_1) &= \hat{e}(g,g_2^a\cdot \alpha(m)^r)   & \\
        &= \hat{e}(g,g_2^a) \cdot \hat{e}(g,\alpha(m)^r)            & \text{(Bilinearity)}\\
        &= \hat{e}(g^a,g_2)\cdot \hat{e}(g^r,\alpha(m))             & \\
        &= \hat{e}(g_1,g_2)\cdot\hat{e}(\sigma_2,\alpha(m))         &
    \end{align*}
    
    We can say that we are moving the exponents from the "private domain" to the "public domain".

\end{proof}

\section{Waters signatures}
\begin{theorem}
    The Waters' signature scheme is \ufcma-secure
\end{theorem}

\begin{proof}
    \begin{quote}
        The trick is to ``program the 'u's'' (Venturi)
    \end{quote}

    \todo{Sequence is incomplete/incorrect, have to study it more...}


    % Original subjects: A, C_WDS, C_CDH
    \begin{cryptoredux}
        {reduxwaters}
        {Reducing Waters' scheme to \cdh}
        {cdh}
        {wds}

        \cseqdelay

        % Instantiate game
        \receive{\shortstack[l]{
            ``params''$ = $ \\
            $ \quad =(\G, \G_t, g, q, \hat{e})$ \\
            $ \quad \pickUAR BilGen(1^\lambda)$ \\
            $a, b \pickUAR \integer_q$
        }}{$(\text{``params''}, g^a, g^b)$}{}

        \cseqdelay

        % Send pk to adversary
        \invoke{Construct $u$ string}{$(\text{``params''}, g^a, g^b, u)$}{}
        
        % Adversary can query for signatures
        \cseqdelay
        \return{$m \in M$}{$m$}{}
        \invoke{$\sigma = \textit{Sign}()$}{$\sigma$}{}
        \cseqdelay

        % Adversary forges
        \return{$m^* \notin M$}{$(m^*, \sigma^*)$}{}
        
    \end{cryptoredux}

    \todo{The following explanation is roundabout, will rectify later}

    The following describes how the Waters' challenger constructs the $u$ string. The main idea is, given $k$ as the message length, to choose each single bit of $u$ from 1 up to $k$ such that:

    \begin{equation*}
        \alpha(m) = g_2^{\beta(m)}g^{\gamma(m)},\quad \beta(m) = x_0 + \sum_{i=1}^{k}m_ix_i,\quad \gamma(m) = y_0 + \sum_{i=1}^{k}m_iy_i
    \end{equation*}

    where $x_0 \pickUAR \{-kl, \dots, 0\}, x_{1\upto k} \pickUAR \{0, \dots, l\}, y_{0\upto k} \pickUAR \integer_q$

    In particular: $l = 2q_s$, where $q_s$ is the number of sign queries made by the adversary.

    Therefore, let $u_i = g_2^{x_i}g^{y_i} \quad \forall i \in [0, k]$. Then:

    \begin{equation*}
        \alpha(m) = g_2^{x_0}g^{y_0} \prod_{i=1}^k (g_2^{x_i}g^{y_i})^{m_i} = g_2^{x_0+\sum_{i=1}^k m_ix_i}g^{y_0+\sum_{i=1}^k m_iy_i} = g_2^{\beta(m)}g^{\gamma(m)}
    \end{equation*}

    \todo{Partizioni, doppi if.... qui non ci ho capito 'na mazza}

    \textbf{Step 1}: $\sigma = (\sigma_1, \sigma_2) = (g_2^a \alpha(m)^{\bar{r}}, g^{\bar{r}})$, for $\bar{r} \pickUAR \integer_q \quad \bar{r} = r - a\beta^{-1}$

    \begin{align*}
        \sigma_1 &= g_2^a\alpha(m)^{\bar{r}} \\
        &= g_2^a\alpha(m)^{r - a\beta^{-1}} \\
        &= g_2^a (g_2^{\beta(m)} g^{\gamma(m)})^{r - a\beta^{-1}} \\
        &= g_2^a g_2^{\beta(m)r-a} g^{\gamma(m)r - \gamma(m)a\beta^{-1}} \\
        &= g_2^{\beta(m)r} g^{\gamma(m)r} g^{-\gamma(m)\beta^{-1}}
    \end{align*}

\end{proof}

%https://eprint.iacr.org/2011/703.pdf
