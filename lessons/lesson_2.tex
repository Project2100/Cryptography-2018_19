\mychapter{2}{Lesson 2} %180928

\section{Message Authentication}

\todo{Msg auth image (bob, eve, tag)}

\subsection{Message Authentication Code (MAC or Tag)}

\textbf{Syntax:}

\begin{itemize}
    \item $\K = $ Key Spac 
    \item $\M = $ Message Space
    \item $\Phi = $ Tag (or signature) Space
    \item $\textsf{Tag}: \K \times \M \to \Phi$
    \item $\textsf{Ver}: \K \times \C \times \Phi \to \binary$
\end{itemize}

\textsf{Tag} and \textsf{Ver} form a cryptographic scheme (an authentication one, in detail), and it must abide by the rule:
\[
    \forall m \in \M, \forall k \in \K \implies \textsf{Ver}(k, m, \textsf{Tag}(k, m)) = 1
\]

In the usual case (deterministic), the verifier consists only of an equality check, reusing the \textsf{Tag} routine: $\textsf{Tag}(k, m) = \phi$

% TODO: to review, idea/phrasing can be incorrect
The security aspect to consider in these schemes is the signatures' \emph{unforgeability}: an attacker cannot effectively define a couple $(m, \phi)$ such that:
\[
    \exists k \in \K \implies \textsf{Ver}(k, m, \phi) = 1
\]

\begin{definition} \emph{($\epsilon$-statistical one-time security)}: A given authentication scheme has $\epsilon$-statistical one-time security iff, given a valid couple $(m_1, \phi_2)$, any adversary cannot \emph{forge} a fresh valid couple $(m_2, \phi_2)$ without knowing the signature key $k$. Formally:
    \[
        \forall m_1, m_2 \in \M, \phi_1, \phi_2 \in \Phi, m_1 \neq m_2 \implies \Pr[\textsf{Tag}(K, m_2) = \phi_2|\textsf{Tag}(K, m_1) = \phi_1] \leq \varepsilon
    \]
    
\end{definition}

\subsubsection{Pairwise-independent hashing}

\begin{definition}
    A family of (Hash) functions $H=\{h_s:\M \to \Phi\}_{s\in S}$ is said to be \emph{pairwise independent} iff, for any two distinct messages $m$ and $m'$, we have that $(h_s(m),h_s(m'))$ is uniform over $\Phi^2$ for random choice of seed $s \in S$. Meaning that:
    \begin{gather*}
        \forall m,m' s.t. m\neq m',\forall \phi,\phi' \in \Phi \\
        Pr[h_s(m)=\phi \wedge h_s(m')=\phi']=\frac{1}{|\Phi|^2}
    \end{gather*}
\end{definition}

\textbf{Construction: } Let $p$ be a prime number.\\
Define $h_{(a,b)}(x)=ax+b \mod p$ now $S=\Z_p^2, M=\Phi=Z_p=\{0,1,...,p-1\}$
Thus the following group will be defined $(\Z_p,+)$

\begin{theorem}
    Above $H$ is Pairwise Independent
\end{theorem}
\begin{proof}
    \begin{gather*}
        \text{Fix} m,m' \in \Z_p s.t. m\neq m', \phi,\phi' \in \Z_p\\
        Pr_{(a,b)}[h_{a,b}(m)=\phi \wedge h_{a,b}(m')=\phi']=\\
        =Pr_{(a,b)}[am+b=\phi \wedge am'+b=\phi']=\\
        =Pr_{(a,b)}[
        \begin{pmatrix}
            m & 1 \\
            m' & 1
        \end{pmatrix}
        \cdot
        \begin{pmatrix}
            a \\
            b
        \end{pmatrix}
        =
        \begin{pmatrix}
            \phi \\
            \phi'
        \end{pmatrix}
        ]=\\
        =Pr_{(a,b)}[
        \begin{pmatrix}
            a \\
            b
        \end{pmatrix}
        =
        \begin{pmatrix}
            m & 1 \\
            m' & 1
        \end{pmatrix}^{-1}
        \cdot
        \begin{pmatrix}
            \phi \\
            \phi'
        \end{pmatrix}
        ]=\\
        =\frac{1}{|\Z_p|^2}=\frac{1}{p^2}
    \end{gather*}
    Where $Pr_{(a,b)}$ denotes the probability over (a,b) random in $\Z_p$
\end{proof}

\begin{theorem}
    Let $Mac(K,m)=h_k(m)$ for $\H=\{h_k:\M \to \Phi \}_k$ be Pairwise Independent. The Mac is $\frac{1}{|\Phi|}$-statistical one-time secure. 
\end{theorem}

\begin{proof}
    \begin{gather*}
        \forall m \in \M, \phi \in \Phi\\
        Pr[Mac(K,m)=\phi]=Pr_k[h_k(m)=\phi]=\frac{1}{|\Phi|}\\
        \text{For any } m\neq m', \phi,\phi' \in \Phi\\
        Pr[Mac(K,m)=\phi \wedge Mac(K,m')=\phi']=\frac{1}{|\Phi|^2}\\
        \implies Pr[Mac(K,m')=\phi'| Mac(K,m)=\phi]=\\
        =\frac{\Pr[Mac(K,m)=\phi \wedge Mac(K,m')=\phi']}{Pr[Mac(K,m)=\phi]}=\\
        =\frac{1}{|\Phi|^2}/\frac{1}{|\Phi|}=\frac{1}{|\Phi|}
    \end{gather*}
\end{proof}
\begin{theorem}
    Any t-time($\epsilon=2^{-\lambda}$)-statistical secure MAC has a key of size $(t+1) \lambda$ for any $\lambda > 0$.\\
    \underline{No proof}
\end{theorem}