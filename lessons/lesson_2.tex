\mychapter{2}{Lesson 2} %180928

\section{Authentic communication}


A simple scenario exposing the problem of authentication is depicted in figure \ref{fig:authentication}. This time, the parties Alice and Bob want to ensure that they are effectively communicating to each other, or in other words, nobody else is \emph{impersonating} one of the two. The objects used here are:
\begin{itemize}
    \item The data to be shared, or \emph{message} $m$;
    \item Some additional secret information, shared by the two parties, that is used to \emph{sign} the message: the \emph{authentication key} or just \emph{key} $k$;
    \item The result of signing a message $m$ using the key $k$: the \emph{signature} or \emph{tag} $\phi$.
\end{itemize}

\begin{figure}[ht]
    \centering
    \begin{tikzpicture}
        \draw
            (0, 0) node (a) [box, fill = white] {Alice} 
            (5, 0) node (b) [box, fill = white] {Bob}
            (2.5, -1) node (e) [box, fill = white] {Eve}
        ;

        \draw[-Stealth] (a) -- node [midway, above] {$(m, \phi)$} (b);
        \draw[-Stealth] (0, -1) node [below] {$k$} -- (a);
        \draw[-Stealth] (5, -1) node [below] {$k$} -- (b);
        \draw[-Stealth] (-1, 0) node [left] {$m$} -- (a);
        \draw[-Stealth] (b) -- (6, 0) node [right] {$1$};
        \draw[-Stealth] (2, -0.1) -- (3, -0.1)
            (2.5, -0.1) -- (e);

    \end{tikzpicture}

    \caption{A depiction of the problem of authentic communication}
    \label{fig:authentication}
\end{figure}

Here is employed a \emph{cryptographic authentication scheme}, or just \emph{authentication scheme}, typically taking the form $\Psi = (\Tag, \Ver)$, where:
\begin{itemize}
    \item $\Tag \in \K \times \M \to \Phi$ is the function that, given a message $m$ in $\M$ and a key $k$ in $\K$ generates the signature $\phi$
    \item $\Ver \in \K \times \M \times \Phi \to \binary$ decides whether $\phi$ is the correct signature for the message $m$ using the key $k$. It is udeful to note that, in various schemes, the verifier simply consists in invoking the tagging function, and comparing the resulting signature with the given one:
    \[
        \Ver(k, m, \phi) = [\Tag(k, m) = \phi]
    \]
\end{itemize}

Given such definitions, an authentication scheme works as intended if and only if:
\[
    \forall m \in \M, \forall k \in \K \implies \Ver(k, m, \Tag(k, m)) = 1
\]

The security aspect to consider in these schemes is the signatures' \emph{unforgeability}: if Eve is able to obtain a signature $\phi$ of a message $m$ of her own choice, then she shall have no better means of \emph{forging} a valid couple $(m', \phi')$ such that $\Ver(k, m', \phi') = 1$ other than random guessing, or knowing the key, of course. The coming definitions will help to give this notion more formal ground:

\begin{definition} \emph{($\varepsilon$-statistical one-time security)}:
    An authentication scheme has $\varepsilon$-statistical one-time security iff, given a valid couple $(m_1, \phi_1)$, any adversary cannot \emph{forge} a fresh valid couple $(m_2, \phi_2)$ without knowing the signature key $k$:
    \[
        \forall m_1 \neq m_2 \in \M \, \forall \phi_1, \phi_2 \in \Phi \quad \Pr[\Tag(K, m_2) = \phi_2 \knowing \Tag(K, m_1) = \phi_1] \leq \varepsilon
    \]
\end{definition}

\subsubsection{Pairwise-independent hashing}

%AP190903: The math stinks here...
\begin{definition}
    Let $\mathcal{S}$ be a seeding space; define a family of \emph{hash functions} to be the following object\footnotemark:
    \[
        H \in \mathcal{S} \to (\M \to \Phi) : s \mapsto h_s
    \]

    \footnotetext{This kind of notation consisting in putting an argument as a subscript to a generic,  typically of higher-order function, is also called ``currying''; it will be used extensively throughout the lessons.}

    Let $S$ be a random variable in the seed space; the hash functions are deemed \emph{pairwise-independent}\footnotemark{} iff, for any two distinct messages $m$ and $m'$, the pair $(h_S(m), h_S(m'))$ distributes evenly in $\Phi^2$. In other words:
    \[
        \forall m, m' \in \M : m \neq m', \forall \phi, \phi' \in \Phi \implies \Pr[h_S(m) = \phi \wedge h_S(m') = \phi'] = \oneover{|\Phi|^2}
    \]
\end{definition}

\footnotetext{Care should be taken to not confuse \emph{pairwise} independency with \emph{mutual} independency: while the former acts only on pairs, the latter considers all possible subsets. The two notions are not necessarily equivalent.}

As an example of such a family, consider the additive group of integers modulo $p$: $(\integer_p, +)$, where $p$ is a prime integer. Define the family:
\[
    h_{(a, b)}(x) = ax + b \mod p
\]

where $\mathcal{S} = \integer_p^2$, and $\M = \Phi = \integer_p$. 

\begin{theorem}
    The functions in the family $H$ are pairwise-independent.
\end{theorem}

\begin{proof}
    Let $S = (a, b)$ be a random seed for $H$; for any distinct messages $m, m'$, and for any tags $\phi, \phi'$:

    \begin{align*}
        & \Pr[h_S(m) = \phi \wedge h_S(m') = \phi'] \\
        =& \Pr[am + b = \phi \wedge am' + b = \phi'] \\
        =& \Pr\left[
        \begin{pmatrix}
            m & 1 \\
            m' & 1
        \end{pmatrix}
        \cdot
        \begin{pmatrix}
            a \\
            b
        \end{pmatrix}
        =
        \begin{pmatrix}
            \phi \\
            \phi'
        \end{pmatrix}
        \right] \\
        =& \Pr\left[
        \begin{pmatrix}
            a \\
            b
        \end{pmatrix}
        =
        \begin{pmatrix}
            m & 1 \\
            m' & 1
        \end{pmatrix}^{-1}
        \cdot
        \begin{pmatrix}
            \phi \\
            \phi'
        \end{pmatrix}
        \right] \\
        =& \oneover{|\integer_p|^2}
    \end{align*}
    which is exactly the definition of pairwise-independency.
\end{proof}

\begin{theorem}
    Define an authentication scheme to be such that its tagging routine is a hash function family: $\Tag(k, m) = h_k(m)$. Let this function family be pairwise-independent. Then the authentication scheme is $\oneover{|\Phi|}$-statistical one-time secure. 
\end{theorem}

\begin{proof}
    For all our purposes, let $K \sim \unifdist(\K)$; then:

    \begin{align*}
        \forall m \in \M, \phi \in \Phi &\implies \Pr[\Tag(K, m) = \phi] = \Pr[h_K(m) = \phi] = \oneover{|\Phi|} \\
        \forall m \neq m' \in \M, \phi, \phi' \in \Phi &\implies \Pr[\Tag(K, m) = \phi \wedge \Tag(K, m') = \phi'] = \oneover{|\Phi|^2} \\
    \end{align*}
    Therefore:
    \begin{align*}
        & \Pr[\Tag(K, m') = \phi' \knowing \Tag(K, m) \evaluatesto \phi] \\
        =& \frac{\Pr[\Tag(K, m') = \phi' \wedge \Tag(K, m) = \phi]}{\Pr[\Tag(K, m) = \phi]} \\
        =& \frac{|\Phi|}{|\Phi|^2} = \oneover{|\Phi|}
    \end{align*}
\end{proof}

\begin{theorem}
    Any $(2^{-\lambda})$-statistical $t$-time secure authentication scheme has a key of size $(t + 1) \lambda$ for any $\lambda > 0$.
\end{theorem}

\begin{proof}
    None given.

    Idea: For each ``time'', use one ``subkey''
\end{proof}
