\mychapter{12}{Lesson 12} %181107
\section{Hashing}

Remember one solution to domain-extension for \prf{}s, as a composition of a \prf{} $F$ with an almost universal hash function $H$. Hash functions compress their arguments to some ``fingerprint'', which is assumed to be unique. However, since this compression in this context inherently introduces information loss, it is not guaranteed that every message gets its own unique fingerprint; indeed, there will be some instances where two messages yield the same hash value, or in other words, the hashes \emph{collide}. It is desirable for a hash function to be \emph{resistant} to these events, meaning that it is hard to reproduce such collisions.

\begin{definition}
    A hash function family $H$ is deemed \emph{collision-resistant}, denoted as $H \in \crh$ iff the probability of finding a collision is negligible, even when knowing the seed $s$. Formally:
    \[
        \forall \adversary \in \ppt \implies \Pr(\cryptog{crh}(\lambda) = 1) \in \negl(\lambda)
    \]

    \begin{cryptogame}
        {crh}
        {The \emph{collision-resistance} game}
        {crh}

        \receive{$s \pickUAR \binary^\lambda$}{$s$}{}

        \send{}{$x, y$}{\textsc{Output 1 iff} $h_s(x) = h_s(y)$}
        
    \end{cryptogame}
\end{definition}

A note: before, we were dealing with unbounded adversaries, and the key was hidden. Now the key is public, but the adversary must be efficient.

\begin{exercise}
    Let $\Pi$ be a \ufcma{} authentication scheme over the message space $\binary^n$. Show that $\Pi' = (\Tag', \Ver') : \Tag'_{k, s}(m) = \Tag_k(h_s(m))$ is \ufcma-secure over $\binary^l$, where $l \in \poly(n)$, as long as $h_s$ itself is \crh.
\end{exercise}

\subsection{Merkle-Damg\r{a}rd construction}

A construct for starters has been defined by Ralph Merkle and Ivan Damg\r{a}rd, which revolves around the use of a \crh{} function $h_s \in \binary^{n + c} \to \binary^n$, where $c$ is the size of one ``block'' of a message. For now, consider the messages to be of an arbitrarily fixed length $l$, and let $c$ be 1.  The steps to follow are:

\begin{enumerate}
    \item Let $\textsc{iv} \in \binary^b$ be an \emph{initialization vector}
    \item Initialize $t_0 := \textsc{iv}$
    \item For each bit $b_i$ of the message to hash $m$:
    \begin{enumerate}
        \item Compute $t_i := h_s(b_i, t_{i - 1})$
    \end{enumerate}
    \item Return $\phi = t_n$
\end{enumerate}

% AP190203: To review, it is somewhat confused
%First step: Compress the original message (assuming fixed size) by one bit

%Let $H$ be a one-bit shrinking function. Then, it can be used to construct a hash function H' that splits an arbitrary-size message into fixed-size blocks, apply H onto them, and return a digest of fixed length. This is exemplified by the diagram in figure \ref{fig:mdbase}

\begin{figure}
    \centering

    \tikzstyle{int}   = [draw, minimum size=2em]
    \tikzstyle{empty} = [minimum size=2em]
    \tikzstyle{init}  = [pin edge={to-,thin,black}]

    \begin{tikzpicture}[node distance = 1.9cm, auto, >=latex']

        \node (a) [empty] {$\textsc{iv}$};
        \node (r) [int, pin={[init]above:$b_1$}] [right of=a] {$h_s$};
        \node (d) [int, pin={[init]above:$b_2$}] [right of=r] {$h_s$};
        \node (e) [int, pin={[init]above:$b_3$}] [right of=d] {$h_s$};
        \node (f) [empty] [right of=e] {$...$};
        \node (g) [int, pin={[init]above:$b_n$}] [right of=f] {$h_s$};
        \node (h) [empty, right of=g] {$\phi$};

        \path[->] (a) edge (r);
        \path[->] (r) edge node {$t_1$} (d);
        \path[->] (d) edge node {$t_2$} (e);
        \path[->] (e) edge node {$t_3$} (f);
        \path[->] (f) edge node {$t_{n-1}$} (g);
        \path[->] (g) edge (h);
    
    \end{tikzpicture}
    \caption{Basic outline of a Merkle-Damg\r{a}rd construction}
    \label{fig:mdbase}
\end{figure}

Figure \ref{fig:mdbase} depicts a general view of the algorithm. Let it be denoted as another hash function $h'_s \in \binary^l \to \binary^n$.

\begin{theorem}
    The construction $H'$ obtained by Merkle-Damg\r{a}rd is a \textsc{crh} function.
\end{theorem}

% AP190204: Tentative
\begin{proof}
    Assume $H'$ can be broken efficiently by a distinguisher $\distinguisher^{\crh}$, meaning that finding two distinct block sequences that give the same hash is easy. Consider the reduction to $H$'s \crh-ness in figure \ref{cryptoredux:mdcrh}

    \begin{cryptoredux}
        {mdcrh}
        {Breaking the underlying \crh-ness of $H$}
        {h-crh}
        {h'-crh}[3]

        \receive{}{$s$}{}
        \invoke{}{$s$}{}
        \return{}{$x, y$}{}

        \cseqdelay

        % AP190911: need a better way to convey meaning here
        \send{\shortstack[r]{
            \textsc{Find} $j :$ \\
            $h'_{s, (j - 1)}(x) \neq h'_{s, (j - 1)}(y)$ \\
            $\wedge\: h'_{s, (j)}(x) = h'_{s, (j)}(y)$
        }}{$h'_{s, (j - 1)}(x), h'_{s, (j - 1)}(y)$}{}

    \end{cryptoredux}

    Ignore same blocks: find the largest j such that:
    \[
        (b_j, x_{j - 1}) \neq (b'_j, y_{j - 1}) \wedge h_s(b_j, x_{j - 1}) = h_s(b'_j, y_{j - 1})
    \]
    this implies the rest of the message is equal, then the resulting final hash will be equal, thus for $j > 0$ we have a collision.

\end{proof}

%AP190911: Not clear at all
\subsubsection{Domain extension}

In order to adapt the MD-construct to messages of variable length, some sort of ``strengthening'' is required: 

% Also called MD-strengthening
\begin{lemma}[Length padding]
    Let $H_s \in \binary^{n + l} \to \binary^n$, then:
    %AP190911: This needs to be rewritten correctly
    \[
        H'_s = H_s(\left<l'\right>, H_s(x_{l'}, \dots , H_s(x_1, 0^n) \dots)), |l'|, |x_i| \in \binary^c 
    \]
\end{lemma}

This is an example of padding which encodes the message length in itself.

\begin{theorem}
    The strengthened construct is collision-resistant for variable-length messages.
\end{theorem}

\begin{proof}
    Hint: similar as above, case by case
\end{proof}

\subsubsection{Merkle trees}

This is an alternative construction to the ``linear'' approach used beforehand: it starts from the message's blocks acting as leaves of a complete binary tree, and using a halving compression function on each node from the bottom up, and outputting the final tag as the image of the root invocation. It also has nice properties, such as easy verification of the presence of a single specific block in the message.

\subsection{Compression functions}

The compression functions are the central point of these kinds of ``fingerprinting'' tag schemes. They are, essentially, hash functions with a smaller codomain than its domain, thereby forcing the existence of collisions.

Let ($\textsf{Gen} : 0 \mapsto (\pk, \sk), f, g)$ be a \pke{} scheme, where the functions $f$ and $g$ are keyed \prp{}s. A \emph{claw} is a couple of values $(x, x')$ such that:
\[
    f(\pk, x) = g(\pk, x')
\]

\begin{cryptogame}
    {cfp}
    {The game of claw-free permutations}
    {cfp}
    
    \receive{$(\pk, \sk) \pickUAR \textsf{Gen}$}{$\pk$}{}
    \send{}{$x, x'$}{\textsc{Output 1 iff} $f(\pk, x) = g(\pk, x')$}

\end{cryptogame}

\begin{theorem}
    Assuming $\mathcal{F}$ is claw-free, then $h_\pk$ is \crh{} from $n + l$ bits to $n$.
\end{theorem}

\subsubsection{Davies-Meyer construct}

% AP190911: How does AES come in here?!
\begin{definition}
    $x_{i + 1} = E_k(x_i) \oplus x_i$, maps n+$\lambda$ to n. E is AES
\end{definition}
