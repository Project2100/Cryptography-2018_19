\mychapter{15}{Lesson 15} %181116
\section{Public key encryption recap} 

$\cryptog{pke-cca}{}$:

\begin{cryptogame}
    {pkecca}
    {\cca{} on a \pke{} scheme}
    {pke-cca}
    
    \receive{\small{$(pk, sk) \pickUAR \mathcal{KG}en(1^\lambda)$}}{$pk$}{}

    \cseqdelay
    \cseqbeginloop

    \send{}{$c'$}{}
    \receive{}{$m'$}{}

    \cseqendloop
    \cseqdelay

    \send{}{$m_0, m_1$}{}

    \receive{\shortstack[l]{
        $ b \pickUAR \binary $ \\
        $ c \pickUAR \enc(pk, m_b) $
    }}{$c$}{}

    \cseqdelay
    \cseqbeginloop

    \send{}{$c'$}{}
    \receive{}{$m'$}{}

    \cseqendloop
    \cseqdelay

    \send{}{$b'$}{\textsc{Output 1 iff} $b' = b$}
\end{cryptogame}


(Reminder): Encryptions are made like this: $Enc(k, m) = (r, F_k(r) \oplus m), r \simeq \mathcal{U}nif(\{0, 1\}^\lambda)$.

Every time an encryption is made, a fresh value $r$ is picked UAR.

\pagebreak
\subsection{Trapdoor permutation}

A \emph{trapdoor permutation} (or \tdp) is a \owp{} family structured has these features:

\begin{itemize}
    \item A key pair is chosen \uar{} by a key generator algorithm:
    \[
        (pk, sk) \pickUAR \mathcal{KG}en(1^\lambda)
    \]
    \item There is a function family $f_{pk} \subseteq (V_{pk} \to V_{pk})$ such that:
    \begin{itemize}
        \item Computing $f_{pk}$ is efficient
        \item Domain sampling ($x \pickUAR V_{pk}$) is efficient
    \end{itemize}
    \item There is an efficient function $g_{sk}$ that ``inverts'' $f_{pk}$ ($sk$ is the ``trapdoor''): 
    \begin{equation*}
        g(sk, f(pk, x)) = x
    \end{equation*}
    \item No efficient adversary is able to invert $f_{pk}$ wihout knowing $sk$
\end{itemize}

Note: Since $pk$ is public, any adversary gets the capability of encrypting messages for free, without requiring an external challenger/oracle!

Therefore, if left deterministic, a \tdp{} is not \cpa-secure.

%The idea is to reuse the message as part of the randomness, or more in detail, use its hardcore predicate. The resulting scheme would the look like this:
Here, in this scheme, we combine randomness and the notion of hardcore predicate $\mathfrak{hc}$:

% AP190123 - IMPORTANT NOTE: This scheme is intended to work for single-bit messages
\begin{itemize}
    \item $(pk, sk) \pickUAR \mathcal{KG}en(1^\lambda)$
    \item $r \pickUAR \Xi_{pk}$
    \item $c := Enc(pk, m) = (f_{pk}(r), \mathfrak{hc}(r) \oplus m)$
    \item Correctness: $Dec(sk, c) = \mathfrak{hc}(g_{sk}(c_1)) \oplus c_2$
    % Exercise: If obscure, do the petty algebra on paper
\end{itemize}

\begin{theorem}
    If $(\mathcal{KG}en, f, g)$ is a TDP, and $\mathfrak{hc}$ is hardcore for $f$, then the above scheme is \cpa-secure.    
\end{theorem}

\begin{proof} The proof is left as exercise

    \todo{Apparently, the reduction here is not easy at all, some hints are needed.}
\end{proof}

\subsection{\textsc{Tdp} examples}

One example stems form the factoring problem: let's look again at $\mathbb{Z}_n^\times$, where $n = pq$, $p, q \in \mathbb{P}$:

\begin{theorem}
    (Chinese remainder, or CRT): The following isomorphisms to $\mathbb{Z}_n^\times$ are true:

\begin{itemize}
    \item $\mathbb{Z}_n \simeq \mathbb{Z}_p \times \mathbb{Z}_q$
    \item $\mathbb{Z}_n^\times \simeq \mathbb{Z}_p^\times \times \mathbb{Z}_q^\times$
\end{itemize}

Note that the theorem is more general, and holds for any $p, q$ that are coprime.
\end{theorem}

How to use this theorem for constructing a \pke{} scheme:

Reminder (Euler's theorem):\todo{ $\forall x \in \mathbb{Z}_n \implies
x^{\varphi(n)} = x \mod n$ \\
maybe the correct one is \\
$\forall x \in \mathbb{Z}_n \implies
x^{\varphi(n)} = 1 \mod n$
}

Reminder: $\forall p, q \in \mathbb{P} \implies \varphi(pq) = (p-1)(q-1)$

So let $a$ be the public key such that $\gcd(a, \varphi(n))=1$, then $\exists! b
\in \mathbb{Z}_n : ab = 1 \mod \varphi(n)$, $b$ will be our private key.

Define encryption as $f(a, m) = m^a \mod n$, and then decryption as $g(b, c) =
c^b \mod n$.

Observe that 
\[
    g(b, f(a, m)) = (m^a)^b =m^{ab}=m^{k\varphi(n)+1}=(m^{\varphi(n)})^{k}m = m \mod n
\]
because $ab = 1 \mod \varphi(n)$.

So we conjecture that the above is a valid \tdp-based \pke{} scheme. This is actually referred to as the \emph{\rsa{} assumption}, and is depicted in figure \ref{cryptogame:rsaass}

\begin{cryptogame}
    {rsaass}
    {The \rsa{} assumption}
    {rsa}

    \receive{\shortstack[l]{
        $b \pickUAR \{0, 1\}$ \\
        $c \pickUAR \textrm{Enc}(pk, m_b)$
    }}{$n, pk, m^{pk}$}{}

    \cseqdelay

    \send{}{$m'$}{\textsc{Output 1 iff} $m' = m$}

\end{cryptogame}


Relation to the factoring problem: $\rsa \implies \textsc{fact}$

Proof: Given $p, q$, an adversary can compute $\varphi(n) = (p-1)(q-1)$, and then find the inverse of the public key in $\mathbb{Z}_{pq}^\times$.

It hasn't been proven that $\textsc{fact} \implies \rsa$


\section{Textbook \rsa}
This is an \underline{insecure} toy example of the more complex \rsa{} (Rivest Shamir Adleman) scheme.

\begin{itemize}
    \item Setup: $pk \pickUAR \integer_n^\times, sk \equiv_{\varphi(n)} pk^{-1}$
    \item Encrypt: $Enc_{pk}(m) = m^{pk} \mod{n}$
    \item Decrypt: $Dec_{sk}(c) = c^{sk} \mod{n}$
    \item Correctness: $Enc_{pk}(Dec_{sk}(m)) = m^{pk \cdot sk} \equiv_n m$
\end{itemize}

Again, since the encryption routine is deterministic, the scheme is not \cpa-secure. However, a hardcore predicate can be inserted to the routine: $\hat{m} = r||m$, where $r \pickUAR \{0, 1\}^l$. Now the encryption is pseudorandom.

\textbf{Facts:}
\begin{enumerate}
    \item $l \in \omega(log(\lambda))$ otherwise it is possible to eficiently bruteforce it.
        \todo{\item If $m \in \{0,1\}$ then I can prove it CPA secure under RSA
        (just use standard TDP)}
    \item If $m$ is "in the middle" ($\{0,1\} \leq m \leq \{0, 1\}^l$) RSA is believed to be secure and is \underline{standardized} (PKCS\#1,5)
        \todo{ \item Still not CCA secure! counterexample?}
\end{enumerate}


\subsection{Trapdoor Permutation from Factoring}
Let's look at $f(x) = x^2\mod{n}$ where $f: \integer_n^\times \to \QR[n] (\subset \integer_n^\times)$, this is not a permutation in general.

Now let's consider the Chinese Reminder Theorem (CRT) representation:

\begin{gather*}
    x=(x_p,x_q) \rightarrow x_p\equiv x\mod{p} , x_q\equiv x\mod{q}\\
    f(x)=x^2\mod{p}; x \pickUAR \integer_p^\times
\end{gather*}

Since $\integer_p^\times$ is cyclic I can always write:

\begin{gather*}
    \integer_p^\times = \{g^0,g^1,g^2,\ldots,g^{\frac{p-1}{2}-1},g^{\frac{(p-1)}{2}},\ldots,g^{p-2} \} \\
    \QR[p]=\{g^0,g^2,g^4,\ldots,\overbrace{g^{p-3}}^{g^{\frac{p-1}{2}-1} \in \integer_p^\times},\underbrace{g^0}_{g^{\frac{p-1}{2}} \in \integer_p^\times},\ldots\} \\
    |\QR[p]|=\frac{p-1}{2}
\end{gather*}

Moreover, since $(g^{\frac{p-1}{2}})^2 \equiv 1 \mod{p} $ and $g^{\frac{p-1}{2}}$ cannot be 1 (since $g^{0}\not=g^{\frac{p-1}{2}}\not=g^{p-1}$ ) but must be one of the $p-1$ elements of $\integer_p^\times$, then $g^{\frac{p-1}{2}} \equiv -1 \mod{p}$.

Now it's possible to show that $f: \QR[p]\to \QR[p] $ is a permutation, and we are going to show a method to invert it, aka $f^{-1}$.

Assume $p\equiv 3 \mod{4}$ ([*]$p=4t+3\Rightarrow t=\frac{p-3}{4}$), then squaring modulo $p$ is a permutation because, given
\underline{$y=x^2 \mod{p}$} if I compute:
\begin{gather*}
    (y^{t+1})^2=\underbrace{y^{2t+2}}_{\text{[*]
    }2t+2=\frac{p-3}{2}+2=\frac{p+1}{2}=\frac{p-1}{2}+1}=(x^2)^{\frac{p-1}{2}+1}=1x^2=x^2\\
    \implies x=\pm y^{t+1}
\end{gather*}

But only 1 among the above $\pm y^{t+1}$ is a square, in particular only the positive one. Since we have that:
\[ 
    p=k4+3 \Rightarrow \frac{p-1}{2}=\frac{4k+2}{2}=2k+1
\]
so $\frac{p-1}{2}$ is odd.

Now, since we are considering just $\QR[p]=\{y \in \mathbb{Z}_{p}^{*} : \exists x \in \mathbb{Z}_{p}^{*} x^{2}=y\}$ and we can write each $x \in \mathbb{Z}_{p}^{*} $ as $g^{z}$ for a $z \in \mathbb{Z}_{p}$, 
\[
    y=x^{2} \Leftrightarrow y=(g^{z})^{2}=g^{2z}
\]

So, $y = g^{z'} \in \QR[p] \Leftrightarrow z'$ is even. If $z'$ is odd, then $y \notin \QR[p]$.

Since $\frac{p-1}{2}$ is odd, then $g^{\frac{p-1}{2}} \not\in \QR[p]$, and since it is possible to generate all of the other numbers with odd exponents 
\[
    g^{odd}=g^{\frac{p-1}{2} \pm even} = g^{\frac{p-1}{2}}g^{ \pm even} \Rightarrow -1(g^{\pm even})
\]
and $g$ powered to odd exponents will have this form.

From that, it's possible to state the following:
\begin{lemma}
    $\forall z, z\in \QR[p] \implies -z \notin \QR[p]$
\end{lemma}

\subsection{Rabin's Trapdoor permutation}

Now we study a one way function built on previous deductions about number theory and modular arithmetic.

The \textit{Rabin trapdoor permutation} is defined as 
\[
    f(x)=x^{2} \text{ mod } n
\]
where $n=p*q$ for primes $p, q = 3 \text{mod} 4$.

We can observe that the image of this function is $\QR[n]$, a subset of $\integer_n^\times$.

For \textbf{Chinese remainder theorem} it is possible to state that $f$ maps as follows
\[
    x = (x_p, x_q) \mapsto (x^2_p, x^2_q)
\]
since each element of $\integer_n$ has always two different forms, in $\integer_p$ and in $\integer_q$.

So
\[
    y \in  \mathbb{QR}_{n} \Leftrightarrow y_{p} \in \mathbb{QR}_{p} \wedge y_{q} \in \mathbb{QR}_{q} 
\]

As before, the image of $f$ is exactly
\[
    \mathbb{QR}_{n} = \{ y: \exists x : y=x^{2} \text{ mod } n\}
\]

If we try to invert the function $f$, even without applying the previous inversion algorithm, we easily note that among the 4 possible values:

\[
    f^{-1}(y)=\{ (x_{p}, x_{q}),(-x_{p}, x_{q}),(x_{p},- x_{q})(-x_{p},- x_{q})\}
\]\label{les15:outoffour}

only 1 is a quadratic residue since we said, in the last lemma, that only one out of $-x_{k}, x_{k}$ is a quadratic residue for $k=q,p$.

Therefore, we have that the Rabin's TDP is a permutation in $\QR[n]$, and that the cardinality of $ \mathbb{QR}_{n} $ is $\frac{|\integer_n^\times|}{4}$.

Furthermore, with the following claim we can state that the Rabin cryptosystem is OWF thanks to the hardness of factoring.

\begin{claim}
    Given $x, z$ such that $x^{2}\text{ mod} n \equiv z^{2} \text{ mod } n
    \equiv y$ mod $n$,
    \[
        x \neq \pm z \Rightarrow \text{ we can factor } n
    \]
\end{claim}

\begin{proof}
    Since $f^{-1}(y)$ has only one value out of four, $x\neq \pm z$ and $z\in \{(x_{p},x_{q}),(-x_{p},-x_{q})\}$ , then $x \in \{(x_{p},-x_{q}),(-x_{p},x_{q})\}$ and 
    \[
        x + z \in \{(0,2x_{q}), (2x_{p}, 0)\}
    \]

    Now assume $x + z = (2x_{p}, 0)$ without loss of generality, since the proof for the other case is the same. We have that $x+z \equiv 0$ mod $q$ and $x+z \not\equiv 0 $ mod $p$. But then $gcd( x+z , n)=q$, and we obtain $q$.
\end{proof}

\begin{theorem}
    Squaring mod $n$, where $n$ is a \emph{Blum integer}\footnotemark is a trapdoor permutation under the factoring assumption.
\end{theorem}

\footnotetext{a Blum integer $n$ is the product of two numbers $p$ and $q$ such that $p, q = 3 \mod 4$, as the definition of Rabin's \tdp}

Since we have already shown tha Rabin's function is a permutation since it is invertible, we have to show that Rabin's function is also \owf.

In other words
\[
    \text{Factoring is hard } \Rightarrow \text{ $f(x)$ is OWF , aka inverting
    it is hard}
\]
The following proof is by contraddiction.
\begin{proof}
    Assume that exists an adversary PPT who, given $y=x^{2} \text{ mod } n$, can find a $z \in \mathbb{Z}_{n} $ such that $z^{2} \text{ mod } n=y$ but $z\neq \pm x$ .

    We can build the following reduction to show that $\A$ choses $x$, here $\mathcal{BG}en$ is a sampler for Blum integers:

    % AP190902: The adversary had a "fact" subscript, is it important?
    \begin{cryptoredux}
        {blumtdp}
        {---}
        {$(p, q, n) \pickUAR \mathcal{BG}en \wedge p,q \equiv 3 \mod 4$}
        {rabin}[1]

        \receive{}{$n = pq$}{}

        \invoke{\shortstack[l]{
            $x \pickUAR \integer^{*}_n$ \\
            $y = x^2 \mod n$
        }}{$n, y$}{}

        \return{}{$z$}{$f(z) = f(x)$}
        
    \end{cryptoredux}

    
    Once obtained $z\neq \pm x$ which $z^{2}=y$ we can use \textbf{Claim 1}(just summing $x$ and $z$ and analyzing the result)  to factorize $n$ in polytime. But factorizing $n$ in polytime is not possible.
\end{proof}
