\mychapter{15}{Lesson 15} %181116
\section{Public key encryption (cont'd)} 

\subsubsection{\textsc{Cca}-security, revisited}

\begin{cryptogame}
    {pkecca}
    {Chosen ciphertext attacks, revisited for \pke{} schemes}
    {cca2}
    
    \receive{$(\pk, \sk) \pickUAR \keygen(1^\lambda)$}{$pk$}{}

    \cseqdelay
    \cseqbeginloop

    \send{}{$c'$}{}
    \receive{$m' = \Dec(\sk, c)$}{$m'$}{}

    \cseqendloop
    \cseqdelay

    \send{}{$m_0, m_1$}{}

    \receive{\shortstack[l]{
        $b \pickUAR \binary$ \\
        $c \pickUAR \Enc(\pk, m_b)$
    }}{$c$}{}

    \cseqdelay
    \cseqbeginloop

    \send{}{$c'$}{}
    \receive{$m' = \Dec(\sk, c)$}{$m'$}{}

    \cseqendloop
    \cseqdelay

    \send{}{$b'$}{\textsc{Output 1 iff} $b' = b$}
\end{cryptogame}

\subsection{Trapdoor permutations}

Let us consider the following scheme $\Pi$, which is an adaption of the old \cpa-secure \ske{} to public keys:
\begin{itemize}
    \item $\Enc(\pk, m) = (r, f_\pk(m \oplus r))$, where $r \sim \unifdist(\binary^\lambda)$
    \item $\Dec(\sk, (c_0, c_1)) = g_\sk(c_1) \oplus c_0$
    \item Correctness: $g_\sk(c_1) \oplus c_0 = g_\sk(f_\pk (m \oplus r)) \oplus r = m \oplus r \oplus r = m$
\end{itemize}

A \emph{trapdoor permutation} (or \tdp) is a \owp{} having the following features:

\begin{itemize}
    \item A key pair is chosen \uar{} by a key generator algorithm:
    \[
        (\pk, \sk) \pickUAR \keygen(1^\lambda)
    \]
    \item There is a function family $F \in \Xi_\pk \to (V_{\pk} \to V_{\pk})$ such that:
    \begin{itemize}
        \item Computing $f_{\pk}$ is efficient
        \item Domain sampling ($x \pickUAR V_{\pk}$) is efficient
    \end{itemize}
    \item There is an efficient function $g_{\sk}$ that efficiently \emph{inverts} $f_{\pk}$, where $sk$ is the ``trapdoor'': 
    \[
        g(\sk, f(\pk, x)) = x
    \]
    \item No efficient adversary is able to invert $f_{\pk}$ wihout knowing $\sk$
\end{itemize}

Note that because $\pk$ is public, an adversary can perform any polynomial number of encryptions with $\pk$, and see the corresponding ciphertext. This is the same characteristic of \pke{} schemes we described just a while ago. It entails that, if left deterministic, a \tdp{} is not \cpa-secure.

Also, the scheme described above is not \cpa-secure form the start: the adversary, by choosing two messages for the challenge, and receiving the ciphertext $(c_0 = r, c_1)$, along with the public key, has everything needed to reconstruct the encryption and check whichever message was encrypted, much like the problem of \ufcma against a deterministic \mac scheme.

%The idea is to reuse the message as part of the randomness, or more in detail, use its hardcore predicate. The resulting scheme would the look like this:
Here, in this scheme, we combine randomness and the notion of hardcore predicate $\hc$:

% AP190123 - IMPORTANT NOTE: This scheme is intended to work for single-bit messages
\begin{itemize}
    \item $(pk, sk) \pickUAR \keygen(1^\lambda)$
    \item $r \pickUAR \Xi_{pk}$
    \item $c := \Enc(\pk, m) = (f_{\pk}(r), \hc(r) \oplus m)$
    \item Correctness: $\Dec(\sk, c) = \hc(g_{\sk}(c_1)) \oplus c_2$
    % Exercise: If obscure, do the petty algebra on paper
\end{itemize}

\begin{theorem}
    If $F$ is a \tdp{} and $\hc$ is hardcore for $f$, then the above scheme is \cpa-secure.    
\end{theorem}

\begin{proof} The proof is left as exercise
    \todo{Apparently, the reduction here is not easy at all, some hints are needed.}
\end{proof}

\subsection{\textsc{Tdp} examples}

One example stems form the factoring problem: let's look again at $\integer_n^\times$, where $n$is the product of two prime numbers $p$, and $q$:

\begin{theorem}[Chinese remainder theorem]
    The following isomorphisms to $\integer_n^\times$ are true:

    \begin{itemize}
        \item $\integer_n \simeq \integer_p \times \integer_q$
        \item $\integer_n^\times \simeq \integer_p^\times \times \integer_q^\times$
    \end{itemize}

    Note that the theorem is more general, and holds for any two numbers $p$ and $q$ that are coprime.
\end{theorem}

How to use this theorem for constructing a \pke{} scheme:

\begin{theorem}[Euler's totient theorem]
    Let $x$, $n$ be two coprime numbers. Then $x^{\varphi(n)} \equiv_n 1$
\end{theorem}

Also, remember that $\forall p, q \in \primenum \implies \varphi(pq) = (p - 1)(q - 1)$

So let $a$ be the public key such that $\gcd(a, \varphi(n)) = 1$, then $\exists! b \in \integer_n : ab \equiv_{\varphi(n)} 1$; $b$ will be our private key. Define encryption as $f(a, m) = m^a \mod n$, and then decryption as $g(b, c) = c^b \mod n$. Observe that 
\[
    g(b, f(a, m)) = (m^a)^b = m^{ab} = m^{k \varphi(n) + 1} = (m^{\varphi(n)})^k m \equiv_n m
\]
because $ab = 1 \mod \varphi(n)$.

So we conjecture that the above is a valid \tdp-based \pke{} scheme. This is actually referred to as the \emph{\rsa{} assumption}, and is depicted in figure \ref{cryptogame:rsaass}

\begin{cryptogame}
    {rsaass}
    {The \rsa{} assumption}
    {rsa}

    \receive{\shortstack[l]{
        $b \pickUAR \binary$ \\
        $c \pickUAR \textrm{Enc}(pk, m_b)$
    }}{$n, pk, m^{pk}$}{}

    \cseqdelay

    \send{}{$m'$}{\textsc{Output 1 iff} $m' = m$}

\end{cryptogame}


Relation to the factoring problem: $\rsa \implies \textsc{fact}$

Proof: Given $p, q$, an adversary can compute $\varphi(n) = (p - 1)(q - 1)$, and then find the inverse of the public key in $\integer_{pq}^\times$.

It hasn't been proven that $\textsc{fact} \implies \rsa$

\section{Textbook \rsa}

This is an insecure toy example of the more complex \rsa{} (Rivest Shamir Adleman) scheme:

\begin{itemize}
    \item Setup: $(\pk, \sk) \pickUAR \keygen(\integer_n^\times) : \sk \equiv_{\varphi(n)} \pk^{-1}$
    \item Encryption: $\Enc(\pk, m) = m^{\pk} \mod n$
    \item Decryption: $\Dec(\sk, c) = c^{\sk} \mod n$
    \item Correctness: $\Enc(\pk, \Dec(\sk, m)) = m^{\pk \cdot \sk} \equiv_n m$
\end{itemize}

Again, since the encryption routine is deterministic, the scheme is not \cpa-secure. However, a hardcore predicate can be inserted to the routine: $\hat{m} = r||m$, where $r \pickUAR \binary^l$. Now the encryption is pseudorandom.

Some interesting facts:
\begin{enumerate}
    \item $l \in \omega(log(\lambda))$ otherwise a brute-force attack becomes viable.
    \item If $m \in \binary$, then the scheme is \cpa-secure under \rsa{} assumption, just use the standard \tdp{}
    \item If $m$ is ``in the middle'': $\binary \leq m \leq \binary^l$; then \rsa{} is believed to be secure (standard PKCS\#1, 5)
    \item However, this construct is not \cca-secure.
    \todo{Counterexample?}
\end{enumerate}

\subsection{Trapdoor Permutation from Factoring}

Here is an attempt to build a \tdp{} over the group $\quadres(n)$: let's look at $f(x) \equiv_n x^2$ where $f \in \integer_n^\times \to \quadres_n (\subset \integer_n^\times)$. Notice that this is not a permutation in general, so let's consider their \textsc{crt}'s representation\footnotemark, in order to restrict $f$'s domain:

\footnotetext{Those well-versed in number representation in computers may know this representation as the \emph{residue number system}}

\[
    x = (x_p \equiv_p x, x_q \equiv_q x), f(x) \equiv_p x^2, x \in \integer_p^\times
\]

Since $\integer_p^\times$ is cyclic:

\begin{align*}
    \integer_p^\times   &= \{g^0, g^1, g^2, \ldots  &&, g^{\frac{p - 1}{2} - 1}              &&, g^{\frac{(p-1)}{2}}                    &&, \ldots, g^{p-2}         &&\} \\
    \quadres_p          &= \{g^0, g^2, g^4, \ldots  &&, g^{2(\frac{p - 1}{2} - 1) = p - 3}   &&, g^{2(\frac{p-1}{2}) = p - 1 \equiv 0}  &&, \ldots, g^{2(p - 2)}    &&\} \\
\end{align*}

Because of this, $|\quadres_p| = \frac{p - 1}{2}$. Moreover, since $g^{2\frac{p-1}{2}} \equiv_p 1$ and $g^{\frac{p-1}{2}}$ cannot be 1 (since $g^0 \neq g^{\frac{p - 1}{2}} \neq g^{p - 1}$ ) but must be one of the $p-1$ elements of $\integer_p^\times$, then $g^{\frac{p-1}{2}} \equiv_p -1$.
    
Now it's possible to show that $f: \quadres_p \to \quadres_p$ is a permutation, and we are going to show how to find $f^{-1}$.

Assume $p \equiv_4 3$, meaning $\exists t : p = 4t + 3 \implies t = \frac{p - 3}{4}$); then squaring modulo $p$ becomes a permutation. Given $y \equiv_p x^2$, observe the following:

\begin{gather*}
    (y^{t + 1})^2 = y^{2t + 2} = y^{2\frac{p - 3}{4} + 2} = y^{\frac{p - 1}{2} + 1} = (x^2)^{\frac{p - 1}{2} + 1} = x^{p - 1} x^2 \equiv_p x^2 \\
    \Downarrow \\
    x = \pm y^{t + 1}
\end{gather*}

But only one among $\pm y^{t + 1}$ is a square: the positive one. Therefore:
\[ 
    p = 4t + 3 \implies \frac{p - 1}{2} = \frac{4t + 2}{2} = 2t + 1
\]
so $\frac{p - 1}{2}$ is odd.

Now, since we are considering just the elements of $\quadres_p$, and we can write each $x \in \integer_p^\times$ as $g^{z}$ for a $z \in \integer_p$:
\[
    y = x^2  = (g^z)^2
\]

So, $y = g^{z'} \in \quadres_p \Leftrightarrow z'$ is even. If $z'$ is odd, then $y \notin \quadres_p$.

Since $\frac{p - 1}{2}$ is odd, then $g^{\frac{p - 1}{2}} \notin \quadres_p$, and since it is possible to generate all of the other numbers with odd exponents 
\[
    g^{odd}=g^{\frac{p-1}{2} \pm even} = g^{\frac{p-1}{2}}g^{ \pm even} \implies -1(g^{\pm even})
\]
and $g$ powered to odd exponents will have this form. From here, it's possible to state the following:

\begin{lemma}
    $\forall z \in \quadres_p \implies -z \notin \quadres_p$
\end{lemma}

\subsection{Rabin's Trapdoor permutation}

Now we study a one way function built on previous deductions about number theory and modular arithmetic. The \emph{Rabin trapdoor permutation} is defined as: 
\[
    f(x) = x^2 \mod n
\]
where $n = pq$ for primes $p, q \equiv_4 3$.

We can observe that the image of this function is $\quadres_n$, a subset of $\integer_n^\times$.

Because of \textsc{crt}, it is possible to state that $f$ maps as follows:
\[
    x = (x_p, x_q) \implies x^2 = (x^2_p, x^2_q)
\]
since each element of $\integer_n$ has always two different forms, in $\integer_p$ and in $\integer_q$. So
\[
    y \in  \quadres_n \Leftrightarrow y_p \in \quadres_p \wedge y_q \in \quadres_q
\]

As before, the image of $f$ is exactly
\[
    \quadres_n = \{ y : \exists x : y \equiv_n x^2\}
\]

If we try to invert the function $f$, even without applying the previous inversion algorithm, we easily note that among the 4 possible values:

\[
    f^{-1}(y) = \{(x_p, x_q),(-x_p, x_q),(x_p, -x_q)(-x_p, -x_q)\}
\]

only 1 is a quadratic residue since we said, in the last lemma, that only one out of $-x_k, x_k$ is a quadratic residue for $k = q,p$.

Therefore, we have that the Rabin's \tdp{} is a permutation in $\quadres_n$, and that the cardinality of $\quadres_n$ is $\frac{|\integer_n^\times|}{4}$. Furthermore, with the following claim we can state that the Rabin cryptosystem is a \owf{} thanks to the \textsc{fact} assumption.

\begin{claim}
    Given $x$ and $z$ such that $x^2 \equiv_n z^2 \equiv_n y$:
    \[
        x \neq \pm z \implies n \text{\textup{ is factorizable}}
    \]
\end{claim}

\begin{proof}
    Since $f^{-1}(y)$ has only one value out of four, $x \neq \pm z$ and $z$ is either $\{(x_p, x_q), (-x_p, -x_q)\}$, then:
    \[
        x \in \{(x_p, -x_q), (-x_p, x_q)\} \implies x + z \in \{(0, 2x_q), (2x_p, 0)\}
    \]
    Now assume $x + z = (2x_p, 0)$ without loss of generality, since the proof for the other case is the same. We have that $x + z \equiv_q 0$ and $x + z \not\equiv_p 0$. But then $\gcd(x + z, n) = q$, and we obtain $q$.
\end{proof}

\begin{theorem}
    Squaring mod $n$, where $n$ is a \emph{Blum integer}\footnotemark is a trapdoor permutation under the factoring assumption.
\end{theorem}

\footnotetext{a Blum integer $n$ is the product of two numbers $p$ and $q$ such that $p, q \equiv_4 3$, as the definition of Rabin's \tdp}

Since we have already shown that Rabin's function is a permutation since it is invertible, we have to show that Rabin's function is also \owf:

\begin{proposition}
\[
    \textsc{fact} \implies f(x) \in \owf \qedhere
\]
\end{proposition}

\begin{proof}
    The proof is by contradiction. Assume there is an adversary \adversary{} who, given $y \equiv_n x^2$, can find an integer $z \in \integer_n$ such that $z^2 \mod n = y$ while $z \neq \pm x$. We can build a reduction as the one in figure \ref{cryptoredux:blumtdp} to show that \adversary{} chooses $x$, here $\mathcal{B}lum$ is a sampler for Blum integers:

    % AP190902: The adversary had a "fact" subscript, is it important?
    \begin{cryptoredux}
        {blumtdp}
        {---}
        {fact}
        {rabin}[1.8]

        \receive{$(p, q, n) \pickUAR \mathcal{B}lum$}{$n = pq$}{}

        \cseqdelay

        \invoke{\shortstack[l]{
            $x \pickUAR \integer_n^\times$ \\
            $y \equiv_n x^2$
        }}{$n, y$}{}

        \cseqdelay

        \return{}{$z$}{$f(z) = f(x)$}
        
    \end{cryptoredux}

    
    Once obtained $z\neq \pm x$ which $z^{2}=y$ we can use \textbf{Claim 1}(just summing $x$ and $z$ and analyzing the result) to factorize $n$ in polynomial time. But factorizing $n$ in polynomial time is not possible.
\end{proof}
