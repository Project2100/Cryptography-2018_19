\mychapter{9}{Lesson 9} %181024

\section{Message Authentication Codes and unforgeability}

After having explored the security concerns and challenges of the \ske{} realm, it is time to turn the attention to symmetric authentication schemes, or \mac{} schemes. Recall that a \mac{} scheme is a couple $(\textit{Tag}, \textit{Verify})$, with the purpose of authenticating the message's source. In this chapter, the tagging function will be denoted as $\textit{Tag}_k$, akin to a \prf.

The desirable property that a \mac{} scheme should hold is to prevent any attacker from generating a valid couple $(m^*, \phi^*)$, even after querying a tagging oracle polynomially many times\footnotemark.
\footnotetext{Do note the similarities (and differences) between unforgeability in this setting and \cpa-security in the secrecy setting}
The act of generating a valid couple from scratch is called \emph{forging}, and the aforementioned property is defined as \emph{unforgeability against chosen-message attacks} (or \ufcma, in short); its game diagram is shown in figure \ref{cryptogame:ufcma}. Do note that $m^*$ is stated to be outside the query set $M$, expressing the ``freshness'' of the forged couple\footnotemark. In formal terms:

\footnotetext{Observe how this setup bears a striking resemblance to the \prf{} game. We're onto something...}

\begin{definition}
    A \textsc{mac} scheme $\Pi$ is \textsc{ufcma}-secure iff:
    \[
        \forall \adversary \in \ppt \implies \Pr[\cryptog{ufcma}(\lambda) = 1] \in \negl(\lambda) \qedhere
    \]
\end{definition}

\begin{cryptogame}
    {ufcma}
    {$\cryptog{ufcma}(\lambda)$}
    {ufcma}

    \cseqchallenger{$k \pickUAR \K$}

    \cseqbeginloop
    \send{}{$m \in M$}{}
    \receive{$\phi = \Tag(k, m)$}{$\phi$}{}
    \cseqendloop
    
    \cseqdelay

    \send{$m^* \notin M$}{$(m^*, \phi^*)$}{\textsc{Output 1 iff} $\Ver(k, m^*, \phi^*) = 1$}
    
\end{cryptogame}

Having defined a good notion of security in the \mac{} scheme domain, we turn our attention to a somewhat trivial scheme, and find out that it is indeed \ufcma-secure:

\begin{theorem}
    Let $f_k$ be a \prf, and define the following \mac{} scheme: 
    \[
        \Pi = (\Tag, \Ver) := (f_k(m), [f_k(m) = \phi])
    \]
    Then $\Pi$ is \ufcma-secure.
\end{theorem}

\begin{proof}
    The usual proof by hybridization to random functions entails. The original game is identical to the \ufcma{} game, where the tagging function is the \prf, whereas the hybrid game will have it replaced with a truly random function, as shown in figure \ref{cryptogame:prfufcmahyb}.

    \begin{cryptogame}
        {prfufcmahyb}
        {$\hybridg{1}(\lambda)$}
        {}

        \cseqchallenger{$R \pickUAR \mathfrak{R}(\lambda, n, l)$}

        \cseqbeginloop
        \send{}{$m_i \in M$}{}
        \receive{$\phi_i = R(m_i)$}{$\phi_i$}{}
        \cseqendloop

        \cseqdelay

        \send{$m^* \notin M$}{$(m^*, \phi^*)$}{\textsc{Output 1 iff} $\Tag(k, m^*) = \phi^*$}

    \end{cryptogame}

    \begin{lemma}
        $\cryptog{ufcma}(\lambda) \compindist \hybridg{1}(\lambda)$  
    \end{lemma}

    \begin{proof}
        By assuming there is a distinguisher $\distinguisher^{\ufcma}$ capable of disproving the lemma, it can be used to distinguish the \prf{} itself, as depicted by the reduction in figure \ref{cryptoredux:prfufcma}
    \end{proof}

    % AP190107: Not so pretty in the last message...
    \begin{cryptoredux}
        {prfufcma}
        {Distinguishing a \prf{} by using $\distinguisher^{\ufcma}$}
        {prf}
        {ufcma}[1.94]

        \cseqchallenger{\shortstack[l]{
            $k \pickUAR \binary^\lambda$ \\
            $R \pickUAR \mathfrak{R}(n, l)$ \\
            $b \pickUAR \binary$
        }}
    
        \cseqbeginloop
        \return{}{$m \in M$}{}
        \send{}{$m$}{}
        \receive{\shortstack[l]{
            $\phi_0 \pickUAR f_k(r)$ \\
            $\phi_1 \pickUAR R(r)$
        }}{$\phi_b$}{}
        \invoke{}{$\phi_b$}{}
        \cseqendloop

        \cseqdelay
    
        \return{$m^* \notin M$}{$(m^*, \phi^*)$}{}
        \send{}{$m^*$}{}
        \receive{\shortstack[l]{
            $\varphi_0 \pickUAR f_k(r)$ \\
            $\varphi_1 \pickUAR R(r)$
        }}{$\varphi_b$}{}

        \cseqdelay

        \send{$b' = \begin{cases}
            0 &\textsc{iff } \varphi_b = \phi^* \\
            1 &\textsc{else}
        \end{cases}$
        }{$b'$}{\textsc{Output 1 iff} $b' = b$}
        
    \end{cryptoredux}

    \begin{lemma}
        $\forall \adversary \in \ppt \implies \Pr[\hybridg{1}(\lambda) = 1] \leq 2^{-l}$
    \end{lemma}

    \begin{proof}
        This is true because the attacker has to predict the output $R(m^*)$ on a fresh input $m^*$ to win the game, which can happen at most with probability $2^{-l}$.
    \end{proof}

    Thus, the conclusion is that $\Pi$ is \ufcma-secure.
\end{proof}

% AP190107: Don't want the next section to creep in before the above figure, so I'm breaking the page. THe best solution would be to put some more text into the previous proof...

\section{Domain extension for \mac{} schemes}

The previous scheme works on fixed-length messages; as in the encryption domain, there are techniques for tagging variable-length messages which are \ufcma-secure. However, before showing them, some other apparently secure modes are described here to give some possible insights on how to tackle the problem. 

Assume the message $m = (m_1, \dots, m_t) \in \binary^{n \cdot t}$ for some $t \geq 1$. Given the tagging function $\Tag_k \in \binary^n \to \binary^l$, an attempt to tag the whole message may be to:

\begin{itemize}    
    \item \textsc{xor} all the message blocks, and then tag: $\phi = \Tag(k, \bigoplus_{i = 1}^{t} m_i)$. This approach opens to some easy forgeries: given an authenticated message $(m, \phi)$, an adversary can create a valid couple $(m', \phi)$, where $m'$ is the original message with two flipped bits in two distinct blocks at the same offset; the resulting \textsc{xor} would be the same.

    \item define the tag to be a $t$-sequence of tags, one for each message block. Again, there is an easy way to forge authenticated messages: an adversary can just flip the position of two arbitrary distinct message blocks along with their relative tags, resulting in a different, yet authentic message.

    \item attempt a variant of the above approach, by adding the block number to the block itself to avoid the previous forging. Yet again, this is not \ufcma-secure: the adversary may just make two queries on two distinct messages, obtain the two tag sequences, and then forge an authenticated message by choosing at each position $i$ whether to pick the message-tag blocks from the first or second query. Possibilities are endless...

\end{itemize}

\subsection{Universal hash functions}

A devised solution which has been proven to be secure relies on the following definition: a function family $H$ which can be used to \emph{shrink} variable length messages and then composed with a \prf{} $f_k$:
\[
   H \in \binary^\lambda \to (\binary^{n \cdot t} \to \binary^n) : s \mapsto h_s
\]
\[
    \Tag((k, s), m) = f_k(h_s(m))
\]

So what are the properties of the such a function family $H \circ F$? The main problem are \emph{collisions}, since for each $m \in \binary^{n \cdot t}$ it should be hard to find $m' \neq m$ such that $h_s(m) = h_s(m')$. But collisions do exist for functions in $H \circ F$, because they map elements from $\binary^{n \cdot t} $ to $\binary^t$, and since the codomain is smaller than the domain, the functions cannot be injective in any way.

To overcome this problem, we can consider two options:
\begin{itemize}
    \item assume collisions are hard to find, even when $s \in \binary^{\lambda}$ is known; in this case we have a family of \textit{collision-resistant hash functions};
    \item let $s$ be secret, and assume collisions are hard to find because it is hard to know how $h_s$ works.
\end{itemize}

\begin{definition}
    A family of hash functions $h_s$ is deemed \emph{$\varepsilon$-universal} iff:
    \[
        \forall x \neq x' \in \binary^{n \cdot t}, S \sim \unifdist(\binary^\lambda) \implies \Pr[h_S(x) = h_S(x')] \leq \varepsilon \qedhere
    \]
\end{definition}

If $\varepsilon = 2^{-n}$, meaning the collision probability is minimized, then the family is also called \emph{perfectly universal}; in the case where $\varepsilon \in \negl(\lambda)$ instead, it is defined as \emph{almost universal} (\textsc{au}). Care should be taken in telling the difference between universality and pairwise independence, which states:
\[
    (h_S(x), h_S(x')) \equiv U_{2n}
\]

\begin{lemma}
    Any pairwise independent hash function is perfectly universal.
\end{lemma}

\begin{proof} The proof is left as exercise.
    \todo{
    % AP190108: Leaving this part as-is
    (should I use $Col$ for solving this? What is the difference and when should I use $Col$ instead of one-shot-probability?)

    \textbf{ASK FOR SOLVING PROPERLY}

    (Thoughts: when I ask \textit{what's the probability that, chosen 2 distinct x-es, their hashes are the same \textbf{on a certain value}?}, maybe I have to use one-shot, because one-shot refers to the prob. that the two inputs collide on a specific value, even if not specified.

    Instead, if I consider \textit{what's the prob. that , chosen 2 distinct x-es, their hashes are the same?}, maybe I have to calculate all the possible collisions, because I want to know if the 2 inputs can collide in general. )
    }
\end{proof}

\begin{theorem}
    Assuming $f_k$ is a \prf{} with $n$-bit domain and $h_s$ is an \textsc{au} hash function family, then $H \circ F$ is a \prf{} on $(n \cdot t)$-bit domain, for $t \geq 1$.
\end{theorem}

\begin{proof}

    This proof too will proceed by hybridizing the original game up to the ideal random one. Consider the three sequences depicted in figures \ref{cryptogame:prfaureal}, \ref{cryptogame:prfauhyb} and \ref{cryptogame:prfaurand}:

    \begin{cryptogame}
        {prfaureal}
        {$Real_{F, \adversary}(\lambda)$}
        {}

        \cseqchallenger{\shortstack[l]{
            $k \pickUAR \binary^\lambda$ \\
            $s \pickUAR \binary^\lambda$
        }}

        \cseqbeginloop
        \send{}{$x$}{}
        \receive{$y = f_k(h_s(x))$}{$y$}{}
        \cseqendloop
        
    \end{cryptogame}

    \begin{cryptogame}
        {prfauhyb}
        {$\hybridg{}[R](\lambda)$}
        {}

        \cseqchallenger{\shortstack[l]{
            $\overline{R} \pickUAR \mathfrak{R}(n, l)$ \\
            $s \pickUAR \binary^\lambda$
        }}

        \cseqbeginloop
        \send{}{$x$}{}
        \receive{$y = \overline{R}(h_s(x))$}{$y$}{}
        \cseqendloop
        
    \end{cryptogame}

    \begin{cryptogame}
        {prfaurand}
        {$Rand_{R', \adversary}(\lambda)$}
        {}

        \cseqchallenger{$R \pickUAR \mathfrak{R}(n, l)$}

        \cseqbeginloop
        \send{}{$x$}{}
        \receive{$y = R(x)$}{$y$}{}
        \cseqendloop
        
    \end{cryptogame}


    \begin{lemma}
        \[
            Real_{F, \adversary}(\lambda) \compindist \hybridg{}[R](\lambda)
        \]
    \end{lemma}

    \begin{proof}
        The proof is left as exercise.
    \end{proof}

    \begin{lemma}
        \[
            \hybridg{}[R](\lambda) \compindist Rand_{R', \adversary}(\lambda)
        \]
    \end{lemma}

    \begin{proof}

        Again, collisions come into play there, but there are multiple cases that complicate things. Given two queries with arguments $x_1, x_2$ returning the same image $y$, the random game can model two scenarios:

        \begin{itemize}
            \item the arguments are equal, but with negligible probability
            \item the arguments are distinct
        \end{itemize}

        while the hybrid can model three of them:

        \begin{itemize}
            \item the arguments are equal, again with negligible probability
            \item the arguments are distinct, \emph{and so are their hashes}
            \item the arguments are distinct, \emph{but not their hashes}
        \end{itemize}
    
        We want to show that the collision at hash level is negligible: as long as they don't happen, the random function $\overline{R}$ is run over a sequence of distinct points, and behaves just as the random game's function $R$ does. So let \textsc{bad} be the event:
        \[
            \exists i \neq j \in [q] : h_s(x_i) = h_s(x_j)
        \]
        where $q$ denotes the adversary's query count. It suffices to show that $\Pr[\textsc{Bad}] \in \negl(\lambda)$.

        % AP190108: Copied verbatim from my notes, maybe this is something Venturi wrote. Doesn't mean I fully understand it...
        Since we don't care what happens after a collision, we can alternatively consider a mental experiment where we answer all queries at random, and only at the end sample $s \pickUAR \binary^\lambda$ and check of collisions: this does not change the value of $\Pr(\textsc{Bad})$. Now queries are independent of $s$, and this eases our proof:
        \begin{align*}
            \Pr[\textsc{Bad}] &= \Pr[\exists x_i \neq x_j, h_S(x_i) = h_S(x_j)] &                               \\
            &\leq \sum_{i \neq j} \Pr[h_S(x_i) = h_S(x_j)]                      & \text{$h_s$ is \textsc{au})}  \\
            &\leq \binom{q}{2} \negl(\lambda) \in \negl(\lambda)                &
        \end{align*}

        By ruling out this event, the lemma is proven.
    \end{proof}

     So now we have $Real \compindist \hybridg{} \compindist Rand$
\end{proof}

\begin{corollary}
    Let $\Pi = (\Tag, \Ver)$ be a variable length message \mac{} scheme where, given a \prf{} $f_k$ and an \textsc{au} hash function family $h_s$, the tagging function is defined as $f_k(h_s)$. Then $\Pi$ is \ufcma-secure.
\end{corollary}

\subsection{Hash function families from finite fields}

% AP190110: A "finite field" aka galois field, is a field with finite elements. (...)
%       - Finite fields exist only for carrier cardinality of q^k, where q is prime and k is nonnegative
%       - Both operations are done modulo q

A generic $2^n$-order finite field has very useful properties: adding two of its elements is equal to \textsc{xor}-ing their binary representations, while multiplying them is done modulo $2^n$. It is possible to define a hash function family that makes good use of these properties, and is suitable for a \ufcma-secure \mac{} scheme.

\begin{construction}
    Let $\mathbb{F} = GF(2^n)$ be a \textit{finite field} (or ``Galois field'') of $2^n$ elements, and let $m = (m_1, \dots, m_t) \in \mathbb{F}^t$ and $s = (s_1, \dots, s_t) \in \mathbb{F}^t$. The desired hash function family will have this form:
    \[
        h_{s}(m)= \sum_{i = 1}^{t} s_i m_i = \left<s, m\right> = q_m(s)
    \]
\end{construction}

\begin{lemma}
    The above function family $h_s$ is almost universal.
\end{lemma}

\begin{proof}

    In order for $h_s$ to be almost universal, collisions must happen negligibly. Suppose we have a collision with two distinct messages $m$ and $m'$:
    \[
        \sum_{i = 1}^t m_i s_i = \sum_{i = 1}^t m'_i s_i
    \]

    Let $\delta_i = m_i - m'_i$ and assume, without loss of generality, that $\delta \neq 0$. Then, by using the previous equation, when a collision happens:
    \[
        0 = \sum_{i = 1}^t m_i s_i - \sum_{i = 1}^t m'_i s_i = \sum_{i = 1}^t \delta_i s_i
    \]

    Since the messages are different from each other, there is at least some $i$-th block that contains some of the differences. Assume, without loss of generality, that some of the differences are contained in the first block ($i = 1$); the sum can then be split between the first block itself $\delta_1 s_1$ and the rest:

    \begin{align*}
        \sum_{i = 1}^t \delta_i s_i                     &= 0                                                \\
        \delta_1 s_1 + \sum_{i = 2}^{t} \delta_i s_i    &= 0                                                \\
        \delta_ 1 s_1                                   &= -\sum_{i = 2}^{t} \delta_i s_i                   \\
        s_1                                             &= \frac{-\sum_{i = 2}^{t} \delta_i s_i}{\delta_1}  \\
    \end{align*}

    which means when a collision happens, $s_1$ must be exactly equal to the sum of the other blocks, which is another element of $\mathbb{F}$. But since every seed is chosen at random among $\mathbb{F}$, the probability of picking the element $s_1$ satisfying the above equation is just $\left|\mathbb{F}\right|^{-1} = 2^{-n} \in \negl(\lambda)$. By repeating this reasoning for every difference-block, a sum of negligible probabilities is obtained, which is in turn negligible; therefore the hash function family $h_s$ is almost universal.

\end{proof}

% AP190111: Leaving this section as-is; will come back to it at a later time
\subsubsection{$H$ with Galois fields elements and polynomials}

\begin{exercise}
    Let $\mathbb{F} = GF(2^n)$ be a finite field of $2^n$ elements, and let $m = (m_1, \dots, m_t) \in \mathbb{F}^t$ and $s \pickUAR \mathbb{F}^t$. Construct the following hash function family:
    \[
        h_{s}(m) = \sum_{i=1}^{t}s^{i-1}m_{i}
    \]

    Prove that this construction is \textsc{au}.
\end{exercise}

\begin{solution}
    (possible proof: to be almost universal, looking at the definition, collisions with $m \not= m'$ must be negligible.

    So consider a collision as above: it must be true that  
    \begin{align*}
        \sum_{i=1}^{t} m_{i}s^{i-1}                                 &= \sum_{i=1}^{t} m'_{i}s^{i-1} \\ 
        \sum_{i=1}^{t} m_{i}s^{i-1}-\sum_{i=1}^{t} m'_{i}s^{i-1}    &= 0                            \\
        q_{m-m'}(s)                                                 &= 0                            \\
    \end{align*}
    How can we make a polynomial equal to 0? We have to find the \textbf{roots} of the polynomial, which we know are at most the polynomial's rank. The rank is $t - 1$, and the probability of picking a root from $\mathbb{F}$ as seed of $h_s$ is:
    \[
        \Pr[s=root] = \frac{t - 1}{2^{n}} \in \negl(\lambda)
    \]
\end{solution}
 
