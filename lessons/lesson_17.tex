\mychapter{17}{Lesson 17} %181123

\section{Construction of a \textsc{cca}-secure \textsc{pke}}

This section exposes a construction of a \textsc{cca}-secure \textsc{pke} scheme, using hash-proof systems, membership-hardness, and the $n$-universality property.

Let $\Pi_1, \Pi_2$ be two distinct hash-proof systems for some \textsc{np} language $L$ and the range of $Prove_2$ supports labels ($L'=L||\binary^\ell$).

Construct the \textsc{cca} scheme as follows: $\Pi := (\mathcal{KG}en, Enc, Dec)$
\begin{itemize}
    \item $ (\overbrace{(\omega_1,\omega_2)}^{pk}, \overbrace{(\tau_1,\tau_2)}^{sk}) \pickUAR \mathcal{KG}en(1^\lambda)\;,\;\;\;\;(\omega_1, \tau_1) \pickUAR Setup_1(1^\lambda), (\omega_2, \tau_2) \pickUAR Setup_2(1^\lambda)$
    \item Encryption routine: $Enc((\omega_1, \omega_2), m)$
    \begin{itemize}
        \item $ (y, x) \pickUAR Sample_1(1^\lambda)$
        \item $ \pi_1 \pickUAR Prove_1(\omega_1, y, x)$
        \item $ l := \pi_1 \cdot m$
        \item $ \pi_2 \pickUAR Prove_2(\omega_2, (y, l), x)$
        \item $ c := (c_1, c_2) = ((y, l), \pi_2)$
    \end{itemize}
    \item Decryption routine: $Dec((\tau_1, \tau_2), (c_1, c_2))$
    \begin{itemize}
        \item $\hat{\pi}_2 = Verify_2(\tau_2, c_1))$
        \item $\textsc{if } \hat{\pi}_2 \neq c_2 \textsc{ then output false}$
        \item Recall: $c_1 = (y, l)$
        \item $\hat{\pi}_1 = Verify_1(\tau_1, y)$
        \item $\textsc{output } l \cdot \hat{\pi}_1^{-1}$
    \end{itemize}
\end{itemize}

Correctness (assume $\hat{\pi}_i = \pi_i \forall i$):
\begin{align*}
    \hat{m} &= Dec(sk, Enc(pk, m)) \\
    &= Dec((\tau_1, \tau_2), Enc((\omega_1, \omega_2), m)) \\
    &= Dec((\tau_1, \tau_2), ((y, l), \pi_2)) \\
    &= l \cdot \hat{\pi}_1^{-1} \\
    &= \pi_1 \cdot m \cdot \hat{\pi}_1^{-1} \\
    &= m
\end{align*}

Some additional notes (may be incorrect):
\begin{itemize}
    \item The message space of the second prover is the range of the first prover.
    \item The message space of the first prover is a multiplicative group
    \item The message space of the second prover is a polylogarithmic language in ($\lambda$)
\end{itemize}





\begin{theorem}
    Assuming $\pi_1$ is 1-universal, $\pi_2$ is 2-universal and $L$ is a membership-hard language; then the above scheme is \textsc{cca}-secure.
\end{theorem}

\begin{proof}
    Five different games will be defined, from $\cryptog{0}$ up to $\cryptog{4}$; the first game will be an analogous formalization of how the above \textsc{pke} scheme works. It shall be proven that, for arbitrarily fixed $b$ in $\binary$:
    \[
        \cryptog{0}(\lambda, b) \equiv \cryptog{1}(\lambda, b) \compindist \cryptog{2}(\lambda, b) \statindist \cryptog{3}(\lambda, b) \equiv \cryptog{4}(\lambda, b)
    \]
    and finally that $\textsc{Game}_{\Pi, A}^4(\lambda, 0) = \textsc{Game}_{\Pi, A}^4(\lambda, 1)$, therefore concluding that $\textsc{Game}_{\Pi, A}^0(\lambda, 0) \compindist \textsc{Game}_{\Pi, A}^0(\lambda, 1)$, and proving this scheme is \textsc{cca}-secure.

    \todo{Non sono per niente sicuro riguardo all'origine di $x$ ed $y$, né tantomeno dove sia definito il sampler per essi}

    The games are defined as follows:

    % Game 0
    \begin{cryptogame}{pkecca0}{Original \textsc{cca} game}{pke-cca}
    
        % Challenger generates key pair and publishes the public key
        \receive{$((\omega_1, \omega_2), (\tau_1, \tau_2)) \pickUAR \mathcal{KG}en(1^\lambda)$}
        {$(\omega_1, \omega_2)$}
        {}
    
        % Adversary can query the challenger for polynomially-many decryptions
        \send{}{}{}
        \receive{}{}{}
    
        \cseqdelay
    
        % Adversary asks for a challenge
        \send{}{$m_0, m_1$}{}
        %Challenger replies with an encrypted message
        \receive{\shortstack[l]{
            $(x, y) \pickUAR Sample_1$ \\
            $\pi_1 \pickUAR Prove_1(\omega_1, y, x)$ \\
            $b \pickUAR \binary$ \\
            $\pi_2 \pickUAR Prove_2(\omega_2, (y, (\pi_1 \cdot m_b)), x)$ }}
        {$c = ((y, (\pi_1 \cdot m_b)), \pi_2)$}
        {}
    
        \cseqdelay
    
        % Adversary can query the challenger for polynomially-many decryptions
        \send{}{}{}
        \receive{}{}{}
    
        \cseqdelay
    
        % Adversary guesses which message has been encrypted
        \send{}{$b'$}{\textsc{Output} 1 \textsc{iff} $b=b'$}
    
    \end{cryptogame}
    
    % Game 1: Use verifiers instead of provers
    \begin{cryptogame}{pkecca1}{Use verifiers}{1}
    
        % Adversary can query the challenger for polynomially-many decryptions
        \send{}{}{}
        \receive{}{}{}
    
        \cseqdelay
    
        % Adversary asks for a challenge
        \send{}{$m_0, m_1$}{}
        %Challenger replies with an encrypted message
        \receive{\shortstack[l]{
            $(x, y) \pickUAR Sample_1$ \\
            $\pi_1 \pickUAR Verify_1(\tau_1, y)$ \\
            $b \pickUAR \binary$ \\
            $\pi_2 \pickUAR Verify_2(\tau_2, (y, (\pi_1 \cdot m_b)))$ }}
        {$c = ((y, (\pi_1 \cdot m_b)), \pi_2)$}
        {}
    
        \cseqdelay
    
        % Adversary can query the challenger for polynomially-many decryptions
        \send{}{}{}
        \receive{}{}{}
    
        \cseqdelay
    
        % Adversary guesses which message has been encrypted
        \send{}{$b'$}{\textsc{Output} 1 \textsc{iff} $b=b'$}
    
    \end{cryptogame}
    
    % Game 2: Sample (y, x(?)) outside the language instead of inside
    \begin{cryptogame}{pkecca2}{Sample statements outside the language}{2}
    
        % Adversary can query the challenger for polynomially-many decryptions
        \send{}{}{}
        \receive{}{}{}
    
        \cseqdelay
    
        % Adversary asks for a challenge
        \send{}{$m_0, m_1$}{}
        %Challenger replies with an encrypted message
        \receive{\shortstack[l]{
            $(x, y) \pickUAR \overline{Sample}_1$ \\
            $\pi_1 \pickUAR Verify_1(\tau_1, y)$ \\
            $b \pickUAR \binary$ \\
            $\pi_2 \pickUAR Verify_2(\tau_2, (y, (\pi_1 \cdot m_b)))$ }}
        {$c = ((y, (\pi_1 \cdot m_b)), \pi_2)$}
        {}
    
        \cseqdelay
    
        % Adversary can query the challenger for polynomially-many decryptions
        \send{}{}{}
        \receive{}{}{}
    
        \cseqdelay
    
        % Adversary guesses which message has been encrypted
        \send{}{$b'$}{\textsc{Output} 1 \textsc{iff} $b=b'$}
    
    \end{cryptogame}
    
    \todo{ Non è stato chiaro sull'origine di x ed y, inoltre mi manca da scrivere le query di decifratura}
    
    % Game 3: Decryption queries return false whenever y is not in the language
    \begin{cryptogame}{pkecca3}{Modify decryption queries}{3}
    
        % Adversary can query the challenger for polynomially-many decryptions
        \send{}{}{}
        \receive{}{}{}
    
        \cseqdelay
    
        % Adversary asks for a challenge
        \send{}{$m_0, m_1$}{}
        %Challenger replies with an encrypted message
        \receive{\shortstack[l]{
            $(x, y) \pickUAR \overline{Sample}_1$ \\
            $\pi_1 \pickUAR Verify_1(\tau_1, y)$ \\
            $b \pickUAR \binary$ \\
            $\pi_2 \pickUAR Verify_2(\tau_2, (y, (\pi_1 \cdot m_b)))$ }}
        {$c = ((y, (\pi_1 \cdot m_b)), \pi_2)$}
        {}
    
        \cseqdelay
    
        % Adversary can query the challenger for polynomially-many decryptions
        \send{}{}{}
        \receive{}{}{}
    
        \cseqdelay
    
        % Adversary guesses which message has been encrypted
        \send{}{$b'$}{\textsc{Output} 1 \textsc{iff} $b=b'$}
    
    \end{cryptogame}
    
    % Game 4: pi1 is chosen UAR
    \begin{cryptogame}{pkecca4}{$\pi_1$ is chosen \textsc{uar}}{4}
    
        % Adversary can query the challenger for polynomially-many decryptions
        \send{}{}{}
        \receive{}{}{}
    
        \cseqdelay
    
        % Adversary asks for a challenge
        \send{}{$m_0, m_1$}{}
        %Challenger replies with an encrypted message
        \receive{ \shortstack[l]{
            $\pi_1 \pickUAR Im(Prove_1)$ \\
            $b \pickUAR \binary$ \\
            $\pi_2 \pickUAR Prove_2(\omega_2, (y, (\pi_1 \cdot m_b)), x)$
        }}
        {$c = ((y, (\pi_1 \cdot m_b)), \pi_2)$}{}
    
        \cseqdelay
    
        % Adversary can query the challenger for polynomially-many decryptions
        \send{}{}{}
        \receive{}{}{}
    
        \cseqdelay
    
        % Adversary guesses which message has been encrypted
        \send{}{$b'$}{\textsc{Output} 1 \textsc{iff} $b=b'$}
    
    \end{cryptogame}

\end{proof}

%==================================================

\begin{lemma}
    \[
        \forall b, G_{0}(\lambda, b) \equiv G_{1}(\lambda, b)   
    \]
\end{lemma}

\begin{proof}
    This follows by the correctness of $\Pi_{1}$ and $\Pi_{2}$
    \begin{gather*}
        \Pi_{1}=\tau_{1}=Ver_{1}(\tau, y)\\
         \Pi_{2}= \tilde{\Pi}_{2}=Ver_{2}(\tau, y)
    \end{gather*}
    with probability 1 over the choice of $(\omega_{1}, \tau_{1}) \pickUAR
    Setup_{1}(1^{\lambda})$
    
    \begin{gather*}
        \Pi_{1} \leftarrow Prove_{1}(\omega, y, x), (\omega_{2},
        \tau_{1})\pickUAR Setup_{2} (1^{\lambda})\\
        \Pi_{2} \leftarrow Prove(\omega_{2}, (y, l), x) \forall y \in L , l \in
        \P_{1}
    \end{gather*}
    
\end{proof}

\begin{lemma}
    \[
        \forall b, G_{1}(\lambda, b) \compindist G_{2}(\lambda, b)   
    \]
\end{lemma}
\begin{proof}
    Straight forward reduction from membership hardness.
\end{proof}

\begin{lemma}
    \[
        \forall b, G_{2}(\lambda , b) \compindist G_{3}(\lambda, b)
    \]
\end{lemma}
\todo{I'm not completely sure the next proof is complete}
\begin{proof}
    Recall that the difference between $G_{2}$ and $G_{3}$ is that 
    \[
        (g^{(i)}, l^{(i)}, \Pi_{2}^{(i)})\text{ such that }y^{(i)}\not\in L 
    \]
    are answered $\perp$ in $G_{3}$, instead in $G_{2}$
    \[
        \perp \text{ comes out as output} \Leftrightarrow
        \tilde{\Pi}^{(i)}_{2}=Ver(\tau, y^{(i)}\not=\Pi^{(i)}_{2})
    \]

    It's possible to distinguish \textbf{two cases} , looking at $c=(y, l,
    \Pi_{2})$ :
\begin{enumerate}
    \item if $(y^{(i)}, l^{(i)})=(y,l)$ and $\tilde{\Pi}^{(i)}_{2}=\Pi^{(i)}_{2}$, it outputs $\perp$ if in the decryption scheme ($\Pi^{(i)}_{2} \not= \tilde{\Pi}^{(i)}_{2}$) it outputs $\perp$.

    \item \footnote{when decryption oracle doesn't output the challenge} otherwise $(y^{(i)}, l^{(i)})\not= (y,l)$ if $y^{(i)}\not\in L$ we want that $\Pi^{(i)}_{2}$ doesn't output exactly $Ver_{2}(\tau, (y^{(i)}, l^{(i)}))$, but it should output $\perp$.
\end{enumerate}

\textbf{EVENT BAD:} If we look at the view of $\adversary$, the only information he
knows is 
\[
    (\omega_{2}, \tilde{\Pi}_{2}=Ver_{2}(\tau_{2}, y))
\]
for $y \not\in L$.\\
The value 
\[
Ver_{2}(\tau_{2}, (y^{(i)}, x^{(i)}))
\]
for $y^{(i)} \in L$ and
$(y^{(i)}, l^{(i)})\not=(y, l)$ is random.\\
So, 
\[
\P [ BAD ] =2^{-|\P_{2}|} 
\]

\end{proof}

\begin{lemma}
    \[
        \forall b , G_{3}(\lambda, b)\equiv G_{4}(\lambda, b)   
    \]
\end{lemma}

\begin{proof}
    If we look at the view of $\adversary$, the only information known about $\tau_{1}$
    is $\omega_{1}$, since the decryption oracle only computes for $y^{(i)} \in
    L$
    \[
        Ver_{1}(\tau, y^{(i)})=Prove_{1}(\omega_{1}, y^{(i)}, x^{(i)})
    \]

    By 1-universality, $\Pi=Ver_{1}(\tau ,y)$ for any $y \in L$ is random.
\end{proof}

\begin{lemma}
    \[
        G_{4}(\lambda, 0)\equiv G_{4}(\lambda, 1)   
    \]
\end{lemma}
\begin{proof}
    The challenge ciphertext is independent of $b$.
\end{proof}





%==================================================
\todo{referencing something from another part of lesson 17}

\subsection{Instantiation of U-HPS (Universal Hash Proof System)}
\textbf{MHL(Membership Hard Language) from DDH}

$\exists r$ using a DDH language such that $L_{DDH}=\{(c_1,c_2), \exists r | c_1=g_1^r, c_2=g_2^r\}$

% impaginare meglio
Given a group $\G$ of order $q$ with $(g_1,g_2)$ as generators, we will have $(g_1,g_2,c_1,c_2)$ but if we impose $g_1=g$ and $g_2=g^a$ then the previous construction becomes $(g,g^a,c_1,c_2)$. But for definition $c_1=g_1^{r_1}$ and $c_2=g_2^{r_2}$ then I can write $(g,g^a,g^r,g^{ar})$.
% Make sense?! no
%$$g_2=g^x , g_1=g , r=g $$ $(g,g^x,g^y,g^{xy}), $

% not sure if this is U-HPS or something random
Now we can define our U-HPS $\Pi:=$(Setup, Prove, Verify)

\begin{itemize}
    \item $Setup(1^\lambda)$: Pick $x_1,x_2 \pickUAR \integer_q$ and define:\\ $\omega=h_1=(g_1^{x_1},g_2^{x_2}$), $\tau=(x_1,x_2)$
    \item $Prove(\omega, \underbrace{(c_1,c_2)}_{y}, r)$ output $\Pi=\omega^2$
    \item $Verify(\tau, \underbrace{(c_1,c_2)}_{y})$ output $\widetilde{\Pi}=c_1^{x_1}c_2^{x_2}$ %pretty shitty tilde here, I need a bigger tilde to cover the full length of Pi -> \usepackage{stackengine} \usepackage{scalerel} added in macro
\end{itemize}

\textbf{Correctness:} $\Pi=\omega^2=(g_1^{x_1}g_2^{x_2})^r=g_1^{rx_1}g_2^{rx_2}=c_1^{x_1}c_2^{x_2}=\widetilde{\Pi}$

\begin{theorem}
    Above construction defines a 1-universal DVNIZK for $L_{DDH}$
\end{theorem}

\begin{proof}

    \textit{We want to prove that if we take any $(c_1,c_2)\notin L_DDH$ the distribution $(\omega=h_1, \widetilde{\Pi}=Verify(\tau,(c_1,c_2)))$ uniformly distributed.}

    Define a \textbf{MAP} $\mu(\overbrace{x_1,x_2}^{random})=(\omega,\Pi)=(g_1^{x_1},g_2^{x_2},c_1^{x_1},c_2^{x_2})$ it suffices to prove that $\mu$ is injective. This can easily be done with some constrains:

    $\mu'(x_1,x_2)=log_{g_1}(\mu(x_1,x_2))=(log_{g_1}(\omega),log_{g_1}(\Pi)$

    For $r_1\neq r_2$ then $c_1=g_1^{r_1},c_2=g_2^{r_2}=g^{\alpha r_2}$. For $\alpha=log_{g_2}g_1$ then $\Pi=c_1^{x_1}c_2^{x_2}=g_1^{r_1x_1}g_2^{r_1x_1+\alpha r_2x_2}$.
    
    \[\mu'(x_1,x_2)=
        \begin{pmatrix}
            z_1\\
            z_2
        \end{pmatrix}
        =
        \begin{pmatrix}
            1 & \alpha \\
            r_1 & r_2\alpha 
        \end{pmatrix}
        \cdot
        \begin{pmatrix}
            x_1\\
            x_2
        \end{pmatrix}
    \]

    \[ \text{Since Det}
    \begin{pmatrix}
        1 & \alpha \\
        r_1 & r_2\alpha 
    \end{pmatrix}=
    \alpha(r_2-r_1)\neq 0
    \text{ the map is injective.}
    \] 

\end{proof}

\begin{itemize}
    \item $Setup(1^\lambda)$: \begin{itemize}
        \item Pick $x_3,x_4,x_5,x_6 \pickUAR \integer_q$ and define:
        \begin{itemize}
            \item $\omega=(h_2,h_3,s)=(g_1^{x_3},g_2^{x_4},g_1^{x_5},g_2^{x_6},s)$ where $s$ is a \textbf{seed} for a CRH$\rightarrow H=\{H_s\}$ %\item $\tau=(x_3,x_4,x_5,x_6)$
        \end{itemize}
    \end{itemize}
    \item $Prove(\omega, (c_1,c_2,l), r)$ \begin{itemize}
        \item Compute $\beta=H_s(c_1,c_2,l)\in \integer_p$
        \item Output $\Pi=h_2^rh_3^{r\beta}$
    \end{itemize}
    \item $Verify(\tau, (c_1,c_2,l))$\begin{itemize}
        \item Compute $\beta=H_s(c_1,c_2,l)\in \integer_q$
        \item Output $\widetilde{\Pi}=c_1^{x_3+\beta x_5}c_2^{x_4+\beta x_6}$
    \end{itemize}
\end{itemize}

\noindent\textbf{Correctness:}\\ $\Pi=h_2^rh_3^{r\beta}=(g_1^{x_3}g_2^{x_4})^r(g_1^{x_5}g_2^{x_6})^r=c_1^{x_3}c_2^{x_4}c_1^{\beta x_5}c_2^{\beta x_6}=c_1^{x_3+\beta x_5}c_2^{x_4+\beta x_6}=\widetilde{\Pi}$

\begin{theorem}
    The above construction define a 2-universal DVNIZK for $L_{DDH}$    
\end{theorem}
\pagebreak
\begin{proof}
    \textit{Same goal and procedure as before}\\
    - Take any $(c_1,c_2) \notin L_{DDH}$\\
    - Fix $(c_1,c_2,l)\neq (c_1',c_2',l')$ s.t. $(c_1,c_2),(c_1',c_2')\notin L_{DDH}$ which means:
    \begin{itemize}
        \item $(c_1,c_2)=(g_1^{r_1}g_2^{r_2})$ \tikzmark{start}
        \item $(c_1',c_2')=(g_1^{r_1'}g_2^{r_2'})$ \tikzmark{end}
        \item $\beta=H_s(c_1,c_2,l)$
        \item $\beta'=H_s(c'_1,c'_2,l')$
    \end{itemize}
    \begin{tikzpicture}[remember picture,overlay]
        \draw[decorate,decoration={brace,raise=12pt}]
          ([yshift=2ex]{{pic cs:end}|-{pic cs:start}}) --
            node[xshift=15pt,anchor=west] {$r_1\neq r_2$ and $r'_1\neq r'_2$} 
          ([yshift=-0.5ex]pic cs:end);
    \end{tikzpicture}
    - Let's define a MAP
    \begin{gather*}
        \mu'(x_3,x_4,x_5,x_6)=(\omega, \widetilde{\Pi}=Ver(\tau,(c_1,c_2,l)),\widetilde{\Pi'}=Ver(\tau,(c_1',c_2',l')))= \\
        =(\underbrace{(h_2,h_3)}_{\omega},c_1^{x_3+\beta x_5}c_2^{x_4+\beta x_6},c_1'^{x_3+\beta'x_5}c_2'^{x_4+\beta'x_6}= \\
        =((g_1^{x_3}g_2^{x4},g_1^{x_5}g_2^{x_6}),g_1^{r_1x_3+\beta r_1x_5}g_2^{r_2x_4+\beta r_2x_6},g_1^{r'_1x_3+\beta' r_1'x_5}g_2^{r_2'x_4+\beta'r_2'x_6})
    \end{gather*}
    But I can rewrite $g_2$ as $g_2=g_1^{\alpha}$ since they are generators. So:
    \begin{gather*}
        ((g_1^{x_3+\alpha x_4},g_1^{x_5+\alpha x_6}),g_1^{r_1x_3+\beta r_1x_5}g_1^{\alpha(r_2x_4+\beta r_2x_6)},g_1^{r'_1x_3+r'_1\beta' x_5}g_1^{\alpha(r'_2x_4+\beta' r'_2x_6)})=\\
        = ((g_1^{x_3+\alpha x_4},g_1^{x_5+\alpha x_6}),g_1^{r_1x_3+\alpha r_2x_4+\beta r_1x_5 +\alpha\beta r_2x_6},g_1^{r'_1x_3+ \alpha r'_2x_4 +r'_1\beta' x_5 + \alpha\beta' r'_2x_6})=
    \end{gather*}
    \[
        \begin{pmatrix}
            z_1\\
            z_2\\
            z_3\\
            z_4
        \end{pmatrix}
        =
        \begin{pmatrix}
            1 & \alpha & 0 & 0 \\
            0 & 0 & 1 & \alpha \\
            r_1 & \alpha r_2 & \beta r_1 & \alpha\beta r_2 \\
            r'_1 & \alpha r'_2 & \beta' r'_1 & \alpha\beta' r'_2
        \end{pmatrix}
        \cdot
        \begin{pmatrix}
            x_3\\
            x_4\\
            x_5\\
            x_6
        \end{pmatrix}
    \]

    \[ \text{Since Det}
    \begin{pmatrix}
        .&.&.&. \\
        .&.&.&.\\
        .&.&.&.\\
        .&.&.&. 
    \end{pmatrix}=
    \alpha^2(r_2-r_1)(r'_2-r'_1)(\beta-\beta')\neq 0
    \]
    \centering\textbf{IFF:}\\
    $\overbrace{r_2\neq r_1 ,r'_2\neq r'_1}^{\text{for construction}},\beta \neq \beta' \rightarrow$ this last condition is true because we picked $H_s$ collision resistant.\\
    Otherwise $H_s$ computed on different elements($(c_1,c_2,l)$ and $(c'_1,c'_2,l')$) will have to output the same $\beta( = \beta')$.
\end{proof}
