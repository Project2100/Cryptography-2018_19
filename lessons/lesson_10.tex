\mychapter{10}{Lesson 10} %181026

% AP190125: Yeah, that's impossible to hyphenate...
\section{Domain extension for \prf-based \mac{} schemes}

\subsection{Hash function families from \prf{}s}

Another way to obtain domain extension for a \mac{} scheme, using the $\F_k(h_s)$ approach, is to construct the hash function family from another \prf{}. We expect to have:
\begin{itemize}
    \item $\Pr[h_s(m) = h_s(m'), s \pickUAR \{0, 1\}^\lambda, (m, m') \pickUAR A(1^\lambda)] \in \negl{\lambda}$;
    \item We need two \prf{}s: one is $F_{k}$, and the other is $F_{s}$
\end{itemize}

\subsection{\textsc{xor}-mode}

Assume that we have this function
\[
    h_{s}(m)=F_{s}(m_{1}||1) \xor ... \xor F_{s}(m_{t}||t)
\]
so that the input to the PRF $F_{s}(.)$ is $n + log_{2} t$ bytes long.

\begin{lemma}
    Above $\H$ is computational AU if $F_s$ is a PRF.
\end{lemma}

\begin{proof} The proof is left as exercise.

    (Hint: The pseudorandom functions can be defined as $F_s = F_k'(0, \dots)$ and $F_k = F_k'(1 \dots)$).
    
    Possible proof: we have to show that
    \[
        \P [ h_{s}(m)=h_{s}(m') ] \in \negl{\lambda}   
    \]
    with $m \neq m'$. This means that 
    \begin{align*}
        &\Pr\left[\left(\bigxor_{i = 1}^{t} F_s(m_i, i)\right) = \left(\bigxor_{i = 1}^{t} F_s(m_i', i)\right)\right] \\
        = &\Pr[\forall i \quad F_s(m_i, i) \xor F_s(m'_i, i) = \bigxor_{j = 1, j \neq i}^{t} F_s(m_j, j) \xor F_s(m'_j, j) = \alpha]
    \end{align*}
    for each $i \in [1,t]$. But $\alpha$ is one unique random number chosen over $2^{n}$ possible candidates, so the collision probability is negligible.
    
\end{proof}

\subsection{\textsc{cbc}-mode \mac{} scheme}
This is part of the standard, used in TLS. It's used with a \prf{} $F_{s}$, setting the starting vector as $\textsc{iv} = 0^n = c_0$ and running this \prf{} as part of \textsc{cbc}. The last block otained by the whole process is the message's signature:
\[
    h_s(m) = F_s(m_t \xor F_s(m_{t-1} \xor F_s(\dots F_s(m_2 \xor F_s(m_1 \xor \textsc{iv})) \dots)))
\]
\begin{lemma}
    CBC MAC defines completely an AU family.
\end{lemma}

\begin{proof}
    (not proven)
\end{proof}

We can use this function to create an \textbf{encrypted CBC}, or \textbf{E-CBC}:
\[
    E-CBC_{K, S}(m)=F_{k}(h_{s}^{CBC}(m))
\]

\begin{theorem}
    Actually if $F_k$ is a \prf, CBC-MAC is already a MAC with domain $n\cdot t$ for arbitrarily fixed $t \in \mathbb{N}$.
\end{theorem}

\begin{proof}
    (not proven)
\end{proof}

\subsection{XOR MAC}
Instead of $\F(\H)$ now the $Tag()$ function outputs $\phi = (\eta , F_{k}(\eta) \xor h_{s}(m))$ where $\eta \pickUAR \{0,1\}^{n}$ is random and it's called \textit{nonce}.

Authentication is done as:
\[
    (m,( \eta, F_{k}(\eta)\xor h_{s}(m)))
\]

When I want to verify a message and I get the couple $(m, (\eta, v))$, I just check that $v=F_{k}(\eta) \xor h_{s}(m)$. It should be hard to find a value called $a$ such that, given $m \not= m'$, 
\[
    h_{s}(m) \xor a= h_{s}(m')
\]

In fact, since an adversary who wants to break this scheme has to send a valid couple $(m^{*}, \phi^{*})$ after some queries, he could:
\begin{itemize}
    \item ask for message $m$ and store the tag $( \eta, F_{k}(\eta)\xor h_{s}(m))$
    \item try to find $a=h_{s}(m) \xor h_{s}(m')$ and modify the previous stored tag adding $v \xor a$, 
\end{itemize}
so now he could send the authenticated message
\[
    (m', (\eta, F_{k}(\eta) \xor h_{s}(m'))) 
\]
which is a valid message.

\todo{AXU property definition is missing}

\begin{lemma}
    XOR mode gives computational AXU (Almost Xor Universal)
\end{lemma}

\begin{proof}
    (not proven)
\end{proof}

\begin{theorem}
    If $\F$ is a PRF and $\H$ is computational AXU, then XOR-MAC is a MAC.
\end{theorem}

\begin{proof}
    (not proven)
\end{proof}

\subsubsection{Summary}

\todo{not sure what to do with this bullet list...}

With variable input lenght:
\begin{itemize}
    \item AXU based XOR mode is secure;
    \item $\F(\H)$ is insecure with polynomial construction $h_{s}(m)=q_{m}(s)$, but can be fixed;
    \item CBC-MAC is not secure, left as exercise;
    \item E-CBC is secure.
\end{itemize}

% AP190111: The cca figure is too big and too important to let it slide through a whole page, so I'm breaking here to ensure it stays with the section title
\pagebreak

\section{\textsc{Cca}-security}

Going back to the encryption realm, a new definition of attack to a \ske{} scheme will be introduced. Now the adversary can query a decryption oracle, along with the \cpa-related encryption oracle, for polynomially many queries. This attack is called the \emph{Chosen Ciphertext Attack}\footnotemark, and schemes that are proven to be \cca-secure are also defined as \emph{non-malleable}, on the reasoning that an attacker cannot craft fresh valid ciphertexts from other valid ones.

\footnotetext{Different versions of the \cca{} notion exist. The one defined here is also called \cca2, or \emph{adaptive Chosen Ciphertext Attack}}

% AP190112: An important rule is forgotten here: the adversary cannot query the decryption oracle on the challenge ciphertext; otherwise the game becomes useless

\begin{cryptogame}
    {cca}
    {The chosen ciphertext attack, on top of \cpa}
    {cca}

    \cseqchallenger{$k \pickUAR \K$}

    %cpa
    \cseqbeginloop
    \send{}{$m$}{}
    \receive{\shortstack[l]{
        $r \text{ chosen \uar}$ \\
        $c = Enc_k(m, r)$
    }}{$c$}{}
    \cseqendloop

    \cseqdelay

    %cca
    \cseqbeginloop
    \send{}{$c$}{}
    \receive{$m = Dec_k(c)$}{$m$}{}
    \cseqendloop

    \cseqdelay

    %challenge
    \send{}{$m_0^*, m_1^*$}{}
    \receive{\shortstack[l]{
        $r^* \text{ chosen \uar}$ \\
        $b \pickUAR \{0, 1\}$ \\
        $c^* = Enc_k(m_b^*, r^*)$
    }}{$c_b^*$}{}

    \cseqdelay
    
    %cpa
    \cseqbeginloop
    \send{}{$m$}{}
    \receive{\shortstack[l]{
        $r \text{ chosen \uar}$ \\
        $c = Enc_k(m, r)$
    }}{$c$}{}
    \cseqendloop

    \cseqdelay

    %cca
    \cseqbeginloop
    \send{}{$c$}{}
    \receive{$m = Dec_k(c)$}{$m$}{}
    \cseqendloop

    \cseqdelay

    %output
    \send{}{$b'$}{\textsc{Output 1 iff} $b' = b$}
\end{cryptogame}

\begin{exercise}
    Show that the scheme $\Pi_{\F}$ defined in theorem \ref{thm:prfcpa}, while \cpa-secure, is not \cca-secure.
\end{exercise}

\begin{proof}

    Let $m_0$ and $m_1$ be the messages the adversary sends to the challenger as the challenge plaintexts; on receiving the ciphertext $c_b = (r, F_k(r \xor m_b) : b \pickUAR \{0, 1\}$, the adversary crafts another ciphertext with an arbitrary value $\alpha$:
    \[
        \widehat{c_b} = (r, F_k(r) \xor m_b \xor \alpha)
    \]
    and queries the decryption oracle on it. The latter will decrypt the new ciphertext and return a plaintext, which can be easily manipulated by the adversary to reveal exactly which message was encrypted during the challenge:

    \begin{align*}
        Dec_k(\widehat{c_b}) &= Dec_k(r, F_k(r) \xor m_b \xor \alpha) \\
        &= F_k(r) \xor F_k(r) \xor m_b \xor \alpha \\
        &= m_b \xor \alpha
    \end{align*}

    Therefore, the adversary certainly wins after just one decryption query, proving the scheme's vulnerability to \cca{} attacks.
   
\end{proof}

\section{Authenticated encryption}

Instead of tackling the \cca-security problem upfront, it might be useful to consider a construction that achieves both secrecy and authentication: that is, a scheme that encrypts messages and authenticates their respective senders at the same time. In the encryption setting, such a scheme is defined as being \cpa-secure with an additional \textsc{auth} property denoting the scheme's resistance to forgeries, much like its \mac{} cousins. The game shown in figure \ref{cryptogame:authske} models this \textsc{auth} property of a scheme $\Pi = (Enc, Dec)$, with an additional quirk to the decryption routine:
\[
    Dec:\K*\C \to M \cup \{\perp\}
\]
where the $\perp$ value is returned whenever the decryption algorithm is supplied an invalid or malformed ciphertext.

\begin{cryptogame}
    {authske}
    {$\cryptog{auth}(\lambda)$}
    {auth}

    \cseqchallenger{$k \pickUAR \K$}

    \cseqbeginloop
    \send{}{$m$}{}
    \receive{\shortstack[l]{
        $r \text{ chosen \uar}$ \\
        $c = Enc_k(m, r)$
    }}{$c$}{$c \in C$}
    \cseqendloop

    \cseqdelay

    \send{$c' \notin C$}{$c'$}{\textsc{Output 1 iff} $Dec_k(c') \neq \perp$}

\end{cryptogame} 

\begin{theorem}
    Let $\Pi$ be a \ske{} scheme. If it is \cpa-secure, and has the \textsc{auth} property, then it is also \cca-secure.
\end{theorem}

\begin{proof} The proof is left as an exercise.

    \emph{Hint}: consider the experiment where $Dec(k, c)$:

    \begin{itemize}
        \item if $c$ not fresh ( i.e. output of previous encryption query $m$, output $m$)
        \item else output $\perp$
    \end{itemize}
    
    The approach would be to reduce cca to cpa; given $D^{cca}$, we can build $D^{cpa}$. $D^{cca}$ will ask decryption queries, but $D^{cpa}$ can answer just with these two properties shown above, so it can reply just if he asked these $(c, m)$ before to its challenger $\C$.

\end{proof}

\subsection{Combining \ske{} \& \mac{} schemes}

Let $\Pi_1 = (Enc, Dec)$ be a \ske{} scheme, and $\Pi_2 = (\textit{Tag}, \textit{Verify})$ be a \mac{} scheme; there are 3 ways to combine them into an authenticated encryption scheme:
% AP190112: A note should be made about notation, for what it means to "pick at random" from an encryption/tagging function (random nonce in the encryption, seeded AU hash function in the tagging)
% Plus, keys aren't curried, but subsubscripting may turn ugly...
\begin{itemize}
    \item \emph{Encrypt-and-Tag}:
    \begin{enumerate}
        \item $c \pickUAR Enc(k_1, m)$
        \item $\phi \pickUAR Tag(k_2, m)$
        \item $c^* = (c, \phi)$
    \end{enumerate}
        
    \item \emph{Tag-then-encrypt}:
    \begin{enumerate}
        \item $\phi \pickUAR Tag(k_2, m)$
        \item $c \pickUAR Enc(k_1, (\phi, m))$
        \item $c^* = c$
    \end{enumerate}

    \item \emph{Encrypt-then-Tag}:
    \begin{enumerate}
        \item $c \pickUAR Enc(k_1, m)$
        \item $\phi \pickUAR Tag(k_2, c)$
        \item $c^* = (c, \phi)$
    \end{enumerate}
\end{itemize}

Of the three options, only the last one is proven to be \cca-secure for arbitrary scheme choices; the other approaches are not secure ``a-priori'', with some couples proven to be secure by themselves. Notable examples are the \emph{Transport Layer Security} (\textsc{tls}) protocol, which employs the second strategy, and has been proven to be secure because of the chosen ecnryption scheme; \emph{Secure SHell} (\textsc{ssh}) instead uses the first strategy.

\begin{theorem}
    If an authenticated encryption scheme $\Pi$ is made by combining a \cpa-secure \ske{} scheme $\Pi_1$ with a \emph{strongly unforgeable} \mac{} scheme $\Pi_{2}$ in the \emph{Encrypt-then-tag} method. Then $\Pi$ is \cpa-secure and \textit{auth}-secure. 
\end{theorem}

% AP190113: Key-related notation issues, again
\begin{proof}

    Assume that $\Pi$ is not \cpa-secure; then an adversary can use the resulting distinguisher $\distinguisher^{\cpa}$ to direct a succesful \cpa{} against $\Pi_1$, as shown in figure \ref{cryptoredux:ettcpa}: the point is to run the two components of $\Pi$ separately.

    \begin{cryptoredux}
        {ettcpa}
        {}
        {cpa$_1$}
        {cpa}

        \cseqchallenger{$k_1 \pickUAR \K_1$}
        \cseqadversary{$k_2 \pickUAR \K_2$}

        \cseqbeginloop
        \return{}{$m$}{}
        \send{}{$m$}{}
        \receive{$c \pickUAR Enc_k(m)$}{$c$}{}
        \invoke{$\phi \pickUAR Tag_k(c)$}{$(c, \phi)$}{}
        \cseqendloop

        \cseqdelay
        
        \return{}{$m_0^*, m_1^*$}{}
        \send{}{$m_0^*, m_1^*$}{}
        \receive{\shortstack[l]{
            $b \pickUAR \{0, 1\}$ \\
            $c^* \pickUAR Enc_k(m_b)$
        }}{$c^*$}{}
        \invoke{$\phi^* \pickUAR Tag_k(c^*)$}{$(c, \phi^*)$}{}

        \cseqdelay

        \cseqbeginloop
        \return{}{$m$}{}
        \send{}{$m$}{}
        \receive{$c \pickUAR Enc_k(m)$}{$c$}{}
        \invoke{$\phi \pickUAR Tag_k(c)$}{$(c, \phi)$}{}
        \cseqendloop

        \cseqdelay

        \return{}{$b'$}{}
        \send{}{$b'$}{\textsc{Output 1 iff} $b' = b$}

    \end{cryptoredux}

    \todo{Professor says that we have to show that $Game^{cpa}(\lambda, 0) \approx_{c} Game^{cpa}(\lambda, 1)$ , but why??? Isn't   this proof enough?}

\end{proof}

Proved for the cpa-security property, now we have to prove, in a similar way, that the auth property must be holded by $\Pi$ if $\Pi_{2}$ is an auth-secure scheme.
\begin{exercise}
    Prove it!

    Similar to the cpa-security proof.
\end{exercise}

