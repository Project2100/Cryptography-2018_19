\mychapter{10}{Lesson 10} %181026

\section{Domain extension for \prf-based authentication schemes}

\subsection{Hash function families from \prf{}s}

Another way to obtain domain extension for a \mac{} scheme, using the $H \circ F$ approach, is to construct the hash function family from another \prf{}. We expect to have:
\begin{itemize}
    \item $\forall \adversary \in \ppt, S \sim \unifdist(\binary^\lambda) \implies \Pr[h_S(m) = h_S(m')] \in \negl(\lambda)$;
    \item We need two \prf{}s: one is $f_k$, and the other is $f_s$
\end{itemize}

\subsection{\textsc{xor}-mode}

Let $f_s$ be a \prf, and assume that we have this function
\[
    h_s(m) = f_s(m_1||1) \oplus ... \oplus f_s(m_t||t)
\]
so that the input to $f_s$ is $n + log_2(t)$ bytes long.

\begin{lemma}
    Above $H$ is computational \textsc{AU}.
\end{lemma}

\begin{proof} The proof is left as exercise.

    (Hint: The pseudorandom functions can be defined as $f_s = f_k'(0, \dots)$ and $f_k = f_k'(1 \dots)$).
    
    Possible proof: we have to show that
    \[
        \forall m \neq m' \implies \Pr[h_s(m) = h_s(m')] \in \negl(\lambda)   
    \]
    This means that:
    \begin{align*}
        & \Pr\left[\left(\bigoplus_{i = 1}^{t} f_s(m_i, i)\right) = \left(\bigoplus_{i = 1}^{t} f_s(m_i', i)\right)\right] \\
        =& \Pr[\forall i \quad f_s(m_i, i) \oplus f_s(m'_i, i) = \bigoplus_{j = 1, j \neq i}^{t} f_s(m_j, j) \oplus f_s(m'_j, j) = \alpha]
    \end{align*}
    for each $i \in [1,t]$. But $\alpha$ is one unique random number chosen over $2^{n}$ possible candidates, so the collision probability is negligible.
    
\end{proof}

\subsection{\textsc{cbc}-mode \mac{} scheme}
This mode is used in practice by the \textsc{tls} standard. It's used with a \prf{} $f_s$, setting the starting vector as $\textsc{iv} = 0^n = c_0$ and running this \prf{} as part of \textsc{cbc}. The last block otained by the whole process is the message's signature:
\[
    h_s(m) = f_s(m_t \oplus f_s(m_{t-1} \oplus f_s(\dots f_s(m_2 \oplus f_s(m_1 \oplus \textsc{iv})) \dots)))
\]
\begin{lemma}
    \textsc{cbc} \mac{} defines completely an \textsc{au} family.
\end{lemma}

\begin{proof}
    (not proven)
\end{proof}

We can use this function to create an \emph{encrypted \textsc{cbc}}, or \textsc{e-cbc}:
\[
    \textsc{e-cbc}_{K, S}(m) = f_{k}(h_s^{\textsc{cbc}}(m))
\]

\begin{theorem}
    If $f_k$ is a \prf, \textsc{cbc} \mac{} is already a \mac{} scheme with domain $n \cdot t$ for arbitrarily fixed $t \in \nonneg$.
\end{theorem}

\begin{proof}
    (not proven)
\end{proof}

\subsection{\textsc{xor} \mac}
Instead of $H \circ F$, now the $\Tag$ routine outputs $\phi = (\eta , f_k(\eta) \oplus h_s(m))$ where $\eta \pickUAR \binary^n$ is random and it's called \textit{nonce}. Authentication is done as:
\[
    (m, (\eta, f_k(\eta) \oplus h_s(m)))
\]

When I want to verify a message and I get the couple $(m, (\eta, v))$, I just check that $v = f_k(\eta) \oplus h_s(m)$. It should be hard to find a value called $a$ such that, given $m \neq m'$, 
\[
    h_s(m) \oplus a = h_s(m')
\]

In fact, since an adversary who wants to break this scheme has to send a valid couple $(m^*, \phi^*)$ after some queries, he could:
\begin{itemize}
    \item ask for message $m$ and store the tag $(\eta, f_k(\eta) \oplus h_s(m))$
    \item try to find $a = h_s(m) \oplus h_s(m')$ and modify the previous stored tag adding $v \oplus a$, 
\end{itemize}
so now he could send the authenticated message
\[
    (m', (\eta, f_k(\eta) \oplus h_s(m'))) 
\]
which is a valid message.

\begin{definition}
    Let $S$ be a uniform seed; a hash function $h_s$ is deemed \emph{Almost \textsc{xor}-universal}, or \textsc{axu} in short, iff:
    \[
        \forall x_1 \neq x_2, a \implies \Pr[h_S(x_1) \oplus h_S(x_2) = a] \leq \varepsilon \qedhere
    \]
\end{definition}

Note that, if $a = 0$, the \textsc{axu} property becomes the \textsc{au} property.

\begin{lemma}
    \textsc{xor} mode gives a computational \textsc{axu} hash function.
\end{lemma}

\begin{proof}
    (not proven)
\end{proof}

\begin{theorem}
    If $F$ is a \prf{} and $H$ is computational \textsc{axu}, then \textsc{xor}-\mac is a \mac.
\end{theorem}

\begin{proof}
    (not proven)
\end{proof}

\subsubsection{Summary}

\todo{not sure what to do with this bullet list...}

With variable input length:
\begin{itemize}
    \item AXU based XOR mode is secure;
    \item $H \circ F$ is insecure with polynomial construction $h_s(m) = q_m(s)$, but can be fixed;
    \item CBC-MAC is not secure, left as exercise;
    \item E-CBC is secure.
\end{itemize}

% AP190111: The cca figure is too big and too important to let it slide through a whole page, so I'm breaking here to ensure it stays with the section title
\pagebreak

\section{\textsc{Cca}-security}

Going back to the encryption realm, a new definition of attack to a \ske{} scheme will be introduced. Now the adversary can query a decryption oracle, along with the \cpa-related encryption oracle, for polynomially many queries. This attack is called the \emph{Chosen Ciphertext Attack}\footnotemark, and schemes that are proven to be \cca-secure are also defined as \emph{non-malleable}, on the reasoning that an attacker cannot craft fresh valid ciphertexts from other valid ones.

\footnotetext{Different versions of the \cca{} notion exist. The one defined here is also called \cca2, or \emph{adaptive Chosen Ciphertext Attack}}

% AP190910: CAUTION - The encryption routine is intrinsically randomized: the returned ciphertexts are composed of a nonce, and the actual encrypted message

\begin{cryptogame}
    {cca}
    {The chosen ciphertext attack, on top of \cpa}
    {cca}

    \cseqchallenger{$k \pickUAR \K$}

    %cpa
    \cseqbeginloop
    \send{}{$m$}{}
    \receive{$c \pickUAR \Enc(k, m)$}{$c$}{}
    \cseqendloop

    \cseqdelay

    %cca
    \cseqbeginloop
    \send{}{$c$}{}
    \receive{$m = \Dec_k(c)$}{$m$}{}
    \cseqendloop

    \cseqdelay

    %challenge
    \send{}{$m_0^*, m_1^*$}{}
    \receive{\shortstack[l]{
        $b \pickUAR \binary$ \\
        $c^* \pickUAR \Enc(k, m_b^*)$
    }}{$c_b^*$}{}

    \cseqdelay
    
    %cpa
    \cseqbeginloop
    \send{}{$m$}{}
    \receive{$c \pickUAR \Enc(k, m)$}{$c$}{}
    \cseqendloop

    \cseqdelay

    %cca
    \cseqbeginloop
    \send{$c \neq c^*$}{$c$}{}
    \receive{$m = \Dec_k(c)$}{$m$}{}
    \cseqendloop

    \cseqdelay

    %output
    \send{}{$b'$}{\textsc{Output 1 iff} $b' = b$}
\end{cryptogame}

\begin{exercise}
    Show that the scheme $\Pi_{F}$ defined in theorem \ref{thm:prfcpa}, while \cpa-secure, is not \cca-secure.
\end{exercise}

\begin{proof}

    Let $m_0$ and $m_1$ be the messages the adversary sends to the challenger as the challenge plaintexts; on receiving the ciphertext $c_b = (r, f_k(r \oplus m_b) : b \pickUAR \binary$, the adversary crafts another ciphertext with an arbitrary value $\alpha$:
    \[
        \widehat{c_b} = (r, f_k(r) \oplus m_b \oplus \alpha)
    \]
    and queries the decryption oracle on it. The latter will decrypt the new ciphertext and return a plaintext, which can be easily manipulated by the adversary to reveal exactly which message was encrypted during the challenge:

    \begin{align*}
        \Dec(k, (\widehat{c_b}) &= \Dec(k, (r, f_k(r) \oplus m_b \oplus \alpha))    \\
                                &= f_k(r) \oplus f_k(r) \oplus m_b \oplus \alpha      \\
                                &= m_b \oplus \alpha                              \\
    \end{align*}

    Therefore, the adversary certainly wins after just one decryption query, proving the scheme's vulnerability to \cca{} attacks.
   
\end{proof}

\section{Authenticated encryption}

Trying to create a \cca-secure scheme form scratch may seem a daunting task. Instead, we will start looking for a scheme that achieves both secrecy and authentication. As one might expect, such a scheme is usually a combination of a valid secrecy scheme with a valid authentication scheme; the result would be a scheme composed by four primitives $\Pi = (\Enc, \Dec, \Tag, \Ver)$. If this scheme turns out to be \cpa-secure, and has an additional property \textsc{auth} denoting the scheme's resistance to forgeries, much like its \mac{} cousins, then it is proven that the scheme is also \cca-secure.

The game shown in figure \ref{cryptogame:authske} models this \textsc{auth} property of a scheme $\Pi = (\Enc, \Dec)$, with an additional quirk to the decryption routine:
\[
    \Dec \in \K \times \C \to M \cup \{\perp\}
\]
where the $\perp$ value is returned whenever the decryption algorithm is supplied an invalid or malformed ciphertext.

\begin{cryptogame}
    {authske}
    {$\cryptog{auth}(\lambda)$}
    {auth}

    \cseqchallenger{$k \pickUAR \K$}

    \cseqbeginloop
    \send{}{$m$}{}
    \receive{$c \pickUAR \Enc(k, m)$}{$c$}{$c \in C$}
    \cseqendloop

    \cseqdelay

    \send{$c' \notin C$}{$c'$}{\textsc{Output 1 iff} $\Dec(k, c') \neq \perp$}

\end{cryptogame} 

\begin{theorem}
    Let $\Pi$ be a \ske{} scheme. If it is \cpa-secure, and has the \textsc{auth} property, then it is also \cca-secure.
\end{theorem}

\begin{proof} The proof is left as an exercise.

    \emph{Hint}: consider the experiment where $\Dec(k, c)$:

    \begin{itemize}
        \item if $c$ is not fresh (i.e. output of previous encryption query $m$),  then output $m$
        \item else output $\perp$
    \end{itemize}
    
    The approach would be to reduce \cca{} to \cpa; given $\distinguisher^{\cca}$, we can build $\distinguisher^{\cpa}$. $\distinguisher^{\cca}$ will ask decryption queries, but $\distinguisher^{\cpa}$ can answer just with these two properties shown above, so it can reply just if he asked these $(c, m)$ before to its challenger $\challenger$.

\end{proof}

\subsection{Combining \ske{} \& \mac{} schemes}

Let $\Pi_1 = (\Enc, \Dec)$ be a \ske{} scheme, and $\Pi_2 = (\Tag, \Ver)$ be a \mac{} scheme; there are 3 ways to combine them into an authenticated encryption scheme:
% AP190112: A note should be made about notation, for what it means to "pick at random" from an encryption/tagging function (random nonce in the encryption, seeded AU hash function in the tagging)
% Plus, keys aren't curried, but subsubscripting may turn ugly...
% AP190910: Keys are part of the signature, nonces are a thing to rethink yet. For now, a pickUAR predicate implies the function is itself randomized, because there are no distributions on the right (Farfetched...)
\begin{itemize}
    \item \emph{Encrypt-and-Tag}:
    \begin{enumerate}
        \item $c \pickUAR \Enc(k_1, m)$
        \item $\phi \pickUAR \Tag(k_2, m)$
        \item $c^* = (c, \phi)$
    \end{enumerate}
        
    \item \emph{Tag-then-encrypt}:
    \begin{enumerate}
        \item $\phi \pickUAR \Tag(k_2, m)$
        \item $c \pickUAR \Enc(k_1, (\phi, m))$
        \item $c^* = c$
    \end{enumerate}

    \item \emph{Encrypt-then-Tag}:
    \begin{enumerate}
        \item $c \pickUAR \Enc(k_1, m)$
        \item $\phi \pickUAR \Tag(k_2, c)$
        \item $c^* = (c, \phi)$
    \end{enumerate}
\end{itemize}

Of the three options, only the last one is proven to be \cca-secure for arbitrary scheme choices; the other approaches are not secure ``a-priori'', with some couples proven to be secure by themselves. Notable examples are the \emph{Transport Layer Security} (\textsc{tls}) protocol, which employs the second strategy, and has been proven to be secure because of the chosen encryption scheme; \emph{Secure shell} (\textsc{ssh}) instead uses the first strategy.

\begin{theorem}
    If an authenticated encryption scheme $\Pi$ is made by combining a \cpa-secure \ske{} scheme $\Pi_1$ with a \emph{strongly unforgeable} \mac{} scheme $\Pi_2$ in the \emph{Encrypt-then-tag} method. Then $\Pi$ is \cpa-secure and \textsc{auth}-secure. 
\end{theorem}

% AP190113: Key-related notation issues, again
\begin{proof}

    Assume that $\Pi$ is not \cpa-secure; then an adversary can use the resulting distinguisher $\distinguisher^{\cpa}$ to direct a successful \cpa{} against $\Pi_1$, as shown in figure \ref{cryptoredux:ettcpa}: the point is to run the two components of $\Pi$ separately.

    \begin{cryptoredux}
        {ettcpa}
        {}
        {cpa$_1$}
        {cpa}

        \cseqchallenger{$k_1 \pickUAR \K_1$}
        \cseqadversary{$k_2 \pickUAR \K_2$}

        \cseqbeginloop
        \return{}{$m$}{}
        \send{}{$m$}{}
        \receive{$c \pickUAR Enc_k(m)$}{$c$}{}
        \invoke{$\phi \pickUAR Tag_k(c)$}{$(c, \phi)$}{}
        \cseqendloop

        \cseqdelay
        
        \return{}{$m_0^*, m_1^*$}{}
        \send{}{$m_0^*, m_1^*$}{}
        \receive{\shortstack[l]{
            $b \pickUAR \binary$ \\
            $c^* \pickUAR Enc_k(m_b)$
        }}{$c^*$}{}
        \invoke{$\phi^* \pickUAR Tag_k(c^*)$}{$(c, \phi^*)$}{}

        \cseqdelay

        \cseqbeginloop
        \return{}{$m$}{}
        \send{}{$m$}{}
        \receive{$c \pickUAR Enc_k(m)$}{$c$}{}
        \invoke{$\phi \pickUAR Tag_k(c)$}{$(c, \phi)$}{}
        \cseqendloop

        \cseqdelay

        \return{}{$b'$}{}
        \send{}{$b'$}{\textsc{Output 1 iff} $b' = b$}

    \end{cryptoredux}

    \todo{Professor says that we have to show that $\cryptog{cpa}(\lambda, 0) \compindist \cryptog{cpa}(\lambda, 1)$, but why? Isn't this proof enough?}

\end{proof}

Proved for the \cpa-security property, now we have to prove, in a similar way, that the \textsc{auth} property must hold by $\Pi$ if $\Pi_2$ is an \textsc{auth}-secure scheme.
\begin{exercise}
    Prove it! Solution is in next lesson.
\end{exercise}
