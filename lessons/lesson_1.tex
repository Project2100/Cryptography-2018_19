\mychapter{1}{Lesson 1} %180926

\section{Secret communication}

The typical setting for the problem of secret communication is depicted in figure \ref{fig:secrecy}. The parties Alice and Bob want to share data in a private fashion, thus preventing a third party Eve from \emph{eavesdropping}.\footnote{That is, getting a hold of the information shared between the parties} The objects of interest here are:
\begin{itemize}
    \item The data to be shared, or \emph{message} $m$;
    \item Some secret information, shared between and known only to Alice and Bob, that is used to \emph{encrypt} the message: the \emph{encryption key} or just \emph{key} $k$;
    \item The result of encrypting a message $m$ using the key $k$: the ciphertext $c$.
\end{itemize}

\begin{figure}[ht]
    \centering
    \begin{tikzpicture}
        \draw
            (0, 0) node (a) [box, fill = white] {Alice} 
            (5, 0) node (b) [box, fill = white] {Bob}
            (2.5, -1) node (e) [box, fill = white] {Eve}
        ;

        \draw[-Stealth] (a) -- node [midway, above] {$c$} (b);
        \draw[-Stealth] (0, -1) node [below] {$k$} -- (a);
        \draw[-Stealth] (5, -1) node [below] {$k$} -- (b);
        \draw[-Stealth] (-1, 0) node [left] {$m$} -- (a);
        \draw[-Stealth] (b) -- (6, 0) node [right] {$m$};
        \draw[-Stealth] (2, -0.1) -- (3, -0.1) (2.5, -0.1) -- (e);

    \end{tikzpicture}

    \caption{A depiction of the problem of secret communication}
    \label{fig:secrecy}
\end{figure}

To complete the picture, the two parties Alice and Bob employ a \emph{cryptographic secrecy scheme}, or \emph{encryption scheme} to convert the message in an encrypted form, and vice versa. It has the form $\Xi = (\Enc, \Dec)$, where:
\begin{itemize}
    \item $\Enc \in \K \times \M \to \C$ is a machine that, given a message $m$ in $\M$ and a key $k$ in $\K$, returns a ciphertext $c$ in $\C$, which is an \emph{encrypted} form of the meesage;
    \item $\Dec \in \K \times \C \to \M$ is a machine that restores the message $m$ encrypted in the given ciphertext $c$ by using the key $k$, effectively \emph{decrypting} the message.
\end{itemize}

For the time being, assume that both \Enc{} and \Dec{} work as normal functions. A fundamental requirement of encryption schemes is that they always completely preserve the message after a whole round of encryption and decryption:
\[
    \forall m \in \M \; \forall k \in \K \quad \Dec(k, \Enc(k, m)) = m
\]

A first definition that formalizes secrecy of an encryption scheme comes from Shannon:

% TODO one-time?!?
% TODO - Is CPA the reason why perfect many-time secrecy isn't even contemplated?
\begin{definition}[\textsc{Perfect secrecy} --- Shannon's perfect secrecy]
    Given an encryption scheme $\Xi: (\Enc, \Dec)$, let $M$ be a generic random variable over the message space $\M$, with the only requirement that all messages $m$ have positive probability (i.e. it is \emph{nonexclusive}), and $K$ be a uniform random variable over the key space $\K$:
    \begin{align*}
        M \in&\ \mathcal{R}and(\M) : \forall m \in \M \quad \Pr[M = m] > 0 \\
        K \in&\ \unifdist(\K)
    \end{align*}
    Then let $\Enc(K, M)$ be the resulting ciphertext from encrypting $M$ using $K$. The scheme $\Xi$ is deemed \emph{perfectly secret} iff such ciphertext is effectively useless in retrieving any information about the original message:
    \[
        \forall m \in \M \; \forall c \in \C \quad \Pr[M = m] = \Pr[M = m \knowing \Enc(K, M) = c]
    \]
\end{definition}

%Perfect one-time secrecy is usually shortened to just \emph{perfect secrecy},\footnote{An explanation for this is in chapter 7} 
This definition can be rephrased in different ways, bringing more details to light:
    
\begin{proposition}
    The following statements are equivalent:
    \begin{enumerate}
        \item \label{def:ps1} $\Pr[M = m] = \Pr[M = m \knowing \Enc(K, M) = c]$
        \item \label{def:ps2} $M \indep \Enc(K, M)$
        \item \label{def:ps3} $\forall m_1, m_2 \in \M \, \forall c \in \C \quad \Pr[\Enc(K, m_1) = c] = \Pr[\Enc(K, m_2) = c]$
    \end{enumerate}
\end{proposition}

\begin{proof}
    The proof is structured as a cyclic implication between the three proposed definitions:
    
    \begin{itemize}
        \item $(\ref{def:ps1}) \implies (\ref{def:ps2})$: Start from one side of the independency definition, and work through the other:
        \begin{align*}
             & \Pr[M = m \wedge \Enc(K, M) = c]                         & \\
            =& \Pr[M = m \knowing \Enc(K, M) = c] \Pr[\Enc(K, M) = c]   & \text{(Conditional prob.)} \\
            =& \Pr[M = m] \Pr[\Enc(K, M) = c]                           & \text{(Using \ref{def:ps1})} \\
        \end{align*}
        This proves that $M$ and $\Enc(K, M)$ are indeed independent random variables.
        
        \item $(\ref{def:ps2}) \implies (\ref{def:ps3})$: Let $M$ be a generic random variable over $\M$. Then:
        \begin{align*}
             & \Pr[\Enc(K, m_1) = c]                                & \\
            =& \Pr[\Enc(K, M) = c \knowing M = m_1]                 & \text{(Conditioning $m$)} \\
            =& \Pr[\Enc(K, M) = c \wedge M = m_1] \Pr[M = m_1]^{-1} & \text{(Conditional prob.)} \\
            =& \Pr[\Enc(K, M) = c]                                  & \text{(Using \ref{def:ps2})} \\
            =& \Pr[\Enc(K, m_2) = c]                                & \mathllap{\text{(Same steps reversed, where $m_1 \mapsto m_2$)}} \\
        \end{align*}

        \item $(\ref{def:ps3}) \implies (\ref{def:ps1})$:
        \begin{align*}
             & \Pr[\Enc(K, M) = c]                                          & \\
            =& \sum_{m \in \M}\Pr[\Enc(K, M) = c \wedge M = m]              & \text{(Total prob.)} \\
            =& \sum_{m \in \M}\Pr[\Enc(K, M) = c \knowing M = m] \Pr[M = m] & \text{(Conditional prob.)} \\
            =& \sum_{m \in \M}\Pr[\Enc(K, m) = c] \Pr[M = m]                & \text{(Condition collapse)} \\
            =& \Pr[\Enc(K, m_0) = c] \sum_{m \in \M} \Pr[M = m]             & \mathllap{\text{(Using \ref{def:ps3} with arbitrary $m_0$})} \\
            =& \Pr[\Enc(K, m_0) = c]                                        & \text{(Total prob.)} \\
            =& \Pr[\Enc(K, M) = c \knowing M = m_0]                         & \text{(Conditioning $m_0$)}
        \end{align*}

        By applying Bayes' theorem, the above result can be turned into the first definition of perfect secrecy:
        \begin{align*}
                    & \Pr[C = c] = \Pr[C = c \knowing M = m]                                        & \\
            \implies& \Pr[C = c] = \Pr[M = m \knowing C = c] \cdot \frac{\Pr[C = c]}{\Pr[M = m]}    & \text{(Bayes' theorem)} \\
            \implies& \Pr[M = m] = \Pr[M = m \knowing C = c]                                        &
        \end{align*}
        where $C = \Enc(K, M)$.
    \end{itemize}
\end{proof}

About definition \ref{def:ps3}, it could be insightful to remark that, for any message $m$ and ciphertext $c$:
\begin{align*}
        & \Pr[\Enc(K, m) = c] \\
       =& \Pr[\Enc(K, M) = c \knowing M = m] \\
    \neq& \Pr[\Enc(K, M) = c]
\end{align*}

which is the difference between \emph{choosing} a specific message $m$, and picking it at random, according to $M$' distribution.

\subsection{One-time Pad}

The One-time Pad, or \otp{} in short, is a simple encryption scheme that leverages the involutory property of the \textsc{xor} operation. Let all the spaces $\K = \M = \C = \binary^l$ be binary strings of some length $l$, and define this scheme $\otp = (\Enc, \Dec)$ as such:
\begin{itemize}
    \item $\Enc(k, m) = k \oplus m$
    \item $\Dec(k, c) = k \oplus c$
    \item[--] Correctness: $\Dec(k, \Enc(k, m)) = \Dec(k, k \oplus m) = k \oplus k \oplus m = m$
\end{itemize}

\begin{theorem}
    \otp{} is perfectly secret.
\end{theorem}
\begin{proof}
    The proof makes use of definition \ref{def:ps3} of perfect secrecy. Let $K$ be a uniformly random key; then for any binary strings $m_1$, $m_2$ and $c$:
    \begin{align*}
        & \Pr[\Enc(K, m_1) = c]     & \\
        =& \Pr[K \oplus m_1 = c]    & \\
        =& \Pr[K = c \oplus m_1]    & \\
        =& |\K|^{-1}                & \text{(K is uniform)} \\
        =& \Pr[\Enc(K, m_2) = c]    & \text{(Same steps reversed, where $m_1 \mapsto m_2$)}  
    \end{align*}
\end{proof}

%TODO OLD NOTE: Observation: k is truly random, but fixed, compare with Pi_\oplus's encryption routine: from a security standpoint, nothing changes! actually, \Pi_\oplus may be weaker brute-force wise than OTP!

At this point, some important observations are in order:
\begin{enumerate}
    \item The key and the message's lengths must always match ($|k| = |m|$);
    \item As the scheme's name suggests, keys are useful just for one encryption. In fact, if the same key is used for two different encryptions, an attacker may exploit the \textsc{xor}'s idempotency to extract valuable information from both ciphertexts:
    \[
        c_1 = k \oplus m_1 \wedge c_2 = k \oplus m_2 \implies c_1 \oplus c_2 = m_1 \oplus m_2
    \]

    % This is actually an example of scheme malleability
    
\end{enumerate}

%\footnote{This vulnerability of applying a function on a ciphertext, and expecting as a result the image of the original message by the same function, is called \emph{malleability}, and is explored further in the notes.}

%\footnote{An encryption algorithm is "malleable" if it is possible to transform a ciphertext into another ciphertext which decrypts to a related plaintext. That is, given an encryption $c$ of a plaintext $m$, it is possible to generate another valid ciphertext $c'$, for a known $Enc$, $Dec$, without necessarily knowing or learning $m$.}

Combined with the fact that keys must be preemptively shared in a secret fashion, these caveats make for an impractical encryption scheme. The first point can actually be generalized for any scheme that is perfeclty secret:

\begin{theorem}
    For an encryption scheme to be perfectly secret, there must be more distinct keys than distinct messages: 
    \[
        \Xi \textrm{ is perfectly secret} \implies |\K| \geq |\M|
    \]
\end{theorem}
\begin{proof}

    % Proof of footnote: if $c$ has zero probability, then $S$ is empty. This means that no matter which key-message pair is chosen, $c$ will never come out. However $c \in \C$, and the latter makes part of \Dec{}'s domain, and if \Dec{} is still regarded to be a function, then $c$ \textbf{must} evaluate to some message, given any key whatsoever. That alone is a contradiction, because it indirectly violates correctness of $\Xi$. Therefore, there cannot be a ciphertext with zero probability.

    Let $M$ be a nonexclusive random variable over the messages, then fix a ciphertext $c$ that has some positive probability to be the result of $M$'s encryption.\footnote{Actually, it can be proven that there cannot be ciphertexts with zero probability.} This means that there are some key-message pairs that result in such ciphertext by means of encryption:
    \[
        \exists (k, m) \in \K \times \M \quad \Enc(k, m) = c
    \]
    Let $S$ be the set collecting all possible message ``sources'' that do encrypt into $c$:
    \[
        S = \{m \in \M : \exists k \in \K \quad \Enc(k, m) = c\}
    \]

    By using $\Xi$'s correctness, this definition can be twisted into using \Dec:
    \[
        S = \{\Dec(k, c) \in \M : k \in \K\}
    \]

    This reveals in a clearer way that there cannot be more sources than keys,\footnote{Remember that \Enc{} and \Dec{} are still considered as mathematical functions, despite them actually being machines} therefore $|S| \leq |\K|$. By also assuming that $|\K| < |\M|$, it follows that $|S| < |\M|$ too; of consequence is that there are some messages that cannot be possibly encrypted into $c$:
    \[
        x \in \M \setminus S \implies \forall k \in \K \quad \Pr[\Enc(k, x) = c] = 0
    \]

    Let $x$ be a message in $\M \setminus S$. Since $M$ is nonexclusive:
    \[
        \Pr[M = x] > 0
    \]
    However, $x$ cannot be encrypted into $c$ by definition, which also means:
    \[
        \Pr[M = x \knowing \Enc(K, M) = c] = 0
    \]
    Therefore:
    \[
        0 = \Pr[M = x \knowing \Enc(K, M) = c] \neq \Pr[M = m] > 0
    \]
    which violates perfect secrecy.
\end{proof}
