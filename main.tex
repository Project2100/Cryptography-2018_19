\documentclass[10pt, a4paper]{report}

\usepackage{nonotation}


%%%%%%%%%%%%%%%%%%%%%%%%% FONTS %%%%%%%%%%%%%%%%%%%%%%%%%%%%%%

\usepackage[utf8]{inputenc}
\usepackage[english]{babel}
\usepackage{bold-extra} % Used for boldface-smallcaps combo in section headers
\usepackage[normalem]{ulem} % For strikethrough text


%%%%%%%%%%%%%%%%%%%%%%%%% HYPERLINKS %%%%%%%%%%%%%%%%%%%%%%%%%

\PassOptionsToPackage{hyphens}{url}
\usepackage{hyperref}

\hypersetup{
    colorlinks,
    linkcolor       = {red!70!black},
    citecolor       = {blue!50!black},
    urlcolor        = {blue!80!black}
}

% Hyperlink dingbat
\usepackage{pifont} 
\newcommand{\linkicon}{\ding{226}}


%%%%%%%%%%%%%%%%%%%%%%%%% TIKZ %%%%%%%%%%%%%%%%%%%%%%%%%%%%%%%

\usepackage{tikz}
\usetikzlibrary{arrows.meta}

% Tikz styles used throughout all the figures
% Nodes
\tikzstyle{box}     = [draw, minimum size = 2em]
\tikzstyle{nobox}   = [minimum size = 2em]
% Pins
\tikzstyle{init}    = [pin edge = {-to, thin, black}]


%%%%%%%%%%%%%%%%%%%%%%%%% THEOREMS %%%%%%%%%%%%%%%%%%%%%%%%%%%

\usepackage{amsthm}

\newtheorem{theorem}                    {Theorem}       [chapter]
\newtheorem{lemma}          [theorem]   {Lemma}
\newtheorem{corollary}      [theorem]   {Corollary}
\newtheorem{claim}          [theorem]   {Claim}
\newtheorem{observation}    [theorem]   {Observation}
\newtheorem{proposition}    [theorem]   {Proposition}
\theoremstyle{definition}
\newtheorem{definition}     [theorem]   {Definition}
\newtheorem{exercise}       [theorem]   {Exercise}
\newtheorem{example}        [theorem]   {Example}
\newtheorem{construction}               {Construction}
\theoremstyle{remark}
\newtheorem{solution}       [theorem]   {Solution}


%%%%%%%%%%%%%%%%%%%%%%%%% TODOS %%%%%%%%%%%%%%%%%%%%%%%%%%%%%%

\usepackage{xcolor}
\usepackage{framed}

\newcounter{todocount}
\definecolor{shadecolor}{rgb}{1, 1, 0}
\newcommand{\todo}[1]{
    \stepcounter{todocount}
    \begin{shaded}
    \textsc{To-do} \arabic{todocount}: #1
    \end{shaded}
}


%%%%%%%%%%%%%%%%%%%%%%%%% OTHER %%%%%%%%%%%%%%%%%%%%%%%%%%%%%%

% Math?lap commands
\usepackage{mathtools}

% Custom chapter title
\newcommand{\mychapter}[2]{
    \setcounter{chapter}{#1}
    \setcounter{section}{0}
    \chapter*{#2}
    \addcontentsline{toc}{chapter}{#2}
}

\newcommand{\upto}{\text{---}} % Range ligature
\renewcommand{\mod}{\mathrel{\textup{mod}}} % Remainder operator
\newcommand{\groupgen}{\mathcal{GG}} % Group generators for standard assumptions
\newcommand{\quadres}{\mathcal{Q}uad} % Quadratic residue group
\newcommand{\keygen}{\textsf{\textup{Keygen}}} % Key generator for public-privte pairs
\newcommand{\C}{\mathcal{C}} % Cipherspace
\newcommand{\G}{\mathcal{G}} % Generally used for generators or groups
\newcommand{\K}{\mathcal{K}} % Keyspace
\newcommand{\M}{\mathcal{M}} % Messagespace
\newcommand{\RO}{\mathrm{RO}} % Random oracle
\newcommand{\Gen}{\mathrm{Gen}} % Generic generator
\newcommand{\otp}{\textsc{otp}} % One-Time Pad


%%%%%%%%%%%%%%%%%%%%%%%%%%%%%%%%%%%%%%%%%%%%%%%%%%%%%%%%%%%%%%

\begin{document}

    \tableofcontents

    \chapter{Introduction}

    Cryptography is the collection of techniques that are designed to enable secure communication between two parties in a setting where third parties intend to listen and/or modify the transmitted information. While encryption schemes are usually the first things that come to mind, it is worthwhile to carefully consider all the aspects that cryptography influences:

    \begin{itemize}
        \item \emph{Confidentiality}: the two parties can safely assume that they are communicating privately; this can be further split into:
              \begin{itemize}
                  \item \emph{Secrecy}: the aspect of communication privacy;
                  \item \emph{Authentication}: the aspect of party identity verification.
              \end{itemize}

        \item \emph{Integrity}: Some techniques employed here are also designed to ensure that the transmitted data is not  altered, a property of which desirability is paramount in safety-critical scenarios.
    \end{itemize}

    Modern cryptography systems are usually designed according to \emph{Kerckhoffs's principle}, which states that a secure system  shall only rely on the encryption keys, and not by the secrecy of the underlying algorithm; as Claude Shannon later summed up: \emph{``the enemy knows the scheme''}. The problem of sharing the key between two parties while retaining communication   confidentiality thus becomes central in developing a good scheme, and is the main focus of almost every scheme described here.

    \part{Mathematical foundations}

    \mychapter{1}{Lesson 1} %180926

Talking cryptography is usually done in the ``confidentiality'' realm, where two characteristics in a communication channel are desirable: it must be \emph{secret}, and \emph{authentic}.

\section{Secret communication}

\todo{Image of Alice,Bob, Eve in ske}

Modern confidentiality/authentication systems are forged under \emph{Kerckhoffs's principle}, which states that a secure system shall only rely on the encryption keys, and not on the underlying algorithm's secrecy; in short, \emph{``No security by obscurity''}. However sharing the key between two parties without the risk of eavesdropping is a costly operation.

The typical objects defined and used troughout cryptography discourse are:
\begin{itemize}
    \item $\K = $ Key Space
    \item $\M = $ Message Space
    \item $\C = $ Ciphertext Space
    \item $\Enc: \K \times \M \to \C$
    \item $\Dec: \K \times \C \to \M$
\end{itemize}

\Enc and \Dec form a cryptographic scheme (a secrecy one, in detail), and it must abide by the rule:
\[
    \forall m \in \M, \forall k \in \K \implies \Dec(k, \Enc(k, m)) = m
\]

\begin{definition}
    (\textit{Shannon's ``Perfect secrecy''}): Let $M$ be any distribution over the message space $\M$, and $K$ a uniform distribution over $\K$. Then, the cryptosystem $\Xi: (\Enc, \Dec)$ is deemed \emph{perfectly secret} iff the ciphertext obtained by applying the encryption routine to a message sampled by $M$ using tke key in $K$ is effectively useless in retrieving any info about the message itself, apart from the key-supplied decryption routine. Formally:
    \[
        \forall M \sim \mathcal{R}and(\M),\, \forall C \sim \mathcal{R}and(\C) \implies \Pr[M = m] = \Pr[M = m \knowing C \evaluatesto c]
    \]
\end{definition}
Note how this definition doesn't involve the encryption key. % Is it important? Why?

The perfect secrecy definition can be rephrased in different ways, bringing more details to light:
\begin{enumerate}
    \item $\Pr[M = m] = \Pr[M = m \knowing C \evaluatesto c]$
    \item $M \indep C$
    %TODO - Better formalize: the variables m, m' and k are 'chosen' by their own distibutions, c is just a result instead
    % Maybe decompose it into smaller bits: use event sets instead
    \item $\forall m_1, m_2 \in \M, c \in \C \implies \Pr[\Enc(K, m_1) = c] = \Pr[\Enc(K, m_2) = c]$
\end{enumerate}

A remark has to be done here:
\begin{align*}
    & \Pr[\Enc(K, m) = c] \\
    =& \Pr[\Enc(K, M) = c \knowing M \evaluatesto m] \\
    \neq& \Pr[\Enc(K, M) = c]
\end{align*}

which is exactly the difference between picking a specific message $m$, and choosing it at random, just as in $M$.
    
\begin{proposition}
    All the previous statements are equivalent.
\end{proposition}

\begin{proof}
    The proof is structured as a cyclic implication between the threee definitions:
    
    \begin{itemize}
        \item $(1) \implies (2)$: Start from one side of the independency ddefinition, and work through the other:
        \begin{align*}
            & \Pr[C = c \wedge M = m]                           & \\
            =& \Pr[C = c] \Pr[M = m \knowing C \evaluatesto c]  & \text{(Conditioned prob.)} \\
            =& \Pr[C = c] \Pr[M = m]                            & \text{(Using 1.)} \\
        \end{align*}
        This proves that $M$ and $C$ are independent distributions.
        
        \item $(2) \implies (3)$: For the proof's purposes, let $M$ be an arbitrary distribution over $\M$; recall that, by definition: $C := \Enc(K, M)$:
        \begin{align*}
            & \Pr[\Enc(K, m_1) = c]                             & \\
            =& \Pr[\Enc(K, M) = c \knowing M \evaluatesto m_1]  & \text{(Introducing $M$)} \\
            =& \Pr[C = c \knowing M \evaluatesto m_1]           & \text{($C$ defintion)} \\
            =& \Pr[C = c]                                       & \text{(Using 2.)} \\
            =& \dots                                            & \mathllap{\text{(Same steps reversed, where $m_1 \mapsto m_2$)}} \\
            =& \Pr[\Enc(K, m_2) = c]                            & \\
        \end{align*}


        \item $(3) \implies (1)$:
        \begin{align*}
            & \Pr[C = c] & \\
            =& \sum_{m}\Pr[C = c \wedge M = m]                                  & \text{(Total prob.)} \\
            =& \sum_{m}\Pr[\Enc(K, M) = c \wedge M = m]                         & \text{($C$ definition)} \\
            =& \sum_{m}\Pr[\Enc(K, M) = c \knowing M \evaluatesto m] \Pr[M = m] & \text{(Cond. prob. def.)} \\
            =& \sum_{m}\Pr[\Enc(K, m) = c] \Pr[M = m]                           & \text{(Cond. collapse)} \\
            =& \Pr[\Enc(K, \overline{m}) = c] \sum_{m} \Pr[M = m]               & \text{(Using 3.)} \\ %TODO Not convinced thoroughly
            =& \Pr[\Enc(K, \overline{m}) = c]                                   & \text{(Total prob.)} \\
            =& \Pr[\Enc(K, M) = c \knowing M \evaluatesto \overline{m}]         & \text{(Introducing $M$)} \\
            =& \Pr[C = c \knowing M \evaluatesto \overline{m}]                  & \text{($C$ definition)} \\
        \end{align*}

        Knowing this, and applying Bayes' theorem, we get back to the first definition:
        \begin{align*}
            & \Pr[C = c] = \Pr[C = c \knowing M \evaluatesto m]                                         & \\
            \implies& \Pr[C = c] = \Pr[M = m \knowing C \evaluatesto c] \frac{\Pr[C = c]}{\Pr[M = m]}   & \text{(Bayes' theorem)} \\
            \implies& \Pr[M = m] = \Pr[M = m \knowing C \evaluatesto c]                                 & \\
        \end{align*}

    \end{itemize}

    Thus, we conclude that all three definitions for perfect secrecy are equivalent.

\end{proof}

\section{One Time Pad (OTP)}

Let $\K = \M = \C = \binary^l$, and define the following cryptographic scheme:
\begin{itemize}
    \item $\Enc(k, m) = k \xor m = c$
    \item $\Dec(k, c) = k \xor c = m$
\end{itemize}

Proof of correctness goes like: $\Dec(k, \Enc(k, m)) = \Dec(k, k \xor m) = k \xor k \xor m = m$.

\begin{theorem}
    The \emph{One-Time Pad} scheme is perfectly secret.
\end{theorem}
\begin{proof}
    Let $K \sim \unifdist(\K)$. Then, $\forall m_1, m_2, c \in \binary^l$:
    \begin{align*}
        & \Pr[\Enc(K, m_1) = c]     & \\
        =& \Pr[K \xor m_1 = c]      & \\
        =& \Pr[K = c \xor m_1]      & \\
        =& |\K|^{-1}                & \text{(K is uniform)} \\
        =& \dots                    & \text{(Same steps reversed, where $m_1 \mapsto m_2$)} \\
        =& \Pr[\Enc(K, m_2) = c]    & \\  
    \end{align*}
    This satisfies the third definition of perfect secrecy.
\end{proof}

%TODO OLD NOTE: Observation: k is truly random, but fixed, compare with Pi_\xor's encryption routine: from a security standpoint, nothing changes! actualy, \Pi_\xor may be weaker bruteforce wise than OTP!

By observing our recent proof, some insights (and problems) arise:
\begin{enumerate}
    \item The key and the message's lengths must always match ($|k| = |m|$);
    \item As the name suggests, keys are useful just for one encryption. Otherwise, given two encryptions witht he same key, an attacker may exploit the \textsc{xor}'s idempotency to extract valuable information from both ciphertexts\footnote{This vulnerability of applying a function on a ciphertext, and expecting as a result the image of the original message by the same function, is called \emph{malleability}, and is explored further in the notes.}:
    \[
        c_1 = k \xor m_1 \wedge c_2 = k \xor m_2 \implies c_1 \xor c_2 = m_1 \xor m_2
    \]
    
\end{enumerate}

%\footnote{An encryption algorithm is "malleable" if it is possible to transform a ciphertext into another ciphertext which decrypts to a related plaintext. That is, given an encryption $c$ of a plaintext $m$, it is possible to generate another valid ciphertext $c'$, for a known $Enc$, $Dec$, without necessarily knowing or learning $m$.}

Combined with the fact that keys must be preemptively shared in a secure fashion, these problems make for a quite impractical cryptographic scheme. One can further this analysis and generalize it to all ``perfect'' schemes, giving a rather delusive conclusion:

\begin{theorem}
    In any perfectly secret \ske{}: $|\K| \geq |\M|$
\end{theorem}
\begin{proof}
    Perfection will be disproved by breaking the first definition. Let $M \sim \unifdist(\M)$, and take $c \in \C : \Pr[C = c] > 0$. Consider $D = \{\Dec(k, c) : k \in \K\}$ as the set of all images of the decryption routine with all keys in $\K$. The offending assumption is that $|\K| < |\M|$. Then, by how $D$ is defined:
    \[
        |D| \leq |\K| < |\M| \implies \exists m \in \M \setminus D
    \]

    \begin{figure}[h]
        \centering
        \def\firstcircle{(0,0) circle (1.5cm)}
        \def\secondcircle{(60:0) circle (0.9cm)}
        \begin{tikzpicture}
            \begin{scope}[shift={(3cm,-5cm)}]
                \draw \firstcircle node[label={[xshift=1.0cm, yshift=0.3cm]$\M$}] { };
                \draw \secondcircle node[label={[xshift=1.2cm, yshift=-0.5cm]$m$}] {$D$};
            \end{scope}
        \end{tikzpicture}
        \caption{Where the messages stand}
    \end{figure}

    Fix this message $m$, and remember that by $M$'s definition, $\Pr[M = m] = |\M|^{-1}$. Since $m \notin D$, there can be no key in $\K$ such that $\Dec(k, c) = m$. By observing that $C$ strictly distributes over $D$, where $m$ is not present, $\Pr[M = m \knowing C \evaluatesto c] = 0$. Summarizing up:
    \[
        0 = \Pr[M = m \knowing C \evaluatesto c] \neq \Pr[M =o m] = |\M|^{-1}
    \]
    This clearly violates the first definition of perfect secrecy.
\end{proof}

    \mychapter{2}{Lesson 2} %180928

\section{Authentic communication}


A simple scenario exposing the problem of authentication is depicted in figure \ref{fig:authentication}. This time, the parties Alice and Bob want to ensure that they are effectively communicating to each other, or in other words, nobody else is \emph{impersonating} one of the two. The objects used here are:
\begin{itemize}
    \item The data to be shared, or \emph{message} $m$;
    \item Some additional secret information, shared by the two parties, that is used to \emph{sign} the message: the \emph{authentication key} or just \emph{key} $k$;
    \item The result of signing a message $m$ using the key $k$: the \emph{signature} or \emph{tag} $\phi$.
\end{itemize}

\begin{figure}[ht]
    \centering
    \begin{tikzpicture}
        \draw
            (0, 0) node (a) [box, fill = white] {Alice} 
            (5, 0) node (b) [box, fill = white] {Bob}
            (2.5, -1) node (e) [box, fill = white] {Eve}
        ;

        \draw[-Stealth] (a) -- node [midway, above] {$(m, \phi)$} (b);
        \draw[-Stealth] (0, -1) node [below] {$k$} -- (a);
        \draw[-Stealth] (5, -1) node [below] {$k$} -- (b);
        \draw[-Stealth] (-1, 0) node [left] {$m$} -- (a);
        \draw[-Stealth] (b) -- (6, 0) node [right] {$1$};
        \draw[-Stealth] (2, -0.1) -- (3, -0.1)
            (2.5, -0.1) -- (e);

    \end{tikzpicture}

    \caption{A depiction of the problem of authentic communication}
    \label{fig:authentication}
\end{figure}

Here is employed a \emph{cryptographic authentication scheme}, or just \emph{authentication scheme}, typically taking the form $\Psi = (\Tag, \Ver)$, where:
\begin{itemize}
    \item $\Tag \in \K \times \M \to \Phi$ is the function that, given a message $m$ in $\M$ and a key $k$ in $\K$ generates the signature $\phi$
    \item $\Ver \in \K \times \M \times \Phi \to \binary$ decides whether $\phi$ is the correct signature for the message $m$ using the key $k$. It is udeful to note that, in various schemes, the verifier simply consists in invoking the tagging function, and comparing the resulting signature with the given one:
    \[
        \Ver(k, m, \phi) = [\Tag(k, m) = \phi]
    \]
\end{itemize}

Given such definitions, an authentication scheme works as intended if and only if:
\[
    \forall m \in \M, \forall k \in \K \implies \Ver(k, m, \Tag(k, m)) = 1
\]

The security aspect to consider in these schemes is the signatures' \emph{unforgeability}: if Eve is able to obtain a signature $\phi$ of a message $m$ of her own choice, then she shall have no better means of \emph{forging} a valid couple $(m', \phi')$ such that $\Ver(k, m', \phi') = 1$ other than random guessing, or knowing the key, of course. The coming definitions will help to give this notion more formal ground:

\begin{definition} \emph{($\varepsilon$-statistical one-time security)}:
    An authentication scheme has $\varepsilon$-statistical one-time security iff, given a valid couple $(m_1, \phi_1)$, any adversary cannot \emph{forge} a fresh valid couple $(m_2, \phi_2)$ without knowing the signature key $k$:
    \[
        \forall m_1 \neq m_2 \in \M \, \forall \phi_1, \phi_2 \in \Phi \quad \Pr[\Tag(K, m_2) = \phi_2 \knowing \Tag(K, m_1) = \phi_1] \leq \varepsilon
    \]
\end{definition}

\subsubsection{Pairwise-independent hashing}

%AP190903: The math stinks here...
\begin{definition}
    Let $\mathcal{S}$ be a seeding space; define a family of \emph{hash functions} to be the following object\footnotemark:
    \[
        H \in \mathcal{S} \to (\M \to \Phi) : s \mapsto h_s
    \]

    \footnotetext{This kind of notation consisting in putting an argument as a subscript to a generic,  typically of higher-order function, is also called ``currying''; it will be used extensively throughout the lessons.}

    Let $S$ be a random variable in the seed space; the hash functions are deemed \emph{pairwise-independent}\footnotemark{} iff, for any two distinct messages $m$ and $m'$, the pair $(h_S(m), h_S(m'))$ distributes evenly in $\Phi^2$. In other words:
    \[
        \forall m, m' \in \M : m \neq m', \forall \phi, \phi' \in \Phi \implies \Pr[h_S(m) = \phi \wedge h_S(m') = \phi'] = \oneover{|\Phi|^2}
    \]
\end{definition}

\footnotetext{Care should be taken to not confuse \emph{pairwise} independency with \emph{mutual} independency: while the former acts only on pairs, the latter considers all possible subsets. The two notions are not necessarily equivalent.}

As an example of such a family, consider the additive group of integers modulo $p$: $(\integer_p, +)$, where $p$ is a prime integer. Define the family:
\[
    h_{(a, b)}(x) = ax + b \mod p
\]

where $\mathcal{S} = \integer_p^2$, and $\M = \Phi = \integer_p$. 

\begin{theorem}
    The functions in the family $H$ are pairwise-independent.
\end{theorem}

\begin{proof}
    Let $S = (a, b)$ be a random seed for $H$; for any distinct messages $m, m'$, and for any tags $\phi, \phi'$:

    \begin{align*}
        & \Pr[h_S(m) = \phi \wedge h_S(m') = \phi'] \\
        =& \Pr[am + b = \phi \wedge am' + b = \phi'] \\
        =& \Pr\left[
        \begin{pmatrix}
            m & 1 \\
            m' & 1
        \end{pmatrix}
        \cdot
        \begin{pmatrix}
            a \\
            b
        \end{pmatrix}
        =
        \begin{pmatrix}
            \phi \\
            \phi'
        \end{pmatrix}
        \right] \\
        =& \Pr\left[
        \begin{pmatrix}
            a \\
            b
        \end{pmatrix}
        =
        \begin{pmatrix}
            m & 1 \\
            m' & 1
        \end{pmatrix}^{-1}
        \cdot
        \begin{pmatrix}
            \phi \\
            \phi'
        \end{pmatrix}
        \right] \\
        =& \oneover{|\integer_p|^2}
    \end{align*}
    which is exactly the definition of pairwise-independency.
\end{proof}

\begin{theorem}
    Define an authentication scheme to be such that its tagging routine is a hash function family: $\Tag(k, m) = h_k(m)$. Let this function family be pairwise-independent. Then the authentication scheme is $\oneover{|\Phi|}$-statistical one-time secure. 
\end{theorem}

\begin{proof}
    For all our purposes, let $K \sim \unifdist(\K)$; then:

    \begin{align*}
        \forall m \in \M, \phi \in \Phi &\implies \Pr[\Tag(K, m) = \phi] = \Pr[h_K(m) = \phi] = \oneover{|\Phi|} \\
        \forall m \neq m' \in \M, \phi, \phi' \in \Phi &\implies \Pr[\Tag(K, m) = \phi \wedge \Tag(K, m') = \phi'] = \oneover{|\Phi|^2} \\
    \end{align*}
    Therefore:
    \begin{align*}
        & \Pr[\Tag(K, m') = \phi' \knowing \Tag(K, m) \evaluatesto \phi] \\
        =& \frac{\Pr[\Tag(K, m') = \phi' \wedge \Tag(K, m) = \phi]}{\Pr[\Tag(K, m) = \phi]} \\
        =& \frac{|\Phi|}{|\Phi|^2} = \oneover{|\Phi|}
    \end{align*}
\end{proof}

\begin{theorem}
    Any $(2^{-\lambda})$-statistical $t$-time secure authentication scheme has a key of size $(t + 1) \lambda$ for any $\lambda > 0$.
\end{theorem}

\begin{proof}
    None given.

    Idea: For each ``time'', use one ``subkey''
\end{proof}

    \mychapter{3}{Lesson 3} %181003

\newcommand{\Extm}{\textup{\textsf{Ext}}}
\newcommand{\Extf}{\textrm{\textup{Ext}}}
\newcommand{\statdist}{\ensuremath{\Delta_\textsc{s}}}
\newcommand{\sdtu}{\ensuremath{\Delta_\textsc{u}}}

\section{Randomness Extraction}

As just seen in both examples of secret and auhentic communications, it is worth observing that they share a critical common mechanism in the form of a preemptively shared key that is picked uniformly at random form the key space. A problem then arises in actually trying to construct something, like a program, that generates ``true'', totally random keys in a timely fashion. The goal is indeed to efficiently extract uniform randomness form things that are not necessarily uniformly random, or worse yet, deterministic.

Suppose to have an unknown coin $C$ that has some nontrivial probability $p$ to evaluate heads. There is a simple algorithm devised by John Von Neumannn that is capable of extracting a fair coin out of it, and it is known as the \emph{Von Neumann extractor}:

\begin{enumerate}
    \item Let $C \sim \berdist(p)$ be a random variable
    \item \label{enum:VNEsample} Sample $c_1 \pickUAR B$
    \item Sample $c_2 \pickUAR B$
    \item If $c_1 = c_2$ go to step \ref{enum:VNEsample}
    \item \label{enum:VNEreturn} Else output $b_1$:
\end{enumerate}

At a high level, this is a loop that in theory could run infinitely; by analyzing it more carefully, the probability of breaking out the loop on an iteration is the same as when the two coin flips $c_1$ and $c_2$ evaluate to different values, which in total is $2p(1 - p)$. There is indeed some probability that a given iteration leads to a new one, but on a complete run's point of view, since all coin flips are independent, the number of iterations that occur before exiting follow a  geometric distribution in $p$, thus the probability of an increased number of failures decrease exponentially.\footnote{Nevertheless, such probability may still be very high when the original coin has a very heavy bias.}

This extractor gives some reasonable hope for constructing a more generic randomness extractor. Before going forward though, it may be best to have a way to measure how much a given random variable is close to a uniform one:

\begin{definition}
    Let $X$ be a random variable from a given probability distribution. A measure of its uniformity is its \emph{min-entropy}, defined as a negative logarithm of the highest probability among all of its outcomes:\footnote{This is a form of the more general definition of R\'enyi entropy}
    \[
        H_{\infty}(X) = -\log(\max_{x \in \Omega}(\Pr[X = x]))
    \]
\end{definition}

To get a feel of this measure, it's worth starting from the trivial case of a ``constant'' random variable $X$ that certainly evaluates to an outcome $x$; by definition, the highest probability on the outcomes of $X$ is 1, therefore: 
\[
    H_{\infty}(X) = -\log(\Pr[X = \overline{x}]) = -\log(1) = 0
\]

Unsurprisingly, a constant variable is useless in creating a uniform distribution: it always gives the same outcome, it is effectively deterministic. On the other hand, let $Y$ be a uniform random variable over some finite outcome space $\Omega$; in this case, all probabilities are the same, thus:
\[
    \forall y \in \Omega \qquad H_{\infty}(Y) = -\log(\Pr[Y = y]) = -\log(|\Omega|^{-1}) = \log(|\Omega|)
\]

\begin{figure}
    \centering
    \begin{tikzpicture}
        \datavisualization[
                scientific axes,
                visualize as smooth line/.list = {first, second},
                x axis = {label = {$\Pr[C = 1]$}},
                y axis = {label = {$H_\infty(C)$}},
                data/format = function]
            data[set = first] {
                var x : interval[0 : 1/2];
                func y = - ln(1 - \value x);
            }
            data[set = second] {
                var x : interval[1/2 : 1];
                func y = - ln(\value x);
            }
        ;
    \end{tikzpicture}
    \caption{The min-entropy of a coin $C$ as a function of how likely it evaluates to heads}
    \label{fig:coinminentropy}
\end{figure}

Figure \ref{fig:coinminentropy} shows how the min-entropy of a coin $C$ variates by tuning its probability of evaluating to heads; it is apparent that the min-entropy reaches the highest point when the coin is fair (i.e. uniform), and goes to zero when it effectively behaves like a constant random variable.

%This point can be further expanded:
%\begin{proposition}
%     The min-entropy of any random variable $X$ is upper bounded by a specific function of its outcome space:
%     \[
%         \forall X \in \mathcal{R}and_\Omega \qquad H_\infty(X) \leq -\log(|\Omega|^{-1})
%     \]
%\end{proposition}
%\begin{proof}
%\end{proof}

Having some grasp on how the measure of min-entropy works, the problem of extracting uniform randomness from non-uniform sources can then be reformulated as the problem of maximizing min-entropy of a given random variable that, in the general case, is not maximal.

In order to have a simplified exposition, the following paragraphs will deal with binary strings of some fixed length $n$ as outcomes, and the definition of min-entropy will be adapted to employ a base-2 logarithm. In this framework, the min-entrpoy of a uniform random variable $U$ over the string space $\binary^n$ becomes the string length itself:\footnote{This also explains how the string length is a recurrent topic in cryptography discussions, as it indirectly expresses a cryptosystem's strength: the greater its min-entropy, the harder it is to find the right key for a ciphertext from scratch.}
\[
    H_\infty(X) = \log_2(|\binary^n|) = \log_2(2^n) = n
\]

Despite such efforts for constructing a universal extractor, a general counterexample can always be found:

\todo{ Lots of unsolved questions here:
\begin{itemize}
    \item Is it possible to put a bound on the min-entropy instead of an equality, and gain some added value out of this claim?
    \item Both the statement and the proof have some hideous caveats that may stem from confusing the extractor as a machine with the extractor as a function, and ultimately lead to a statement that blindly treats two RVs $X$ and $Y$ as "equivalent"
\end{itemize}
... actually, is a statement about min-entropy even needed?
}

\begin{claim}
    There is no universal algorithm $\Extm$ that, given a random variable $X$ with relatively high min-entropy:
    \[
        H_\infty(X) \leq n - 1
    \]
    returns a uniform random variable $U$:
    \[
        \Extm \in \binary \to \binary : \forall X \in \mathcal{R}and_{\binary} : H_\infty(X) = n - 1 \qquad \Extm(X) \sim U
    \]
\end{claim}

\begin{proof}
    Assume there exists such a universal extractor $\Extm$, and let $C$ be a coin with min-entropy $n - 1$. The general gist of $\Extm$ is that it uses $C$ multiple times, each time $i$ obtaining an outcome $q_i$, in order to generate the random output. Let $q$ be the generic sequence of outcomes that $\Extm$ receives; in this perspectve, it is possible to reformulate the extractor as a function $\Extf$, where its domain $Q$ is composed of the sequences $q$ for which $\Extm$ does stop and gives an output:\footnote{Not to be confused with the \emph{recognized language} of $\Extm$, as $Q$ also includes the sequences for which the extractor yieds 0}
    \[
        Q \subseteq \binary^* \qquad \Extf \in Q \to \binary : q \in Q \Longleftrightarrow \Extm \textrm{ stops}
    \]

    From there, $Q$ can be seen as bipartitioned by the set $Q_1$ that collects the sequences that evaluate to 1,\footnote{And that can be called the language recognized by $\Extf$} and $Q_0$ that collects the sequences evaluating to 0.

    \todo{And here's the supposed plot hole: $Y$ and $X$ cannot be in such a direct relationship, as $Y$ destroys the independencies of each toss of $X$}

    Let $Y$ be a random variable in $\binary^*$ such that it distributes uniformly over the greatest block between $Q_1$ and $Q_0$. Then the min-entropy of $Y$ is $\log(|Q_1|)$; however, since all the inputs in $Q_1$ are mapped to 1 by the extractor function:
    \[
        \forall q \notin Q_1 \qquad \Pr[Y = q] = 0 \implies \Pr[\Extf(Y) = 1] = 1
    \]

    which contradicts the property of the original $\Extm$ to be a universal extractor.
\end{proof}

% This is probably the original version, but that is really unrelated to some thing as the Von Neumann Extractor:
%
% Let Ext be a universal extractor from 2^n to 2, and X be a nontrivial RV in 2^n; then Ext(X) sim unifdist by definition.
% Since ext is a machine (determinitstic!), it can be equivalent to a math function, and in this setting, its domain can be split parts Q0 and Q1 according to Ext. Now let Y be a RV that uniformly distributes over the greatest of Q0 and Q1. Then, the minentroy of Y is > n-1, however Ext(Y) will always return the same value, i.e. it has become constant. Contradiction; thus ext cannot be universal

So there can be no construct that transforms any unknown distribution into the uniform one. This can be changed if, along with the unknown randomness, a smaller amount of uniform randomness is introduced to produce the desired uniform distribution. Before formalizing such idea, another notion of measure related to probability distributions can help:

\begin{definition}[Statistical distance]
    Let $X$ and $Y$ be two random variables on the same probability space. Then, for each outcome, take half the absolute difference of probabilities between the two, and sum all of them up. The result is called the \emph{statistical distance} between $X$ and $Y$:
    \[
        \statdist(X, Y) = \half\sum_{x \in \Omega}|\Pr[X = x] - \Pr[Y = x]|
    \]
\end{definition}

\begin{figure}
    \centering
    \begin{tikzpicture}
        \datavisualization[
                scientific axes = clean,
                visualize as scatter/.list = {first, second},
                style sheet = cross marks,
                first = {label in legend = {text = $X$}},
                second = {label in legend = {text = $Y$}},
                x axis = {label = {$\omega$}},
                y axis = {label = {$\Pr$}, include value = {0, 1}}]
            data[set = first] {
                x, y
                0, 0.25
                1, 0.15
                2, 0.35
                3, 0.05
                4, 0.2
            }
            data[set = second] {
                x, y
                0, 0.05
                1, 0.2
                2, 0.1
                3, 0.3
                4, 0.35
            }
        ;
    \end{tikzpicture}
    \caption{Two random distributions denoted espectively by random variables $X$ and $Y$}
    \label{fig:statdist}
\end{figure}

Figure \ref{fig:statdist} shows a visual interpretation of such measure with two arbitrary random variables, where their statistical distance is half the sum of the distances of the points on each column.

In most scenarios, given a random variable $X$, it is valuable to know how much it is distant to a uniform random variable $U$ over the same space $\Omega$. To this purpose, the notation $\statdist(X, U)$ will be shortened to $\sdtu(X)$, making any definition of uniform variables implicit.

This measure contributes in building a slightly more sophisticated definition of a randomness extractor:

\begin{definition}[Seeded extractor]
    Let $\Extm$ be a machine having two binary inputs of length $d$ and $n$, and binary output length $l$:\footnote{Spoiler: The first input, of length $d$, is also called a \emph{seed}}
    \[
        \Extm: \binary^d \times \binary^n \to \binary^l
    \]
    and let $S$ and be a uniform random variable in the irst input $\binary^d$. Then $\Extm$ is called a ($k$, $\varepsilon$)-extractor iff, for any random variable $X$ with min-entropy at least $k$, the output distribution of $\Extm(S, X)$ is statistically close to a uniform distribution within a bound of $\varepsilon$:
    \[
        \forall X : H_{\infty}(X) \geq k \implies \statdist((S, \Extm(S, X)), (S, \unifdist(\binary^l))) \leq \varepsilon
    \]
\end{definition}

% AP210312: Consider putting the conversion to SDTU here

\subsubsection{Universal hash functions}

The hash function families introduced earlier turn out to be a straightforward tool in constructing seeded extractors; they work even under weaker assumptions. For this, consider the probability notion of \emph{collision}:

\begin{definition}[Collision probability]
    Let $X$ and $Y$ be two \iid{} random variables; a \emph{collision} is the event of both variables evaluating to the same outcome. Such event probabilities are denoted as:
    \[
        Col(X) = Col(Y) = \Pr[X = Y]
    \]
\end{definition}

A different formulation of collision probability can be reached by simple manipulations, with the added benefit that it refers to only one of the two \iid{} random variables:
\begin{align*}
    \Pr(X = Y) =& \sum_{x \in \Omega} \Pr[X = x \wedge Y = x] & \text{(Total probability)} \\
               =& \sum_{x \in \Omega} \Pr[X = x] \Pr[Y = x]   & \text{(Independency)}      \\
               =& \sum_{x \in \Omega} \Pr[X = x]^2                                         \\
\end{align*}

\begin{definition}
    Let $S$ be a uniform seed. A hash function family $H$ is deemed \emph{universal} iff:
    \[
        \forall a \neq b \in \Omega \implies \Pr[h_S(a) = h_S(b)] = \oneover{\binary^l}
    \]
\end{definition}

\begin{lemma}[Collision bound] \label{lem:colbound}
    Let $X$ be a random variable such that its collision probability is upper bounded by the following function of some positive value $\varepsilon$ arbitrarily close to $0$:
    \[
        Col(X) \leq \frac{1 + 4\varepsilon^2}{|\Omega|}
    \]
    Then $\sdtu(X) \leq \varepsilon$, meaning that $X$ is almost uniform over $\Omega$.
\end{lemma}


\begin{proof}
    By definition of statistical distance:
    \[
        \sdtu(X) = \half \sum_{x \in \Omega} \left| \Pr[X = x] - \oneover{|\Omega|} \right|
    \]

    Decompose each of the above addends into the couple $q_x$ and $s_x$:
    \[
        q_x = \Pr[X = x] - \oneover{|\Omega|} \qquad s_x =
        \begin{cases}
            1  & \text{if $q_x \geq 0$} \\
            -1 & \text{otherwise}
        \end{cases}
    \]
    
    Then, by the Euclidean variant of the Cauchy-Schwarz inequality:
    \begin{align*}
        \sdtu(X) =&\ \half \sum_{x \in \Omega} q_x s_x                                                                & \text{(Decomposition)}     \\
              \leq&\ \half \sqrt{ \left( \sum_{x \in \Omega} q_x^2 \right) \left( \sum_{x \in \Omega} s_x^2 \right) } & \text{(Cauchy-Schwarz)}    \\
                 =&\ \half \sqrt{|\Omega| \sum_{x \in \Omega} q_x^2}                                                  & (\forall x \; (s_x^2 = 1)) \\
    \end{align*}

    Focusing on the sum $\sum_{x \in \Omega} q_x^2$:
    \begin{align*}
        \sum_{y \in \Omega} q_x^2 =&\ \sum_{x \in \Omega} \left( \Pr[X = x] - \oneover{|\Omega|} \right)^2                                   &                        \\
                                  =&\ \sum_{x \in \Omega} \left( \Pr[X = x]^2 - \frac{2}{|\Omega|} \Pr[X = x] + \oneover{|\Omega|^2} \right) &                        \\
                                  =&\ Col(X) - \frac{2}{|\Omega|} + \frac{1}{|\Omega|}                                                       &                        \\
                               \leq&\ \frac{1 + 4 \varepsilon^2}{|\Omega|} - \frac{2}{|\Omega|} + \oneover{|\Omega|}                         & \text{(By hypothesis)} \\
                                  =&\ \frac{4 \varepsilon^2}{|\Omega|}                                                                       &                        \\
    \end{align*}

    Thus, getting back to the statistical distance evaluation:
    \begin{align*}
        \sdtu(X) \leq&\ \half \sqrt{|\Omega| \sum_{x \in \Omega} q_x^2}        \\
                 \leq&\ \half \sqrt{|\Omega| \frac{4 \varepsilon^2}{|\Omega|}} \\
                    =&\ \half 2 \varepsilon = \varepsilon                      \\
    \end{align*}

    and by stating that $\varepsilon$ is arbitrarily close to zero, $X$ is statistically close to the uniform distribution over $\Omega$.
\end{proof}

\subsubsection{Leftover hash lemma}

This lemma has been proven by Russell Impagliazzo, Leonid Levin and Michael Luby:

\begin{lemma}[Leftover hash]
    Let $h_s$ be a pairwise-independent hash function with uniform seed $S$, and $\Extm \in \binary^d \times \binary^n \to \binary^l$ be a seeded randomness extractor. Then if $\Extm(S, x) = h_S(x)$, and $x$ is governed by a random variable $X$ with min-entropy $H_{\infty}(X) \geq k$, where:
    \[
       k = l - 2 \log_2 \varepsilon - 2
    \]
    then $\Extm$ is a ($k$, $\varepsilon$)-seeded randomness extractor.
\end{lemma}

\begin{proof}
    The extractor definition requires that:
    \[
        \statdist(Y, (S, U_l)) \leq \varepsilon
    \]
    where $U_l$ distributes uniformly over $\binary^l$. Although the appearance of $S$ in both operands may seem to complicate things, the distance formulation can be changed to remove such constraint, and become simpler:
    \begin{align*}
         &\ \statdist((S, h_S(X))), (S, U_l))                                                                                               &                \\
        =&\ \half \sum_{(s, t) \in \binary^{d + l}} \left| \Pr[(S, h_S(X))) = (s, t)] - \Pr[(S, U_l) = (s, t)] \right|                      &                \\
        =&\ \half \sum_{(s, t) \in \binary^{d + l}} \left| \Pr[(S, h_S(X))) = (s, t)] - \oneover{|\binary|^d} \oneover{|\binary^l|} \right| & (S \indep U_l) \\
        =&\ \half \sum_{(s, t) \in \binary^{d + l}} \left| \Pr[(S, h_S(X))) = (s, t)] - \Pr[U_{d + l} = (s, t)] \right|                     &                \\
         &                                                                                          & \mathllap{(U_{d + l} \sim \unifdist(\binary^{d + l}))} \\
        =&\ \statdist((S, h_S(X)), U_{d + l})                                                                                               &                \\
        =&\ \sdtu((S, h_S(X)))                                                                                                              &                \\
    \end{align*}
    
    At this point the \hyperref[lem:colbound]{collision bound} can be used to put an upper bound on this statistical distance. To this end, let $Y = (S, h_S(X))$ and $Y = (S, h_S(X))$ be two \iid{} random variables; the collision probability of $Y$ equates to:
    \begin{align*}
         &\ Col(Y)                                               &                                               \\
        =&\ \Pr[Y = Y']                                          &                                               \\
        =&\ \Pr[S = S' \wedge h_S(X) = h_{S'}(X')]               & \mathllap{\text{(By definition)}}             \\
        =&\ \Pr[S = S'] \Pr[h_S(X) = h_{S'}(X') \knowing S = S'] & \mathllap{\text{(Conditional prob.)}}         \\
        =&\ \Pr[S = S'] \Pr[h_S(X) = h_S(X')]                    & \mathllap{\text{(Condition collapse)}}        \\
        =&\ 2^{-d} \Pr[h_S(X) = h_S(X')]                         & \mathllap{\text{(Uniform collision prob.)}}   \\
        =&\ 2^{-d} \left( \Pr[h_S(X) = h_S(X') \wedge X = X'] + \Pr[h_S(X) = h_S(X') \wedge X \neq X'] \right) & \\
         &                                                       & \mathllap{\text{(Total probability)}}         \\
    \end{align*}

    The two addends are tackled separately. For the first one: 

    \begin{align*}
         &\ \Pr[h_S(X) = h_S(X') \wedge X = X']               &                               \\
        =&\ \Pr[X = X'] \Pr[h_S(X) = h_S(X') \knowing X = X'] & \text{(Conditional prob.)}    \\
        =&\ Col(X) \Pr[h_S(X) = h_S(X') \knowing X = X']      & \text{(Collision def.)}       \\
        =&\ Col(X) \Pr[h_S(X) = h_S(X)]                       & \text{(Condition collapse)}   \\
        =&\ Col(X)                                            & \text{(Prob. of a tautology)} \\
    \end{align*}

    \begin{proposition}
        Given any random variable $X$:
        \[
            H_\infty(X) \geq k \implies Col(X) \leq 2^{-k} \qedhere
        \]
    \end{proposition}

    \begin{proof}
        By definition of min-entropy\footnote{The definition specifies explicitly a base-2 logarithm, but can actually be any base; in case, the statement to be proved shall correct the collision probability bound accordingly by changing its exponential basis}:
        \begin{align*}
            -\log \max_{x \in \Omega}(\Pr [X = x]) &\geq k      \\
            \log \max_{x \in \Omega}(\Pr [X = x])  &\leq -k     \\
            \max_{x \in \Omega}(\Pr [X = x])       &\leq 2^{-k} \\
        \end{align*}

        On the other hand:
        \begin{align*}
            Col(X) =&\ \sum_{x \in \Omega} \Pr[X = x]^2                               \\
                \leq&\ \sum_{x \in \Omega} \Pr[X = x] \max_{y \in \Omega}(\Pr[X = y]) \\
                   =&\ \max_{y \in \Omega}(\Pr[X = y]) \sum_{x \in \Omega} \Pr[X = x] \\
                   =&\ \max_{y \in \Omega}(\Pr[X = y])                                \\
        \end{align*}

        Therefore:
        \[
            Col(X) \leq \max_{x \in \Omega}(\Pr[X = x]) \leq 2^{-k}
        \]
    \end{proof}

    Shifting the focus to the second addend:

    \begin{align*}
            &\ \Pr[h_S(X) = h_S(X') \wedge X \neq X']                  &                                                  \\
           =&\ \Pr[X \neq X'] \Pr[h_S(X) = h_S(X') \knowing X \neq X'] & \text{(Conditional prob.)}                       \\
        \leq&\ \Pr[h_S(X) = h_S(X') \knowing X \neq X']                & \mathllap{\text{(Prob. is not greater than 1)}}  \\
    \end{align*}

    \begin{proposition}
        If $h_S$ is a pairwise independent hash function, then:
        \[
            \Pr[h_S(X) = h_S(X') \knowing X \neq X'] \leq 2^{-l}
        \]
    \end{proposition}
    
    \begin{proof}
        \todo{Left as an exercise.}
    \end{proof}

    Going back to the collision probability of $Y$, and using the two propositions above:
    \begin{align*}
        Col(Y) \leq&\ 2^{-d} \left( Col(X) + \Pr[h_S(X) = h_S(X') \knowing X \neq X'] \right) & \\
               \leq&\ 2^{-d} (2^{-k} + 2^{-l}) & \\
                  =&\ 2^{-d - l} (2^{-k + l} + 2^{-l + l}) & \\
                  =&\ 2^{-d - l} (2^{-(l - 2 \log_2(\varepsilon) - 2) + l} + 1)  & \\
                  =&\ 2^{-d - l} (2^{2 \log_2(\varepsilon) + 2} + 1)  & \\
                  =&\ 2^{-d - l} (4\varepsilon^2 + 1) & \\
    \end{align*}

    Applying the collision bound entails that $\sdtu(Y) \leq \varepsilon$, therefore $h_S$ is a $(k, \varepsilon)$-seeded randomness extractor.
\end{proof}

Note that for smaller values of $\varepsilon$ we have greater values of min-entropy. A problem that is still open is to find an extractor with $k \approx l$.

    \mychapter{4}{Lesson 4} %181005

\section{Negligible function}

What is exactly a negligible function? Below here there is a possible interpretation of this notion, taken from an answer to a question in the Cryptography Stack Exchange website:
\begin{quotation}
    ``[...] in modern cryptographic schemes, we generally do not try to achieve perfect secrecy [...]. Instead, we define security against a specific set of adversaries whose computational power is bounded. Generally, we assume an adversary that is bounded to run in time polynomial to $n$, where $n$ is the security parameter given to the key generation algorithm [...].

    So consider a scheme $\Pi$ where the only attack against it is brute-force attack. We consider $\Pi$ to be secure if it cannot be broken by a brute-force attack in polynomial time.

    The idea of \emph{negligible probability} encompasses this exact notion. In $\Pi$, let's say that we have a polynomial-bounded adversary. Brute force attack is not an option. But instead of brute force, the adversary can try (a polynomial number of) random values and hope to guess the right one. In this case, we define security using negligible functions: The probability of success has to be smaller than the reciprocal of any polynomial function.

    And this makes a lot of sense: if the success probability for an individual guess is a reciprocal of a polynomial function, then the adversary can try a polynomial amount of guesses and succeed with high probability. If the overall success rate is $\oneover{\poly(n)}$ then we consider this attempt a feasible attack to the scheme, which makes the latter insecure.

    So, we require that the success probability must be less than the reciprocal of every polynomial function. This way, even if the adversary tries $\poly(n)$ guesses, it will not be significant since it will only have tried:
    \[
        \frac{\poly(n)}{superpoly(n)}
    \]
    As $n$ grows, the denominator grows far faster than the numerator and the success probability will not be significant.''\footnote{\linkicon \href
        {https://crypto.stackexchange.com/questions/5832/what-exactly-is-a-negligible-and-non-negligible-function}
        {\textsf{``What exactly is a negligible (and non-negligible) function?'' --- Cryptography Stack Exchange}}}
\end{quotation}


\begin{definition}
    Let $f : \nonneg \to \nonneg$ be a function. Then it is deemed \emph{polynomial}, and denoted as $f \in \poly(\lambda)$, iff:
    \[
        \exists c \in \nonneg : f(\lambda) \in O(\lambda^c) \qedhere
    \]
\end{definition}

% AP190904: Big O or small O?!?
% ^^^^^^^^: Big O, as specified here: https://en.wikipedia.org/wiki/Randomness_extractor#Randomness_extractors_in_cryptography
\begin{definition}
    Let $\nu : \nonneg \to \real$ be a function. Then it is deemed \emph{negligible}, and denoted as $\nu \in \negl(\lambda)$, iff:
    \[
        \forall f \in \poly(\lambda)  \implies \nu(\lambda) \in O\left(\oneover{f(\lambda)}\right) \qedhere
    \]
\end{definition}

Note that these actually represent upper bounds for functions: a negligible function adheres to the polynomial function definition, whereas the opposite isn't true. To sum it up: $\negl \subset \poly$.

\begin{exercise}
    Let $p(\lambda), p'(\lambda) \in \poly(\lambda)$ and $\nu(\lambda), \nu'(\lambda) \in \negl(\lambda)$. Then prove the following:

    \begin{enumerate}
        \item $p(\lambda) \cdot p'( \lambda) \in  \poly(\lambda)$
        \item \label{ex:negl} $\nu(\lambda) + \nu'(\lambda) \in \negl(\lambda)$
    \end{enumerate} 
\end{exercise}

\begin{solution}[\ref{ex:negl}]

    \todo{Questa soluzione usa disuguaglianze deboli; per essere negligibile una funzione dev'essere strettamente minore di un polinomiale inverso. Da approfondire}
    
    We need to show that for any $c \in \nonneg $, then there is $n_0$ such that $\forall n > n_{0} \implies \nu(n) + \nu'(n) < \oneover{n^c}$.
    
    Consider an arbitrary $c \in \nonneg$. Then, since $c + 1 \in \nonneg$, and both $\nu$ and $\nu'$ are negligible, there exist $n_\nu$ and $n_{\nu'}$ such that:
    
    \begin{align*}
        \forall n \geq n_{\nu} \implies& \nu(n) \leq n^{-(c+1)} \\
        \forall n \geq n_{\nu'} \implies& \nu'(n) \leq n^{-(c+1)}
    \end{align*}

    Fix $n_{0} = \max(n_{\nu}, n_{\nu'})$. Then, since $n_0 \geq 2$, $\forall n \geq n_0$ we have:
    \begin{align*}
        & \nu(n) + \nu'(n) \\
        \leq& n^{-(c+1)} + n^{-(c+1)} \\
        =& 2n^{-(c+1)} \\
        \leq& n^{-c}
    \end{align*}
    Therefore, we conclude that $\nu(n) + \nu'(n) \in \negl(\lambda)$.
\end{solution}


\section{One-way functions}

From here, we start defining an object that is fundamental to everyday cryptography: the \emph{one-way} function, or \owf{} in short. Colloquially, a one-way function is a function that is ``easy to compute'', while being ``hard to invert'' a the same time, the concept of hardness being borrowed by complexity theory.

\begin{definition}    
    Let $f : \binary^{n(\lambda)} \to \binary^{n(\lambda)}$ be a function. Then it is a \owf{} iff:
    \[
        \forall \adversary \in \ppt\, \exists \nu(\lambda) \in \negl(\lambda) : \Pr\left[\cryptog{owf}[f](\lambda) = 1 \right] \leq \nu(\lambda) \qedhere
    \]
\end{definition}


\begin{cryptogame}
    {owfdef}
    {One-Way Function hardness}
    {owf}

    \receive{\shortstack[l]{
        $x \pickUAR \binary^n$ \\
        $y = f(x)$
    }}{$y$}{}

    \cseqdelay

    \send{}{$x'$}{\textsc{Output 1 iff} $f(x') = y$}

\end{cryptogame}

The structure of the ``game'' appearing in the definition is depicted in figure \ref{cryptogame:owfdef}. Do note that the game does not check for $x = x'$, but rather for $f(x) = f(x')$; in a sense, the adversary is not trying to guess what the original $x$ was: its goal is to find any value such that its image is $y$ according to $f$, and such value may very well not be unique.

\begin{exercise} \label{ex:owf}
    Prove the following claims:
    \begin{enumerate}
        \item There exists an inefficient adversary that wins $\cryptog{owf}[f]$ with probability $1$
        \item There exists an efficient adversary that wins $\cryptog{owf}[f]$ with probability $2^{-n}$
    \end{enumerate}
\end{exercise}

\begin{solution}[\ref{ex:owf}]\
    \begin{enumerate}
        \item Adversary uses a brute-force attack.
        \item Adversary makes a random guess.
    \end{enumerate}
\end{solution}

A one-way function can be thought as a function which is very efficient in generating ``puzzles'' that are very hard to solve from scratch. Furthermore, given a candidate solution, one can efficiently verify its validity. In a twist of perspective, for a given couple $(\mathcal{P}_\textsc{gen},\mathcal{P}_\textsc{ver})$ of a puzzle generator and a puzzle verifier, another ``game'' can be drawn as in figure \ref{cryptogame:owpuzzle}.

\begin{cryptogame}
    {owpuzzle}
    {The puzzle game}
    {puzzle}

    \receive{$(x, y) \pickUAR \mathcal{P}_\textsc{gen}$}{$y$}{}

    \cseqdelay

    \send{}{$x'$}{\textsc{Output} $\mathcal{P}_\textsc{ver}(x', y)$}

\end{cryptogame}

It can also be said that the one-way puzzle problem is in \textsc{np}, because witness checking is easy, but not in \textsc{p} because finding a solution to begin with is hard.

\subsubsection{Impagliazzo's Worlds}

Suppose to have Gauss, a genius child, and his professor. The professor gives to Gauss some mathematical problems, and Gauss wants to solve them all.

Imagine now that, if using one-way functions, the problem is $f(x)$, and its solution is $x$. According to Impagliazzo, we live in one of these possible worlds:
\begin{itemize}
    \item \textit{Algorithmica}: $\textsc{P} = \textsc{NP}$, meaning all efficiently verifiable problems are also efficiently solvable. 
    
    The professor can try as hard as possible to create a hard scheme, but he won't succeed because Gauss will always be able to efficiently break it using the verification procedure to compute the solution

    \item \textit{Heuristica}: \textsc{NP} problems are hard to solve in the worst case but easy on average. 
    
    The professor, with some effort, can create a game difficult enough, but Gauss will solve it anyway; here there are some problems that the professor cannot find a solution to

    \item \textit{Pessiland}: \textsc{NP} problems are hard on average but no one-way functions exist
    
    \item \textit{Minicrypt}: One-way functions exist but public-key cryptography is impractical
    
    \item \textit{Cryptomania}: Public-key cryptography is possible: two parties can exchange secret messages over open channels
\end{itemize}
    
\section{Computational Indistinguishability}

Distribution ensembles $X = \{X_{\lambda \in \mathbb{N}}\}$ and $Y = \{Y_{\lambda \in \mathbb{N}}\}$ are distribution sequences.

\begin{definition}[\emph{Comp. indist.}]
    Let $X$ and $Y$ be two distribution sequences; they are deemed \emph{computationally indistinguishable}, written as ``$X \compindist Y$'' iff:
    \[
        \forall \adversary \in \ppt \exists \nu(\lambda) \in \negl(\lambda) : \left|\Pr[\adversary(X_\lambda) = 1] - \Pr[\adversary(Y_\lambda) = 1]\right| \leq \nu(\lambda) \qedhere
    \]
\end{definition}

In words: any \emph{efficient} adversary attempting to distinguish outputs between the two ensembles will succeed with a probability that is negligibly different than randomly guessing. Note the emphasis on ``efficient'', which makes this relationship between ensembles weaker than what would be a purely statistical one.

With the purpose of making these new concepts clearer, it is presented this mental game.

\todo{AP181129-2344: There may be room for improvement, but I like how it's worded: it puts some unusual perspective into the cryptographic game, and it could be a good thing since it closely precedes our first reduction, and the whole hybrid argument mish-mash.}

A \emph{challenger} \challenger{} chooses a value $z$ among $X_\lambda$ and $Y_\lambda$, and gives it to a \emph{distinguisher} \distinguisher{}. In turn, \distinguisher{} has to correctly guess which was the source of $z$: either $X_\lambda$ or $Y_\lambda$. 

If we let $X_\lambda$ and $Y_\lambda$ to be \emph{computationally indistinguishable}, then, fixed 1 as one of the sources, the probability that \distinguisher{} says ``1!'' when \challenger{} picks $z$ from $X_\lambda$ is \emph{not so far} from the probability that \distinguisher{} says ``1!'' when \challenger{} picks $z$ from $Y_\lambda$.

So, this means that, when this property is verified by two random variables, there isn't too much \textit{difference} between the two variables in terms of information avaliable to \distinguisher{}, otherwise the distance between the two probabilities should be much more than a negligible quantity.

\begin{lemma} \label{lem:compmall}
    Let $f$ be a function that has polynomial time-complexity. Then, for any two ensembles $X$ and $Y$:
    \[
        X \compindist Y \implies f(x) \compindist f(y) \qedhere
    \]
\end{lemma}

\begin{proof}
    This proof is by contradiction and uses a reduction. Let $X \compindist Y$ be two indistinguishable ensembles, and $f \in \ppt$ an arbitrary poly-time complex function. Assume there exists an adversary \adversary{} to the challenge of distinguishing the ensembles' images $f(X)$ from $f(Y)$ that does efficiently succeed, as shown in figure \ref{cryptogame:fdistin}. 
    
    \begin{cryptogame}
        {fdistin}
        {A distinguisher for $f$}
        {$f$-\textsc{dist}}

        % Adversary asks for the challenge. Challenger chooses evenly between X and Y, and samples a value; then it sends that value's image by f to the adversary
        \receive{\shortstack[l]{
            $z_0 \pickUAR X$ \\
            $z_1 \pickUAR Y$ \\
            $b \pickUAR \binary$
        }}{$f(z_b)$}{}

        \cseqdelay

        % Adversary attempts to guess from which distribution the value's image came from
        \send{}{$b'$}{\textsc{Output 1 iff} $b' = b$}

    \end{cryptogame}
    
    Fix this adversary to be the distinguisher $\distinguisher_f$. From here, another adversary $\adversary \in \ppt$ can use $\distinguisher_f$ to effectively distinguish the original ensembles, as depicted in figure \ref{cryptoredux:fdistin}:
    \begin{enumerate}
        \item \adversary{} asks for the original sample from the challenger
        \item \adversary{} applies $f$ on the sample
        \item \adversary{} relays the resulting image to $\distinguisher_f$
        \item $\distinguisher_f$ replies with his outcome
        \item \adversary{} relays the outcome to the challenger
    \end{enumerate}
    All of this is done in polynomial time, since all functions and machines involved in the process operate in \ppt. This contradicts the computational indistinguishablilty of $X$ and $Y$.

    \begin{cryptoredux}
        {fdistin}
        {Distinguisher reduction}
        {dist}
        {f-\textsc{dist}}
        
        % The distinguisher asks for the challenge
        \receive{\shortstack[l]{
            $z_0 \pickUAR X$ \\
            $z_1 \pickUAR Y$ \\
            $b \pickUAR \binary$
        }}{$z_b$}{}
      
        % The distinguisher applies f to z and relays the image to A
        \invoke{}{$f(z_b)$}{}
        
        % Adversary distinguishes between the two distributions
        \return{}{$b'$}{}

        \send{}{$b'$}{\textsc{Output 1 iff} $b' = b$}
        
    \end{cryptoredux}

\end{proof}


\section{Pseudo-random generators}

A deterministic function $G \in \binary^\lambda \to \binary^{\lambda + l(\lambda)}$ is called a \emph{pseudo-random generator}, or \prg{} in short, iff:

\begin{itemize}
    \item $G \in \ppt(\lambda)$
    \item $|G(s)| = \lambda + l(\lambda)$ % Well duh, we can see it by def...
    \item Given $U_n$ to be a distribution ensemble of $n$ uniform random variables:
    \[
        G(U_{\lambda}) \compindist U_{\lambda + l(\lambda)}
    \]
\end{itemize}

\begin{cryptogame}
    {prg}
    {The pseudorandom game}
    {prg}
    
    \receive{\shortstack[l]{
        $x_0 \pickUAR G(U_\lambda)$ \\
        $x_1 \pickUAR U_{\lambda+l(\lambda)}$ \\
        $b \pickUAR \binary$
    }}{$x_b$}{}

    \cseqdelay

    \send{}{$b'$}{\textsc{Output 1 iff} $b = b'$}
    
\end{cryptogame}

So, if we take $s \pickUAR U_\lambda$, the output of $G$ will be indistinguishable from a random pick from $U_{\lambda + l(\lambda)}$.

    \mychapter{5}{Lesson 5} %181010

This chapter/lesson is devoted in constructing \prg{}s. We begin by assuming to have already a \prg{} $G \in \binary^\lambda \to \binary^{\lambda + 1}$, that extends the string length by one bit, and prove that it is possible to extend such string by an indefinite amount while preserving pseudo-randomness.

\section{Stretching a \prg}

Consider this algorithm that uses $G$ to construct $G^l$, as depicted in figure \ref{fig:gpowerl}:

\begin{enumerate}
    \item Let $s_0 \pickUAR \binary^\lambda$
    \item % AP190904: Ehhhhhhhhh....
    \item $\forall i \in [l(\lambda)]$
    \begin{enumerate}
        \item let $G(s_{i-1}) = (s_i, b_i)$, where $b_i$ is the extra bit generated by a single use of $G$
    \end{enumerate}
    \item Compose $(b_{1}, b_{2}, ..., b_{{l(\lambda)}}, s_{{l(\lambda)}})$. This will be the returned string, which is $\lambda + l(\lambda)$ bits long
\end{enumerate}

\begin{figure}[h]
    \begin{tikzpicture}[node distance = 1.7cm, auto, >=latex']

        \node (a) {};
        \node (b) [box, right of=a, pin={[init]above:$b_{1}$}] {$G$};
        \node (c) [box, right of=b, pin={[init]above:$b_{2}$}] {$G$};
        \node (d) [nobox, right of=c] {$\dots$};
        \node (e) [box, right of=d, pin={[init]above:$b_{i+1}$}] {$G$};
        \node (f) [nobox, right of=e] {$\dots$};
        \node (g) [box, right of=f, pin={[init]above:$b_{l(\lambda)}$}] {$G$};
        \node (h) [right of=g] {};

        \path[->] (a) edge node {$s_{0}$} (b);
        \path[->] (b) edge node {$s_{1}$} (c);
        \path[->] (c) edge node {$s_{2}$} (d);
        \path[->] (d) edge node {$s_{i}$} (e);
        \path[->] (e) edge node {$s_{i+1}$} (f);
        \path[->] (f) edge node {$s_{l(\lambda)-1}$} (g);
        \draw[->] (g) edge node {$s_{l(\lambda)}$} (h);

    \end{tikzpicture}
    \caption{Constructing $G^{l(\lambda)}$ from $G(\lambda)$}
    \label{fig:gpowerl}
\end{figure}

To prove that this construct is a valid \prg, we will make use of a known technique for proving many other results, which relies heavily on reductions like the one employed back in the \owf{} topic, and is commonly called the \emph{``hybrid argument''}.
    
\begin{lemma}[\emph{Hybrid argument}] \label{lem:hybrid}
    Let $X$, $Y$ and $Z$ be three any distribution ensembles of the same length. The following is true:
    \[
        X \compindist Y \wedge Y \compindist Z \implies X \compindist Z
    \] 
\end{lemma}
\begin{proof}
    $\forall \distinguisher \in \ppt$, by using the triangle inequality:
    \begin{align*}
        & |\Pr[\distinguisher(X) = 1] - \Pr[\distinguisher(Z) = 1]| \\
        =\:& |\Pr[\distinguisher(X) = 1] - \Pr[\distinguisher(Y) = 1] + \Pr[\distinguisher(Y) = 1] - \Pr[\distinguisher(Z) = 1]| \\
        \leq\:& |\Pr[\distinguisher(X) = 1] - \Pr[\distinguisher(Y) = 1]| + |\Pr[\distinguisher(Y) = 1] - \Pr[\distinguisher(Z) = 1]| \\
        \leq\:& \nu(n) + \nu'(n)
    \end{align*}
    where $\nu, \nu' \in \negl(n)$. By the sum property of negligible functions, the result is still negligible, proving the lemma.
\end{proof}

In essence, the hybrid argument proves that computational indistinguishability is a transitive relationship, which enables us to design ``hybrid'' games in order to bridge differences two arbitrary ones. This property will be very useful in all future proofs, as it will be shown for the coming theorem:

\begin{theorem}
    If there exists a \prg{} $G(\lambda)$ with one bit stretch, then there exists a \prg{} $G^{l(\lambda)}$ with polynomial stretch relative to its input length:
    \begin{equation*}
        G : \binary^\lambda \to \binary^{\lambda + 1} \implies \forall l(\lambda) \in \poly(\lambda)\ \exists G^l \in \binary^\lambda \to \binary^{\lambda + l(\lambda)} \qedhere
    \end{equation*} 
\end{theorem}

% AP190904: This proof may be overcomplicated
\begin{proof}

    First off, do observe that, since both $G$ and $l$ are polynomial in $\lambda$, then so is $G^{l(\lambda)}$, because it combines $G$ $l(\lambda)$-many times. To prove that $G^{l(\lambda)}$ is indeed a \prg, we will apply the hybrid argument. The hybrids are defined as:
    \begin{itemize}
        \item $H_{\lambda}^{0} := G^{l(\lambda)}(U_{\lambda})$, which is the original construct
        \item $H_{\lambda}^{i} :=
            \begin{cases}
                b_1 , ..., b_{i} \pickUAR \{0,1\} \\
                s_{i} \leftarrow \{0,1\}^{\lambda+i} \\
                (b_{i+1}, ..., b_{l(\lambda)}, s_{l(\lambda)}) := G^{l(\lambda)-i}(s_{i})
            \end{cases}$
        \item $H_{\lambda}^{l(\lambda)} := U_{\lambda + l}$
    \end{itemize}

    Focusing on two subsequent generic hybrids, as shown in figures \ref{fig:prgi} and \ref{fig:prgiplusone}, it can be observed that the only difference between the two resides in how $b_{i + 1}$ is generated: in $H^i$ it comes from an instance of $G$, whereas in $H^{i + 1}$ is chosen at random. $H_\lambda^0$ is the starting point where all bits are pseudorandom, which coincides with the $G^{l(\lambda)}$, and $H_\lambda^{l(\lambda)}$ will generate a totally random string.

    \begin{figure}[ht]

        \begin{tikzpicture}[node distance = 1.9cm, auto, >=latex']

            \node (a) [nobox, pin={[init]above:$b_{1}$}] {};
            \node (c) [nobox, pin={[init]above:$\dots$}] [right of=a,node distance=1cm] {};
            \node (r) [box, pin={[init]above:$b_{i}$}] [right of=c,node distance=1cm] {$U_{\lambda+1}$};
            \node (d) [box, pin={[init]above:$b_{i+1}$}] [right of=r] {$G$};
            \node (e) [box, pin={[init]above:$b_{i+2}$}] [right of=d] {$G$};
            \node (f) [nobox] [right of=e] {$\dots$};
            \node (g) [box, pin={[init]above:$b_{{l(\lambda)}}$}] [right of=f] {$G$};
            \node (h) [right of=g, node distance=2cm]{};

            \path[->] (r) edge node {$s_{i}$} (d);
            \path[->] (d) edge node {$s_{i+1}$} (e);
            \path[->] (e) edge node {$s_{i+2}$} (f);
            \path[->] (f) edge node {$s_{{l(\lambda)}-1}$} (g);
            \path[->] (g) edge node {$s_{{l(\lambda)}}$} (h);

        \end{tikzpicture}

        \caption{$H_{\lambda}^{i}$}
        \label{fig:prgi}

        \begin{tikzpicture}
            \draw[line width=0.2 mm] (0,0) -- (12,0);
        \end{tikzpicture}

        \begin{tikzpicture}[node distance = 1.9cm, auto, >=latex']
            
            \node (a) [nobox, pin={[init]above:$b_{1}$}] {};
            \node (c) [nobox, pin={[init]above:$...$}] [right of=a,node distance=1cm] {};
            \node (r) [box, pin={[init]above:$b_{i}$}] [right of=c,node distance=1cm] {$U_1$};
            \node (d) [box, pin={[init]above:$b_{i+1}$}] [right of=r] {$U_{\lambda+1}$};
            \node (e) [box, pin={[init]above:$b_{i+2}$}] [right of=d] {$G$};
            \node (f) [nobox] [right of=e] {$...$};
            \node (g) [box, pin={[init]above:$b_{l(\lambda)}$}] [right of=f] {$G$};
            \node (h) [right of=g, node distance=2cm]{};

            \path[->] (d) edge node {$s_{i+1}$} (e);
            \path[->] (e) edge node {$s_{i+2}$} (f);
            \path[->] (f) edge node {$s_{{l(\lambda)}-1}$} (g);
            \path[->] (g) edge node {$s_{{l(\lambda)}}$} (h);

        \end{tikzpicture}
        \caption{$H_{\lambda}^{i+1}$}
        \label{fig:prgiplusone}
    \end{figure}

    So let's fix a step $i$ in the gradual substitution, and define the following function $f_i$:

    \begin{equation*}
        f_i(s_{i+1}, b_{i+1}) = (b_1, \dots, b_i, b_{i+1}, b_{i+2}, \dots, b_{l(\lambda)}, s_{l(\lambda)})
    \end{equation*}

    where the first i bits are chosen uniformly at random, and the remaining ones are obtained by subsequent applications of $G$. It can be observed that:

    \begin{itemize}
        \item $f_i(G(U_\lambda))$ has the exact same distribution of $H_\lambda^i$
        \item $f_i(U_{\lambda + 1})$ has the exact same distribution of $H_\lambda^{i + l}$
    \end{itemize}

    Since by \prg{} definition $G(U_\lambda) \compindist U_{\lambda + 1}$, by using the lemma \ref{lem:compmall} we can deduce that $f_i(G(U_\lambda)) \compindist f_{i}(U_{\lambda + 1})$, which in turn, by how $f$ is defined, implies $H^{i} \compindist H^{i + 1}$. This holds for an arbitrary choice of $i$, so by extension:

    \begin{equation*}
        G^{l(\lambda)}(U_\lambda) = H_{\lambda}^{0} \compindist H_{\lambda}^{1} \compindist \dots \compindist H_{\lambda}^{{l(\lambda)}} = U_{\lambda + l(\lambda)}
    \end{equation*}

    which proves that $G^l$ is indeed a \prg.
\end{proof}

\begin{proof}
    \textit{(Contradiction)}: This is an alternate proof that, instead of looking for a function $f$ to model hybrid transitioning, aims for a contradiction.
    
    Suppose $G^l$ is not a \prg; then there must be a point in the hybrid chain $H_\lambda^0 \compindist \dots \compindist H_\lambda^l$ where $H_\lambda^i \not\compindist H_\lambda^{i + 1}$. Thus there exists a distinguisher $\distinguisher^{\textsc{i-th}}$ able to tell apart $H_\lambda^i$ from $H_\lambda^{i + 1}$, as shown in figure \ref{cryptogame:prghybdist}:

    \[
        \exists i \in [0, l], \exists \distinguisher^{\textsc{i-th}} \in \ppt : |\Pr[\distinguisher^{\textsc{i-th}}(H_\lambda^i) = 1] - \Pr[\distinguisher^{\textsc{i-th}}(H_\lambda^{i + 1}) = 1]| \notin \negl(\lambda)
    \]

    \begin{cryptogame}
        {prghybdist}
        {Distinguisher for $H_{\lambda}^{i}$ and $H_{\lambda}^{i+1}$}
        {i-th}

        \receive{\shortstack[l]{
            $Z \pickUAR \{H_{\lambda}^{i}, H_{\lambda}^{i+1}\} \simeq \{0, 1\}$ \\
            $z \pickUAR Z$}}
        {$z = (b_{1}, \dots ,b_{l(\lambda)}, s_{l(\lambda)})$}{}

        \cseqdelay
        
        \send{}{$a$}{\textsc{Output 1 iff} $a \simeq Z$}

    \end{cryptogame}

    If such a distinguisher exists, it can be also used to distinguish an output of $G$ from a $\lambda+1$ uniform string by ``crafting'' a suitable bit sequence, which will distribute exactly as the hybrids in question, as shown in the reduction in figure \ref{cryptoredux:prghyb}. This contradicts the hypothesis of $f$ being a \prg, which by definition is to be indistinguishable from a truly random distribution. Therefore, $G^l$ is indeed a \prg.

    \begin{cryptoredux}
        {prghyb}
        {Reducing to a distinguisher for $G$, where $\beta = (b_1, \dots, b_{i-1})$ and $\sigma = (b_{i+1}, \dots, b_{l(\lambda)}, s_{l(\lambda))}$}
        {prg}
        {i-th}

        \receive{\shortstack[l]{
            $s \pickUAR U_\lambda$ \\
            $z_0 = G(s)$ \\
            $z_1 \pickUAR U_{\lambda+1}$ \\
            $a \pickUAR \{0, 1\}$}}
        {$z_a = (b, s)$}{}

        \cseqdelay

        \invoke{\shortstack[l]{
            $\beta \pickUAR U_{i}$ \\
            $\sigma = G^{l(\lambda)-i+1}(s)$
        }}
        {$(\beta, b, \sigma)$}{}
        
        \cseqdelay

        \return{}{$a'$}{}
        \send{}{$a'$}{\textsc{Output 1 iff} $a' = a$}
    
    \end{cryptoredux}

\end{proof}

\section{Hardcore predicates}

Now that we've seen how to reuse a one-bit stretch \prg{} in order to obtain an arbitrary length of pseudorandom bits, we turn to the problem of constructing a 1-bit stretch \prg{} itself. Let $f$ be a \owf{}, and consider the following questions:
\begin{itemize}
    \item Given an image $f(x)$, which bits of the input $x$ are hard to extract?
    \item Is it always true that, given $f$, the first bit of $f(x)$ is hard to compute for any choice of $x$?
\end{itemize}

\begin{example}
    Given an OWF $f$, then $f'(x) = x_0 || f(x)$ is a OWF.
\end{example}

Two definitions for hardcore predicates are given:

\begin{definition}
    Let $f : \binary^n \to \binary^n$ be a poly-time complex function. A poly-time complex predicate $\hc : \binary^n \to \binary $ is said to be \emph{hard-core} for $f$ iff:
    \[
        \forall \adversary \in \ppt \implies \Pr(\adversary(f(x)) = \hc(x) \knowing x \pickUAR \binary^n) \in \negl(\lambda) 
    \]
\end{definition}

\begin{cryptogame}
    {hcpreddef1}
    {The hardcore game, $f$ and $\hc$ are known}
    {hc(1)}
    \receive{$x \pickUAR \binary^n$}{$f(x)$}{}

    \send{}{$b$}{\textsc{Output 1 iff} $b = \hc(x)$}
\end{cryptogame}

\begin{definition}
    A polynomial time function $\hc : \binary^n \to \binary$ is hard-core for a function $f$ iff:
    \[
        (f(X), h(X)) \compindist (f(X), U_2)
    \]
    where $X$ is a uniform distribution ensemble over $\binary^n$, and $U_2 \sim \unifdist(\binary)$.
\end{definition}


\begin{cryptogame}{hcpreddef2}{Another hardcore game, $f$ and $\mathfrak{hc}$ are known}{hc(2)}
    \cseqdelay
    \receive{\shortstack[l]{
        $x \pickUAR \{0, 1\}^n$ \\
        $z_0 = \mathfrak{hc}(x)$ \\
        $z_1 \pickUAR \{0, 1\}$ \\
        $b \pickUAR \{0, 1\}$
    }}{$(f(x), z_b)$}{}

    \cseqdelay
    \cseqdelay

    \send{}{$b'$}{\textsc{Output 1 iff} $b' = b$}
\end{cryptogame}

% Theorem-as-exercise: The two definitions are equivalent

Having made this definition, some observations are in order: we're going to rule out a cheesy solution

\begin{claim}
    There is no \textit{universal} hardcore predicate $\mathfrak{HC}$ for all functions.
\end{claim}

\begin{proof}
    Suppose there exists such a predicate $\mathfrak{HC}$. Let $f'(x) = \mathfrak{HC}(x) || f(x)$ for a given function $f$. Then $\mathfrak{HC}$ cannot be a hardcore predicate of $f'$, because any image obtained by $f$ reveals the predicate's image itself. This contradicts the universality of $\mathfrak{HC}$.
\end{proof}

However, it is always possible to construct a hardcore predicate for a \owf{}, from another \owf:

\begin{theorem}[Goldreich-Levin, '99]
    Let $f$ be a \owf{} and consider $g(x, r) = (f(x), r)$ for $r \in \binary^n$. Then $g$ is a \owf{}, and:
    \[
        h(x,r) = \left<x, r\right> = \bigxor_{i = 1}^n x_i \xor r_i = \sum_{i = 1}^n x_i \xor r_i \mod 2
    \]
    is hardcore for $g$.
\end{theorem}

\begin{proof}
    \todo{TO BE COMPLETED (...did we actually do this? è una bella menata dimostrare questo)}
\end{proof}

\begin{exercise} \label{ex:owfcomp}
    Prove that $f \in \owf \implies g \in \owf$ (Hint: do a reduction).
\end{exercise}

\begin{solution}[\ref{ex:owfcomp}]
    Let $\distinguisher^{\textsc{g-owf}}$ be a machine that is efficient in inverting $g$, and consider the reduction shown in figure \ref{cryptoredux:owfowf}. By how $g$ is defined, $r'$ must be equal to $r$; therefore $x'$ must be a valid pre-image of $y$ in $f$. This contradicts the property of $f$ being a \owf.

    \begin{cryptoredux}
        {owfowf}
        {Efficiently inverting $f$}
        {f-owf}
        {g-owf}[2]
        
        \receive{\shortstack[l]{
            $x \pickUAR \binary^n$ \\
            $y = f(x)$
        }}{$y$}{}

        \invoke{$r \pickUAR \binary^m$}{$(y, r)$}{}

        \return{}{$(x', r')$}{}

        \send{}{$x'$}{\textsc{Output 1 iff} $f(x') = y$}

    \end{cryptoredux}

\end{solution}

\subsection{One-way permutations}

A \emph{one-way permutation}, or \owp{} in short, is defined exactly as the name itself suggests: a bijective \owf{}.

\[
    f \in \binary^n \leftrightarrow \binary^n \wedge f \in \owf \implies f \in \owp
\]

\begin{corollary}
    If $f \in \binary^n \to \binary^n$ is a \owp{} then, by the theorem of Goldreich-Levin defining $g(), h()$:
    \[
        G(s) = (g(s), h(s))
    \]
    is a \prg.
\end{corollary}

\begin{proof}
    The theorem states that if $f$ is an \owf, then so is $g$. It's trivial to prove the analogue for \owp{}s. Moreover $h$ is hardcore for $g$, tus:
    \begin{align*}
        G(U_{2n}) &\equiv (g(U_{2n}), h(U_{2n})) & \\
        &\equiv (f(U_n), U_n, h(U_{2n})) & \\
        &\compindist (f(U_{n}), U_{n}, U_{1}) & \text{(definition 1 of hardcore predicate)} \\
        &\equiv U_{2n+1} &
    \end{align*}

\end{proof}

We've been successful in constructing a 1-bit stretch \prg; from here, by using the results in the previous section, we can construct a \prg{} that returns binary strings of an arbitrary length that are also pseudo-random.


    \part{Symmetric schemes}

    \mychapter{6}{Lesson 6} %181012

\section{Computationally secure encryption}

Having a better idea of what can and can't be accomplished in the cryptographic world, by means of theorems and proofs, we can focus now on the goal of defining a cryptographic system that meets our requirements. In this lesson, we focus specifically on the secrecy-oriented schemes, thus dealing with encryption and decryption of messages.

The requirements of a ``good'' encryption scheme are collectivey called those of \emph{computationally secure encryption}: the characerizing requirement is to design a task, or routine, that is \emph{computationally hard} for an attacker to revert.
In detail: this task usually involves a secret key\footnotemark, and is accomplished in polynomial time, and any attacker who wishes to revert it has no efficient means of doing it wothout knowing such key. Other properties include:

\footnotetext{This is the case for symmetric-key schemes, though many other kinds exist: some involving ``public'' keys, some others not having any key at all}

\begin{enumerate}
    \item \label{prop:owk} \emph{one-way}ness with respect to the encryption key: given $c = \Enc(k, m)$, it should be hard to recover $k$
    \item \label{prop:owm} \emph{one-way}ness with respect to the original message: given $c = \Enc(k, m)$, it should be hard to recover $m$
    \item \label{prop:nol} In a stricter sense: no information whatsoever must ``leak'' from the message
\end{enumerate}

To start visualizing these concepts, let $\Pi = (\Enc, \Dec)$ be a secrecy scheme, and consider the game depicted in figure \ref{cryptogame:otindist}  where the adversary ``wins'' the game when the challenger outputs 1.

\begin{cryptogame}
    {otindist}
    {$\cryptog{ind}(\lambda, b)$}
    {ind}

    \send{}{$m_0, m_1 \in \M$}{}

    \receive{\shortstack[l]{
        $k \pickUAR \K$ \\
        $b \pickUAR \binary$ \\
        $c \pickUAR \Enc(k, m_b)$ }}
    {$c$}{}

    \cseqdelay

    \send{}{$b'$}{\textsc{Output 1 iff} $b = b'$}
    
\end{cryptogame}

\begin{definition}
    The scheme $\Pi$ is said to be \emph{computationally one-time secure} iff:
    \[
        \forall \adversary \in \ppt \implies \cryptog{ind}(\lambda, 0) \compindist \cryptog{ind}(\lambda, 1)\footnotemark
    \]  
    or, rephrased in probability terms:
    \[
        \forall \adversary \in \ppt \implies \lvert\Pr[\cryptog{ind}(\lambda, 0) = 1] - \Pr[\cryptog{ind}(\lambda, 1) = 1]\rvert \in \negl(\lambda) \qedhere
    \]
\end{definition}
\footnotetext{$\cryptog{ind}$ refers to the indistinguishability of the messages sent by \adversary{} during the game}

This last definition shows how such a scheme is compliant with the three properties exposed beforehand. In particular:

\begin{enumerate}
    \item \emph{It is hard to recover the key}. If not, then an adversary \adversary{} can efficiently recover the key and use it to decrypt the ciphertext, which in turn enables him to perfectly distinguish $m_{0}$ from $m_{1}$ on any instance;
    \item \emph{It is hard to recover the message}. This is analogous, and even more obvious than the preceding point. Nevertheless, this is a necessary condition for a secrecy scheme to be ``good'', and it mustn't be forgotten;
    \item \emph{No information about the message whatsoever may leak from the ciphertext}. This may seem subtler than the previous point, but it warrants caution. Observe how an adversary \adversary{}, if it has the ability to extract even a tiny bit of information of the original message from the ciphertext, then it is actually able to make an educated guess on which message was encrypted in the first place, putting him at an advantage. This leads the probabilities described in the definition to be sensibly more unbalanced than negligible, forfeiting the desired secrecy.
\end{enumerate}

By extension, we may ask ourselves what scheme may or may not be \emph{computationally two-time secure}. For instance, let $\Pi_{\xor} = (\Enc, \Dec)$ be a secrecy scheme using a \prg{} $G : \binary^\lambda \to \binary^n$, structured as follows:

\begin{itemize}
    \item $\K = \binary^\lambda$, $\M = \C= \binary^n$
    \item $\Enc(k, m) = G(k) \xor m$
    \item $\Dec(k, c) = c \xor G(k) = m$
\end{itemize}

To be two-time secure means that, even if an adversary \adversary{} gets hold of a valid plaintext-ciphertext couple $(\overline{m}, \overline{c})$, he is unable to decrypt any future ciphertexts\footnotemark, apart from the obvious $\overline{c}$. However, observe that \adversary{} is now able to extract valuable information for decrypting future ciphertexts:

\footnotetext{This example models a technique called ``Chosen Plaintext Attack'', which will be discussed in depth later}

\[
    \overline{c} = \Enc(k, \overline{m}) = G(k) \xor \overline{m} \implies \overline{c} \xor \overline{m} = G(k)
\]
so now, for any second ciphertext \adversary{} receives, he can ``mimick'' the decryption routine, and efficiently uncover the underlying plaintext. This proves that $\Pi_\xor$ is not two time-secure; nevertheless, it is still one-time secure:

\begin{theorem}
    If $G$ is a \prg, then $\Pi_\xor$ is computationally one-time secure
\end{theorem}

\begin{proof}
    This proof is another example that showcases the use of hybrid games. Recalling the one-time security definition, we need to show that:
    \[
        \forall \adversary \in \ppt \implies \cryptog{ind}[\Pi_\xor](\lambda, 0) \compindist \cryptog{ind}[\Pi_\xor](\lambda, 1)
    \]
    Consider the hybrid game in figure \ref{cryptogame:xorothybrid}, where the original encryption routine is changed to use a completely random value, instead of using $G(k)$\footnotemark. As an exercise, compare it with the original one-time secure definition in figure \ref{cryptogame:otindist}, to check that it perfectly matches.

    \footnotetext{The observant student may recognize that this modification yields exactly the ``one-time pad'' secrecy scheme discussed in lesson 1}

    \begin{cryptogame}
        {xorothybrid}
        {$\hybridg{}[\Pi_\xor](\lambda, b)$}
        {r-ind}

        \send{$m_0 \neq m_1 \in \binary^l$}{$m_0, m_1$}{}

        \receive{\shortstack[l]{
            $r \pickUAR \binary^l$ \\   % R \sim \unifdist{\binary^l}
            $b \pickUAR \binary$ \\     % B \sim \unifdist{\binary}
            $c = r \xor m_b$            % C = R \xor m_B
        }}
        {$c$}{}

        \cseqdelay

        \send{}{$b'$}{\textsc{Output 1 iff} $b' = b$}
        
    \end{cryptogame}

    The proof begins by affirming that:
    \begin{claim}
        \[
            \forall \adversary \in \ppt \implies \hybridg{}[\Pi_\xor](\lambda, 0) \equiv \hybridg{}[\Pi_\xor](\lambda, 1)
        \]
    \end{claim}

    To prove it, notice that $r$ is chosen uniformly at random, and independently of $b$. Thus, no matter how the messages $m_0$ and $m_1$ are structured, $r$ will effectively make the chosen message completely unrecognizable. In formal terms, let $B$ and $R$ be the random variables for \challenger{}'s picks of $b$ and $r$ respectively; then:
    \begin{align*} % AP190906: Totally original, need a peer review here
        & \left| \Pr(B = 0 \knowing C = c) - \Pr(B = 1 \knowing C \evaluatesto c) \right| & \\
        =& \left| \Pr(B = 0 \knowing R \xor m_B \evaluatesto c) - \Pr(B = 1 \knowing R \xor m_B \evaluatesto c) \right| & \text{(C definition)} \\
        =& \mathrlap{\frac{\left| \Pr(R \xor m_B = c \knowing B \evaluatesto 0) \Pr(B = 0) - \Pr(R \xor m_B = c \knowing B \evaluatesto 1) \Pr(B = 1) \right|}{\Pr(R \xor m_B = c)}} & \\
        & & \text{(Bayes' theorem)} \\
        =& \frac{\left| \Pr(R \xor m_0 = c) \Pr(B = 0) - \Pr(R \xor m_1 = c) \Pr(B = 1) \right|}{\Pr(R \xor m_B = c)} & \text{(Cond. collapse)} \\
        =& \frac{\left| \oneover{2^l} \half - \oneover{2^l} \half \right|}{\Pr(R \xor m_B = c)} = 0 & \\
    \end{align*}

    Having proven that \adversary's success is equivalent to straight guessing in $\hybridg{}[\Pi_\xor]$, we now relate the hybrid game to the original one, affirming that:
    
    \begin{claim}
        \[
            \forall \adversary \in \ppt,\, \forall b \in \binary \implies \hybridg{}[\Pi_\xor](\lambda, b) \compindist \cryptog{ind}[\Pi_\xor](\lambda, b) \qedhere
        \]
    \end{claim}

    The proof proceeds by reduction as depicted in figure \ref{cryptoredux:xorotprg}, by assuming the existence of a distinguisher $\textsf{D}^{\textsc{ind}}$ for $c = G(k) \xor m_{b}$ and $c = r \xor m_{b}$, and using it to break $G$'s pseudo-random generation property.

    \begin{cryptoredux}
        {xorotprg}
        {Reducing to breaking a \prg}
        {prg}
        {G-ind}[2]

        \receive{\shortstack[l]{
            $k \pickUAR \binary^\lambda$ \\
            $x_0 \pickUAR G(k)$ \\
            $x_1 \pickUAR \binary^l$ \\
            $b \pickUAR \binary$
        }}{$x_b$}{}

        \return{}{$m_0, m_1$}{}

        \invoke{\shortstack[l]{
            $\beta \pickUAR \binary$ \\
            $c = x_b \xor m_\beta$
        }}{$c$}{}

        \return{}{$\beta'$}{}

        \cseqdelay
        \send{$b' = \begin{cases}
            0 &\textsc{iff } \beta' = \beta \\
            1 &\textsc{else}
        \end{cases}$
        }{$b'$}{\textsc{Output 1 iff} $b' = b$}

    \end{cryptoredux}

    Again, notice how the games of indistinguishability and pseudo-random generation are reliably reproduced on their respective sides. This construct's value centers on how $\distinguisher^\textsc{ind}$ will perform in its own challenge:
    \begin{itemize}
        \item if $x_b$ is a random value, then $\distinguisher^\textsc{ind}$ will perform as depicted in the hybrid game, thus giving right or wrong answers at random;
        \item if b is the result of $G$, then $\distinguisher^\textsc{ind}$ has a better chance in finding the right answer by its own design;
    \end{itemize}
    From these observations, especially the second point, the adversary \adversary{} has a better chance of winning the \prg{} game by asserting that $x_b$ comes from $G$ whenever $\distinguisher^\textsc{ind}$ makes a correct guess; conversely, \adversary{} will preferrably declare that $x_b$ is truly random whenever $\distinguisher^\textsc{ind}$ fails a guess, as the probability of the latter getting fooled by a random value is sensibly greater than by a pseudo-random one.

    Either way, by the existence of $\distinguisher{}^\textsc{ind}$, \adversary{} gains an edge in efficiently recognizing $G$, which cannot happen by $G$'s definition; the claim is proven. The theorem's proof can be completed by putting the pieces together, as is usual in the hybrid argument:
    \[
        \cryptog{ind}[\Pi_\xor](\lambda, 0) \compindist
        \hybridg{}[\Pi_\xor](\lambda, 0) \equiv
        \hybridg{}[\Pi_\xor](\lambda, 1) \compindist
        \cryptog{ind}[\Pi_\xor](\lambda, 1)
    \]

    which finally states that $\cryptog{ind}[\Pi_\xor](\lambda, 0) \compindist \cryptog{ind}[\Pi_\xor](\lambda, 1)$.
\end{proof}
 

\section{Pseudorandom functions}

\textsc{Prg}s are used in practice as a stepping stone for building \emph{pseudo-random functions}, \prf{} henceforth, which are the principal construct in several cryptographic schemes. Before formally introducing what a \prf{} is, we begin instead by defining what a \emph{truly random function} is:

\begin{definition}
    A random function $R : \binary^n \to \binary^l$ is a function that, depending on what is known about its previous applications:

    \begin{itemize}
        \item if $x$ is ``fresh'' (in formal terms, $R$ has never been applied to $x$ beforehand), then a value $y$ is chosen \uar{} from $R$'s codomain, and it is permanently associated as the image of $x$ in $R$\footnote{This property is also called \emph{lazy sampling}.};

        \item if $x$ is not fresh, then $R(x)$ is directly returned instead. \qedhere
    \end{itemize}
\end{definition}

It should be noted that, if such functions are to be implemented in computers, they would occupy too much space in memory. Suppose all the possible outputs of $R$ have been generated and stored as an array in memory; then its total size in bits will be $l \cdot 2^n$:

\[  
    \color{black!50}
    \overbracket[0.5pt][5pt]{\;\color{black}\printarray[4em]{0010...}\;}^{l \text{ bits}}_1\;
    \overbracket[0.5pt][5pt]{\;\color{black}\printarray[4em]{1110...}\;}^{l \text{ bits}}_2\;
    \overbracket[0.5pt][5pt]{\;\color{black}\printarray[4em]{0011...}\;}^{l \text{ bits}}_3\;\;
    \color{black!25}
    \printarray[2em]{..., ..., ..., ...}\;\;
    \color{black!50}
    \overbracket[0.5pt][5pt]{\;\color{black}\printarray[4em]{1110...}\;}^{l \text{ bits}}_{2^n}
\]

Such a function becomes cumbersome and difficult to maintain in practice; therefore, it is desirable to find a kind of function which looks as a random function possible, but does not require to be wholly memorized, while not forgetting to maintain poly-time complexity. Pseudo-randomness comes to the rescue here:

\begin{definition}
    Let $f$ be a function, then it is deemed pseudo-random (therefore, $f$ is a \prf{}) iff it is computationally indistinguishable from a true random function.
\end{definition}

In detail, \prf{}s are actually designed as function families $f_k$\footnotemark, where $k$ is a parameter that indexes the funcitions inside the family. To model the \prf{}s' indistinguishability from random functions, let $F \in \binary^\lambda \to (\binary^{n(\lambda)} \to \binary^{l(\lambda)} )$, usually denoted simply by $f_k$, be a \prf{}, and define $\mathfrak{R}(n, l)$ to be the domain that collects the random functions from $\binary^{n(\lambda)}$ to $\binary^{l(\lambda)}$.

\footnotetext{Does this remind you of something else? If not, look back in lesson 2 and 3.}

Consider the indistinguishability game drawn in figure \ref{cryptogame:prf}; although one may thing that \prg{}s and \prf{}s aren't much different, the game tells a different story, which is best put by an introductory paragraph about \prf{}s in their Wikipedia page:

\begin{quotation}
    ``Pseudorandom functions are not to be confused with pseudorandom generators (\prg{}s). The guarantee of a \prg{} is that a single output appears random if the input was chosen at random. On the other hand, the guarantee of a \prf{} is that all its outputs appear random, regardless of how the corresponding inputs were chosen, as long as the function was drawn at random from the \prf{} family.''\footnote{\ding{226} \href
        {https://en.wikipedia.org/wiki/Pseudorandom_function_family}
        {\textsf{Pseudorandom function family --- Wikipedia}}}
\end{quotation}

\begin{cryptogame}
    {prf}
    {The \prf{} indistinguishability game}
    {prf}

    \cseqchallenger{\shortstack[l]{
        $k \pickUAR \binary^\lambda$ \\
        $R \pickUAR \mathfrak{R}(n, l)$ \\
        $b \pickUAR \binary$
    }}

    \cseqdelay
    \cseqbeginloop

    \send{}{$x$}{}

    \receive{\shortstack[l]{
        $y_0 = f_k(x)$ \\
        $y_1 = R(x)$
    }}{$y_b$}{}

    \cseqendloop
    \cseqdelay

    \send{}{$b'$}{\textsc{Output 1 iff} $b = b'$}

\end{cryptogame}

In this game, \adversary{} is allowed to make multiple queries to $\challenger^\prf$, as opposed to the \prg{} indistinguishability game where it can perform just one query before making its guess. To reiterate the \prf{} definition in terms of this game:

\begin{definition}
    A function family $F = f_k$ is a \prf{} iff: 
    \[
        \cryptog{prf}[f_k](\lambda, 0) \compindist \cryptog{prf}[f_k](\lambda, 1) \qedhere
    \]
\end{definition}

\begin{exercise}
    Prove the following statements:
    \begin{itemize}
        \item No \prg{} is secure against unbounded attackers;
        \item No \prf{} is secure against unbounded attackers.
    \end{itemize}
   
\end{exercise}


\subsection{\textsc{Ggm}-tree}

This section is dedicated to a concrete example of a \prf{} which is built from the ground up using a \prg. This construct has been designed from Oded Goldreich, Shafi Goldwasser and Silvio Micali, and its structure is akin to a binary tree, hence its name: \emph{\ggm-tree}.

\begin{construction}
    Let $G \in \binary^\lambda \to \binary^{2\lambda}$ be a \prg{} such that it doubles the length of its argument, and denote the images' first and second halves as $G_0(k)$ and $G_1(k)$ respectively, so that: 
    \[
        k \mapsto (G_0(k), G_1(k))
    \]

    Since the principal mechanism makes use of the halves being the same length of the argment, in the same spirit, we will denote the action of using one half of an image of $G$ as argument of $G$ itself in a shorter fashion, as demonstrated in the following example:
    \[
        G_a(G_b(G_c(k))) =: G_{abc}(k)
    \]

    This leads to the final step: let $f_k$ be a function family, where $k \in \binary^\lambda$, such that:
    \[
        f_k(r) = G_r(k)
    \]

    This is our candidate \prf. To visualize it, consider the tree structure depicted in figure \ref{fig:ggmtree}: at each level of the tree, a single bit of $r$ is used to decide which half of $G$'s image will be used in the next level. For example, $f_k(01 \dots 10)$ would evaluate as $G_0(G_1( \dots G_1(G_0(k))))$.

    \begin{figure}
        \centering
        \begin{tikzpicture}[
            level 1/.style={sibling distance=16em},
            level 2/.style={sibling distance=8em},
            level 3/.style={sibling distance=4em}]

            \node{$k$}
                child{ node {$G_0(k)$}
                    child { node {$G_{00}(k)$}
                        child { node (a) {$G_{000}(k)$} node [below of = a] {$\vdots$}}
                        child { node (a) {$G_{001}(k)$} node [below of = a] {$\vdots$}}
                    }
                    child { node {$G_{01}(k)$}
                        child { node (a) {$G_{010}(k)$} node [below of = a] {$\vdots$}}
                        child { node (a) {$G_{011}(k)$} node [below of = a] {$\vdots$}}
                    }
                }
                child { node {$G_1(k)$}
                    child { node {$G_{10}(k)$}
                        child { node (a) {$G_{100}(k)$} node [below of = a] {$\vdots$}}
                        child { node (a) {$G_{101}(k)$} node [below of = a] {$\vdots$}}
                    }
                    child { node {$G_{11}(k)$}
                        child { node (a) {$G_{110}(k)$} node [below of = a] {$\vdots$}}
                        child { node (a) {$G_{111}(k)$} node [below of = a] {$\vdots$}}
                    }
                };
        \end{tikzpicture}
        \caption{The \textsc{ggm}-tree for $G$}
        \label{fig:ggmtree}
    \end{figure}

\end{construction}

The proof of \prf-ness will be discussed in the next lesson.

    \mychapter{7}{Lesson 7} %181017

\subsubsection{\textsc{Ggm}-tree (cont'd)}

% Reminders: k is the scheme key
As stated in the previous lesson, given a \prg{} $G \in \binary^\lambda \to \binary^{2\lambda}$, we can build a function family $f_k$ by repeatedly taking halves of $G$'s images, and plugging them back into $G$. Our goal is to prove the following theorem:

\begin{theorem}
    If $G$ is a \prg, then $f_k$ is a \prf.
\end{theorem}

% AP190103: All notation is still left in an inconsistent state, it's difficult here to establish which things can go where...
\begin{proof}

    Before starting to prove the \prf-ness of $f_k$, we make a brief consideration about its time complexity: computing $f_k(x)$ consists in computing $G$ and taking half of the resulting image as many times as is the length of $x$. Since the length of $x$ is polynomial in $\lambda$, so is the number of $G$'s iterations; combine this with the fact that $G$ is itself polynomial by definition, and we conclude that $f_k$ is polynomial too.

    Having cleared any doubts about $f_k$'s time complexity, we now turn to the essential point of interest: its pseudo-randomness. The proof will proceed by induction over the lentgth of $x$, which is also the height of the tree-like structure modeling the algorithm.
    
    \textbf{Base case} ($n = 1$): $f_k$'s domain is restricted to $\binary$, meaning that its images will be respectively the two halves on a single iteration of $G(k)$; they are, of course, pseudorandom by $G$'s definition:
    \[
        (f_k(0), f_k(1)) = (G_0(k), G_1(k)) \compindist U_{2\lambda}
    \]
    therefore, in this case, $F_k$ is pseudorandom.

    % AP190103: Here too...
    \textbf{Inductive step}: Let $f'_k : \binary^{n - 1} \to \binary^\lambda$ be a \prf. Define $f_k$ as follows:
    \[
        f_k : \binary^n \to \binary^\lambda : (b, x) \in \binary \times \binary^{n - 1} \mapsto G_b(f'_k(x))
    \]
    It must be proven that if $f_k'$ is a \prf, then so is $f_k$. To help ourselves, we'll define some hybrid games as usual. These are depicted in figures \ref{cryptogame:ggmhyb1} and \ref{cryptogame:ggmhyb2}.

    \begin{cryptogame}
        {ggmhyb1}
        {$\hybridg{1}[f_k](\lambda, b)$: The \ggm{} construct is put against randomly driven \prg}
        {n-th}

        \cseqchallenger{\shortstack[l]{
            $k \pickUAR \binary^\lambda$ \\
            $\overline{R} \pickUAR \mathfrak{R}(n-1, \lambda)$ \\
            $b \pickUAR \binary$
        }}

        \cseqdelay
        \cseqbeginloop
        \send{}{$r = (r_0, r_{1 \upto n})$}{}
        \receive{\shortstack[l]{
            $y_0 = G_{r_0}(f'_k(r_{1 \upto n})) = f_k(r)$ \\
            $y_1 = G_{r_0}(\overline{R}(r_{1 \upto n}))$
        }}{$y_b$}{}
        \cseqendloop
        \cseqdelay

        \send{}{$b'$}{\textsc{Output 1 iff} $b = b'$}

    \end{cryptogame}

    \begin{cryptogame}
        {ggmhyb2}
        {$\hybridg{2}[\overline{R} \circ G](\lambda, b)$: The randomly driven \prg{} against a true random function}
        {n-th}

        \cseqchallenger{\shortstack[l]{
            $\overline{R} \pickUAR \mathfrak{R}(n-1, \lambda)$ \\
            $R \pickUAR \mathfrak{R}(n, \lambda)$ \\
            $b \pickUAR \binary$
        }}

        \cseqdelay
        \cseqbeginloop
        \send{}{$r = (r_0, r_{1 \upto n})$}{}
        \receive{\shortstack[l]{
            $y_0 = G_{r_0}(\overline{R}(r_{1 \upto n}))$ \\
            $y_1 = R(r)$
        }}{$y_b$}{}
        \cseqendloop
        \cseqdelay

        \send{}{$b'$}{\textsc{Output 1 iff} $b = b'$}

    \end{cryptogame}
    
    We are going to prove, in the scope of a single induction step, that these hybrids bridge indistinguishability for the original game. It is all that's needed to complete the proof.

    \begin{lemma}
        $\hybridg{1}[f_k](\lambda, 0) \compindist \hybridg{1}[f_k](\lambda, 1)$       
    \end{lemma}

    \begin{proof}
        Assume $\exists \distinguisher^{\textsc{n-th}} \in \ppt$ that can distinguish $f_k$ from $\overline{R} \circ G$ at the $n$-th step; then an adversary \adversary{} can use it to break in turn $f'_k$'s \prf-ness, and distingush it from $\overline{R} \circ G$ as shown in figure \ref{cryptoredux:ggmhyb1}:

        % AP190103: It is not immediately clear by the sequence that we're simulating two games in two different "inductive" steps, instead of trying to break one standard assumption
        % The reasoning is: distinguish the hybrids, simulate the induction steps
        \begin{cryptoredux}
            {ggmhyb1}
            {Using $\distinguisher^{\textsc{n-th}}$ to break $f'_k$}
            {f'-prf}
            {n-th}

            \cseqchallenger{\shortstack[l]{
                $k \pickUAR U_{\lambda - 1}$ \\
                $\overline{R} \pickUAR \mathfrak{R}(n - 1, \lambda)$ \\
                $b \pickUAR \binary$
            }}

            \cseqbeginloop
            \return{}{$(r_0, r_{1 \dots n})$}{}

            \send{}{$r_{1 \dots n}$}{}

            \receive{\shortstack[l]{
                $z_0 = f_k'(r_{1 \dots n})$ \\
                $z_1 = \overline{R}(r_{1 \dots n})$
            }}{$z_b$}{}

            \cseqdelay

            \invoke{\shortstack[l]{
                $y_b = G_{r_0}(z_b)$  % AP190909 - IMPORTANT : Make a note here: y is either f_k or RcircG
            }}{$y_b$}{}

            \cseqendloop

            \return{}{$b'$}{}

            \send{}{$b'$}{\textsc{Output 1 iff} $b' = b$}

        \end{cryptoredux}

    \end{proof}

    Before tackling $\hybridg{2}[\overline{R}](\lambda, b)$, it is best to introduce another lemma:

    \begin{lemma}
        If $G : \binary^\lambda \to \binary^{2 \lambda}$ is a \prg, then:
        \[
            \forall K_i \sim \unifdist(\lambda) \iid \implies (G(K_1), \dots , G(K_t)) \compindist (U_{2 \lambda}, \dots , U_{2\lambda}) \qedhere
        \]
    \end{lemma}

    \begin{proof}
        \todo{Idea: all values are independent and pseudorandom on their own, hybridize progressively...}
        % AP190909: Wait: isn't that a property of a PRG already? No, the definition states indist for a single sample, not for an arbitrary sequence; hence the idea, btw
    \end{proof}

    Now for the final lemma:

    \begin{lemma}
        $\hybridg{2}[\overline{R} \circ G](\lambda, 0) \compindist \hybridg{2}[\overline{R} \circ G](\lambda, 1)$       
    \end{lemma}
    
    %AP190912: May need to flesh out some images here, especially a cryptoredux drawing...
    \begin{proof}

        \todo{Need figures here}

        This proof is trickier: the problem lies in the inherent difference between the task of detecting a \prg{} from that of detecting a \prf{}. The aforemenioned lemma helps us with the aspect of polynomial queries on a \prg{}, bringing it in line with the \prf{} game; yet, we're far from done.

        Let's delve into the details: from \adversary's perspective, the game of distinguishing $\overline{R} \circ G$ from $R$ , where \distinguisher{} performs a number of queries $q$ polynomial in $\lambda$, is perfectly modeled by the act of distinguishing a sequence of evaluations $(G(K_1), \dots, G(K_q))$ from $(U_{2\lambda}, \dots, U_{2\lambda})$. Therefore, the game can be twisted in a way that \adversary{} receives either one of the two whole sequences beforehand, and respects the rules by sending to $D$ the right piece of information on each of its queries, in order to simulate the $\overline{R} \circ G$/$R$ game correctly.

        At this point, the simple way for A to send the right info on each query $i$ would be to check whether the query's first bit $r_0$ would be $0$ or $1$, and give back to \distinguisher{} either the first or the second half of the i-th value of the sequence he received from \challenger{}; this is exactly how \ggm{} operates. However, there's a catch: \distinguisher{} might make two queries where the r1---n parts are the same, the only difference is in the first bit: essentially, D is asking for both parts of G(r) from two distinct queries. In this case, if \adversary{} just sends the i-th value's corresponding half, the simulation would break, because the halves come from different samples.
        
        Thus, the adversary must keep track of which suffixes he has been already queried about, and make sure to send the value's other half (there are only two) whenever he is queried on a suffix for a second time. This final touch solves the problem of simulating the game for D, and now the task of A is changed to an equivalent one of breaking the lemma stated at the beginning.
        
        \todo{Prata's notes:
        
        Let T1, T2 be empty tables.

        Given query x input x1---n let xbar=x1---i

        if xbar notin T1 then x <-\$ binarylambda and add k\_xbar to T2

        if xbar in T2 let k\_xbar = T2[T1[xbar]]

        output y = G\_x\_n(g\_x\_n-1(...G\_x\_i+1(k\_xbar)))

        H\_0: GGM treee
        
        H\_n: random function

        exploit lemma prg => prf
        }
        
    \end{proof}

    In the end the hybrids are proven to mutually indistinguishable, therefore the inductive step is correct, proving the theorem.

\end{proof}

\section{\textsc{Cpa}-security}

Now it's time to define a stronger notion of security, which is widely used in cryptology for first assessments of cryptographic strength. Let $\Pi := (\Enc, \Dec)$ be a \ske{} scheme, and consider the game depicted in figure \ref{cryptogame:cpadef}. Observe that this time, the adversary can ``query'' the challenger for the ciphertexts of any messages of his choice, with the only reasonable restriction that the query amount must be polynomially bound by $\lambda$. This kind of game/attack is called the \emph{Chosen Plaintext Attack}, because of the adversary's capability of obtaining ciphertexts from messages. The usual victory conditions found in n-time security games, which are based on ciphertext distinguishability, apply.

\begin{cryptogame}
    {cpadef}
    {The \cpa-security game: $\cryptog{cpa}(\lambda, b)$}
    {cpa}

    \cseqchallenger{$k \pickUAR \binary^\lambda$}

    \cseqbeginloop
    \send{}{$m$}{}
    \receive{$c \pickUAR \Enc(k, m)$}{$c$}{}
    \cseqendloop

    \cseqdelay

    \send{}{$m_0^*, m_1^*$}{}
    \receive{\shortstack[l]{
        $b \pickUAR \binary$ \\
        $c^* \pickUAR \Enc(k, m_b^*)$
    }}{$c^*$}{}

    \cseqdelay

    \cseqbeginloop
    \send{}{$m$}{}
    \receive{$c \pickUAR \Enc(k, m)$}{$c$}{}
    \cseqendloop

    \cseqdelay

    \send{}{$b'$}{\textsc{Output 1 iff} $b' = b$}

\end{cryptogame}

\begin{definition}
    A scheme is \cpa-secure if $\cryptog{cpa}(\lambda, 0) \compindist \cryptog{cpa}(\lambda, 1)$
\end{definition}


Having given this definition of security, recall the $\Pi_\xor$ scheme defined in the previous lesson. It is easy to see that $\Pi_\xor$ is not \cpa-secure for the same reasons that it is not computationally 2-time secure; however this example sheds some new light about a deeper problem:

%AP181230: Refer to page 72 of Katz-Lindell for a good explanation
\begin{observation}
    No deterministic scheme can achieve \cpa-security.
\end{observation}

This is true, because nothing prevents the adversary from asking the challenger to encrypt either $m_0$ or $m_1$, or even both, before starting the actual challenge; just as in the 2-time case for $\Pi_\xor$, he will know the messages' ciphertexts in advance, so he will be able to tell which message the challenger has encrypted every time. The solution for obtaining a \cpa-secure encryption scheme consists of returning different ciphertexts for the same message, even better if they look random. This can be achieved by using \prf{}s.

Consider the following \ske{} scheme $\Pi_{f_k}$, where $f_k$ is a \prf{} structured as follows:

\begin{itemize}
    \item $\Enc(k, m) = (c_1, c_2) = (r, f_k(r) \xor m)$, where $k \pickUAR \binary^\lambda$ and $r \pickUAR \binary^n$
    
    \item $\Dec(k, (c_1, c_2)) = f_k(c_1) \xor c_2$
\end{itemize}

Observe that the random value $r$ is part of the ciphertext, making it long $n+l$ bits; also more importantly, the adversary can and will always see $r$. The key $k$ though, which gives a \textit{flavour} to the \prf, is still secret.

\begin{theorem} \label{thm:prfcpa}
    If $f_k$ is a \prf, then $\Pi_{f_k}$ is \cpa-secure.
\end{theorem}

\begin{proof}
    We have to prove that $\cryptog{cpa}[\Pi_{f_k}](\lambda, 0) \compindist \cryptog{cpa}[\Pi_{f_k}](\lambda, 1)$; to this end, the hybrid argument will be used. Let the first hybrid $\hybridg{0}$ be the original game, the second hybrid $\hybridg{1}$ will have a different encryption routine:

    \begin{itemize}
        \item $r \pickUAR \binary^n$
        \item $R \pickUAR \mathfrak{R}(n, l)$
        \item $c = (r, R(r) \xor m)$, where $m$ is the plaintext to be encrypted
    \end{itemize}

    and then the last hybrid $\hybridg{2}$ will simply output $(r_1, r_2) \pickUAR U_{n + l}$.

    \begin{lemma}
        $\forall b \in \binary \implies \hybridg{0}(\lambda, b) \compindist \hybridg{1}(\lambda, b)$.
    \end{lemma}

    \begin{proof}
        As usual, the proof is by reduction: suppose there exists a distinguisher \distinguisher{} capable of telling the two hybrids apart; then \distinguisher{} can be used to break $f_k$'s property of being a \prf. The way to use \distinguisher{} is to make it play a \cpa-like game, as shown in figure \ref{cryptoredux:prfcpa}\footnotemark, where the adversary attempting to break $f_k$ decides which message to encrypt between $m_0$ and $m_1$ beforehand, and checks whether it guesses which message has been encrypted. Either way, the adversary can get a sensible probability gain in guessing if the received values from the challenger were random, or generated by $f_k$. Thus, assuming such \distinguisher{} exists, \adversary{} can efficiently break $f_k$, contradictiong its \prf-ness.

        \footnotetext{An observant student may notice a striking similarity with a previously exposed reduction in figure \ref{cryptoredux:xorotprg}}

        % AP190101: Potrei includere una definizione alternativa di prf più su, in modo tale che questa riduzionie sia un po' più chiara

        \begin{cryptoredux}
            {prfcpa}
            {Breaking a \prf, for fixed message choice of $m_0$}
            {prf}
            {cpa}

            \cseqchallenger{\shortstack[l]{
                $k \pickUAR \binary^\lambda$ \\
                $R \pickUAR \mathfrak{R}(n, l)$ \\
                $b \pickUAR \binary$
            }}
        
            \cseqbeginloop
            \return{}{$m$}{}
            \send{$r \pickUAR \binary^n$}{$r$}{}
            \receive{\shortstack[l]{
                $z_0 \pickUAR f_k(r)$ \\
                $z_1 \pickUAR R(r)$
            }}{$z_b$}{}
            \invoke{$c = (r, z_b \xor m)$}{$c$}{}
            \cseqendloop

            \cseqdelay
        
            \return{}{$m_0^*, m_1^*$}{}
            \send{$r^* \pickUAR \binary^n$}{$r^*$}{}
            \receive{\shortstack[l]{
                $z_0^* \pickUAR f_k(r)$ \\
                $z_1^* \pickUAR R(r)$
            }}{$z_b^*$}{}
            \invoke{$c^* = (r^*, z_b^* \xor m_0^*)$}{$c^*$}{}

            \cseqdelay
        
            \cseqbeginloop
            \return{}{$m$}{}
            \send{$r \pickUAR \binary^n$}{$r$}{}
            \receive{\shortstack[l]{
                $z_0 \pickUAR f_k(r)$ \\
                $z_1 \pickUAR R(r)$
            }}{$z_b$}{}
            \invoke{$c = (r, z_b \xor m)$}{$c$}{}
            \cseqendloop
            
            \cseqdelay
        
            \return{}{$b'$}{}

            \cseqdelay

            \send{$b'' = \begin{cases}
                0 &\textsc{iff } b' = 0 \\
                1 &\textsc{else}
            \end{cases}$
            }{$b''$}{\textsc{Output 1 iff} $b'' = b$}
        
        \end{cryptoredux}

    \end{proof}

    \begin{lemma}
        $\forall b \in \binary \implies \hybridg{1}(\lambda, b) \compindist \hybridg{2}(\lambda, b)$.
    \end{lemma}

    \begin{proof}
        Firstly, it can be safely assumed that any ciphertext $(r_i, R(r_i) \xor m_b)$ distributes equivalently with its own sub-value $R(r_i)$, because of $R$'s true randomness, and independency from $m_b$.

        %AP190102: An image visualizing the repeat event might be very useful
        Having said that, the two hybrids apparently distribute uniformly, making them perfectly equivalent; however there is a caveat: if both games are run and one value $\overline{r}$ is queried twice in both runs, then on the second query the adversary will receive the same image in $\hybridg{1}$, but almost certainly a different one in $\hybridg{2}$. This is because the first hybrid uses a function, which is deterministic by its nature, whereas the image in the second hybrid is picked completely randomly from the codomain. Nevertheless, this sneaky issue about "collisions" can be proven to happen with negligible probability.

        Call \textsc{Repeat} this collision event on $\overline{r}$ between 2 consecutive games. Then:
        
        % AP190102: Could use some clarifications? Especially about what the indices represent
        \begin{align*}
        \Pr[\textsc{Repeat}] &= \Pr[\exists i, j \in q \text{ such that } r_i = r_j] \\
            &\leq \sum_{i \neq j} \Pr[r_i=r_j] \\
            &= Col(U_n) \\
            &= \sum_{i \neq j} \sum_{e \in \binary^n} \Pr[r_1 = r_2 = e] \\
            &= \sum_{i \neq j} \sum_{e \in \binary^n} \Pr[r = e]^{2} \\
            &= \binom{q}{2} 2^{n} \frac{1}{2^{2n}} \\
            &= \binom{q}{2} 2^{-n} \\
            &\leq q^{2}2^{-n} \in \negl{\lambda} 
        \end{align*} 


        which proves that the \textsc{Repeat} influences negligibly on the two hybrids' equivalence. Thus $\hybridg{1}(\lambda, b) \compindist \hybridg{2}(\lambda, b)$\footnotemark.
        
        \footnotetext{Do note that the hybrids lose their originally supposed perfect equivalence ($\hybridg{1}(\lambda, b) \equiv \hybridg{2}(\lambda, b)$) because of the \textsc{Repeat} event. Nevertheless, the lemma is still proven because it includes computational bounds into \adversary}
    \end{proof}

    With the above lemmas, and observing that $\hybridg{2}(\lambda, 0) \equiv \hybridg{2}(\lambda, 1)$, we can reach the conclusion that $\hybridg{0}(\lambda, 0) \compindist \hybridg{0}(\lambda, 1)$, which is what we wanted to demonstrate.

\end{proof}

    \mychapter{8}{Lesson 8} %181019

\section{Domain extension}

Up until now, encryption has been dealt with messages of fixed size around a polynomial function to $\lambda$. How to deal with messages with aritrary size? Setting a maximum bound to message length seems impractical, both for waste reasons when messages are too short, and for practicality when messages eventually get too long. The solution takes the form of a ``block-cipher'', where a message of a given size is split into equally-sized blocks, and then encrypted using a fixed-size encryption scheme. Various instances of this technique, called \emph{modes}, have been devised.

\subsection{Electronic Codebook mode}

The operation of \textsc{ecb}-mode is straightforward: Given a message split into blocks $(m_1, \dots, m_t)$, apply the scheme's encryption routine to each block, as shown in figure \ref{fig:ecb}:
\[
    c_i = F_k(r) \xor m_i \quad \forall i \in \{0, \dots, t\}
\]
Decryption is trivially implemented by \textsc{xor}-ing the ciphered blocks with $F_k(r)$.


\begin{figure}[ht]
    \centering
    \begin{tikzpicture}[node distance = 12mm, >=latex']

        \path[->]
            node (m1) {$m_1$}
            node (x1) [right of = m1] {$\xor$}
            node (c1) [below of = x1] {$c_1$}
            node (f1) [box, above of = x1] {$f_k$}
            node (r1) [above of = f1] {$r$}

            node (m2) [right of = x1] {$m_2$}
            node (x2) [right of = m2] {$\xor$}
            node (c2) [below of = x2] {$c_2$}
            node (f2) [box, above of = x2] {$f_k$}
            node (r2) [above of = f2] {$r$}

            node (xd) [right of = x2, node distance = 15mm] {$\dots$}
            node (cd) [right of = c2, node distance = 15mm] {$\dots$}
            node (fd) [right of = f2, node distance = 15mm] {$\dots$}
            node (rd) [right of = r2, node distance = 15mm] {$\dots$}

            node (mt) [right of = xd] {$m_t$}
            node (xt) [right of = mt] {$\xor$}
            node (ct) [below of = xt] {$c_t$}
            node (ft) [box, above of = xt] {$f_k$}
            node (rt) [above of = ft] {$r$}

            (r1) edge (f1)
            (r2) edge (f2)
            (rt) edge (ft)

            (f1) edge (x1)
            (f2) edge (x2)
            (ft) edge (xt)

            (m1) edge (x1)
            (m2) edge (x2)
            (mt) edge (xt)
            
            (x1) edge (c1)
            (x2) edge (c2)
            (xt) edge (ct)
        ;

    \end{tikzpicture}
    \caption{\textsc{ecb}-mode block-cipher in action, using a \prf{} as the encryption routine}
    \label{fig:ecb}
\end{figure}

This approach has the advantage of being completely parallelizable, as each block can clearly be encrypted separately; however there is a dangerous flaw in being not \cpa-secure, even when using a \prf-based encryption scheme. To understand why, observe that random nonces for ciphertext randomization are chosen per-message; this means the encryption of message blocks become deterministic in the message scope, enabling an adversary to attack the scheme within a single plaintext. It is sufficient to choose an all-0 or all-1 message to realize that all its blocks would encrypt to the same ciphered block.

\subsection{Cipher block chaining mode (\textsc{cbc})}

% AP190105: The explanation anticipates PRPs, which may be a problem. Should be reviewed at later times...
This mode serializes block encryption by using the preceding ciphered block in the formula:
\[
    c_i = P_k(r) \xor m_i \quad \forall i \in \{0, \dots, t\}
\]
This time, a \emph{pseudorandom permutation}(\prp) is used instead of a \prf; they will be discussed later on. The diagram in figure \ref{fig:cbc} shows a general view of \textsc{cbc}-mode's operation. The decryption process is analogous but in a reversed fashion, by computing the preimage of a ciphered block and \textsc{xor}-ing it with the preceding ciphered block:
\[
    m_i = P_k^{-1}(c_i) \xor c_{i-1}
\]

\begin{figure}[ht]
    \centering
    \begin{tikzpicture}[node distance = 12mm, > = latex']

        \draw[->]
            node (r) {$r$}
            node (c0) [below of = r, node distance = 24mm] {$c_0$}

            node (c1) [right of = c0, node distance = 24mm] {$c_1$}
            node (p1) [box, above of = c1] {$P_k$}
            node (x1) [above of = p1] {$\xor$}
            node (m1) [above of = x1] {$m_1$}
            
            node (c2) [right of = c1, node distance = 24mm] {$c_2$}
            node (p2) [box, above of = c2] {$P_k$}
            node (x2) [above of = p2] {$\xor$}
            node (m2) [above of = x2] {$m_2$}

            node (cd) [right of = c2, node distance = 24mm] {$\dots$}
            node (pd) [above of = cd] {$\dots$}
            node (xd) [above of = pd] {$\dots$}
            node (md) [above of = xd] {$\dots$}

            node (ct) [right of = cd, node distance = 24mm] {$c_t$}
            node (pt) [box, above of = ct] {$P_k$}
            node (xt) [above of = pt] {$\xor$}
            node (mt) [above of = xt] {$m_t$}
            
            (r) edge (c0)

            (m1) edge (x1)
            (m2) edge (x2)
            (mt) edge (xt)

            (x1) edge (p1)
            (x2) edge (p2)
            (xt) edge (pt)

            (p1) edge (c1)
            (p2) edge (c2)
            (pt) edge (ct)

            (c0) -| +(12mm, 24mm) edge (x1)
            (c1) -| +(12mm, 24mm) edge (x2)
            (c2) -| +(12mm, 24mm) edge (xd)
            (cd) -| +(12mm, 24mm) -- (xt)
        ;

    \end{tikzpicture}
    \caption{\textsc{cbc}-mode block-cipher in action, using a \prp{} as the encryption routine}
    \label{fig:cbc}
\end{figure}

\subsection{Counter mode}

Also denoted as \textsc{ctr} in short, this mode closely resembles \textsc{ecb}-mode but uses a ``rolling'' nonce instead of a static one, as shown in figure \ref{fig:ctr}. At each successive block, the nonce is incremented by 1 and then used in a single block encryption. Since the nonce is in $\binary^n$, the increment is done modulo $2^n$ so that the value will wrap around to 0 it it ever overflows. Decryption is analogous.

\begin{figure}[ht]
    \centering

    \begin{tikzpicture}[node distance = 12mm, > = latex']

        \draw[->]
            node (c0) {$c_0$}

            node (c1) [right of = c0, node distance = 24mm] {$c_1$}
            node (x1) [above of = c1] {$\xor$}
            node (m1) [left of = x1] {$m_1$}
            node (f1) [box, above of = x1] {$f_k$}
            node (r1) [above of = f1] {$r$}

            node (c2) [right of = c1, node distance = 24mm] {$c_2$}
            node (x2) [above of = c2] {$\xor$}
            node (m2) [left of = x2] {$m_2$}
            node (f2) [box, above of = x2] {$f_k$}
            node (r2) [above of = f2] {$r + 1$}

            node (cd) [right of = c2, node distance = 19mm] {$\dots$}
            node (xd) [above of = cd] {$\dots$}
            node (fd) [above of = xd] {$\dots$}
            node (rd) [above of = fd] {$\dots$}

            node (ct) [right of = cd, node distance = 24mm] {$c_t$}
            node (xt) [above of = ct] {$\xor$}
            node (mt) [left of = xt] {$m_t$}
            node (ft) [box, above of = xt] {$f_k$}
            node (rt) [above of = ft] {$r + t - 1$}
            
            (r1) edge (r2)
            (r2) edge (rd)
            (rd) edge (rt)

            (r1) edge (f1)
            (r2) edge (f2)
            (rt) edge (ft)

            (f1) edge (x1)
            (f2) edge (x2)
            (ft) edge (xt)
            
            (m1) edge (x1)
            (m2) edge (x2)
            (mt) edge (xt)

            (x1) edge (c1)
            (x2) edge (c2)
            (xt) edge (ct)

            (r1) -| (c0)
        ;

    \end{tikzpicture}
    \caption{Counter-mode block-cipher in action, using a \prf{} as the encryption routine}
    \label{fig:ctr}
\end{figure}

This apparently innocuous change to \textsc{ebc} is enough to ensure \cpa-security, at the cost of perfect parallelization.

\begin{theorem}
    Assume $f_k$ is a \prf, then the counter-mode block cipher is \cpa-secure for variable length messages\footnotemark.
\end{theorem}
\footnotetext{\emph{Variable length messages} exactly means every message $m = (m_1, \dots, m_t)$ is made of $t$ blocks, and $t$ can change from any message to a different one.}

\begin{proof}
    Figure \ref{cryptogame:ctrcpa} models a \cpa{} attack to a counter-mode block-cipher. The proof will proceed by hybrid argument starting from this game, therefore the statement to verify will be $\cryptog{cpa}[\textsc{ctr}](\lambda, 0) \compindist \cryptog{cpa}[\textsc{ctr}](\lambda, 1)$.

    \begin{cryptogame}
        {ctrcpa}
        {A chosen plaintext attack to counter-mode block-cipher}
        {cpa}
        
        \cseqbeginloop

        \send{}{$m = (m_1, \dots, m_t)$}{}
        \receive{\shortstack[l]{
            $r = c_0 \pickUAR \binary^n$ \\
            $c_i = f_k(r + i - 1) \xor m_i \quad \forall i$
        }}
        {$c = (c_0, c_1, \dots, c_t)$}{}

        \cseqendloop

        \cseqdelay
        
        \send{$|\mu_0| = |\mu_1| \in \mathcal{M}$}{$\mu_0, \mu_1$}{}
        \receive{\shortstack[l]{
            $\rho = \gamma_0 \pickUAR \binary^n$ \\
            $b \pickUAR \binary$ \\
            $(\gamma_b)_i = f_k(\rho + i - 1) \xor (\mu_b)_i \quad \forall i$
        }}
        {$\gamma_b = (\gamma_0, \gamma_1, \dots, \gamma_t)$}{}

        \cseqdelay
        
        \cseqbeginloop

        \send{}{$m = (m_1, \dots, m_t)$}{}
        \receive{\shortstack[l]{
            $r = c_0 \pickUAR \binary^n$ \\
            $c_i = f_k(r + i - 1) \xor m_i \quad \forall i$
        }}
        {$c = (c_0, c_1, \dots, c_t)$}{}

        \cseqendloop

        \cseqdelay
        
        \send{}{$b'$}{\textsc{Output 1 iff} $b' = b$}

    \end{cryptogame}

    Define the two hybrid games from the original \cpa{} game as follows:

    \begin{itemize}
        \item $\hybridg{1}[\textsc{ctr}](\lambda, b)$: A random function $R$ is chosen \uar{} from $\mathfrak{R}(n, n)$ at the beginning of the game, and is used in place of $F_k$ in all block encryptions;
        \item $\hybridg{2}[\textsc{ctr}](\lambda, b)$: The challenger will pick random values from $\binary^n$ as ciphered blocks, disregarding any encryption routine.
    \end{itemize}

    \begin{lemma}
        $\cryptog{cpa}[\textsc{ctr}](\lambda, b) \compindist \hybridg{1}[\textsc{ctr}](\lambda, b) \quad \forall b \in \binary$
    \end{lemma}

    \begin{proof} The proof is left as exercise.
        
        \emph{Hint}: Since the original game and the first hybrid are very similar, we can use a distinguisher which plays the \cpa-game; since this is a lemma, our goal in the reduction is to break the precondition contained in the theorem statement.
    \end{proof}

    \begin{lemma}
        $\hybridg{1}[\textsc{ctr}](\lambda, b) \compindist \hybridg{2}[\textsc{ctr}](\lambda, b) \quad \forall b \in \binary$
    \end{lemma}

    \begin{proof}
    
        Since $m_i$ doesn't affect the distribution of the result at all, for any $i$, if $R(r^{*})$ behaves like a true random extractor, then the two hybrids are indistinguishable in the general case ($R(r + i) \xor m_i \approx R(r + i)$). However, there is a sneaky issue: if in both games it happens that a given nonce $r_i$ is used in both one query encryption and the challenge mesage encryption at any step, the subsequent encrypted blocks will be completely random in the second hybrid, whereas in the first hybrid the function's images, albeit random, become predictable, enabling a \cpa.

        Nevertheless, it can be proved that these ``collisions'' happen with negligible probability within $\hybridg{1}[\textsc{ctr}]$. Let:
        \begin{itemize}
            \item $q$ = number of encryption queries in a game run
            \item $t_i$ = number of blocks for the $i$-th query
            \item $\tau$ = number of blocks for the challenge ciphertext
            \item \textsc{Overlap} event: $\exists i, j, \iota : r_i + j = \rho + \iota$
        \end{itemize}

        The \textsc{Overlap} event exaclty models our problematic scenario. Now it suffices to show that it occurs negligibly. For simplicity, assume the involved messages are of same length, that is $t_i = \tau =: t$. Denote with $\textsc{Overlap}_i$ to be the event that the $i$-th query overlaps the challenge sequence as specified above.

        Fix some $\rho$. One can see that $\textsc{Overlap}_i$ happens if:
        \[
            \rho - t + 1 \leq r_i \leq \rho + t - 1
        \]
        which means that $r_i$ should be chosen \emph{at least} in a way that:
        \begin{itemize}
            \item the sequence $\rho, \dots, \rho + t - 1$ comes before the sequence $r_i, \dots, r_i + t - 1$, and they overlap just for the last element $\rho + t - 1 = r_i$ or
            \item the sequence $r_i, \dots, r_i + t - 1$ comes before the sequence the sequence $\rho, \dots, \rho + t - 1$, and they overlap just for the last element $r_i + t - 1 = \rho$.
        \end{itemize}

        Then:

        \begin{align*}
            \Pr[\textsc{Overlap}_i] &= \frac{(\rho + t - 1) - (\rho - t + 1) + 1}{2^n} \\
            &= \frac{2t-1}{2^n} \\
            \Pr[\textsc{Overlap}] &\leq \sum_{i = 1}^{t} \Pr[\textsc{Overlap}_i] \\
            &\leq 2 \frac{t^2}{2^n} \in \negl{\lambda} 
        \end{align*}
            
        which proves that our collision scenario happens with negligible pobability, thus the two hybrids are indistinguishable.

    \end{proof}
    
    Having proven the indistinguishability between the hybrids, the conclusion is reached:
    \[
        \cryptog{cpa}[\textsc{ctr}](\lambda, 0) \compindist \cryptog{cpa}[\textsc{ctr}](\lambda, 1)
    \]
    
\end{proof}

    \mychapter{9}{Lesson 9} %181024

\section{Message Authentication Codes and unforgeability}

After having explored the security concerns and challenges of the \ske{} realm, it is time to turn the attention to symmetric authentication schemes, or \mac{} schemes. Recall that a \mac{} scheme is a couple $(\textit{Tag}, \textit{Verify})$, with the purpose of authenticating the message's source. In this chapter, the tagging function will be denoted as $\textit{Tag}_k$, akin to a \prf.

The desirable property that a \mac{} scheme should hold is to prevent any attacker from generating a valid couple $(m^*, \phi^*)$, even after querying a tagging oracle polynomially many times\footnotemark.
\footnotetext{Do note the similarities (and differences) between unforgeability in this setting and \cpa-security in the secrecy setting}
The act of generating a valid couple from scratch is called \emph{forging}, and the aforementioned property is defined as \emph{unforgeability against chosen-message attacks} (or \ufcma, in short); its game diagram is shown in figure \ref{cryptogame:ufcma}. Do note that $m^*$ is stated to be outside the query set $M$, expressing the ``freshness'' of the forged couple\footnotemark. In formal terms:

\footnotetext{Observe how this setup bears a striking resemblance to the \prf{} game. We're onto something...}

\begin{definition}
    A \textsc{mac} scheme $\Pi$ is \textsc{ufcma}-secure iff:
    \[
        \forall \adversary \in \ppt \implies \Pr[\cryptog{ufcma}(\lambda) = 1] \in \negl(\lambda) \qedhere
    \]
\end{definition}

\begin{cryptogame}
    {ufcma}
    {$\cryptog{ufcma}(\lambda)$}
    {ufcma}

    \cseqchallenger{$k \pickUAR \K$}

    \cseqbeginloop
    \send{}{$m \in M$}{}
    \receive{$\phi = \Tag(k, m)$}{$\phi$}{}
    \cseqendloop
    
    \cseqdelay

    \send{$m^* \notin M$}{$(m^*, \phi^*)$}{\textsc{Output 1 iff} $\Ver(k, m^*, \phi^*) = 1$}
    
\end{cryptogame}

Having defined a good notion of security in the \mac{} scheme domain, we turn our attention to a somewhat trivial scheme, and find out that it is indeed \ufcma-secure:

\begin{theorem}
    Let $f_k$ be a \prf, and define the following \mac{} scheme: 
    \[
        \Pi = (\Tag, \Ver) := (f_k(m), [f_k(m) = \phi])
    \]
    Then $\Pi$ is \ufcma-secure.
\end{theorem}

\begin{proof}
    The usual proof by hybridization to random functions entails. The original game is identical to the \ufcma{} game, where the tagging function is the \prf, whereas the hybrid game will have it replaced with a truly random function, as shown in figure \ref{cryptogame:prfufcmahyb}.

    \begin{cryptogame}
        {prfufcmahyb}
        {$\hybridg{1}(\lambda)$}
        {}

        \cseqchallenger{$R \pickUAR \mathfrak{R}(\lambda, n, l)$}

        \cseqbeginloop
        \send{}{$m_i \in M$}{}
        \receive{$\phi_i = R(m_i)$}{$\phi_i$}{}
        \cseqendloop

        \cseqdelay

        \send{$m^* \notin M$}{$(m^*, \phi^*)$}{\textsc{Output 1 iff} $\Tag(k, m^*) = \phi^*$}

    \end{cryptogame}

    \begin{lemma}
        $\cryptog{ufcma}(\lambda) \compindist \hybridg{1}(\lambda)$  
    \end{lemma}

    \begin{proof}
        By assuming there is a distinguisher $\distinguisher^{\ufcma}$ capable of disproving the lemma, it can be used to distinguish the \prf{} itself, as depicted by the reduction in figure \ref{cryptoredux:prfufcma}
    \end{proof}

    % AP190107: Not so pretty in the last message...
    \begin{cryptoredux}
        {prfufcma}
        {Distinguishing a \prf{} by using $\distinguisher^{\ufcma}$}
        {prf}
        {ufcma}[1.94]

        \cseqchallenger{\shortstack[l]{
            $k \pickUAR \binary^\lambda$ \\
            $R \pickUAR \mathfrak{R}(n, l)$ \\
            $b \pickUAR \binary$
        }}
    
        \cseqbeginloop
        \return{}{$m \in M$}{}
        \send{}{$m$}{}
        \receive{\shortstack[l]{
            $\phi_0 \pickUAR f_k(r)$ \\
            $\phi_1 \pickUAR R(r)$
        }}{$\phi_b$}{}
        \invoke{}{$\phi_b$}{}
        \cseqendloop

        \cseqdelay
    
        \return{$m^* \notin M$}{$(m^*, \phi^*)$}{}
        \send{}{$m^*$}{}
        \receive{\shortstack[l]{
            $\varphi_0 \pickUAR f_k(r)$ \\
            $\varphi_1 \pickUAR R(r)$
        }}{$\varphi_b$}{}

        \cseqdelay

        \send{$b' = \begin{cases}
            0 &\textsc{iff } \varphi_b = \phi^* \\
            1 &\textsc{else}
        \end{cases}$
        }{$b'$}{\textsc{Output 1 iff} $b' = b$}
        
    \end{cryptoredux}

    \begin{lemma}
        $\forall \adversary \in \ppt \implies \Pr[\hybridg{1}(\lambda) = 1] \leq 2^{-l}$
    \end{lemma}

    \begin{proof}
        This is true because attacker has to predict the output $R(m^*)$ on a fresh input $m^*$ to win the game, which can happen at most with probability $2^{-l}$.
    \end{proof}

    Thus, the conclusion is that $\Pi$ is \ufcma-secure.
\end{proof}

% AP190107: Don't want the next section to creep in before the above figure, so I'm breaking the page. THe best solution would be to put some more text into the previous proof...

\section{Domain extension for \mac{} schemes}

The previous scheme works on fixed length messages; as in the encryption domain, there are techniques for tagging variable length messages which are \ufcma-secure. However, before showing them, some other apparently secure modes are described here to give some possible insights on how to tackle the problem. 

Assume the message $m = (m_1, \dots, m_t) \in \binary^{n \cdot t}$ for some $t \geq 1$. Given the tagging function $\Tag_k \in \binary^n \to \binary^l$, an attempt to tag the whole message may be to:

\begin{itemize}    
    \item \textsc{xor} all the message blocks, and then tag: $\phi = \Tag(k, \bigoplus_{i = 1}^{t} m_i)$. This approach opens to some easy forgeries: given an authenticated message $(m, \phi)$, an adversary can create a valid couple $(m', \phi)$, where $m'$ is the original message with two flipped bits in two distinct blocks at the same offset; the resulting \textsc{xor} would be the same.

    \item define the tag to be a $t$-sequence of tags, one for each message block. Again, there is an easy way to forge authenticated messages: an adversary can just flip the position of two arbitrary distinct message blocks along with their relative tags, resulting in a different, yet authentic message.

    \item attempt a variant of the above approach, by adding the block number to the block itself to avoid the previous forging. Yet again, this is not \ufcma-secure: the adversary may just make two queries on two distinct messages, obtain the two tag sequences, and then forge an authenticated message by choosing at each position $i$ whether to pick the message-tag blocks from the first or second query. Possibilities are endless...

\end{itemize}

\subsection{Universal hash functions}

A devised solution which has been proven to be secure relies on the following definition: a function family $H$ which can be used to \emph{shrink} variable length messages and then composed with a \prf{} $f_k$:
\[
   H \in \binary^\lambda \to (\binary^{n \cdot t} \to \binary^n) : s \mapsto h_s
\]
\[
    \Tag((k, s), m) = f_k(h_s(m))
\]

So what are the properties of the such a function family $H \circ F$? The main problem are \emph{collisions}, since for each $m \in \binary^{n \cdot t}$ it should be hard to find $m' \neq m$ such that $h_s(m) = h_s(m')$. But collisions do exist for functions in $H \circ F$, because they map elements from $\binary^{n \cdot t} $ to $\binary^t$, and since the codomain is smaller than the domain, the functions cannot be injective in any way.

To overcome this problem, we can consider two options:
\begin{itemize}
    \item assume collisions are hard to find, even when $s \in \binary^{\lambda}$ is known; in this case we have a family of \textit{collision-resistant hash functions};
    \item let $s$ be secret, and assume collisions are hard to find because it is hard to know how $h_s$ works.
\end{itemize}

\begin{definition}
    A family of hash functions $h_s$ is deemed \emph{$\varepsilon$-universal} iff:
    \[
        \forall x \neq x' \in \binary^{n \cdot t}, S \sim \unifdist(\binary^\lambda) \implies \Pr[h_S(x) = h_S(x')] \leq \varepsilon \qedhere
    \]
\end{definition}

If $\varepsilon = 2^{-n}$, meaning the collision probability is minimized, then the family is also called \emph{perfectly universal}; in the case where $\varepsilon \in \negl(\lambda)$ instead, it is defined as \emph{almost universal} (\textsc{au}). Care should be taken in telling the difference between universality and pairwise independence, which states:
\[
    (h_S(x), h_S(x')) \equiv U_{2n}
\]

\begin{lemma}
    Any pairwise independent hash function is perfectly universal.
\end{lemma}

\begin{proof} The proof is left as exercise.
    \todo{
    % AP190108: Leaving this part as-is
    (should I use $Col$ for solving this? What is the difference and when should I use $Col$ instead of one-shot-probability?)

    \textbf{ASK FOR SOLVING PROPERLY}

    (Thoughts: when I ask \textit{what's the probability that, chosen 2 distinct x-es, their hashes are the same \textbf{on a certain value}?}, maybe I have to use one-shot, because one-shot refers to the prob. that the two inputs collide on a specific value, even if not specified.

    Instead, if I consider \textit{what's the prob. that , chosen 2 distinct x-es, their hashes are the same?}, maybe I have to calculate all the possible collisions, because I want to know if the 2 inputs can collide in general. )
    }
\end{proof}

\begin{theorem}
    Assuming $f_k$ is a \prf{} with $n$-bit domain and $h_s$ is an \textsc{au} hash function family, then $H \circ F$ is a \prf{} on $(n \cdot t)$-bit domain, for $t \geq 1$.
\end{theorem}

\begin{proof}

    This proof too will proceed by hybridizing the original game up to the ideal random one. Consider the three sequences depicted in figures \ref{cryptogame:prfaureal}, \ref{cryptogame:prfauhyb} and \ref{cryptogame:prfaurand}:

    \begin{cryptogame}
        {prfaureal}
        {$Real_{F, \adversary}(\lambda)$}
        {}

        \cseqchallenger{\shortstack[l]{
            $k \pickUAR \binary^\lambda$ \\
            $s \pickUAR \binary^\lambda$
        }}

        \cseqbeginloop
        \send{}{$x$}{}
        \receive{$y = f_k(h_s(x))$}{$y$}{}
        \cseqendloop
        
    \end{cryptogame}

    \begin{cryptogame}
        {prfauhyb}
        {$\hybridg{}[R](\lambda)$}
        {}

        \cseqchallenger{\shortstack[l]{
            $\overline{R} \pickUAR \mathfrak{R}(n, l)$ \\
            $s \pickUAR \binary^\lambda$
        }}

        \cseqbeginloop
        \send{}{$x$}{}
        \receive{$y = \overline{R}(h_s(x))$}{$y$}{}
        \cseqendloop
        
    \end{cryptogame}

    \begin{cryptogame}
        {prfaurand}
        {$Rand_{R', \adversary}(\lambda)$}
        {}

        \cseqchallenger{$R \pickUAR \mathfrak{R}(n, l)$}

        \cseqbeginloop
        \send{}{$x$}{}
        \receive{$y = R(x)$}{$y$}{}
        \cseqendloop
        
    \end{cryptogame}


    \begin{lemma}
        \[
            Real_{F, \adversary}(\lambda) \compindist \hybridg{}[R](\lambda)
        \]
    \end{lemma}

    \begin{proof}
        The proof is left as exercise.
    \end{proof}

    \begin{lemma}
        \[
            \hybridg{}[R](\lambda) \compindist Rand_{R', \adversary}(\lambda)
        \]
    \end{lemma}

    \begin{proof}

        Again, collisions come into play there, but there are multiple cases that complicate things. Given two queries with arguments $x_1, x_2$ returning the same image $y$, the random game can model two scenarios:

        \begin{itemize}
            \item the arguments are equal, but with negligible probability
            \item the arguments are distinct
        \end{itemize}

        while the hybrid can model three of them:

        \begin{itemize}
            \item the arguments are equal, again with negligible probability
            \item the arguments are distinct, \emph{and so are their hashes}
            \item the arguments are distinct, \emph{but not their hashes}
        \end{itemize}
    
        We want to show that the collision at hash level is negligible: as long as they don't happen, the random function $\overline{R}$ is run over a sequence of distinct points, and behaves just as the random game's function $R$ does. So let \textsc{bad} be the event:
        \[
            \exists i \neq j \in [q] : h_s(x_i) = h_s(x_j)
        \]
        where $q$ denotes the adversary's query count. It suffices to show that $\Pr[\textsc{Bad}] \in \negl(\lambda)$.

        % AP190108: Copied verbatim from my notes, maybe this is something Venturi wrote. Doesn't mean I fully understand it...
        Since we don't care what happens after a collision, we can alternatively consider a mental experiment where we answer all queries at random, and only at the end sample $s \pickUAR \binary^\lambda$ and check ofr collisions: this does not change the value of $\Pr(\textsc{Bad})$. Now queries are independent of $s$, and this eases our proof:
        \begin{align*}
            \Pr[\textsc{Bad}] &= \Pr[\exists x_i \neq x_j, h_S(x_i) = h_S(x_j)] &                               \\
            &\leq \sum_{i \neq j} \Pr[h_S(x_i) = h_S(x_j)]                      & \text{$h_s$ is \textsc{au})}  \\
            &\leq \binom{q}{2} \negl(\lambda) \in \negl(\lambda)                &
        \end{align*}

        By ruling out this event, the lemma is proven.
    \end{proof}

     So now we have $Real \compindist \hybridg{} \compindist Rand$
\end{proof}

\begin{corollary}
    Let $\Pi = (\Tag, \Ver)$ be a variable length message \mac{} scheme where, given a \prf{} $f_k$ and an \textsc{au} hash function family $h_s$, the tagging function is defined as $f_k(h_s)$. Then $\Pi$ is \ufcma-secure.
\end{corollary}

\subsection{Hash function families from finite fields}

% AP190110: A "finite field" aka galois field, is a field with finite elements. (...)
%       - Finite fields exist only for carrier cardinality of q^k, where q is prime and k is nonnegative
%       - Both operations are done modulo q

A generic $2^n$-order finite field has very useful properties: adding two of its elements is equal to \textsc{xor}-ing their binary representations, while multiplying them is done modulo $2^n$. It is possible to define a hash function family that makes good use of these properties, and is suitable for a \ufcma-secure \mac{} scheme.

\begin{construction}
    Let $\mathbb{F} = GF(2^n)$ be a \textit{finite field} (or ``Galois field'') of $2^n$ elements, and let $m = (m_1, \dots, m_t) \in \mathbb{F}^t$ and $s = (s_1, \dots, s_t) \in \mathbb{F}^t$. The desired hash function family will have this form:
    \[
        h_{s}(m)= \sum_{i = 1}^{t} s_i m_i = \left<s, m\right> = q_m(s)
    \]
\end{construction}

\begin{lemma}
    The above function family $h_s$ is almost universal.
\end{lemma}

\begin{proof}

    In order for $h_s$ to be almost universal, collisions must happen negligibly. Suppose we have a collision with two distinct messages $m$ and $m'$:
    \[
        \sum_{i = 1}^t m_i s_i = \sum_{i = 1}^t m'_i s_i
    \]

    Let $\delta_i = m_i - m'_i$ and assume, without loss of generality, that $\delta \neq 0$. Then, by using the previous equation, when a collision happens:
    \[
        0 = \sum_{i = 1}^t m_i s_i - \sum_{i = 1}^t m'_i s_i = \sum_{i = 1}^t \delta_i s_i
    \]

    Since the messages are different from each other, there is at least some $i$-th block that contains some of the differences. Assume, without loss of generality, that some of the differences are contained in the first block ($i = 1$); the sum can then be split between the first block itself $\delta_1 s_1$ and the rest:

    \begin{align*}
        \sum_{i = 1}^t \delta_i s_i                     &= 0                                                \\
        \delta_1 s_1 + \sum_{i = 2}^{t} \delta_i s_i    &= 0                                                \\
        \delta_ 1 s_1                                   &= -\sum_{i = 2}^{t} \delta_i s_i                   \\
        s_1                                             &= \frac{-\sum_{i = 2}^{t} \delta_i s_i}{\delta_1}  \\
    \end{align*}

    which means when a collision happens, $s_1$ must be exactly equal to the sum of the other blocks, which is another element of $\mathbb{F}$. But since every seed is chosen at random among $\mathbb{F}$, the probability of picking the element $s_1$ satisfying the above equation is just $\left|\mathbb{F}\right|^{-1} = 2^{-n} \in \negl(\lambda)$. By repeating this reasoning for every difference-block, a sum of negligible probabilities is obtained, which is in turn negligible; therefore the hash function family $h_s$ is almost universal.

\end{proof}

% AP190111: Leaving this section as-is; will come back to it at a later time
\subsubsection{$H$ with Galois fields elements and polynomials}

\begin{exercise}
    Let $\mathbb{F} = GF(2^n)$ be a finite field of $2^n$ elements, and let $m = (m_1, \dots, m_t) \in \mathbb{F}^t$ and $s \pickUAR \mathbb{F}^t$. Construct the following hash function family:
    \[
        h_{s}(m) = \sum_{i=1}^{t}s^{i-1}m_{i}
    \]

    Prove that this construction is \textsc{au}.
\end{exercise}

\begin{solution}
    (possible proof: to be almost universal, looking at the definition, collisions with $m \not= m'$ must be negligible.

    So consider a collision as above: it must be true that  
    \begin{align*}
        \sum_{i=1}^{t} m_{i}s^{i-1}                                 &= \sum_{i=1}^{t} m'_{i}s^{i-1} \\ 
        \sum_{i=1}^{t} m_{i}s^{i-1}-\sum_{i=1}^{t} m'_{i}s^{i-1}    &= 0                            \\
        q_{m-m'}(s)                                                 &= 0                            \\
    \end{align*}
    How can we make a polynomial equal to 0? We have to find the \textbf{roots} of the polynomial, which we know are at most the polynomial's rank. The rank is $t - 1$, and the probability of picking a root from $\mathbb{F}$ as seed of $h_s$ is:
    \[
        \Pr[s=root] = \frac{t - 1}{2^{n}} \in \negl(\lambda)
    \]
\end{solution}
 
    \mychapter{10}{Lesson 10} %181026

\section{Domain extension for \prf-based authentication schemes}

\subsection{Hash function families from \prf{}s}

Another way to obtain domain extension for a \mac{} scheme, using the $H \circ F$ approach, is to construct the hash function family from another \prf{}. We expect to have:
\begin{itemize}
    \item $\forall \adversary \in \ppt, S \sim \unifdist(\binary^\lambda) \implies \Pr[h_S(m) = h_S(m')] \in \negl(\lambda)$;
    \item We need two \prf{}s: one is $f_k$, and the other is $f_s$
\end{itemize}

\subsection{\textsc{xor}-mode}

Let $f_s$ be a \prf, and assume that we have this function
\[
    h_s(m) = f_s(m_1||1) \xor ... \xor f_s(m_t||t)
\]
so that the input to $f_s$ is $n + log_2(t)$ bytes long.

\begin{lemma}
    Above $H$ is computational \textsc{AU}.
\end{lemma}

\begin{proof} The proof is left as exercise.

    (Hint: The pseudorandom functions can be defined as $f_s = f_k'(0, \dots)$ and $f_k = f_k'(1 \dots)$).
    
    Possible proof: we have to show that
    \[
        \forall m \neq m' \implies \Pr[h_s(m) = h_s(m')] \in \negl(\lambda)   
    \]
    This means that:
    \begin{align*}
        & \Pr\left[\left(\bigxor_{i = 1}^{t} f_s(m_i, i)\right) = \left(\bigxor_{i = 1}^{t} f_s(m_i', i)\right)\right] \\
        =& \Pr[\forall i \quad f_s(m_i, i) \xor f_s(m'_i, i) = \bigxor_{j = 1, j \neq i}^{t} f_s(m_j, j) \xor f_s(m'_j, j) = \alpha]
    \end{align*}
    for each $i \in [1,t]$. But $\alpha$ is one unique random number chosen over $2^{n}$ possible candidates, so the collision probability is negligible.
    
\end{proof}

\subsection{\textsc{cbc}-mode \mac{} scheme}
This mode is used in practice by the \textsc{tls} standard. It's used with a \prf{} $f_s$, setting the starting vector as $\textsc{iv} = 0^n = c_0$ and running this \prf{} as part of \textsc{cbc}. The last block otained by the whole process is the message's signature:
\[
    h_s(m) = f_s(m_t \xor f_s(m_{t-1} \xor f_s(\dots f_s(m_2 \xor f_s(m_1 \xor \textsc{iv})) \dots)))
\]
\begin{lemma}
    \textsc{cbc} \mac{} defines completely an \textsc{au} family.
\end{lemma}

\begin{proof}
    (not proven)
\end{proof}

We can use this function to create an \emph{encrypted \textsc{cbc}}, or \textsc{e-cbc}:
\[
    \textsc{e-cbc}_{K, S}(m) = f_{k}(h_s^{\textsc{cbc}}(m))
\]

\begin{theorem}
    Actually if $f_k$ is a \prf, \textsc{cbc} \mac{} is already a \mac{} scheme with domain $n \cdot t$ for arbitrarily fixed $t \in \nonneg$.
\end{theorem}

\begin{proof}
    (not proven)
\end{proof}

\subsection{\textsc{xor} \mac}
Instead of $H \circ F$, now the $\Tag$ routine outputs $\phi = (\eta , f_k(\eta) \xor h_s(m))$ where $\eta \pickUAR \binary^n$ is random and it's called \textit{nonce}. Authentication is done as:
\[
    (m, (\eta, f_k(\eta) \xor h_s(m)))
\]

When I want to verify a message and I get the couple $(m, (\eta, v))$, I just check that $v = f_k(\eta) \xor h_s(m)$. It should be hard to find a value called $a$ such that, given $m \neq m'$, 
\[
    h_s(m) \xor a = h_s(m')
\]

In fact, since an adversary who wants to break this scheme has to send a valid couple $(m^*, \phi^*)$ after some queries, he could:
\begin{itemize}
    \item ask for message $m$ and store the tag $(\eta, f_k(\eta) \xor h_s(m))$
    \item try to find $a = h_s(m) \xor h_s(m')$ and modify the previous stored tag adding $v \xor a$, 
\end{itemize}
so now he could send the authenticated message
\[
    (m', (\eta, f_k(\eta) \xor h_s(m'))) 
\]
which is a valid message.

\todo{AXU property definition is missing, rest of the section is left as-is}

\begin{lemma}
    XOR mode gives computational AXU (Almost Xor Universal)
\end{lemma}

\begin{proof}
    (not proven)
\end{proof}

\begin{theorem}
    If $F$ is a \prf{} and $H$ is computational AXU, then XOR-MAC is a MAC.
\end{theorem}

\begin{proof}
    (not proven)
\end{proof}

\subsubsection{Summary}

\todo{not sure what to do with this bullet list...}

With variable input lenght:
\begin{itemize}
    \item AXU based XOR mode is secure;
    \item $H \circ F$ is insecure with polynomial construction $h_s(m) = q_m(s)$, but can be fixed;
    \item CBC-MAC is not secure, left as exercise;
    \item E-CBC is secure.
\end{itemize}

% AP190111: The cca figure is too big and too important to let it slide through a whole page, so I'm breaking here to ensure it stays with the section title
\pagebreak

\section{\textsc{Cca}-security}

Going back to the encryption realm, a new definition of attack to a \ske{} scheme will be introduced. Now the adversary can query a decryption oracle, along with the \cpa-related encryption oracle, for polynomially many queries. This attack is called the \emph{Chosen Ciphertext Attack}\footnotemark, and schemes that are proven to be \cca-secure are also defined as \emph{non-malleable}, on the reasoning that an attacker cannot craft fresh valid ciphertexts from other valid ones.

\footnotetext{Different versions of the \cca{} notion exist. The one defined here is also called \cca2, or \emph{adaptive Chosen Ciphertext Attack}}

% AP190910: CAUTION - The encryption routine is intrisically rnadomized: the returned ciphertexts are composed of a nonce, and the actual encrypted message

\begin{cryptogame}
    {cca}
    {The chosen ciphertext attack, on top of \cpa}
    {cca}

    \cseqchallenger{$k \pickUAR \K$}

    %cpa
    \cseqbeginloop
    \send{}{$m$}{}
    \receive{$c \pickUAR \Enc(k, m)$}{$c$}{}
    \cseqendloop

    \cseqdelay

    %cca
    \cseqbeginloop
    \send{}{$c$}{}
    \receive{$m = \Dec_k(c)$}{$m$}{}
    \cseqendloop

    \cseqdelay

    %challenge
    \send{}{$m_0^*, m_1^*$}{}
    \receive{\shortstack[l]{
        $r^* \text{ chosen \uar}$ \\
        $b \pickUAR \binary$ \\
        $c^* \pickUAR \Enc(k, m_b^*)$
    }}{$c_b^*$}{}

    \cseqdelay
    
    %cpa
    \cseqbeginloop
    \send{}{$m$}{}
    \receive{$c \pickUAR \Enc(k, m)$}{$c$}{}
    \cseqendloop

    \cseqdelay

    %cca
    \cseqbeginloop
    \send{$c \neq c^*$}{$c$}{}
    \receive{$m = \Dec_k(c)$}{$m$}{}
    \cseqendloop

    \cseqdelay

    %output
    \send{}{$b'$}{\textsc{Output 1 iff} $b' = b$}
\end{cryptogame}

\begin{exercise}
    Show that the scheme $\Pi_{F}$ defined in theorem \ref{thm:prfcpa}, while \cpa-secure, is not \cca-secure.
\end{exercise}

\begin{proof}

    Let $m_0$ and $m_1$ be the messages the adversary sends to the challenger as the challenge plaintexts; on receiving the ciphertext $c_b = (r, f_k(r \xor m_b) : b \pickUAR \binary$, the adversary crafts another ciphertext with an arbitrary value $\alpha$:
    \[
        \widehat{c_b} = (r, f_k(r) \xor m_b \xor \alpha)
    \]
    and queries the decryption oracle on it. The latter will decrypt the new ciphertext and return a plaintext, which can be easily manipulated by the adversary to reveal exactly which message was encrypted during the challenge:

    \begin{align*}
        \Dec(k, (\widehat{c_b}) &= \Dec(k, (r, f_k(r) \xor m_b \xor \alpha))    \\
                                &= f_k(r) \xor f_k(r) \xor m_b \xor \alpha      \\
                                &= m_b \xor \alpha                              \\
    \end{align*}

    Therefore, the adversary certainly wins after just one decryption query, proving the scheme's vulnerability to \cca{} attacks.
   
\end{proof}

\section{Authenticated encryption}

Instead of tackling the \cca-security problem upfront, it might be useful to consider a construction that achieves both secrecy and authentication: that is, a scheme that encrypts messages and authenticates their respective senders at the same time. In the encryption setting, such a scheme is defined as being \cpa-secure with an additional \textsc{auth} property denoting the scheme's resistance to forgeries, much like its \mac{} cousins. The game shown in figure \ref{cryptogame:authske} models this \textsc{auth} property of a scheme $\Pi = (\Enc, \Dec)$, with an additional quirk to the decryption routine:
\[
    \Dec \in \K \times \C \to M \cup \{\perp\}
\]
where the $\perp$ value is returned whenever the decryption algorithm is supplied an invalid or malformed ciphertext.

\begin{cryptogame}
    {authske}
    {$\cryptog{auth}(\lambda)$}
    {auth}

    \cseqchallenger{$k \pickUAR \K$}

    \cseqbeginloop
    \send{}{$m$}{}
    \receive{$c \pickUAR \Enc(k, m)$}{$c$}{$c \in C$}
    \cseqendloop

    \cseqdelay

    \send{$c' \notin C$}{$c'$}{\textsc{Output 1 iff} $\Dec(k, c') \neq \perp$}

\end{cryptogame} 

\begin{theorem}
    Let $\Pi$ be a \ske{} scheme. If it is \cpa-secure, and has the \textsc{auth} property, then it is also \cca-secure.
\end{theorem}

\begin{proof} The proof is left as an exercise.

    \emph{Hint}: consider the experiment where $\Dec(k, c)$:

    \begin{itemize}
        \item if $c$ is not fresh (i.e. output of previous encryption query $m$),  then output $m$
        \item else output $\perp$
    \end{itemize}
    
    The approach would be to reduce \cca{} to \cpa; given $\distinguisher^{\cca}$, we can build $\distinguisher^{\cpa}$. $\distinguisher^{\cca}$ will ask decryption queries, but $\distinguisher^{\cpa}$ can answer just with these two properties shown above, so it can reply just if he asked these $(c, m)$ before to its challenger $\challenger$.

\end{proof}

\subsection{Combining \ske{} \& \mac{} schemes}

Let $\Pi_1 = (\Enc, \Dec)$ be a \ske{} scheme, and $\Pi_2 = (\Tag, \Ver)$ be a \mac{} scheme; there are 3 ways to combine them into an authenticated encryption scheme:
% AP190112: A note should be made about notation, for what it means to "pick at random" from an encryption/tagging function (random nonce in the encryption, seeded AU hash function in the tagging)
% Plus, keys aren't curried, but subsubscripting may turn ugly...
% AP190910: Keys are part of the signature, nonces are a thing to rethink yet. For now, a pickUAR predicate implies the function is itself randomized, because there are no distributions on the right (Farfetched...)
\begin{itemize}
    \item \emph{Encrypt-and-Tag}:
    \begin{enumerate}
        \item $c \pickUAR \Enc(k_1, m)$
        \item $\phi \pickUAR \Tag(k_2, m)$
        \item $c^* = (c, \phi)$
    \end{enumerate}
        
    \item \emph{Tag-then-encrypt}:
    \begin{enumerate}
        \item $\phi \pickUAR \Tag(k_2, m)$
        \item $c \pickUAR \Enc(k_1, (\phi, m))$
        \item $c^* = c$
    \end{enumerate}

    \item \emph{Encrypt-then-Tag}:
    \begin{enumerate}
        \item $c \pickUAR \Enc(k_1, m)$
        \item $\phi \pickUAR \Tag(k_2, c)$
        \item $c^* = (c, \phi)$
    \end{enumerate}
\end{itemize}

Of the three options, only the last one is proven to be \cca-secure for arbitrary scheme choices; the other approaches are not secure ``a-priori'', with some couples proven to be secure by themselves. Notable examples are the \emph{Transport Layer Security} (\textsc{tls}) protocol, which employs the second strategy, and has been proven to be secure because of the chosen ecnryption scheme; \emph{Secure shell} (\textsc{ssh}) instead uses the first strategy.

\begin{theorem}
    If an authenticated encryption scheme $\Pi$ is made by combining a \cpa-secure \ske{} scheme $\Pi_1$ with a \emph{strongly unforgeable} \mac{} scheme $\Pi_2$ in the \emph{Encrypt-then-tag} method. Then $\Pi$ is \cpa-secure and \textit{auth}-secure. 
\end{theorem}

% AP190113: Key-related notation issues, again
\begin{proof}

    Assume that $\Pi$ is not \cpa-secure; then an adversary can use the resulting distinguisher $\distinguisher^{\cpa}$ to direct a succesful \cpa{} against $\Pi_1$, as shown in figure \ref{cryptoredux:ettcpa}: the point is to run the two components of $\Pi$ separately.

    \begin{cryptoredux}
        {ettcpa}
        {}
        {cpa$_1$}
        {cpa}

        \cseqchallenger{$k_1 \pickUAR \K_1$}
        \cseqadversary{$k_2 \pickUAR \K_2$}

        \cseqbeginloop
        \return{}{$m$}{}
        \send{}{$m$}{}
        \receive{$c \pickUAR Enc_k(m)$}{$c$}{}
        \invoke{$\phi \pickUAR Tag_k(c)$}{$(c, \phi)$}{}
        \cseqendloop

        \cseqdelay
        
        \return{}{$m_0^*, m_1^*$}{}
        \send{}{$m_0^*, m_1^*$}{}
        \receive{\shortstack[l]{
            $b \pickUAR \binary$ \\
            $c^* \pickUAR Enc_k(m_b)$
        }}{$c^*$}{}
        \invoke{$\phi^* \pickUAR Tag_k(c^*)$}{$(c, \phi^*)$}{}

        \cseqdelay

        \cseqbeginloop
        \return{}{$m$}{}
        \send{}{$m$}{}
        \receive{$c \pickUAR Enc_k(m)$}{$c$}{}
        \invoke{$\phi \pickUAR Tag_k(c)$}{$(c, \phi)$}{}
        \cseqendloop

        \cseqdelay

        \return{}{$b'$}{}
        \send{}{$b'$}{\textsc{Output 1 iff} $b' = b$}

    \end{cryptoredux}

    \todo{Professor says that we have to show that $\cryptog{cpa}(\lambda, 0) \compindist \cryptog{cpa}(\lambda, 1)$, but why? Isn't this proof enough?}

\end{proof}

Proved for the cpa-security property, now we have to prove, in a similar way, that the auth property must hold by $\Pi$ if $\Pi_2$ is an auth-secure scheme.
\begin{exercise}
    Prove it!

    Similar to the cpa-security proof.
\end{exercise}


    \mychapter{11}{Lesson 11} %181031

% "The age of ULTRON" - wanted to keep that somewhere...
\section{Authenticated encryption (continued)}

Having proven that an authenticated encryption scheme in an \emph{Encrypt-then-Tag} mode is \cpa-secure, it remains to prove that it has the \textsc{auth} property. Before this, a new unforgeability definition is needed:

\begin{definition}
    Let $\Pi = (\Tag, \Ver)$ be a \mac{} scheme. Then $\Pi$ is \emph{strongly} resistant to \emph{existential forgery under chosen message attack}, or s\textsc{ef-cma}-secure in short, iff:
    \[
        \Pr[\cryptog{ufcma}(\lambda) = 1] \in \negl(\lambda)
    \]
    i.e. it is \ufcma, but with the additional restriction that the tag $\phi^*$ of the forged message must be ``fresh'' itself.
\end{definition}

Note the small difference in security between \ufcma{} and s\textsc{ef-cma}.

\begin{theorem}
    Let $\Pi = (\Enc, \Dec, \Tag, \Ver)$ be an authenticated encryption scheme, composed by a \ske{} scheme $\Pi_1$ and a \mac{} scheme $\Pi_2$. If $\Pi_2$ is s\textsc{ef-cma}, then $\Pi$ has the \textsc{auth} property.
\end{theorem}

% AP190114: Need to phrase theorems differently
\begin{proof}
    The proof is analogous to the previous proof regarding the scheme's \cpa{} security. Suppose that $\Pi$ has not the \textsc{auth} property; then an adversary can use the distinguisher $\distinguisher^{\textsc{auth}}$ to successfully forge authenticated messages with fresh signatures against $\Pi_2$, as depicted in figure \ref{cryptoredux:ettauth}.
    
    \begin{cryptoredux}
        {ettauth}
        {Breaking authenticity of $\Pi_2$}
        {sef-cma$_2$}
        {auth}[1.7]

        \cseqbeginloop

        \return{}{$m$}{}
        \send{$c \pickUAR \Enc(k_1, m)$}{$c$}{}
        \receive{$\phi \pickUAR \Tag(k_2, c)$}{$\phi$}{}
        \invoke{}{$(c, \phi)$}{\shortstack[r]{
            $c \in C$ \\
            $\phi \in \Phi$
        }}

        \cseqendloop

        \cseqdelay

        \return{\shortstack[r]{
            $c^* \notin C$ \\
            $\phi^* \notin \Phi$
        }}{$(c^*, \phi^*)$}{}

        \send{}{$(c^*, \phi^*)$}{\shortstack[l]{
            \textsc{Output 1 iff} \\
            $\quad \Ver(k_2, c^*, \phi^*) = 1$
        }}

    \end{cryptoredux}

From $\distinguisher^{\textsc{auth}}$'s perspective, all the couples $(c_i, \phi_i)$ received are made from $m$ as follows:

\[
    c_i \pickUAR \Enc(k_1, m) \wedge \phi_i \pickUAR \Tag(k_2, c_i)
\]

Since $\distinguisher^{\textsc{auth}}$ is able to break the \textsc{auth} property of $\Pi$, the couple it outputs $(c^*, \phi^*)$ will be such that:

\[
    \Dec(k_1, c^*) = \widehat{m^*} \wedge \Ver(k_2, c^*, \phi^*) = 1
\]

Note that it is not important that the ciphertext decodes to the original plaintext; the mater at hand is that \adversary{} can now use the same couple to break $\Pi_2$'s \ufcma{} property, which is a contradiction.

It could happen that, for $c^* = c$ previously seen, $\phi^*$ is a new fresh tag, never seen before. Just in this case the $auth$ game would be valid because $(c^{*}, \phi^{*})$ would have never been seen before, but \textbf{not } the eufcma game, because $c^{*}$ was previously sent to the challenger.
\end{proof}
    
Now we want an ufcma secure scheme able to resist against message-tag challenge couples where the tag is fresh but the message has been already requested to the challenger.

\section{Pseudorandom permutations}

\todo{Luby-Rackoff is cited here, but it's related to both \prf{}s-\prp{}s and the Feistel network analysis} 

Nothing prevents a \prf{} $g_k$ to be bijective; in this case, it is referred to as a \emph{pseudorandom permutation}, or \prp{} in short. Their definition is analogous to a generic \prf, as shown in figure \ref{cryptogame:prp}: \prp{}s are computationally indistinguishable from a random permutation.

\begin{cryptogame}
    {prp}
    {$\cryptog{prp}[g_k](\lambda, b)$}
    {prp}

    \cseqchallenger{\shortstack[l]{
        $k \pickUAR \binary^\lambda$ \\
        $P \pickUAR \mathfrak{P}(n)$ \\
        $b \pickUAR \binary$
    }}

    \cseqdelay
    \cseqbeginloop
    \send{}{$x$}{}
    \receive{\shortstack[l]{
        $y_0 = g_k(x)$ \\
        $y_1 = P(x)$
    }}{$y_b$}{}
    \cseqendloop
    \cseqdelay

    \send{}{$b'$}{\textsc{Output 1 iff} $b' = b$}
    
\end{cryptogame}

\[
    \cryptog{prp}[g_k](\lambda, 0) \compindist \cryptog{prp}[g_k](\lambda, 1)
\]

An important difference is that $g_k$ is efficiently invertible, although knowledge of $k$ is reqiuired in order to do so.

\subsection{Feistel network}

\textsc{Prp}s have been successfully constructed by using existing \prf{}s into what is called a \emph{Feistel network}. As a starting point, let $f_k \in \binary^n \to \binary^n$ be a \prf, and define the function $\psi_f$ as follows:

\begin{align*}
    \psi_f(x, y)        &= (y, x \xor F(y)) = (x', y')                                  \\
    \psi^{-1}_f(x', y') &= (F(x') \xor y', x') = (F(y) \xor x \xor F(y), y) = (x, y)    \\
\end{align*}

\begin{figure}
    \centering

    \begin{tikzpicture}[auto]
        
        \node (x) {$x$};
        \node (y) [below of = x, node distance = 3cm] {$y$};

        \node (xor) [right of = y] {$\xor$};
        \node (F) [box, above of = xor, node distance = 1.5cm] {$F_k$};

        \draw[->] (x) -- +(2cm, 0) -- +(3cm, -3cm) -- +(4cm, -3cm);
        \draw[->] (y) -- (xor);
        \draw[->] (xor) -- +(1cm, 0) -- +(2cm, 3cm) -- +(3cm, 3cm);

        \draw[->] (x)+(1cm, 0) -- (F);
        \draw[->] (F) -- (xor);
        
    \end{tikzpicture}
    
    \label{fig:feistel}
    \caption{A single-round Feistel network}
\end{figure}

% AP190121: Some more intermediate steps might be useful
While this construct is invertible and uses a \prf{}, it is not pseudorandom itself, because the first $n$ bits of $\psi_F$'s image are always equal to $y$, and thus visible to any adversary. A first attempt at fixing this vulnerability would be to apply the construct two times on two different \prf{}s $\psi^2_{F, F'}$, in an attempt to ``hide'' $y$. Yet, this approach still leaks valuable information:
\[
    \psi_{F, F'}(x, z) \xor \psi_{F, F'}(y, z) = (x \xor y, \dots)
\]

However, this example with additional restrictions will be useful very soon, so it is reworded as the following lemma:

\begin{lemma}
    For any unbounded adversary making $q \in \poly(\lambda)$ queries, he is unable to consistently win the game depicted in \ref{cryptogame:randfeis}:
    \[
        \cryptog{r-feis}(\lambda, 0) \statindist \cryptog{r-feis}(\lambda, 1)
    \]
    as long as $y_1, \ldots, y_q$ are mutually distinct.
\end{lemma}

\begin{cryptogame}
    {randfeis}
    {The randomic-Feistel distinguishing game: $\cryptog{r-feis}s$}
    {r-feis}

    \cseqchallenger{\shortstack[l]{
        $\overline{R}, \overline{R'} \pickUAR \mathfrak{R}(n, n)$ \\
        $R \pickUAR \mathfrak{R}(2n, 2n)$ \\
        $b \pickUAR \binary$
    }}

    \cseqdelay
    \cseqbeginloop
    \send{}{$(x, y)$}{}
    \receive{\shortstack[l]{
        $z_0 = \psi^2_{\overline{R}, \overline{R'}}(x, y)$ \\
        $z_1 = R(x, y)$
    }}{$z_b$}{}
    \cseqendloop
    \cseqdelay

    \send{}{$b'$}{\textsc{Output 1 iff} $b' = b$}

\end{cryptogame}

\begin{proof}
    \todo{Idea: Hybridize over the queries before the challenge, from PR to random; prove that the stat distance between i and i+1 is negligible}
\end{proof}


Going back to the Feistel networks in general, it should be easy to see that they can be made of an arbitrary number of rounds, by simply chaining output with input. The $l$-th iteration is denoted as: 
\[
    \psi^l_\Phi(x, y) = \psi_{F}(\psi_{F''}( \ldots \psi_{F^{(l)}}(x, y) \ldots ))
\]
where $\Phi$ is the sequence of \prf{}s used at each single step. It can be shown that the rounds neeed to obtain a network that is indeed pseudorandom is just 3; also, the same \prf{} can be used, it is sufficient to change the seed on each iteration\footnotemark:

\begin{theorem}[Michael Luby - Charles Rackoff]
    Let $F_i, F_j, F_k$ be a \prf{} over three seeds. Then $\psi^3_{i, j, k}$ is a \prp.
\end{theorem}

\footnotetext{That is actually the purpose of using a \prf}

\begin{proof}
    \todo{Idea: Four total games: original, swap prfs with random functions, swap the three functions with a single one (use previous lemma), swap random function with random permutation (avoid bad events generated by injection property)}
\end{proof}
    \mychapter{12}{Lesson 12} %181107
\section{Hashing}

Remember one solution to domain-exetension for \prf{}s, as a composition of a \prf{} $F$ with an almost universal hash function $H$. Hash functions compress their arguments to some ``fingerprint'', which is asssumed to be unique. However, since this compression in this context inherently introduces information loss, it is not guaranteed that every message gets its own unique fingerprint; indeed, there will be some instances where two messages yield the same hash value, or in other words, the hashes \emph{collide}. It is desirable for a hash function to be \emph{resistant} to these events, meaning that it is hard to reproduce such collisions.

\begin{definition}
    A hash function family $H$ is deemed \emph{collision-resistant}, denoted as $H \in \crh$ iff the probability of finding a collision is negligible, even when knowing the seed $s$. Formally:
    \[
        \forall \adversary \in \ppt \implies \Pr(\cryptog{crh}(\lambda) = 1) \in \negl(\lambda)
    \]

    \begin{cryptogame}
        {crh}
        {The \emph{collision-resistance} game}
        {crh}

        \receive{$s \pickUAR \binary^\lambda$}{$s$}{}

        \send{}{$x, y$}{\textsc{Output 1 iff} $h_s(x) = h_s(y)$}
        
    \end{cryptogame}
\end{definition}

A note: before, we were dealing with unbounded adversaries, and the key was hidden. Now the key is public, but the adversary must be efficient.

\begin{exercise}
    Let $\Pi$ be a \ufcma{} authentication scheme over the message space $\binary^n$. Show that $\Pi' = (\Tag', \Ver') : \Tag'_{k, s}(m) = \Tag_k(h_s(m))$ is \ufcma-secure over $\binary^l$, where $l \in \poly(n)$, as long as $h_s$ itself is \crh.
\end{exercise}

\subsection{Merkle-Damg\r{a}rd construction}

A construct for starters has been defined by Ralph Merkle and Ivan Damg\r{a}rd, which revolves around the use of a \crh{} function $h_s \in \binary^{n + 1} \to \binary^n$. For now, consider the messages to be of an arbitrarily fixed length $l$. The steps to follow are:

\begin{enumerate}
    \item Let $\textsc{iv} \in \binary^b$ be an \emph{initialization vector}
    \item Initialize $t_0 := \textsc{iv}$
    \item For each bit $b_i$ of the message to hash $m$:
    \begin{enumerate}
        \item Compute $t_i := h_s(b_i, t_{i - 1})$
    \end{enumerate}
    \item Return $\phi = t_n$
\end{enumerate}

% AP190203: To review, it is somewhat confused
%First step: Compress the original message (assuming fixed size) by one bit

%Let $H$ be a one-bit shrinking function. Then, it can be used to construct a hash function H' that splits an arbitrary-size message into fixed-size blocks, apply H onto them, and return a digest of fixed length. This is exemplified by the diagram in figure \ref{fig:mdbase}

\begin{figure}
    \centering

    \tikzstyle{int}   = [draw, minimum size=2em]
    \tikzstyle{empty} = [minimum size=2em]
    \tikzstyle{init}  = [pin edge={to-,thin,black}]

    \begin{tikzpicture}[node distance = 1.9cm, auto, >=latex']

        \node (a) [empty] {$\textsc{iv}$};
        \node (r) [int, pin={[init]above:$b_1$}] [right of=a] {$h_s$};
        \node (d) [int, pin={[init]above:$b_2$}] [right of=r] {$h_s$};
        \node (e) [int, pin={[init]above:$b_3$}] [right of=d] {$h_s$};
        \node (f) [empty] [right of=e] {$...$};
        \node (g) [int, pin={[init]above:$b_n$}] [right of=f] {$h_s$};
        \node (h) [empty, right of=g] {$\phi$};

        \path[->] (a) edge (r);
        \path[->] (r) edge node {$t_1$} (d);
        \path[->] (d) edge node {$t_2$} (e);
        \path[->] (e) edge node {$t_3$} (f);
        \path[->] (f) edge node {$t_{n-1}$} (g);
        \path[->] (g) edge (h);
    
    \end{tikzpicture}
    \caption{Basic outline of a Merkle-Damg\r{a}rd construction}
    \label{fig:mdbase}
\end{figure}

Figure \ref{fig:mdbase} depicts a general view of the algorithm. Let it be denoted as another hash function $h'_s \in \binary^l \to \binary^n$.

\begin{theorem}
    The construction $H'$ obtained by Merkle-Damg\r{a}rd is a \textsc{crh} function.
\end{theorem}

% AP190204: Tentative
\begin{proof}
    Assume $H'$ can be broken efficiently by a distinguisher $\distinguisher^{\crh}$, meaning that finding two distinct block sequences that give the same hash is easy. Consider the reduction to $H$'s \crh-ness in figure \ref{cryptoredux:mdcrh}

    \begin{cryptoredux}
        {mdcrh}
        {Breaking the underlying \crh-ness of $H$}
        {h-crh}
        {h'-crh}[3]

        \receive{}{$s$}{}
        \invoke{}{$s$}{}
        \return{}{$x, y$}{}

        \cseqdelay

        % AP190911: need a better way to convey meaning here
        \send{\shortstack[r]{
            \textsc{Find} $j :$ \\
            $h'_{s, (j - 1)}(x) \neq h'_{s, (j - 1)}(y)$ \\
            $\wedge\: h'_{s, (j)}(x) = h'_{s, (j)}(y)$
        }}{$h'_{s, (j - 1)}(x), h'_{s, (j - 1)}(y)$}{}

    \end{cryptoredux}

    Ignore same blocks: find largest j such that:
    \[
        (b_j, x_{j - 1}) \neq (b'_j, y_{j - 1}) \wedge h_s(b_j, x_{j - 1}) = h_s(b'_j, y_{j - 1})
    \]
    this implies the rest of th message is equal, then the resulting final hash will be equal, thus for $j > 0$ we have a collision.

\end{proof}

%AP190911: Not clear at all
\subsubsection{Domain extension}

In order to adapt the MD-construct to messages of variable length, some sort of ``strengthening'' is required: 

% Also called MD-strengthening
\begin{lemma}[Length padding]
    Let $H_s \in \binary^{n + l} \to \binary^n$, then:
    %AP190911: This needs to be rewritten correctly
    \[
        H'_s = H_s(\left<l'\right>, H_s(x_{l'}, \dots , H_s(x_1, 0^n) \dots)), |l'|, |x_i| \in \binary^c 
    \]
\end{lemma}

This is an example of padding which encodes the message length in itself.

\begin{theorem}
    The strengthened construct is collision resistant for variable-length messages.
\end{theorem}

\begin{proof}
    Hint: similar as above, case by case
\end{proof}

\subsubsection{Merkle trees}

---

\subsection{Compression functions}

Let ($\textsf{Gen} : 0 \mapsto (\pk, \sk), f, g)$ be a \pke{} scheme, where the functions $f$ and $g$ are keyed \prp{}s. A \emph{claw} is a couple of values $(x, x')$ such that:
\[
    f(\pk, x) = g(\pk, x')
\]

\begin{cryptogame}
    {cfp}
    {The game of claw-free permutations}
    {cfp}
    
    \receive{$(\pk, \sk) \pickUAR \textsf{Gen}$}{$\pk$}{}
    \send{}{$x, x'$}{\textsc{Output 1 iff} $f(\pk, x) = g(\pk, x')$}

\end{cryptogame}

\begin{theorem}
    Assuming $\mathcal{F}$ is claw-free, then $h_\pk$ is \crh{} from $n + l$ bits to $n$.
\end{theorem}

\subsubsection{Davies-Meyer construct}

% AP190911: How does AES come in here?!
\begin{definition}
    $x_{i + 1} = E_k(x_i) \xor x_i$, maps n+$\lambda$ to n. E is AES
\end{definition}


    \part{Asymmetric schemes}

    \mychapter{13}{Lesson 13} %181109

\section{Number theory}

\begin{theorem}[Fermat's last theorem]
    \[
        \forall x, y, z \in \integer, n > 2 \implies x^n + y^n \neq z^n
    \]
\end{theorem}

\begin{lemma}
    \[
        \forall a \in \integer_n : \gcd(a, n) > 1 \implies a \notin \integer_n^\times
    \]
\end{lemma}

\begin{proof}
    Assume there exists $b \in \integer_n$ such that $ab \equiv_n 1$. Then, there exists a quotient for the division $by$ nbetween a and b with remainder 1. Observe that $\gcd(a, n)$ divides $ab + qn$, which is equal to 1. It entails that $\gcd(a, n) = 1$, which is a contradiction.
\end{proof}

\begin{lemma}
    \[
        \forall a, b \in \nonneg : a \geq b \neq 0 \implies \gcd(a, b) = \gcd(b, a \mod b)
    \]
\end{lemma}

\begin{proof}
    Not given.
\end{proof}

\begin{theorem}
    Given two integers $a$ and $b$, their greatest common divisor can be computed efficiently with respect to their lengths. Additionally, two other numbers u and v can be efficiently computed in order to satisfy B\'ezout's identity: $\gcd(a, b) = au + bv$
\end{theorem}

\begin{proof}
    Hint: Use previous lemma recursively; we stop at $r_{t + 1} = 0$ for some $t$, thus:
    \[
        \gcd(a, b) = \dots = \gcd(r_t, r_{t + 1}) = r_t
    \]
\end{proof}

\begin{claim}
    $r_{i + 2} \leq r_i/2 \forall 0 \leq i \leq t - 2 \implies \#steps = \lambda - 1 \text{ if } |b| \in \binary^\lambda$
\end{claim}

The hypohesis is a natural consequence of the repeated modulo operations.

\begin{observation}[Exponentiation mod n: Square and multiply]
    Let $b \in \binary^l$, where by writing $b_i$ we denote b's i-th bit. Then:
    \[
        a^b \equiv_n a^{\sum_{i = 0}^l 2^ib_i} \equiv_n \prod_{i = 0}^l a^{2^ib_i} 
    \]
\end{observation}

%AP190913 - Unproven, can't find it anywhere
\begin{theorem}
    The number of primes lesser than or equal to $x$ is a number greater than or equal to $\frac{x}{3\log_2 x}$
\end{theorem}

There exist many algorithms that solve the problem of primality testing, with time complexity polynomial in the length of their numerical representation; of most relevance are those of Miller-Rabin, which is probabilistic, but consistently used in practice, and the completely deterministic Agrawal-Kayal-Saxena (\textsc{aks}), which has a much greater polynomial rank, and has been deemed impractical for most uses.

What remained as a conjecture is the intractability of determining the factors of a $\lambda$-bit composite number, which is widely known as the \emph{factorization hardness}; this in turn would imply that integer multiplication of two $\lambda$-bit primes is a \owf.

\begin{definition}
    Given a group $G$, its order is the least $i$ such that $a^i \equiv_n 1$ 
\end{definition}

\begin{corollary}
    \[
        \forall a \in \integer_m^\times \implies a^{\phi(n)} \equiv_n 1 \wedge a^b \equiv_n a^{b \mod \phi(n)} \wedge a^{p - 1} \equiv_p 1
    \]
\end{corollary}

\begin{proof}
    $(\integer_n^\times, \cdot)$ is a group with $\phi(n)$ elements. Take $\left< a \right> = \{a^0, \dots, a^{d + 1}\}$, where $d$ is the order. We must have $\phi(n) = kd$ for some $k$. Therefore, $a^{\phi(n)} = a^{dk} \equiv_n 1$.

    Also, $a^b \equiv_n a^{b\phi(n) + b \mod \phi(n)} \equiv_n 1 \cdot 2^{b \mod \phi(n)}$
\end{proof}

\begin{theorem}
    Let G, H be two groups such that $H < G$, meaning the order of H divides the order of G
\end{theorem}

\section{Standard model assumptions}

We now turn our attention to some conjectures that form the basis for most of the cryptographic schemes that will follow. We will start from the weakest, and go up to the strongest. For all our purposes, let $\groupgen(1^\lambda)$ be a ``group generator'' with security parameter $\lambda$, and let a random sample $(G, g, q) \pickUAR \groupgen(1^\lambda)$ be a triplet composed of the group itself $G$, one of its generators $g$, and its order $q$.

% CDH -> DDH -> DL
\subsubsection{Discrete logarithm}

Given $g$ and $g^x$ in a $\lambda$-bit group, there is no efficient algorithm for computing $y$ such that $g^y = g^x$ withouut knowing $x$ beforehand:
\[
    \forall \adversary \in \ppt \implies \Pr[\cryptog{dl}(\lambda) = 1] \in \negl(\lambda)
\]

\begin{cryptogame}
    {dlass}
    {$\cryptog{dl}(\lambda)$}
    {dl}

    \receive{\shortstack[l]{
        %AP190913: Actually, the adversary knows this!
        $(G, g, q) \pickUAR \groupgen(1^\lambda)$ \\
        $x \pickUAR \integer_q$
    }}{$y = g^x$}{}

    \send{}{$x'$}{\textsc{Output 1 iff} $y^{x'} \equiv_G y$}
    
\end{cryptogame}

This means that the \dl{} for a generic group G yields a \owf, whereas in a multiplicative group $\integer_p^\times$, we obtain a \owp.

\subsubsection{Computational Diffie-Hellman}

%AP190913: Probably it is not necessary to compute it directly, we may only need a congruent value
Given a group $G$ and two elements in it $g^x$, $g^y$, it is impractical to compute $g^{xy}$ without knowing $x$ or $y$.

\begin{cryptogame}
    {cdhass}
    {$\cryptog{cdh}(\lambda)$}
    {cdh}

    \receive{\shortstack[l]{
        %AP190913: Actually, the adversary knows this!
        $(G, g, q) \pickUAR \groupgen(1^\lambda)$ \\
        $x, y \pickUAR \integer_q$
    }}{$(g^x, g^y)$}{}

    \send{}{$h$}{\textsc{Output 1 iff} $h = g^{xy}$}
    
\end{cryptogame}

\subsubsection{Decisional Diffie-Hellman}

Given a group $G$ and three elements in it $g^x$, $g^y$, $g^z$, it is impractical to distinguish $g^{xy}$ from $g^z$ by only knowing $g^x$ and $g^y$, along with the originating triad $(G, g, q))$.

\begin{cryptogame}
    {ddhass}
    {$\cryptog{ddh}(\lambda)$}
    {ddh}

    \cseqdelay

    \receive{\shortstack[l]{
        %AP190913: Actually, the adversary knows this!
        $(G, g, q) \pickUAR \groupgen(1^\lambda)$ \\
        $x, y, z_1 \pickUAR \integer_q$ \\
        $z_0 = xy$ \\
        $b \pickUAR \binary$
    }}{$(g^x, g^y, g^{z_b})$}{}

    \cseqdelay

    \send{}{$b'$}{\textsc{Output 1 iff} $b' = b$}
    
\end{cryptogame}

All these assumptions helped in constructind the \emph{Diffie-Hellman key exchange} protocol, which is a way to establish a \ske{} channel from an unsafe channel, with any adversary unable to efficiently break the channel's secrecy. Do note that authentication is left out of the picture here.

\begin{cryptosequence}
    {dhkex}
    {The Diffie-Hellman Key Exchange protocol}
    
    \cseqentity{A}{Alice}
    \cseqentity[2.2]{B}{Bob}

    \cseqmessagel{B}{\shortstack[l]{
        $(G, g, q) \pickUAR \Group\Gen(1^\lambda)$ \\
        $x \pickUAR G$ 
    }}{$(G, g, q, g^x)$}{A}{}

    \cseqdelay

    \cseqmessager{A}{\shortstack[l]{
        $y \pickUAR G$ \\
        $k = (g^x)^y$
    }}{$g^y$}{B}{$k = (g^y)^x$}

\end{cryptosequence}

Some relationships have been established between these assumptions: it is known that $\ddh \implies \cdh \implies \dl$; and that $\cdh \notimplies \ddh$.

    \mychapter{14}{Lesson 14} %181114

\subsubsection{Decisional Diffie-Hellman (cont'd)}

%AP190913: Is p supposed to be prime?
\begin{claim}
    \ddh{} is not hard for groups $\integer_p^\times$
\end{claim}

\begin{proof}
    Let $\quadres_p$ be the group of quadratic residues modulo $p$, where the group operation is multiplication:
    \[
        \quadres(p) = \{y : \exists x \in \integer_p \implies y \equiv_p x^2\} = \{g^z : \forall z\}
    \]
    where $g$ generates $\integer_p^\times$.Then, we can test if a give number $y$ is in $\quadres_p$ by checking if $y^{(p-1)/2} \equiv_p 1$, because:
    \[
        \exists z : y = g^{2z} \implies y^{(p-1)/2} = g^{\frac{2z(p-1)}{2}} = g^{z(p-1)} \equiv_p 1
    \]
    Otherwise: % AP190208: Not sure at all that this is correct
    \[
        \nexists z : y = g^{2z} \implies y^{(p-1)/2} \equiv_p g^{z(p-1)} \cdot g^{(p-1)/2} \not\equiv_p 1
    \]

    Furthermore: $g^{xy} \in \quadres_p \implies x \equiv_2 0 \vee y \equiv_2 0$. With this in mind, given a random choice of $x$ and $y$, the probability that $g^{xy}$ falls in $\quadres(p)$ is $\frac{3}{4}$, and the probability of it being outside the group is $\frac{1}{4}$. This is an advantage available to a polynomial adversary.
\end{proof}

Nevertheless, some otehr groups are believed to harden quadratic residue membership; examples of such groups are $\quadres_p$ itself for $p = 2q + 1$, where $q$ is prime, or the elliptic curve groups.

\subsubsection{Extended \ddh}

This is a construction that takes the \ddh{} assumption to an extreme using groups. The \ddh{} assumption is reported here:
\[
    (G, g, q) \pickUAR \groupgen(1^\lambda), \forall i\: X_i, Y_i, Z_i \sim \unifdist(G) \implies (g^X, g^Y, g^{XY}) \compindist (g^X, g^Y, g^Z)
\]
where $X$, $Y$ and $Z$ are distribution ensembles.

The construction extends this concept of hardness in detecting $xy$ by replicating it $n$-times:
\begin{theorem}
    Let $X$, $Y_{1\upto n}$ and $Z_{1 \upto n}$ be distribution ensembles over a group $(G, g, q)$ Then:
    \[
        (g^X, g^{Y_1}, g^{XY_1}, \dots, g^{Y_n}, g^{XY_n}) \compindist (g^X, g^{Y_1}, g^{Z_1}, \dots, g^{Y_n}, g^{Z_n})
    \]
\end{theorem}

\begin{proof}
    \todo{The game is modified a bit to ease analysis of the reduction, must take care...
    
    Also, this kind of proof is ``tight''; the hybridization would have introduced a negligible difference instead}

    The proof can be structured by progressive hybridization over the sequences, and breaking the \ddh{} assumption at any step $i$. Here instead, we will simulate the \textsc{ext} sequence directly in the distinguishing game, and assume there is $\distinguisher^{\textsc{ext}}$ capable of telling them apart; figure \ref{cryptoredux:extddh} shows the steps to take.

    \begin{cryptoredux}
        {extddh}
        {Breaking \ddh{} by distinguishing the \ddh{} sequence}
        {ddh}
        {ext}
        
        \cseqdelay
        \receive{\shortstack[l]{
            %AP190913: Actually, the adversary knows this!
            $(G, g, q) \pickUAR \groupgen(1^\lambda)$ \\
            $x, y \pickUAR \integer_q$ \\
            $b \pickUAR \binary$ \\
            $\beta_b \pickUAR \{0, \integer_q\setminus\{0\}\}$ \\
            $z_b = xy + \beta_b$
        }}{$(g^x, g^y, g^{z_b})$}{}

        \cseqbeginloop
        \cseqdelay
        \cseqadversary{\shortstack[r]{
            $u_i, v_i \pickUAR \integer_p$ \\
            $g^{y_i} = (g^y)^{u_j}g^{v_j}$ \\
            $g^{z_i} = (g^z)^{u_j}(g^x)^{v_j}$
        }}
        \cseqdelay
        \cseqendloop
        \cseqdelay
        
        \invoke{}{$(g^x, g^{y_1}, g^{z_1}, \dots)$}{}

        \return{}{$b'$}{}

        \send{}{$b'$}{\textsc{Output 1 iff} $b' = b$}

    \end{cryptoredux}

\end{proof}

Notice that $(g^y)^{u_j}g^{v_j}= g^{yu_j + v_j}$, and $(g^z)^{u_j}(g^x)^{v_j} = g^{zu_j + xv_j}$, which depending on what $z$ is, becomes either $g^{x(yu_j + v_j)}$, matching perfectly the $y_i$ case, or remains as a random linear combination, matching the $z_i$ case.

\subsubsection{Naor-Reingold \prf}

This is an alternative to the extended \ddh{} seen above, designed by Moni Naor and Omer Reingold. It constructs a \prf{} as follows:
\[
    F^{\textsc{nr}} \in \integer_q^{(n+1)} \times \binary^n \to (G, g, q) : f^{\textsc{nr}}_k(x_{1\upto n}) \mapsto g^{k_0\prod_{i = 1}^{n}k_i^{x_i}}
\]

\todo{Transcribing notes directly:
\{
This is \ggm{} with $G^{g, q, a}(g^b) = G_0(g^b) \mathrel{||} G_1(g^b) = (g^b, g^{ab})$

E.g.: $011 \mapsto g^{a_0a_1^0a_2^1a_3^1}$

\}

Notes about claw-free permutations follow; also, hash functions make an appearance

}


\begin{figure}
    \centering
    \begin{tikzpicture}[
        level 1/.style={sibling distance=16em},
        level 2/.style={sibling distance=8em},
        level 3/.style={sibling distance=4em}]

        \node{$g^{a_0}$}
            child{ node {}
                child { node {}
                    child { node (a) {$g^{a_0}$} node [below of = a] {$\vdots$}}
                    child { node (a) {} node [below of = a] {$\vdots$}}
                }
                child { node {}
                    child { node (a) {} node [below of = a] {$\vdots$}}
                    child { node (a) {} node [below of = a] {$\vdots$}}
                }
            }
            child { node {}
                child { node {}
                    child { node (a) {} node [below of = a] {$\vdots$}}
                    child { node (a) {} node [below of = a] {$\vdots$}}
                }
                child { node {}
                    child { node (a) {} node [below of = a] {$\vdots$}}
                    child { node (a) {$g^{a_0a_1a_2a_3}$} node [below of = a] {$\vdots$}}
                }
            };
    \end{tikzpicture}
    \caption{Depicting Naor-Reingold method as a tree-like structure}
    \label{fig:nrtree}
\end{figure}

\section{Public key encryption schemes}

Now we discuss a substantially different way of encrypting messages, which involves two separate keys per party instead of a single, shared key; hence the names \emph{Public key scheme}, or \pke{}, or \emph{Asymmetric key scheme}.


\subsubsection{\textsc{Cpa}-security, revisited}

In a \pke{} setting, the adversary knows the public key by design, which in turn is all that he needs to preform encryptions; therefore, the queries in a hypotetical \cpa{} game are moot, because \adversary{} can make the encryptions on its own; figure \ref{cryptogame:pkecpa} shows the changed game for a \pke{}.


\begin{cryptogame}
    {pkecpa}
    {Chosen plaintext attacks, revisited for \pke{} schemes}
    {cpa2}

    \receive{$(\pk, \sk) \pickUAR \keygen(1^\lambda)$}{$\pk$}{}

    \cseqbeginloop
    \cseqadversary{$c = \Enc(\pk, m)$}
    \cseqendloop

    \send{}{$m_0, m_1$}{}

    \receive{\shortstack[l]{
        $b \pickUAR \binary$ \\
        $c_b = \Enc(\pk, m_b)$
    }}{$m_b$}{}

    \cseqbeginloop
    \cseqadversary{$c = \Enc(\pk, m)$}
    \cseqendloop

    \send{}{$b'$}{\textsc{Output 1 iff} $b' = b$}

\end{cryptogame}

\todo{Another verbatim transcription:

Abstract construction (inefficient) for trapdoor permutation

$(Gen, f, f')$, where $gen= \keygen : \singleton \to \Xi_\pk, f(\pk, \cdot) \in \Xi_\pk \to \Xi_\pk, f' = f^{-1}(\sk, \cdot) \in \Xi_\pk \to \Xi_\pk$

Correctness: $f^{-1}(\sk, f(\pk, m)) = m$

$TDP \implies PKE$, Easy but not trivial

$\cryptog{tdp}$... same as \cpa2?

Why $f, f'$ is not cpa-secure? Because they are deterministic

Use $f$'s hardcore predicate

}

    \mychapter{15}{Lesson 15} %181116
\section{Public key encryption (cont'd)} 

\subsubsection{\textsc{Cca}-security, revisited}

\begin{cryptogame}
    {pkecca}
    {Chosen ciphertext attacks, revisited for \pke{} schemes}
    {cca2}
    
    \receive{$(\pk, \sk) \pickUAR \keygen(1^\lambda)$}{$pk$}{}

    \cseqdelay
    \cseqbeginloop

    \send{}{$c'$}{}
    \receive{$m' = \Dec(\sk, c)$}{$m'$}{}

    \cseqendloop
    \cseqdelay

    \send{}{$m_0, m_1$}{}

    \receive{\shortstack[l]{
        $b \pickUAR \binary$ \\
        $c \pickUAR \Enc(\pk, m_b)$
    }}{$c$}{}

    \cseqdelay
    \cseqbeginloop

    \send{}{$c'$}{}
    \receive{$m' = \Dec(\sk, c)$}{$m'$}{}

    \cseqendloop
    \cseqdelay

    \send{}{$b'$}{\textsc{Output 1 iff} $b' = b$}
\end{cryptogame}

\subsection{Trapdoor permutations}

Let us cnsider the following scheme $\Pi$, which is an adaption of the old \cpa-secure \ske{} to public keys:
\begin{itemize}
    \item $\Enc(\pk, m) = (r, f_\pk(m \xor r))$, where $r \sim \unifdist(\binary^\lambda)$
    \item $\Dec(\sk, (c_0, c_1)) = g_\sk(c_1) \xor c_0$
    \item Correctness: $g_\sk(c_1) \xor c_0 = g_\sk(f_\pk (m \xor r)) \xor r = m \xor r \xor r = m$
\end{itemize}

A \emph{trapdoor permutation} (or \tdp) is a \owp{} having the following features:

\begin{itemize}
    \item A key pair is chosen \uar{} by a key generator algorithm:
    \[
        (\pk, \sk) \pickUAR \keygen(1^\lambda)
    \]
    \item There is a function family $F \in \Xi_\pk \to (V_{\pk} \to V_{\pk})$ such that:
    \begin{itemize}
        \item Computing $f_{\pk}$ is efficient
        \item Domain sampling ($x \pickUAR V_{\pk}$) is efficient
    \end{itemize}
    \item There is an efficient function $g_{\sk}$ that efficiently \emph{inverts} $f_{\pk}$, where $sk$ is the ``trapdoor'': 
    \[
        g(\sk, f(\pk, x)) = x
    \]
    \item No efficient adversary is able to invert $f_{\pk}$ wihout knowing $\sk$
\end{itemize}

Note that because $\pk$ is public, an adversary can perform any polynomial number of encryptions with $\pk$, aand see the corresponding ciphertext. This is the same characteristic of \pke{} schemes we described just a while ago. It entails that, if left deterministic, a \tdp{} is not \cpa-secure.

Also, the scheme described above is not \cpa-secure form the start: the adversary, by choosing two messages for the challenge, and receiving the ciphertext $(c_0 = r, c_1)$, along with the public key, has everything needed to reconstruct the encryption and chec whichever message was encrypted, much like the problem of \ufcma against a deterministic \mac scheme.

%The idea is to reuse the message as part of the randomness, or more in detail, use its hardcore predicate. The resulting scheme would the look like this:
Here, in this scheme, we combine randomness and the notion of hardcore predicate $\hc$:

% AP190123 - IMPORTANT NOTE: This scheme is intended to work for single-bit messages
\begin{itemize}
    \item $(pk, sk) \pickUAR \keygen(1^\lambda)$
    \item $r \pickUAR \Xi_{pk}$
    \item $c := \Enc(\pk, m) = (f_{\pk}(r), \hc(r) \oplus m)$
    \item Correctness: $\Dec(\sk, c) = \hc(g_{\sk}(c_1)) \oplus c_2$
    % Exercise: If obscure, do the petty algebra on paper
\end{itemize}

\begin{theorem}
    If $F$ is a \tdp{} and $\hc$ is hardcore for $f$, then the above scheme is \cpa-secure.    
\end{theorem}

\begin{proof} The proof is left as exercise
    \todo{Apparently, the reduction here is not easy at all, some hints are needed.}
\end{proof}

\subsection{\textsc{Tdp} examples}

One example stems form the factoring problem: let's look again at $\integer_n^\times$, where $n$is the product of two prime numbers $p$, and $q$:

\begin{theorem}[Chinese remainder theorem]
    The following isomorphisms to $\integer_n^\times$ are true:

    \begin{itemize}
        \item $\integer_n \simeq \integer_p \times \integer_q$
        \item $\integer_n^\times \simeq \integer_p^\times \times \integer_q^\times$
    \end{itemize}

    Note that the theorem is more general, and holds for any two numbers $p$ and $q$ that are coprime.
\end{theorem}

How to use this theorem for constructing a \pke{} scheme:

\begin{theorem}[Euler's totient theorem]
    Let $x$, $n$ be two coprime numbers. Then $x^{\varphi(n)} \equiv_n 1$
\end{theorem}

Also, remember that $\forall p, q \in \primenum \implies \varphi(pq) = (p - 1)(q - 1)$

So let $a$ be the public key such that $\gcd(a, \varphi(n)) = 1$, then $\exists! b \in \integer_n : ab \equiv_{\varphi(n)} 1$; $b$ will be our private key. Define encryption as $f(a, m) = m^a \mod n$, and then decryption as $g(b, c) = c^b \mod n$. Observe that 
\[
    g(b, f(a, m)) = (m^a)^b = m^{ab} = m^{k \varphi(n) + 1} = (m^{\varphi(n)})^k m \equiv_n m
\]
because $ab = 1 \mod \varphi(n)$.

So we conjecture that the above is a valid \tdp-based \pke{} scheme. This is actually referred to as the \emph{\rsa{} assumption}, and is depicted in figure \ref{cryptogame:rsaass}

\begin{cryptogame}
    {rsaass}
    {The \rsa{} assumption}
    {rsa}

    \receive{\shortstack[l]{
        $b \pickUAR \binary$ \\
        $c \pickUAR \textrm{Enc}(pk, m_b)$
    }}{$n, pk, m^{pk}$}{}

    \cseqdelay

    \send{}{$m'$}{\textsc{Output 1 iff} $m' = m$}

\end{cryptogame}


Relation to the factoring problem: $\rsa \implies \textsc{fact}$

Proof: Given $p, q$, an adversary can compute $\varphi(n) = (p - 1)(q - 1)$, and then find the inverse of the public key in $\integer_{pq}^\times$.

It hasn't been proven that $\textsc{fact} \implies \rsa$

\section{Textbook \rsa}

This is an \underline{insecure} toy example of the more complex \rsa{} (Rivest Shamir Adleman) scheme.

\begin{itemize}
    \item Setup: $(\pk, \sk) \pickUAR \keygen(\integer_n^\times) : \sk \equiv_{\varphi(n)} \pk^{-1}$
    \item Encryption: $\Enc(\pk, m) = m^{\pk} \mod n$
    \item Decryption: $\Dec(\sk, c) = c^{\sk} \mod n$
    \item Correctness: $\Enc(\pk, \Dec(\sk, m)) = m^{\pk \cdot \sk} \equiv_n m$
\end{itemize}

Again, since the encryption routine is deterministic, the scheme is not \cpa-secure. However, a hardcore predicate can be inserted to the routine: $\hat{m} = r||m$, where $r \pickUAR \binary^l$. Now the encryption is pseudorandom.

Some interesting facts:
\begin{enumerate}
    \item $l \in \omega(log(\lambda))$ otherwise a brute-force attack becomes viable.
    \item If $m \in \binary$, then the scheme is \cpa-secure under \rsa{} assumption, just use the standard \tdp{}
    \item If $m$ is ``in the middle'': $\binary \leq m \leq \binary^l$; then \rsa{} is believed to be secure (standard PKCS\#1, 5)
    \item However, this construct is not \cca-secure.
    \todo{Counterexample?}
\end{enumerate}

\subsection{Trapdoor Permutation from Factoring}

Here is an attempt to build a \tdp{} over the group $\quadres(n)$: let's look at $f(x) \equiv_n x^2$ where $f \in \integer_n^\times \to \quadres_n (\subset \integer_n^\times)$. Notice that this is not a permutation in general, so let's consider their \textsc{crt}'s representation\footnotemark, in order to restrict $f$'s domain:

\footnotetext{Those well-versed in number representation in computers may know this representation as the \emph{residue number system}}

\[
    x = (x_p \equiv_p x, x_q \equiv_q x), f(x) \equiv_p x^2, x \in \integer_p^\times
\]

Since $\integer_p^\times$ is cyclic:

\begin{align*}
    \integer_p^\times   &= \{g^0, g^1, g^2, \ldots  &&, g^{\frac{p - 1}{2} - 1}              &&, g^{\frac{(p-1)}{2}}                    &&, \ldots, g^{p-2}         &&\} \\
    \quadres_p          &= \{g^0, g^2, g^4, \ldots  &&, g^{2(\frac{p - 1}{2} - 1) = p - 3}   &&, g^{2(\frac{p-1}{2}) = p - 1 \equiv 0}  &&, \ldots, g^{2(p - 2)}    &&\} \\
\end{align*}

Because of this, $|\quadres_p| = \frac{p - 1}{2}$. Moreover, since $g^{2\frac{p-1}{2}} \equiv_p 1$ and $g^{\frac{p-1}{2}}$ cannot be 1 (since $g^0 \neq g^{\frac{p - 1}{2}} \neq g^{p - 1}$ ) but must be one of the $p-1$ elements of $\integer_p^\times$, then $g^{\frac{p-1}{2}} \equiv_p -1$.
    
Now it's possible to show that $f: \quadres_p \to \quadres_p$ is a permutation, and we are going to show how to find $f^{-1}$.

Assume $p \equiv_4 3$, meaning $\exists t : p = 4t + 3 \implies t = \frac{p - 3}{4}$); then squaring modulo $p$ becomes a permutation. Given $y \equiv_p x^2$, observe the following:

\begin{gather*}
    (y^{t + 1})^2 = y^{2t + 2} = y^{2\frac{p - 3}{4} + 2} = y^{\frac{p - 1}{2} + 1} = (x^2)^{\frac{p - 1}{2} + 1} = x^{p - 1} x^2 \equiv_p x^2 \\
    \Downarrow \\
    x = \pm y^{t + 1}
\end{gather*}

But only one among $\pm y^{t + 1}$ is a square: the positive one. Therefore:
\[ 
    p = 4t + 3 \implies \frac{p - 1}{2} = \frac{4t + 2}{2} = 2t + 1
\]
so $\frac{p - 1}{2}$ is odd.

Now, since we are considering just the elements of $\quadres_p$, and we can write each $x \in \integer_p^\times$ as $g^{z}$ for a $z \in \integer_p$:
\[
    y = x^2  = (g^z)^2
\]

So, $y = g^{z'} \in \quadres_p \Leftrightarrow z'$ is even. If $z'$ is odd, then $y \notin \quadres_p$.

Since $\frac{p - 1}{2}$ is odd, then $g^{\frac{p - 1}{2}} \notin \quadres_p$, and since it is possible to generate all of the other numbers with odd exponents 
\[
    g^{odd}=g^{\frac{p-1}{2} \pm even} = g^{\frac{p-1}{2}}g^{ \pm even} \implies -1(g^{\pm even})
\]
and $g$ powered to odd exponents will have this form. From here, it's possible to state the following:

\begin{lemma}
    $\forall z \in \quadres_p \implies -z \notin \quadres_p$
\end{lemma}

\subsection{Rabin's Trapdoor permutation}

Now we study a one way function built on previous deductions about number theory and modular arithmetic. The \emph{Rabin trapdoor permutation} is defined as: 
\[
    f(x) = x^2 \mod n
\]
where $n = pq$ for primes $p, q \equiv_4 3$.

We can observe that the image of this function is $\quadres_n$, a subset of $\integer_n^\times$.

Because of \textsc{crt}, it is possible to state that $f$ maps as follows:
\[
    x = (x_p, x_q) \implies x^2 = (x^2_p, x^2_q)
\]
since each element of $\integer_n$ has always two different forms, in $\integer_p$ and in $\integer_q$. So
\[
    y \in  \quadres_n \Leftrightarrow y_p \in \quadres_p \wedge y_q \in \quadres_q
\]

As before, the image of $f$ is exactly
\[
    \quadres_n = \{ y : \exists x : y \equiv_n x^2\}
\]

If we try to invert the function $f$, even without applying the previous inversion algorithm, we easily note that among the 4 possible values:

\[
    f^{-1}(y) = \{(x_p, x_q),(-x_p, x_q),(x_p, -x_q)(-x_p, -x_q)\}
\]

only 1 is a quadratic residue since we said, in the last lemma, that only one out of $-x_k, x_k$ is a quadratic residue for $k = q,p$.

Therefore, we have that the Rabin's \tdp{} is a permutation in $\quadres_n$, and that the cardinality of $\quadres_n$ is $\frac{|\integer_n^\times|}{4}$. Furthermore, with the following claim we can state that the Rabin cryptosystem is a \owf{} thanks to the \textsc{fact} assumption.

\begin{claim}
    Given $x$ and $z$ such that $x^2 \equiv_n z^2 \equiv_n y$:
    \[
        x \neq \pm z \implies n \text{\textup{ is factorizable}}
    \]
\end{claim}

\begin{proof}
    Since $f^{-1}(y)$ has only one value out of four, $x \neq \pm z$ and $z$ is either $\{(x_p, x_q), (-x_p, -x_q)\}$, then:
    \[
        x \in \{(x_p, -x_q), (-x_p, x_q)\} \implies x + z \in \{(0, 2x_q), (2x_p, 0)\}
    \]
    Now assume $x + z = (2x_p, 0)$ without loss of generality, since the proof for the other case is the same. We have that $x + z \equiv_q 0$ and $x + z \not\equiv_p 0$. But then $\gcd(x + z, n) = q$, and we obtain $q$.
\end{proof}

\begin{theorem}
    Squaring mod $n$, where $n$ is a \emph{Blum integer}\footnotemark is a trapdoor permutation under the factoring assumption.
\end{theorem}

\footnotetext{a Blum integer $n$ is the product of two numbers $p$ and $q$ such that $p, q \equiv_4 3$, as the definition of Rabin's \tdp}

Since we have already shown that Rabin's function is a permutation since it is invertible, we have to show that Rabin's function is also \owf:

\begin{proposition}
\[
    \textsc{fact} \implies f(x) \in \owf \qedhere
\]
\end{proposition}

\begin{proof}
    The proof is by contradiction. Assume there is an adversary \adversary{} who, given $y \equiv_n x^2$, can find an integer $z \in \integer_n$ such that $z^2 \mod n = y$ while $z \neq \pm x$. We can build a reduction as the one in figure \ref{cryptoredux:blumtdp} to show that \adversary{} chooses $x$, here $\mathcal{B}lum$ is a sampler for Blum integers:

    % AP190902: The adversary had a "fact" subscript, is it important?
    \begin{cryptoredux}
        {blumtdp}
        {---}
        {fact}
        {rabin}[1.8]

        \receive{$(p, q, n) \pickUAR \mathcal{B}lum$}{$n = pq$}{}

        \cseqdelay

        \invoke{\shortstack[l]{
            $x \pickUAR \integer_n^\times$ \\
            $y \equiv_n x^2$
        }}{$n, y$}{}

        \cseqdelay

        \return{}{$z$}{$f(z) = f(x)$}
        
    \end{cryptoredux}

    
    Once obtained $z\neq \pm x$ which $z^{2}=y$ we can use \textbf{Claim 1}(just summing $x$ and $z$ and analyzing the result)  to factorize $n$ in polytime. But factorizing $n$ in polytime is not possible.
\end{proof}

    \mychapter{16}{Lesson 16} %181121

\section{\textsc{Pke} schemes over \ddh{} assumption}

\subsection{El Gamal scheme}

Let's define a new $\Pi = (\keygen, \Enc, \Dec)$. Generate the needed public parameters $(G,g,q) \pickUAR \groupgen(1^{\lambda})$\footnotemark, then:

\footnotetext{G could be any valid group such as $\quadres_p$, or an elliptic curve group}

\begin{itemize}
    \item Key generation: $(\pk, \sk) = (g^x, x)$, where $x \pickUAR \integer_q$
    \item Encryption: $\Enc(\pk, m) = (g^r, \pk^r \cdot m)$, where $r \pickUAR \integer_q$\footnote{We need $r$ because we want to re-randomize $c$}
    \item Decryption: $\Dec(\sk, (c_1, c_2))= c_1^{-\sk} \cdot c_2$
\end{itemize}

Correctness of this scheme follows from some algebric steps:
\begin{align*}
    \hat{m} &= \Dec(\sk, \Enc(\pk, m))          \\
            &= \Dec(x, \Enc(g^x, m))            \\
            &= \Dec(x, (g^r, (g^x)^r \cdot m))  \\
            &= (g^r)^{-x} \cdot (g^x)^r \cdot m \\
            &= m                                \\
\end{align*}

\begin{theorem}
    Assuming \textsc{ddh}, the El Gamal scheme is \textsc{cpa}-secure.
\end{theorem}

\begin{proof}
    Consider the two following games $\hybridg{0}(\lambda, b)$ and $\hybridg{1}(\lambda, b)$ defined as follows. Observe that $b$ can be fixed without loss of generality.
    
    \begin{cryptogame}
        {pke1}
        {$\hybridg{0}(\lambda, b) = \cryptog{gamal}(\lambda, b)$}
        {gamal}

        \receive{\shortstack[l]{
            $x \pickUAR \integer_q$ \\
            $\pk = g^x$
        }}{$\pk$}{}

        \cseqdelay
        \send{}{$m_0, m_1$}{}
        \cseqdelay
        \receive{\shortstack[l]{
            $r \pickUAR \integer_q$ \\
            $b \pickUAR \binary$ \\
            $c = \Enc(\pk, m_b) = (g^r, (g^x)^r m_b)$
        }}{$c$}{}
        \cseqdelay

        \send{}{$b'$}{\textsc{Output 1 iff} $b' = b$}

    \end{cryptogame}

    \begin{cryptogame}
        {pke2}
        {$\hybridg{1}(\lambda, b)$}
        {$\neg$gamal}

        \receive{\shortstack[l]{
            $x \pickUAR \integer_q$ \\
            $\pk = g^x$
        }}{$\pk$}{}

        \cseqdelay
        \send{}{$m_0, m_1$}{}
        \cseqdelay
        \receive{\shortstack[l]{
            $r, z \pickUAR \integer_q$ \\
            $c = (g^r, g^z m_b)$
        }}{$c$}{}
        \cseqdelay

        \send{}{$b'$}{\textsc{Output 1 iff} $b' = b$}

    \end{cryptogame}
    

    % The reasoning was a bit cloudy, hope now it is clearer
    %AP190915: Not so sure...
    \textbf{Note:} it is important to note that we can measure the advantage of $\adversary$, so fixed its output $Adv_{\adversary}(\lambda) = |\underbrace{\Pr[\overbrace{\adversary = 1}^{b' = 1} \knowing b = 0}_{\adversary\ loses}] - \underbrace{\Pr[\overbrace{\adversary = 1}^{b' = 1} \knowing b = 1]}_{\adversary\ wins}|$. Since $b$ is fixed the above formula will give a value $\lambda \in negl$, generally the advantage of an adversary is: $\frac{1}{2}+\lambda$ (random guessing + a negligible factor). %maybe this is a repetition but I think I never wrote it anywhere else

    \todo{Here the 1/2 value may refer to how the Katz-Lindell book exposes their proofs by leaving arbitrary choice of $b$ to the challengers, thus limiting the probability of the adversaries' success by 1/2.
    
    Venturi prefers to fix $b$ beforehand, meaning the whole proof can be stated by setting b to either 0 or 1, and mantain its validity without changing anything else. In this way, he somehow bounds the probability of success to be $\leq \negl(\lambda)$}

    We will prove that:
    \[
        \hybridg{0}(\lambda, 0) \compindist \hybridg{1}(\lambda, 0) \equiv \hybridg{1}(\lambda, 1) \compindist \hybridg{0}(\lambda, 1)
    \]

    \begin{claim}
        $\forall b \in \binary \implies \hybridg{0}(\lambda, b) \compindist \hybridg{1}(\lambda, b)$
    \end{claim}

    \begin{proof}
        This proof reduces to disproving \ddh. Fix $b$ without loss of generality, and assume there exists a distinguisher $\distinguisher$ able to distinguish $H_0(\lambda,b)$ and $H_1(\lambda,b)$. Consider the game in figure \ref{cryptoredux:gamalddh}:

        \begin{cryptoredux}
            {gamalddh}
            {---}
            {ddh}
            {gamal}

            \receive{\shortstack[l]{
                $X = g^x, Y = g^y$ \\
                $Z_0 = g^{xy}, Z_1 = g^z$ \\
                $b \pickUAR \binary$
            }}{$(X, Y, Z_b)$}{}

            \invoke{}{$\pk = X$}{}
            \cseqdelay
            \return{}{$(m_0, m_1)$}{}
            \invoke{$\boxed{a = 0}$ }{$(Y, Z_b \cdot m_a)$}{}
            \cseqdelay
            \return{}{$a'$}{}

            \cseqdelay

            \send{$b' = \begin{cases}
                    0 & \textsc{iff} \boxed{a' = 0} \\
                    1 (coin?) & \textsc{else} \\
                \end{cases}
            $}{$b'$}{\textsc{Output 1 iff} $b' = b$}

        \end{cryptoredux}

        If $\distinguisher^{\textsc{gamal}}$ is able to break the cipher, then \adversary{} is in turn able to distinguish a \textsc{dh} triple from a random one, falsifying \ddh.
    \end{proof}

    \begin{claim}
        $\hybridg{1}(\lambda, 0) \equiv \hybridg{1}(\lambda, 1)$
    \end{claim}

    \begin{proof}
        This follows from the fact that:
        \[
            (g^x, (g^r, g^z m_0)) \equiv (g^x, (g^r, U_\lambda) \equiv (g^x, (g^r, g^z m_1)))
        \]
    \end{proof}
    
    Composing the two claims, the proof is complete.

\end{proof}

\subsubsection{Properties of of El Gamal \textsc{pke} scheme}

Some useful observations can be made about this scheme:
\begin{itemize}
    \item It is \textbf{homomorphic}: Given two ciphertexts $(c_1, c_2)$ and $(c_1', c_2')$, then doing the product between them yields another valid ciphertext:
    % I'd like to expand a bit more here, or else it may stay a bit obscure...
    \begin{align*}
        & (c_1 \cdot c_1', c_2 \cdot c_2') \\
        =& (g^{r + r'}, h^{r + r'}(m \cdot m'))
    \end{align*}
    thus, decrypting $c \cdot c'$, gives $m \cdot m'$.
    
    \item It is \textbf{re-randomizable}: Given a ciphertext $(c_1, c_2)$, and $r' \pickUAR \integer_q$, then computing $(g^{r'} \cdot c_1, h^{r'} \cdot c_2)$ results in a ``fresh'' encryption for the same message: the random value used at the encryption step will change from the original $r$ to $r + r'$
\end{itemize}
These observations lead to the conclusion that this scheme is not \cca-secure. However, such properties can be desirable in some use cases, where a message must be kept secret to the second party. In fact, there are some \pke{} schemes which are designed to be \textbf{fully homomorphic}, i.e. they are homomorphic for any kind of function.

Consider the following use case: a client $C$ has an object $x$ and wants to apply a function $f$ over it, but it lacks the computational power to execute it. There is another subject $S$, which is able to efficiently compute $f$, so the goal is to let it compute $f(x)$ but the client wishes to keep $x$ secret from him. This can be achieved using a \textsc{fh-pke} scheme as follows:

% QUESTION: Why use a PKE scheme (and not a SKE)? Are there some advantages-shortcomings?

\begin{cryptosequence}
    {delseccomp}
    {Delegated secret computation}

    \cseqentity{C}{Client}
    \cseqentity[2.2]{S}{Server}

    \cseqdelay

    % Client encrypts object and sends application request
    \cseqmessager{C}{\shortstack[r]{
        Object: $x$ \\
        Function: $f$ \\
        $(\pk, \sk) \pickUAR \mathcal{KG}en$ \\
        $c \pickUAR \Enc(\pk, x)$
    }}{$f, c$}{S}{}

    \cseqdelay
    \cseqdelay

    \cseqmessagel{S}{}{$f(c)$}{C}{$f(x) = \Dec(\sk, f(c))$}

\end{cryptosequence}


However one important consideration must be made: All these useful characteristics expose an inherent malleability of any fully homomorphic scheme: any attacker can manipulate ciphertexts efficiently, and with some predictable results. This compromises even \textsc{cpa} security of such schemes.

%\subsection{Cramer-Shoup \textsc{pke} scheme}
\section{Proof systems}

% Not sure if CS is defined over DDH, or this is the DDH variant of CS, making CS more abstract
This new scheme assumes \ddh{} like its El Gamal cousin, and has the advantage of being \cca-secure. A powerful tool, called \emph{Designated Verifier Non-Interactive Zero-Knowledge} (\textsc{dv-nizk} in short), or alternatively \emph{Hash-Proof System}, is used here.


\todo{the Ali Baba example would be useful here. Also, the coloured balls is a good example too (long live Wikipedia)}

Let $L \subseteq Y$ be a Turing-recognizable language in \textsc{np}, and a predicate $V \in X \times Y \to \binary$ such that:
\[
    L := \{y \in Y : \exists x \in X \implies V(x, y) = 1\}
\]
where $x$ is called a \emph{witness} of $y$.

\todo{to review and understand/better}

% AP181122-1554: CAUTION FRAGMENTED
In our case, let $x = (p, q), y = pq$, and define a scheme $\Pi$ as follows:

\[
    \Pi = (\mathcal{S}etup, \mathcal{P}rove, \mathcal{V}erify)
\]

% AP190914: Is (omega, tau) the (pk, sk) couple?
\begin{itemize}
    \item Setting up: $(\omega, \tau) \pickUAR \mathcal{S}etup(1^\lambda)$, where $\omega$ is the \emph{common reference string}, and $\tau$ is the \emph{trapdoor}
    \item Proving a statement: $\pi = \mathcal{P}rove(\omega, y, x)$
    \item Verifying: $\widehat{\pi} = \mathcal{V}erify(\tau, y)$
\end{itemize}

Correctness is defined as $\mathcal{P}rove(\omega, y, x) = \mathcal{V}erify(\tau, y)$. Some more observations:

\begin{itemize}
    \item $\omega$ is public ($ = pk$)
    \item $\tau$ is part of the secret key
    \item $\tau = (x, y) : V(x, y) = 1$ %May be false
    \item There is presumably a common third-party, which samples from the setup and publishes $\omega$, while giving $\tau$ to only B.
    \todo{Why? Can't the verifier do the setup and publish omega directly? Is honesty an issue here?}
\end{itemize}

\begin{cryptosequence}
    {csoverview}
    {Overview of Cramer-Shoup operation}
    
    \cseqentity{P}{Prover}
    \cseqentity[2.2]{V}{Verifier}

    \cseqdelay

    \cseqmessager{P}{\shortstack[r]{
        $\omega$\\
        $(x,y)$ s.t. $R(x,y)=1$
    }}{$\pi = \mathcal{P}rove(\omega, y, x)$}{V}{\shortstack[l]{
        $\tau, y' \in L$\\
        $Ver(\tau,y)=\tilde{\pi}\in P$\\
        check if $\pi = \tilde{\pi}$
        %Object: $x$ \\
        %Function: $f$ \\
        %$(pk, sk) \pickUAR \mathcal{KG}en$ \\
        %$c \pickUAR Enc(pk, x)$
    }}
    
\end{cryptosequence}

Purpose of a proof system is to give a way to convince someone (the ``verifier'') that someone else (the ``prover'') knows something (the ``statement'' $y$), and nothing more, to no one else. The proof can be computed in two different ways, this is the core notion of \emph{zero-knowledge}, $\neg \tau \implies \textsc{zk}$

\subsubsection{Properties}

\begin{itemize}
    \item \textit{honest people} \textbf{Completeness}:
    \[
        \forall y \in L, \forall (\omega, \tau) \pickUAR \mathcal{S}etup(1^\lambda) \implies \mathcal{P}rove(\omega, y, x) = \mathcal{V}erify(\tau, y)
    \]
    \item \textit{(stronger, against malicious prover)} \textbf{Soundness}: It is hard to produce a valid proof for any $y \notin L$
    \item \textit{(characteristic, against malicious verifier)} \textbf{Zero-knowledge}: Proof for $x$ can be simulated without knowing $x$ itself
\end{itemize}


\begin{definition}[$t$-universality, or $t - 1$ simulation soundness]
    % Designated verifier non interactive zero knowledge
    Let $\Pi$ be a \textsc{dv-nizk} scheme; it is \textit{t-universal} iff for any distinct $y_{1 \upto t} \notin L$ we have: 
    \[
        (\omega, \mathcal{V}erify(\tau, y_1), \ldots, \mathcal{V}erify(\tau, y_{t})) = (\omega, v_1, \ldots, v_t)
    \]
    where $(\omega, \tau) \pickUAR \mathcal{S}etup(1^\lambda)$ and $v_{1 \upto t} \pickUAR \mathcal{P}roof$, where $\mathcal{P}roof$ is the proofs' space.
\end{definition}

\subsubsection{Enriching a proof system}

We can ``enrich'' a \textsc{dv-nizk} scheme with labels $l \in \binary^*$. Suppose to have the following:
\[
    L' = L \| \binary^* = \{(y, l) \in L \times \binary^*\}
\]
Then our scheme changes, because the statements are now compositions of a label $l$ along with te actual statement $y$; for the \textit{t-universality} property, we can now consider two distinct $(y_i, l_i)$.

\subsubsection{Membership-hard language}

\begin{definition}
    Language $L$ is \emph{membership-hard} iff there exists a language $\overline{L}$ such that:
    \begin{enumerate}
        \item $L \cap \overline{L} = \emptyset$
        \item $\exists \textsf{\textup{Sample}} \in \ppt$ outputting $y \pickUAR \mathcal{Y}$ together with $x \in \overline{X}$ such that $R(y, x) = 1$
        
        (therefore $(y, x) \pickUAR \textsf{\textup{Sample}}(1^{\lambda})$)
        \item $\exists \overline{\textsf{\textup{Sample}}} \in \ppt$ outputting $y \pickUAR \overline{L}$
        \item $\{y : (y, x) \pickUAR \textsf{\textup{Sample}}(1^{\lambda})\} \compindist \{y : y \pickUAR \overline{\textsf{\textup{Sample}}}(1^{\lambda})\}$
    \end{enumerate}
    
\end{definition}


    \input{lessons/lesson_17.tex}
    \mychapter{18}{Lesson 18} %181128

\subsubsection{Cramer-Shoup scheme construction}

From the above two proof systems we can construct a \pke{} scheme, which is attributed to Cramer and Shoup:

\todo{split definition from correctness}

\begin{itemize}
    \item $(\pk, \sk) \pickUAR \keygen$, where:
    \begin{itemize}
        \item $\pk := (h_1, h_2, h_3) = (g_1^{x_1} g_2^{x_2}, g_1^{x_3} g_2^{x_4}, g_1^{x_5} g_2^{x_6})$
        \item $\sk := (x_1, x_2, x_3, x_4, x5, x_6)$
    \end{itemize}
    \item Encryption procedure:
    \begin{itemize}
        \item $r \pickUAR \integer_q$
        \item $\beta = h_s(c_1, c_2, c_3) = (g_1^r, g_2^r, h_1^r m)$
        \item $\Enc(\pk, m) = (c_1, c_2, c_3, (h_2 h_3^\beta)^r)$
    \end{itemize}
    \item Decryption procedure:
    \begin{itemize}
        \item Check that $c_1^{x_3 + \beta x_5} c_2^{x_4 + \beta x_6} = c_4$. If not, output $\perp$.
        \item Else, output $\widehat{m} = c_3 c_1^{-x_1} c_2^{-x_2}$
    \end{itemize}
\end{itemize}

\todo{All the proofs here...}

\section{Digital signatures}

In this section we explore the solutions to the problem of authentication using an asymmetric key method. Some observations are in order:

\begin{figure}
    \centering

    \begin{tikzpicture}[node distance = 2cm, auto, >=latex']

        \node (s) {};
        \node (a) [box] [right of = s] {Alice};
        \node (k) [above of = a, node distance = 1cm] {$sk$};
        \node (b) [box] [right of = a, node distance = 5cm] {Bob};
        \node (p) [above of = b, node distance = 1cm] {$pk$};
        \node (e) [right of = b] {};

        \path[->] (s) edge node {$m$} (a);
        \path[->] (k) edge (a);
        \path[->] (a) edge node {$(m, \sigma)$} (b);
        \path[->] (p) edge (b);
        \path[->] (b) edge node {$b$} (e);

    \end{tikzpicture}
    \caption{Asymmetric authentication}
    \label{fig:digisign}
\end{figure}

\begin{itemize}
    \item In a symmetric setting, a verifier routine could be banally implemented as recomputing the signature using the shared secret key and the message. Here, Bob cannot recompute $\sigma$ as he's missing Alice's secret key (and for good reasons too...). Thus, the verifying routine must be defined otherwise;
    \item In a vaguely similar manner to how an attacker could encrypt messages by itself in the asymmetric scenario, because the public key is known to everyone, any attacker can verify any signed messages, for free.
\end{itemize}

Nevertheless, proving that a \textsc{ds} scheme is secure is largely defined in the same way as in the symmetric scenario, with the \textsc{uf-cma} property:

\begin{cryptogame}{dsufcma}{Unforgeable digital signatures}{uf-cma}
    
    \receive{$(\pk, \sk) \pickUAR \keygen(1^\lambda)$}{$\pk$}{}

    \cseqdelay

    \send{$m \in M$}{$m$}{}
    \receive{$\sigma \pickUAR \textsf{\textup{Sign}}(sk, m)$}{$\sigma$}{}

    \cseqdelay

    \send{$m^* \notin M$}{$(m^*, \sigma^*)$}{\textsc{Output} $\textsf{\textup{Verify}}(\pk, m^*, \sigma^*)$}

\end{cryptogame}

\subsection{Public Key Infrastructure}

The problem now is that Alice has a public key, but she wants some sort of ``certificate of validity'' for it, so that Bob will be sure that whenever he receives Alice's public key, he can be sure it's the right one by checking such certificate.

For certificates to be useful, the parties need an universally-trusted third party, called \textit{Certification Authority}. It will provide a special \textit{signature} to Alice for proving her identity to Bob, as exemplified by the sequence in figure \ref{cryptosequence:certissue}.

Whenever Bob wants to check the validity of the Alice's public key, he can query the authority for the certificate, and verify the public key he just received, as shown in figure \ref{cryptosequence:certcheck}

\begin{cryptosequence}
    {certissue}
    {}

    \cseqentity{CA}{CA}
    \cseqentity[2.2]{A}{Alice}

    \cseqmessager{CA}{$(\pk_C, \sk_C) \pickUAR \keygen$}{$\pk_C$}{A}{}

    \cseqdelay

    \cseqmessagel{A}{\shortstack[l]{
        $(\pk_A, \sk_A) \pickUAR \keygen$ \\
        $\textsc{id}_A = \textup{``Alice''} \| \pk_A$
        }}{$\textsc{id}_A$}{CA}{}
    
    \cseqdelay

    \cseqmessager{CA}{$\textsc{cert}_A \pickUAR \textsf{\textup{Sign}}(\sk_C, \textsc{id}_A)$}{$\textsc{cert}_A$}{A}{}
    
\end{cryptosequence}

\begin{cryptosequence}
    {certcheck}
    {}

    \cseqentity{CA}{CA}
    \cseqentity[0.8]{A}{Alice}
    \cseqentity[1.4]{B}{Bob}

    \cseqmessager{CA}{\shortstack[r]{
        \small{$(\pk_C, \sk_C) \pickUAR$} \\
        \small{$\pickUAR \keygen$}
    }}{$\pk_C$}{A}{}
    \cseqmessager{CA}{}{$\pk_C$}{B}{}

    \cseqdelay

    \cseqmessagel{A}{\shortstack[l]{
        $(\pk_A, \sk_A) \pickUAR \keygen$ \\
        $\textsc{id}_A = \textup{``Alice''} \| \pk_A$
    }}{$\textsc{id}_A$}{CA}{}

    \cseqdelay

    \cseqmessager{CA}{\shortstack[r]{
        \small{$\textsc{cert}_A \pickUAR$} \\
        \small{$\pickUAR \textsf{\textup{Sign}}(\sk_C, \textsc{id}_A)$}
    }}{$\textsc{cert}_A$}{A}{}
    
    \cseqdelay

    \cseqmessager{A}{}{$(\textsc{id}_A, \textsc{cert}_A)$}{B}{\small{$b = \textsf{\textup{Auth}}(\pk_C, \textsc{id}_A, \textsc{cert}_A)$}}

\end{cryptosequence}

How can Bob recognize a valid certificate from an expired/invalid one? The infrastructure provides some servers which contain the lists of the currently valid certificates, such as $\textsc{cert}_A$, in the case of Alice.

\begin{theorem}
    Signatures are in \textit{\textbf{Minicrypt}}.
\end{theorem}

This is a counterintuitive result, not proven during the lesson, but very interesting because it implies that we can create valid signatures only with hash functions, without considering at all public key encryption.


Up next:
\begin{itemize}
    \item Digital Signatures from TDP*
    \item Digital Signatures from ID Scheme*
    \item Digital Signatures from CDH
\end{itemize}

Where * appears, something called \textit{Random Oracle Model} is used in the proof. Briefly, this model assumes the existance of an ideal hash function which behaves like a truly random function (outputs a random y as long as x was never queried, otherwise gives back the already taken y).

    \mychapter{19}{Lesson 19} %181130

\todo{E' venuto fuori un casino in questa lezione, ho cercato di riordinare le cose}

\subsubsection{Bilinear Map}

\begin{definition}
    
    Let's define a \emph{bilinear group} as $(G, G_t, q, g, \hat{e}) \pickUAR \mathcal{B}ilin(1^\lambda)$, where:
    \begin{itemize}
        \item $G, G_t$ are prime order groups (order $q$).
        \item $g$ is a generator of $G$ picked \uar.
        \item $(G, \cdot)$ is a multiplicative group and $G_t$ is the ``target'' group.
        \item $\hat{e}$ is an efficiently computable \emph{bilinear} map: $G \times G \rightarrow G_t$ defined as such:
        \[
            \forall h \in G \forall a, b \in \integer_q \implies \hat{e}(g^a, h^b) = \hat{e}(g, h)^{ab} = \hat{e}(g^{ab}, h)
        \]
        provided that $\hat{e}(g, g) \neq 1$, that is, $\hat{e}$ is \emph{non-degenerative}
    \end{itemize}

    To put it in simple words, the exponents can move.
\end{definition}

\todo{Venturi said something here related to Weil pairing over an elliptic curve. I found \href{https://www.math.auckland.ac.nz/~sgal018/crypto-book/ch26.pdf}{this}. Interesting but not useful.}

It must be noted that the \ddh{} assumption is easy for the group $G$: suffice to see that $\hat{e}(g^a, g^b) = \hat{e}(g, g^c)$ is true iff $c = ab$. On the other hand:

\begin{proposition}
    \cdh{} is hard for $G$.
\end{proposition}

\begin{proof}

    Now $\keygen(1^\lambda)$ will:
    \begin{itemize}
        \item Generate some params: $(G, G_t, g, q, \hat{e}) \pickUAR \mathcal{B}ilin(1^\lambda)$ 
        \item $a \pickUAR \integer_q$, then $g_1 = g^a$
        \item Pick $g_2 = g^b$ and $g_2, u_{0 \upto k} \pickUAR G$.
        \item Then output: 
        \begin{itemize}
            \item $\pk = (\text{params}, g_1, g_2, u_0, ..., u_k)$
            \item $\sk = g_2^a = g^{ab}$
        \end{itemize}
    \end{itemize}

    $\textsf{\textup{Sign}}(\sk, m)$:
    \begin{itemize}
        \item Divide the message $m$ of length $k$ in single bits as follows: $m = (m_{1 \upto k})$
        \item Now define $\alpha(m) = u_0\prod_{i = 1}^k u_i^{m_i}$
        \item Pick $r \pickUAR \integer_q$ and output the signature $\sigma = (\sk \cdot \alpha(m)^r,g^r)=(\sigma_1,\sigma_2)$
    \end{itemize}

    $\textsf{\textup{Ver}}(\pk, m, (\sigma_1, \sigma_2))$:
    \begin{itemize}
        \item Check $\hat{e}(g, \sigma_1) = \hat{e}(\sigma_2, \alpha(m)) = \hat{e}(g_1, g_2)$
    \end{itemize}

    The scheme's correctness is proven my ``moving'' the exponents:
    \begin{align*}
        \hat{e}(g, \sigma_1) &= \hat{e}(g,g_2^a\cdot \alpha(m)^r)   & \\
        &= \hat{e}(g,g_2^a) \cdot \hat{e}(g,\alpha(m)^r)            & \text{(Bilinearity)}\\
        &= \hat{e}(g^a,g_2)\cdot \hat{e}(g^r,\alpha(m))             & \\
        &= \hat{e}(g_1,g_2)\cdot\hat{e}(\sigma_2,\alpha(m))         &
    \end{align*}
    
    We can say that we are moving the exponents from the "private domain" to the "public domain".

\end{proof}

\section{Waters signatures}
\begin{theorem}
    The Waters' signature scheme is \ufcma-secure
\end{theorem}

\begin{proof}
    \begin{quote}
        The trick is to ``program the 'u's'' (Venturi)
    \end{quote}

    \todo{Sequence is incomplete/incorrect, have to study it more...}


    % Original subjects: A, C_WDS, C_CDH
    \begin{cryptoredux}
        {reduxwaters}
        {Reducing Waters' scheme to \cdh}
        {cdh}
        {wds}

        \cseqdelay

        % Instantiate game
        \receive{\shortstack[l]{
            ``params''$ = $ \\
            $ \quad =(\G, \G_t, g, q, \hat{e})$ \\
            $ \quad \pickUAR BilGen(1^\lambda)$ \\
            $a, b \pickUAR \integer_q$
        }}{$(\text{``params''}, g^a, g^b)$}{}

        \cseqdelay

        % Send pk to adversary
        \invoke{Construct $u$ string}{$(\text{``params''}, g^a, g^b, u)$}{}
        
        % Adversary can query for signatures
        \cseqdelay
        \return{$m \in M$}{$m$}{}
        \invoke{$\sigma = \textit{Sign}()$}{$\sigma$}{}
        \cseqdelay

        % Adversary forges
        \return{$m^* \notin M$}{$(m^*, \sigma^*)$}{}
        
    \end{cryptoredux}

    \todo{The following explanation is roundabout, will rectify later}

    The following describes how the Waters' challenger constructs the $u$ string. The main idea is, given $k$ as the message length, to choose every single bit of $u$ from 1 up to $k$ such that:

    \begin{equation*}
        \alpha(m) = g_2^{\beta(m)}g^{\gamma(m)},\quad \beta(m) = x_0 + \sum_{i=1}^{k}m_ix_i,\quad \gamma(m) = y_0 + \sum_{i=1}^{k}m_iy_i
    \end{equation*}

    where $x_0 \pickUAR \{-kl, \dots, 0\}, x_{1\upto k} \pickUAR \{0, \dots, l\}, y_{0\upto k} \pickUAR \integer_q$

    In particular: $l = 2q_s$, where $q_s$ is the number of sign queries made by the adversary.

    Therefore, let $u_i = g_2^{x_i}g^{y_i} \quad \forall i \in [0, k]$. Then:

    \begin{equation*}
        \alpha(m) = g_2^{x_0}g^{y_0} \prod_{i=1}^k (g_2^{x_i}g^{y_i})^{m_i} = g_2^{x_0+\sum_{i=1}^k m_ix_i}g^{y_0+\sum_{i=1}^k m_iy_i} = g_2^{\beta(m)}g^{\gamma(m)}
    \end{equation*}

    \todo{Partizioni, doppi if.... qui non ci ho capito 'na mazza}

    \textbf{Step 1}: $\sigma = (\sigma_1, \sigma_2) = (g_2^a \alpha(m)^{\bar{r}}, g^{\bar{r}})$, for $\bar{r} \pickUAR \integer_q \quad \bar{r} = r - a\beta^{-1}$

    \begin{align*}
        \sigma_1 &= g_2^a\alpha(m)^{\bar{r}} \\
        &= g_2^a\alpha(m)^{r - a\beta^{-1}} \\
        &= g_2^a (g_2^{\beta(m)} g^{\gamma(m)})^{r - a\beta^{-1}} \\
        &= g_2^a g_2^{\beta(m)r-a} g^{\gamma(m)r - \gamma(m)a\beta^{-1}} \\
        &= g_2^{\beta(m)r} g^{\gamma(m)r} g^{-\gamma(m)\beta^{-1}}
    \end{align*}

\end{proof}

%https://eprint.iacr.org/2011/703.pdf


    \part{Proof-based schemes}

    \mychapter{20}{Lesson 20} %181205

\section{Random Oracle Model (ROM)}

The Random Oracle Model treats a given hash function $H$ as a truly random function. As a reminder: a truly random function $R$ is defined to have a specific evaluation behaviour. They do act, in fact, as truth tables\footnotemark:

\begin{itemize}
    \item if the argument hasn't been submitted to the function beforehand, then a value is chosen \uar{} from the codomain, and assigned as the image of said argument in the function;
    \item otherwise, the function will return the image as assigned in the corresponding previous evaluation.
\end{itemize}

\footnotetext{Such tables are also aptly called \emph{rainbow tables}.}

\subsection{Full domain hashing}

Let $(f, f^{-1}, \Gen)$ be a \tdp{} scheme over some domain $\mathcal{X}_{\pk}$

Take \rsa:
\begin{itemize}
    \item $(m, \pk, \sk) \pickUAR \Gen\rsa(1^\lambda)$
    \item $f(\pk, x) = x^{\pk} \mod n$
    \item $f^{-1}(\sk, y) = y^{\sk} \mod n$
\end{itemize}

Build a similar asymmetric-authentication scheme as such:

\begin{itemize}
    \item $(m, \pk, \sk) \pickUAR \Gen\rsa(1^\lambda)$
    \item $Sign_{\sk, H}(x): \sigma = f^{-1}(\sk, H(m))$
    \item $Verify_{\pk, H}: H(m) = f(pk, \sigma)$
\end{itemize}

\begin{exercise}
    Show \rsa-sign is not secure without $H$. (Hint: The scheme becomes malleable)
\end{exercise}

\begin{theorem}
    If the above scheme (full-domain hash) uses a \tdp{} for $f, f^{-1}$, then it is (asymmetric)-\ufcma under the random oracle model.
\end{theorem}

\begin{proof}
    Idea: Reduce to \tdp, program the random oracle.

    \begin{cryptoredux}
        {fdhufcma}
        {}
        {tdp}
        {ufcma(a)}

        \receive{\shortstack[l]{
            $(pk, sk) \pickUAR \Gen(1^\lambda)$ \\
            $x \pickUAR \mathcal{X}_{pk}$ \\
            $y = f(pk, x)$
        }}{$pk, y$}{}

        \invoke{$*$}{$pk$}{}

        \cseqdelay
        \cseqbeginloop
        \return{}{$m$}{}
        \invoke{}{$\sigma$}{}
        \cseqendloop
        \cseqdelay

        \return{}{$(m^*, \sigma^*)$}{}

        \send{}{$\sigma^*$}{}
    \end{cryptoredux}

    Notes: this is a loose reduction

    Some assumptions are made:

    \begin{itemize}
        \item The adversary makes the same number of RO queries as the number of signing queries done by the distinguisher (without loss of generality)
        \item The RO query must be done \emph{before} the corresponding sign query, otherwise the adversary cannot sign the messages, as specified by the scheme
    \end{itemize}

    The RO queries are actually an analogue of the definition of a random function, and it is the \emph{programming} step of the oracle itself; then if the signing queries do not correspond to any RO query, abort the game.

\end{proof}

\section{ID Scheme}

\subsubsection{``Sigma'' protocol}

\subsection{Fiat-Shamir scheme}

\subsubsection{Honest Verifier Zero-Knowledge (HVZK)}

\subsubsection{Special Soundness (SS)}

    \input{lessons/lesson_21.tex}
    \input{lessons/lesson_22.tex}
    \input{lessons/lesson_23.tex}
    \input{lessons/lesson_24.tex}
\end{document}
