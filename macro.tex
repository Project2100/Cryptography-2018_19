\usepackage{titletoc}


\usepackage{amsmath}
\usepackage{amsfonts}
\usepackage{amsthm}
\usepackage{amssymb}
\usepackage{thmtools}
\usepackage{acronym}
\usepackage{multirow}
\usepackage{cleveref}
\usepackage{graphicx}
\usepackage{mathtools}
\usepackage{cancel}
\usepackage{ifthen}
\usepackage{etoolbox}
\usepackage{xcolor}
\usepackage{subcaption}
\usepackage{calc}
\usepackage{preview}
\usepackage{color}
\usepackage{framed}
\usepackage{stackengine}
\usepackage{scalerel}
\usepackage{enumerate}


\newcommand{\mychapter}[2]{
    \setcounter{chapter}{#1}
    \setcounter{section}{0}
    \chapter*{#2}
    \addcontentsline{toc}{chapter}{#2}
}

% Range ligature
\newcommand{\upto}{\text{---}}

% Remainder operator, redefined in a sane way
\renewcommand{\mod}{\mathrel{\textup{mod}}}

% Group generators for standard assumptions
\newcommand{\groupgen}{\mathcal{GG}}

% Quadratic residue group
\newcommand{\quadres}{\mathcal{Q}uad}

% Key generator for public-privte pairs
\newcommand{\keygen}{\mathcal{KG}}

% Grouped or
\newcommand{\biglor}{\bigvee}

% XORs
\newcommand{\bigxor}{\bigoplus}
\newcommand{\xor}{\oplus}

\newcommand{\C}{\mathcal{C}} % Cipherspace
\newcommand{\G}{\mathcal{G}} % Generally used for generators or groups
\newcommand{\K}{\mathcal{K}} % Keyspace
\newcommand{\M}{\mathcal{M}} % Messagespace
\newcommand{\X}{\mathcal{X}} % May be a RV X's domain, will soon change to the more common \Omega

\newcommand{\RO}{\mathrm{RO}} % Random oracle

%\newcommand{\QR}[1][p]{\mathbb{QR}_{#1}} % Quadratic residue group

\newcommand{\Gen}{\mathrm{Gen}} % Generic generator
%\newcommand{\Group}{\mathbb{G}} % Generic group


%========================================================================================================


\theoremstyle{plain}
\declaretheorem[qed=\ensuremath{\diamond}]{theorem}
\declaretheorem[qed=\ensuremath{\diamond}]{lemma}
\declaretheorem[qed=\ensuremath{\diamond}]{corollary}
\declaretheorem[qed=\ensuremath{\diamond}]{fact}
\declaretheorem[qed=\ensuremath{\diamond}]{claim}
\declaretheorem[qed=\ensuremath{\diamond}]{observation}
\declaretheorem[qed=\ensuremath{\diamond}]{proposition}

\crefname{thm}{theorem}{theorems}
\Crefname{thm}{Theorem}{Theorems}

\crefname{lem}{lemma}{lemmas}
\Crefname{lem}{Lemma}{Lemmas}

\crefname{cor}{corollary}{corollarys}
\Crefname{cor}{Corollary}{Corollarys}

\crefname{fct}{fact}{facts}
\Crefname{fct}{Fact}{Facts}

\crefname{clm}{claim}{claims}
\Crefname{clm}{Claim}{Claims}

\crefname{obs}{observation}{observations}
\Crefname{obs}{Observation}{Observations}

\crefname{prop}{proposition}{propositions}
\Crefname{prop}{Proposition}{Propositions}


\theoremstyle{definition}
\declaretheorem[qed=\ensuremath{\diamond}]{definition}
\declaretheorem[qed=\ensuremath{\diamond}]{construction}

\crefname{defn}{definition}{definitions}
\Crefname{defn}{Definition}{Definitions}

\crefname{cons}{construction}{constructions}
\Crefname{cons}{Construction}{Constructions}


% \theoremstyle{plain}
\newtheorem{thm}{Theorem}
\newtheorem{lem}{Lemma}
\newtheorem{cor}{Corollary}
\newtheorem{fct}{Fact}
\newtheorem{clm}{Claim}
\newtheorem{obs}{Observation}
\newtheorem{prop}{Proposition}
\newtheorem{question}{Question}
\newtheorem{exercise}[theorem]{Exercise}
\newtheorem{solution}{Solution}
\newtheorem{example}{Example}
% \theoremstyle{definition}
\newtheorem{defn}{Definition}
\newtheorem{cons}{Construction}


%----------------------IMAGES-----------------------------------
\usetikzlibrary{tikzmark,decorations.pathreplacing,positioning,shapes,fit,arrows}

\newcounter{listcount} \newcounter{totcount}
\newcommand{\printarray}[2][1em]{% \printarray[<width>]{<array list>}
    \unskip \setcounter{totcount}{0}% Reset totcount counter
    \renewcommand*{\do}[1]{\stepcounter{totcount}}% Count elements
    \docsvlist{#2}% Process list a first time to obtain # of elements
    \setcounter{listcount}{0}% Reset listcount counter
    \renewcommand*{\do}[1]{%
        \stepcounter{listcount}% Move to next element
        \framebox[#1][c]{\rule{0pt}{1.5ex}\smash{\ensuremath{##1}}}%
        \ifnum\value{listcount}<\value{totcount}\thickspace\fi
    }
    \docsvlist{#2}% Process list a second time to typeset each element
}
