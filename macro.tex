% Theoremstyle
\usepackage{amsthm}

% Math?lap commands
\usepackage{mathtools}

% docsvlist
\usepackage{etoolbox}

% Shaded environment
\usepackage{xcolor}
\usepackage{framed}


\newcommand{\mychapter}[2]{
    \setcounter{chapter}{#1}
    \setcounter{section}{0}
    \chapter*{#2}
    \addcontentsline{toc}{chapter}{#2}
}

% Range ligature
\newcommand{\upto}{\text{---}}

% Remainder operator, redefined in a sane way
\renewcommand{\mod}{\mathrel{\textup{mod}}}

% Group generators for standard assumptions
\newcommand{\groupgen}{\mathcal{GG}}

% Quadratic residue group
\newcommand{\quadres}{\mathcal{Q}uad}

% Key generator for public-privte pairs
\newcommand{\keygen}{\textsf{\textup{Keygen}}}

\newcommand{\C}{\mathcal{C}} % Cipherspace
\newcommand{\G}{\mathcal{G}} % Generally used for generators or groups
\newcommand{\K}{\mathcal{K}} % Keyspace
\newcommand{\M}{\mathcal{M}} % Messagespace

\newcommand{\RO}{\mathrm{RO}} % Random oracle

\newcommand{\Gen}{\mathrm{Gen}} % Generic generator

% One-Time Pad
\newcommand{\otp}{\textsc{otp}}


%========================================================================================================


\newtheorem{theorem}                    {Theorem}       [chapter]
\newtheorem{lemma}          [theorem]   {Lemma}
\newtheorem{corollary}      [theorem]   {Corollary}
\newtheorem{claim}          [theorem]   {Claim}
\newtheorem{observation}    [theorem]   {Observation}
\newtheorem{proposition}    [theorem]   {Proposition}
\theoremstyle{definition}
\newtheorem{definition}     [theorem]   {Definition}
\newtheorem{exercise}       [theorem]   {Exercise}
\newtheorem{example}        [theorem]   {Example}
\newtheorem{construction}               {Construction}
\theoremstyle{remark}
\newtheorem{solution}       [theorem]   {Solution}


%----------------------IMAGES-----------------------------------
\usetikzlibrary{tikzmark,decorations.pathreplacing,positioning,shapes,fit}

\newcounter{listcount} \newcounter{totcount}
\newcommand{\printarray}[2][1em]{% \printarray[<width>]{<array list>}
    \unskip \setcounter{totcount}{0}% Reset totcount counter
    \renewcommand*{\do}[1]{\stepcounter{totcount}}% Count elements
    \docsvlist{#2}% Process list a first time to obtain # of elements
    \setcounter{listcount}{0}% Reset listcount counter
    \renewcommand*{\do}[1]{%
        \stepcounter{listcount}% Move to next element
        \framebox[#1][c]{\rule{0pt}{1.5ex}\smash{\ensuremath{##1}}}%
        \ifnum\value{listcount}<\value{totcount}\thickspace\fi
    }
    \docsvlist{#2}% Process list a second time to typeset each element
}
